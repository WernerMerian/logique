% file: base-preamble-fr.tex



%%%%%%%%%%%%%%%%%%%%%%%%%%%%%%%%%%%%%%%%%%%%%%%%
%----- MISE EN FORME DES CARACTÈRES (1/2) -----%
%%%%%%%%%%%%%%%%%%%%%%%%%%%%%%%%%%%%%%%%%%%%%%%%

%----- FONTES -----%

\usepackage{lmodern}



%%%%%%%%%%%%%%%%%%%%%%%
%----- ENCODAGES -----%
%%%%%%%%%%%%%%%%%%%%%%%

%----- ENCODAGE EN SORTIE -----%

\usepackage[T1]{fontenc}

\usepackage[T1]{tipa}


%----- ENCODAGE EN ENTRÉE -----%

\usepackage[utf8]{inputenc}



%%%%%%%%%%%%%%%%%%%%%%%%%%%%%%%%%%%%%%%%%%%%%%%%
%----- FORMATAGE DU CONTENU TEXTUEL (1/2) -----%
%%%%%%%%%%%%%%%%%%%%%%%%%%%%%%%%%%%%%%%%%%%%%%%%

%----- FORMATAGE DES CODES SOURCES -----%

\usepackage{minted}
\usemintedstyle{emacs}

\usepackage[linesnumbered, ruled, vlined, nofillcomment]{algorithm2e}



%%%%%%%%%%%%%%%%%%%%%%%%%%%%%%%%%%%%%%%%%%%%%%%%
%----- MISE EN FORME DES CARACTÈRES (2/2) -----%
%%%%%%%%%%%%%%%%%%%%%%%%%%%%%%%%%%%%%%%%%%%%%%%%

%----- FONTES -----%

\usepackage{lettrine}


%----- TYPOGRAPHIE -----%

\usepackage[french]{babel}

\usepackage{microtype}

\usepackage{csquotes}



%%%%%%%%%%%%%%%%%%%%%%%%%%%
%----- MISE EN AVANT -----%
%%%%%%%%%%%%%%%%%%%%%%%%%%%

%----- SOULIGNEMENT -----%

\usepackage{soul}



%%%%%%%%%%%%%%%%%%%%%%%%%%%%%%%%%%%%%%
%----- MISE EN FORME DES TITRES -----%
%%%%%%%%%%%%%%%%%%%%%%%%%%%%%%%%%%%%%%

%----- STYLE DES TITRES -----%

\usepackage[Lenny]{fncychap}


%----- NUMÉROTATION DES TITRES -----%

\setcounter{secnumdepth}{3}



%%%%%%%%%%%%%%%%%%%%%%%%%%%%%%%%%%%%%%%%%%%
%----- MISE EN FORME DES PARAGRAPHES -----%
%%%%%%%%%%%%%%%%%%%%%%%%%%%%%%%%%%%%%%%%%%%

%----- LISTES -----%

\usepackage[inline]{enumitem}
\setlist[itemize, 1]{label=\textbullet}
\setlist[itemize, 2]{label=$\circ$}
\setlist[itemize, 3]{label={\tiny$\blacksquare$}}
\setlist[itemize, 4]{label={\tiny$\square$}}

\usepackage{moreenum}


%----- NUMÉROTATION DES LISTES ORDONNÉES DANS LES THÉORÈMES -----%

\newlist{enumeratedef}{enumerate}{1}
\setlist[enumeratedef]{
	label = {\normalfont(\arabic*)},
	ref   = \thetheorem.{\normalfont(\arabic*)},
}

\newlist{enumeratethm}{enumerate}{1}
\setlist[enumeratethm]{
	label = {\normalfont(\textit{\roman*})},
	ref   = \thetheorem.{\normalfont(\textit{\roman*})},
}

\newlist{enumerateex}{enumerate}{1}
\setlist[enumerateex]{
	label = {\normalfont(\alph*)},
	ref   = \thetheorem.{\normalfont(\alph*)},
}



%%%%%%%%%%%%%%%%%%%%%%%%%%%%%%%%%%%%%%%%%
%----- MISE EN FORME DES ÉQUATIONS -----%
%%%%%%%%%%%%%%%%%%%%%%%%%%%%%%%%%%%%%%%%%

%----- NUMÉROTATION DES ÉQUATIONS -----%

\numberwithin{equation}{subsection}



%%%%%%%%%%%%%%%%%%%%%%%%%%%%%
%----- MISE EN COULEUR -----%
%%%%%%%%%%%%%%%%%%%%%%%%%%%%%

%----- COLORATION DU TEXTE -----%

\usepackage[
	x11names*,
	svgnames*,
	dvipsnames,
]{xcolor}



%%%%%%%%%%%%%%%%%%%%%%%%%%%%%%%%%%%%%%%%%%%%%%%%
%----- FORMATAGE DU CONTENU TEXTUEL (2/2) -----%
%%%%%%%%%%%%%%%%%%%%%%%%%%%%%%%%%%%%%%%%%%%%%%%%

%----- FORMATAGE DES THÉORÈMES -----%

\usepackage{amsthm}



%%%%%%%%%%%%%%%%%%%%%%%%%%%%%%%%%%%%%%%%%%%%%%%%%%%%
%----- FORMATAGE DU CONTENU PARATEXTUEL (2/2) -----%
%%%%%%%%%%%%%%%%%%%%%%%%%%%%%%%%%%%%%%%%%%%%%%%%%%%%

%----- ÉPIGRAPHES -----%

\usepackage{epigraph}


%----- NOTES DE BAS DE PAGE -----%

\usepackage{footnote}


%----- NOTES MARGINALES -----%

\usepackage{marginnote}


%----- NOTES DE FIN -----%

\usepackage{endnotes} 



%%%%%%%%%%%%%%%%%%%%%%%%%%%%%%%%%%%%%%%%%%%%%%%%%%%%%%%%
%----- SYMBOLES SPÉCIAUX ET NOTATIONS SPÉCIFIQUES -----%
%%%%%%%%%%%%%%%%%%%%%%%%%%%%%%%%%%%%%%%%%%%%%%%%%%%%%%%%

%----- SYMBOLES ET NOTATIONS MATHÉMATIQUES DE L'AMS -----%

\usepackage{amsmath}

\usepackage{amssymb}


%----- AUTRES SYMBOLES ET NOTATIONS MATHÉMATIQUES -----%

\usepackage{latexsym} 

\usepackage{mathtools}

\usepackage{mathrsfs}

\usepackage{dsfont}

\usepackage{xfrac}

\usepackage{cancel}

\usepackage{centernot}

\usepackage{empheq}


%----- SYMBOLES ET NOTATIONS INFORMATIQUES -----%

\usepackage[
	classfont = sanserif, % pour les classes de complexité des problèmes algorithmiques (P, NP, PSPACE, ENUM, #P...) on veut du textsf
	langfont  = caps, % pour les problèmes algorithmiques (Sat, SubsetSum, VerteCover...) on veut du textsc
	funcfont  = roman, % pour les classes de croissance assymptotique des fonctions (poly, subexp...) [par exemple utilisé dans P/poly] on veut du textrm
	full,
	disableredefinitions,
]{complexity}


%----- SYMBOLES ET NOTATIONS LOGIQUES -----%

\usepackage{cmll}

\usepackage{bussproofs}

\usepackage{formal-grammar}


%----- AUTRES SYMBOLES ET NOTATIONS -----%

\usepackage{stmaryrd}

\usepackage{bbding}

\usepackage{accents}

\usepackage{esint}

\usepackage{pifont}



%%%%%%%%%%%%%%%%%%%%%%%%%%%%%%%%%%%%%%%%%%%%%%%%%%%%%%%%%%%%%%%%%
%----- TABLES, FIGURES, ORNEMENTS ET DISPOSITION GRAPHIQUE -----%
%%%%%%%%%%%%%%%%%%%%%%%%%%%%%%%%%%%%%%%%%%%%%%%%%%%%%%%%%%%%%%%%%

%----- FIGURES -----%

\usepackage{graphicx}
\graphicspath{{./images/}}

\usepackage{svg}
\svgpath{{./images/}}
\svgsetup{inkscapelatex=false}

\usepackage{tikz}
% file: config-tikz.tex



%%%%%%%%%%%%%%%%%%%%%
%----- GRAPHES -----%
%%%%%%%%%%%%%%%%%%%%%

%----- NŒUDS -----%


%----- ARÊTES -----%

\tikzset{>=stealth}



%%%%%%%%%%%%%%%%%%%%%%%
%----- STRUCTUES -----%
%%%%%%%%%%%%%%%%%%%%%%%



\usepackage{tikz-3dplot}

\usepackage{tikz-cd}
% file: config-tikz-cd.tex



\tikzcdset{arrow style = tikz}

\usepackage{pgfplots}


%----- TABLES -----%

\usepackage{tabularx}

\usepackage{longtable}

\usepackage{array}

\usepackage{booktabs}

\usepackage{diagbox}


%----- LEGENDES -----%

\usepackage[%
	labelfont = {normalfont},%
]{caption}

\usepackage{subcaption}


%----- NUMÉROTATION DES TABLES ET FIGURES -----%

\numberwithin{table}{subsection}
\numberwithin{figure}{subsection}


%----- DISPOSITION GRAPHIQUE -----%

\usepackage{float}

\usepackage{wrapfig}

\input{insbox.tex}

\usepackage{adjustbox}

\usepackage{graphbox}


%----- ORNEMENTS -----%

\usepackage{pgfornament}



%%%%%%%%%%%%%%%%%%%%%%%%%%%%%%%%%%%%%%%%%%%%%%%%%%
%----- INDEXATION D'ÉLÉMENTS DU TEXTE (1/2) -----%
%%%%%%%%%%%%%%%%%%%%%%%%%%%%%%%%%%%%%%%%%%%%%%%%%%

%----- REFERENCES BIBLIOGRAPHIQUES -----%

\usepackage[
	backend     = biber,
	style       = alphabetic,
	sorting     = nty,
	maxbibnames = 99,
	backref     = true,
]{biblatex}
\addbibresource{./preamble/references.bib}
\DefineBibliographyStrings{french}{
	backrefpage  = {cf. p.},
	backrefpages = {cf. pp.}
}


%----- INDEX -----%

\usepackage[
	splitindex,%
]{imakeidx}

\makeindex[
	program = makeindex,
	columns = 2,
	intoc   = true,
	options = -s ./preamble/index-style.ist,
]



%%%%%%%%%%%%%%%%%%%%%%%%%%%%%%%%%%%%%%%%%%%%%%%%%%
%----- INDEXATION D'ÉLÉMENTS DE TITRE (1/2) -----%
%%%%%%%%%%%%%%%%%%%%%%%%%%%%%%%%%%%%%%%%%%%%%%%%%%

%----- PROFONDEUR D'INDEXATION -----%

\setcounter{tocdepth}{2}



%%%%%%%%%%%%%%%%%%%%%%%%%%%%%%%%
%----- MISE EN PAGE (1/2) -----%
%%%%%%%%%%%%%%%%%%%%%%%%%%%%%%%%

%----- PAGINATION -----%

\usepackage{changepage}


%----- GESTION DES BLANCS -----%

\usepackage{geometry}
% file: config-geometry.tex



\newgeometry{
	vmargin        = {3cm, 3cm},
	hmargin        = {3cm, 3cm},
	marginparwidth = 2.5cm,
}


%----- EN-TÊTES ET PIEDS DE PAGE -----%

\usepackage{fancyhdr}
% file: config-fancyhdr.tex



\fancyhead[RE]{\sc\nouppercase{\leftmark}}%
\fancyhead[LO]{\sc\nouppercase{\rightmark}}%
\fancyhead[LE, RO]{\thepage}%


%----- INCLUSION DE DOCUMENTS EXTERNES -----%

\usepackage{pdfpages}



%%%%%%%%%%%%%%%%%%%%%%%%
%----- MÉTA LaTeX -----%
%%%%%%%%%%%%%%%%%%%%%%%%

%----- MÉTA KERNEL -----%

\usepackage{morewrites}


%----- MÉTA MACROS -----%

\usepackage{xspace}

\usepackage{xparse}

\usepackage{ifthen}

\usepackage{etoolbox}


%----- MÉTA ENVIRONNEMENTS -----%

\usepackage{environ}



%%%%%%%%%%%%%%%%%%%%%%%%%%%%%%%%%%%%%%%%%%%%%%%%%%%
%----- CORRECTION, RÉVISION ET RÉCTIFICATION -----%
%%%%%%%%%%%%%%%%%%%%%%%%%%%%%%%%%%%%%%%%%%%%%%%%%%%

%----- NOTES DE RÉVISION -----%

\usepackage{todonotes}



%%%%%%%%%%%%%%%%%%%%%%%
%----- DÉBUGGAGE -----%
%%%%%%%%%%%%%%%%%%%%%%%

%----- TEXTES DE DÉBUGGAGE -----%

\usepackage{lipsum}
\usepackage{blindtext}


%----- MODALITÉS DE COMPILATION -----%

\usepackage[
	useregional,
]{datetime2}
% file: config-datetime2.tex



\newcommand*{\DTMfrenchfulltimesep}{\,}%

\newcommand*{\DTMfrenchfullhourstring}{h}%
\newcommand*{\DTMfrenchfullminstring}{min}%
\newcommand*{\DTMfrenchfullsecstring}{s}%

\DTMnewtimestyle{frenchfull}%
{%
	\renewcommand*\DTMdisplaytime[3]{%
		\number##1%
		\DTMfrenchfulltimesep%
		\DTMfrenchfullhourstring%
		\DTMfrenchfulltimesep%
		\DTMtwodigits{##2}%
		\DTMfrenchfulltimesep%
		\DTMfrenchfullminstring%
		\DTMfrenchfulltimesep
		\DTMtwodigits{##3}%
		\DTMfrenchfulltimesep%
		\DTMfrenchfullsecstring%
	}%
}%

\AtBeginDocument{%
	\DTMsettimestyle{frenchfull}%
}%



%%%%%%%%%%%%%%%%%%%%%%%%%%%%%%%
%----- LIENS HYPERTEXTES -----%
%%%%%%%%%%%%%%%%%%%%%%%%%%%%%%%

%----- NOTATIONS -----%

\usepackage{knowledge}


%----- HYPER-LIENS -----%

\usepackage{hyperref}
\hypersetup{
	plainpages  = false,
	linktoc     = section,
	hyperindex  = true,
	colorlinks  = true,
	breaklinks  = true,
	pageanchor  = true,
	linkcolor   = Red,
	anchorcolor = black,
	citecolor   = Green,
	filecolor   = Cyan,
	menucolor   = Red,
	runcolor    = Cyan,
	urlcolor    = WildStrawberry,
}
\makeatletter
\AtBeginDocument{
	\hypersetup{
		pdftitle  = {\@title},
		pdfauthor = {\@author},
	}
}
\makeatother


%----- RENVOIES -----%

\usepackage[%
symbol       = $\uparrow\;$,%
numberlinked = false,%
]{footnotebackref}



%%%%%%%%%%%%%%%%%%%%%%%%%%%%%%%%%%%%%%%%%%%%%%%%%%
%----- GESTION DES LABELS ET DES RÉFÉRENCES -----%
%%%%%%%%%%%%%%%%%%%%%%%%%%%%%%%%%%%%%%%%%%%%%%%%%%

%----- LABELS -----%

\usepackage[
	nameinlink,
	noabbrev,
]{cleveref}
% file: config-amsthm.tex



%%%%%%%%%%%%%%%%%%%%%%%%%%%%%
%----- DEFINITION LIKE -----%
%%%%%%%%%%%%%%%%%%%%%%%%%%%%%

\newtheoremstyle{mydefinitionstyle} % Name
{}          % Space above
{}          % Space below
{}          % Body font
{}          % Indent amount
{\bfseries} % Theorem head font
{.}         % Punctuation after theorem head
{ }         % Space after theorem head, ' ', or \newline
{\thmname{#1}\thmnumber{ #2}\thmnote{ (#3)}} % Theorem head spec (can be left empty, meaning `normal')

\theoremstyle{mydefinitionstyle}

\newtheorem{definition}{Définition}[subsection]
\newtheorem{notation}[definition]{Notation}
\newtheorem{remark}[definition]{Remarque}
\newtheorem{exercise}[definition]{Exercice}



%%%%%%%%%%%%%%%%%%%%%%%%%%
%----- THEOREM LIKE -----%
%%%%%%%%%%%%%%%%%%%%%%%%%%

\newtheoremstyle{mytheoremstyle} % Name
{}          % Space above
{}          % Space below
{\slshape}  % Body font
{}          % Indent amount
{\bfseries} % Theorem head font
{.}         % Punctuation after theorem head
{ }         % Space after theorem head, ' ', or \newline
{\thmname{#1}\thmnumber{ #2}\thmnote{ (#3)}} % Theorem head spec (can be left empty, meaning `normal')

\theoremstyle{mytheoremstyle}

\newtheorem{property}[definition]{Propriété}
\newtheorem{proposition}[definition]{Proposition}
\newtheorem{claim}[definition]{Fait}
\newtheorem{lemma}[definition]{Lemme}
\newtheorem{theorem}[definition]{Théorème}
\newtheorem{corollary}[definition]{Corollaire}
\newtheorem{axiom}[definition]{Axiome}
\newtheorem{conjecture}[definition]{Conjecture}



%%%%%%%%%%%%%%%%%%%%%%%%%%
%----- EXAMPLE LIKE -----%
%%%%%%%%%%%%%%%%%%%%%%%%%%

\makeatletter
\newcommand{\newexample}[2]{%
	\newenvironment{#1}[1][]{%
		\par%
		\def \param{##1}%
		\normalfont \topsep6\p@\@plus6\p@\relax%
		\trivlist%
		\item \relax%
		{%
			\itshape%
			#2%
			\ifx\param\empty%
				\relax%
			\else%
				\ (\param)%
			\fi%
			\@addpunct{.}
		}%
		\hspace%
		\labelsep%
		\ignorespaces%
	}{%
		\endtrivlist\@endpefalse
	}
}
\makeatother

\newexample{example}{Exemple}
\newexample{solution}{Solution}
\newexample{question}{Question}
\newexample{answer}{Réponse}
% file: config-cleveref.tex



%%%%%%%%%%%%%%%%%%%%%%%%%%%%%
%----- DEFINITION LIKE -----%
%%%%%%%%%%%%%%%%%%%%%%%%%%%%%

\crefname{definition}{définition}{définitions}
\crefname{notation}{notation}{notations}
\crefname{remark}{remarque}{remarques}
\crefname{exercise}{exercice}{exercices}



%%%%%%%%%%%%%%%%%%%%%%%%%%
%----- THEOREM LIKE -----%
%%%%%%%%%%%%%%%%%%%%%%%%%%

\crefname{property}{propriété}{propriétés}
\crefname{proposition}{proposition}{propositions}
\crefname{claim}{fait}{faits}
\crefname{lemma}{lemme}{lemmes}
\crefname{theorem}{théorème}{théorèmes}
\crefname{corollary}{corollaire}{corollaires}
\crefname{axiom}{axiome}{axiomes}
\crefname{conjecture}{conjecture}{conjectures}



%%%%%%%%%%%%%%%%%%%%
%----- TITLES -----%
%%%%%%%%%%%%%%%%%%%%

\crefname{part}{partie}{parties}
\crefname{chapter}{chapitre}{chapitres}
\crefname{section}{section}{sections}
\crefname{subsection}{sous-section}{sous-sections}
\crefname{subsubsection}{sous-sous-section}{sous-sous-sections}
\crefname{paragraph}{paragraphe}{paragraphes}
\crefname{subparagraph}{sous-paragraphe}{sous-paragraphes}



%%%%%%%%%%%%%%%%%%%%%%%%%%%%%%%%
%----- FIGURES AND TABLES -----%
%%%%%%%%%%%%%%%%%%%%%%%%%%%%%%%%

\crefname{figure}{figure}{figures}
\crefname{subfigure}{sous-figure}{sous-figures}

\crefname{table}{table}{tables}
\crefname{subtable}{sous-table}{sous-tables}



%%%%%%%%%%%%%%%%%%%%%%%%%%%
%----- FOOTNOTE LIKE -----%
%%%%%%%%%%%%%%%%%%%%%%%%%%%

\crefname{footnote}{note de bas de page}{notes de bas de page}
\crefname{marginenote}{note marginale}{notes marginales}
\crefname{endnote}{note de fin}{notes de fin}



%%%%%%%%%%%%%%%%%%%%%%%%%%%%%%%%
%----- PARAGRAPH ELEMENTS -----%
%%%%%%%%%%%%%%%%%%%%%%%%%%%%%%%%

\crefname{item}{item}{items}



%%%%%%%%%%%%%%%%%%%%%%%%%%%%%%%
%----- EQUATION ELEMENTS -----%
%%%%%%%%%%%%%%%%%%%%%%%%%%%%%%%

\crefname{equation}{équation}{équations}


%----- RÉFÉRENCES -----%

\usepackage{zref}



%%%%%%%%%%%%%%%%%%%%%%%%%%%%%%%%%%%%%%%%%%%%%%%%%%
%----- INDEXATION D'ÉLÉMENTS DE TITRE (2/2) -----%
%%%%%%%%%%%%%%%%%%%%%%%%%%%%%%%%%%%%%%%%%%%%%%%%%%

%----- TABLE DES SOUS-MATIÈRES -----%

\usepackage{minitoc}
% file: config-minitoc.tex



\mtcsettitle{minitoc}{Table des sous-matières}
\mtcsetdepth{minitoc}{3}



%%%%%%%%%%%%%%%%%%%%%%%%%%%%%%%%
%----- MISE EN PAGE (2/2) -----%
%%%%%%%%%%%%%%%%%%%%%%%%%%%%%%%%

%----- FILIGRANE -----%

\usepackage[some]{background}



%%%%%%%%%%%%%%%%%%%%%%%%%%%%%%%%%%%%%%%%%%%%%%%%%%
%----- INDEXATION D'ÉLÉMENTS DU TEXTE (2/2) -----%
%%%%%%%%%%%%%%%%%%%%%%%%%%%%%%%%%%%%%%%%%%%%%%%%%%

%----- GLOSSAIRE -----%

\usepackage[
	toc,
	translate = true,
]{glossaries}
\makeglossaries
\newglossaryentry{enculer-les-mouches}
{
	name=Enculer les mouches,
	description={(apparition : fin du XXe siècle, variation possible : sodomiser les drosophiles) Composé de enculer et de mouche. Permet probablement d'imager une grande difficulté ou déployer de grands efforts pour un but dérisoire. Décrit également une démarche consistant à s'attarder inutilement sur des points de détail, en faisant preuve d'une méticulosité extrême, voire excessive, au détriment de l'essentiel.}
}

\newglossaryentry{honnete-et-frequentable}
{
	name=Honnête et fréquentable,
	description={Se dit d'une fonction manipulée par un physicien, c'est-à-dire continues et dérivables autant que les nécessités de calcul l'exigeront.}
}

\newglossaryentry{poussage-de-symboles}
{
	name=Poussage de symboles,
	description={Activité passionnante à laquelle s'attèlent certains logiciens, qui consiste à dérouler les définitions, les notations et les abréviations des symboles jusqu'à arriver au résultat, et ce, sans avoir besoin d'ajouter aucune conjonction de coordination de la langue française.}
}



%%%%%%%%%%%%%%%%%%%%%%%%%%%%%%%%%%%%%%%%%%%
%----- PACKAGES SAUVEGARDÉS EN LOCAL -----%
%%%%%%%%%%%%%%%%%%%%%%%%%%%%%%%%%%%%%%%%%%%

%----- PACKAGES PERSONNELS -----%



%----- PACKAGES TIERS -----%






%%%%%%%%%%%%%%%%%%%%%
%----- PATCHES -----%
%%%%%%%%%%%%%%%%%%%%%


