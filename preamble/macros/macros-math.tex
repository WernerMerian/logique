% file: macros-math.tex



%%%%%%%%%%%%%%%%%%%%%%%
%----- COMMANDES -----%
%%%%%%%%%%%%%%%%%%%%%%%

%----- NOUVEAUX SYMBOLS -----%

\newcommand{\powerset}{\mathcal{P}} % ensemble des parties
\newcommand{\partialto}{\rightharpoonup} % fonction partielle
\newcommand{\defeq}{\triangleq}
           %{\overset{{\normalfont\text{déf}}}{=}} % égalité définitionnelle
\newcommand{\symdiff}{\triangle} % différence symétrique
\newcommand{\subfin}{\subseteq_{\mathrm{fin}}}


%----- PROBLEMES ALGORITHMIQUES -----%

\newcommand{\problemStatement}[3]{
	\begin{center}
		\begin{tabularx}{.8\linewidth}{lX}
			\toprule
			\multicolumn{2}{c}{#1}\\
			\midrule
			\bfseries Instance:  & #2\\
			\bfseries Sortie: & #3\\
			\bottomrule
		\end{tabularx}
	\end{center}
}



%%%%%%%%%%%%%%%%%%%%%%%%%%%%%%%%%%%%%%
%----- OPERATEURS MATHÉMATIQUES -----%
%%%%%%%%%%%%%%%%%%%%%%%%%%%%%%%%%%%%%%

%----- FONCTIONS -----%

\DeclareMathOperator{\id}{id}
\DeclareMathOperator{\im}{Im}
\DeclareMathOperator{\dom}{Dom}

%----- PROPOSITIONS ET TERMES -----%

\DeclareMathOperator{\Prop}{Prop}
\DeclareMathOperator{\Var}{Var}
\DeclareMathOperator{\VL}{VL}
\DeclareMathOperator{\Term}{Term}

%----- THÉORIE DES MODÈLES -----%

\DeclareMathOperator{\Const}{Const}
\DeclareMathOperator{\Fun}{Fun}
\DeclareMathOperator{\Rel}{Rel}
\DeclareMathOperator{\ar}{ar}

\DeclareMathOperator{\Struct}{Struct}
\DeclareMathOperator{\Val}{Val}


%----- THÉORIE DES ENSEMBLES -----%

\DeclareMathOperator{\Card}{Card}
\DeclareMathOperator{\Funct}{Funct}
\DeclareMathOperator{\trans}{trans}
\DeclareMathOperator{\trcl}{trcl}

%----- STRUCTURES DE DONNEES -----%

\DeclareMathOperator{\List}{List}
\DeclareMathOperator{\cons}{cons}
\DeclareMathOperator{\nil}{nil}

%%%%%%%%%%%%%%%%%%%%%%%%%
%----- LaTeX HACKS -----%
%%%%%%%%%%%%%%%%%%%%%%%%%

%----- REDIMENSIONNE LES PARENTHÈSES AUTOURS UNDERBRACE/UNDERBRACKET/UNDERARROW -----%

%% good formating for underbrace/overbrace when used inside parenthesis

\newcommand{\underbracein}[2]{\vphantom{#1}\smash{\underbrace{#1}_{#2}}}
\newcommand{\underbraceinclap}[2]{\vphantom{#1}\smash{\underbrace{#1}_{\mathclap{#2}}}}
\newcommand{\underbraceout}[2]{\vphantom{\underbrace{#1}_{#2}}}

\newcommand{\overbracein}[2]{\vphantom{#1}\smash{\overbrace{#1}_{#2}}}
\newcommand{\overbraceinclap}[2]{\vphantom{#1}\smash{\overbrace{#1}_{\mathclap{#2}}}}
\newcommand{\overbraceout}[2]{\vphantom{\overbrace{#1}_{#2}}}

%% good formating for underbracket/overbracket when used inside parenthesis

\newcommand{\underbracketin}[2]{\vphantom{#1}\smash{\underbracket[.6pt][3pt]{#1}_{#2}}}
\newcommand{\underbracketinclap}[2]{\vphantom{#1}\smash{\underbracket[.6pt][3pt]{#1}_{\mathclap{#2}}}}
\newcommand{\underbracketout}[2]{\vphantom{\underbracket[.6pt][3pt]{#1}_{#2}}}

\newcommand{\overbracketin}[2]{\vphantom{#1}\smash{\overbracket[.6pt][3pt]{#1}_{#2}}}
\newcommand{\overbracketinclap}[2]{\vphantom{#1}\smash{\overbracket[.6pt][3pt]{#1}_{\mathclap{#2}}}}
\newcommand{\overbracketout}[2]{\vphantom{\overbracket[.6pt][3pt]{#1}_{#2}}}

%% arrows above and under equation pointing a particular index in a list
%% see: https://tex.stackexchange.com/questions/513452/up-arrow-under-a-sequence

\newcommand{\underarrowin}[2]{\vphantom{#1}\smash{\underset{\substack{\uparrow\\#2}}{#1}}}
\newcommand{\underarrowinclap}[2]{\vphantom{#1}\smash{\underset{\mathclap{\substack{\uparrow\\#2}}}{#1}}}
\newcommand{\underarrowout}[2]{\vphantom{\underset{\substack{\uparrow\\#2}}{#1}}}

\newcommand{\overarrowin}[2]{\vphantom{#1}\smash{\underset{\substack{#2\\\downarrow}}{#1}}}
\newcommand{\overarrowinclap}[2]{\vphantom{#1}\smash{\underset{\mathclap{\substack{#2\\\downarrow}}}{#1}}}
\newcommand{\overarrowout}[2]{\vphantom{\underset{\substack{#2\\\downarrow}}{#1}}}



%%%%%%%%%%%%%%%%%%%%%%%%%%%%%%%%%%%%%
%----- FORMATAGE DES ÉQUATIONS -----%
%%%%%%%%%%%%%%%%%%%%%%%%%%%%%%%%%%%%%

%----- ÉTIQUETTES -----%

\makeatletter
\newcommand{\leqnomode}{\tagsleft@true\let\veqno\@@leqno}
\newcommand{\reqnomode}{\tagsleft@false\let\veqno\@@eqno}
\makeatother



%%%%%%%%%%%%%%%%%%%%%%%%%%
%----- ENVIRONMENTS -----%
%%%%%%%%%%%%%%%%%%%%%%%%%%

%----- HACKS -----%

\NewEnviron{centerwide}{
	\par
	\noindent
	\makebox[\textwidth]{\parbox{1.3\textwidth}{\BODY}}
}
\NewEnviron{centerwideequation}{\[\makebox[\displaywidth]{$\displaystyle\BODY$}\]}
