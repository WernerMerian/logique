% file: macros-math.tex



%%%%%%%%%%%%%%%%%%%%%%%
%----- COMMANDES -----%
%%%%%%%%%%%%%%%%%%%%%%%

%----- NOUVEAUX SYMBOLES -----%

\newcommand{\powerset}{\mathcal{P}} % ensemble des parties
\newcommand{\powerdef}{\mathcal{D}} % ensemble des parties définissables
\newcommand{\powerfin}{\powerset_{\mathrm{fin}}} % ensemble des parties finies
\newcommand{\powermul}{\powerset_{\mathcal{M}}} % ensemble des multi ensembles
\newcommand{\vois}{\mathcal{V}} % ensemble des voisinages d'un point
\newcommand{\partialto}{\rightharpoonup} % fonction partielle
%\newcommand{\defeq}{\triangleq}
           %{\overset{{\normalfont\text{déf}}}{=}} % égalité définitionnelle
\newcommand{\symdiff}{\triangle} % différence symétrique
\newcommand{\subfin}{\subseteq_{\mathrm{fin}}} % inclusion d'une partie finie
\newcommand{\lexl}{<_{\mathrm{lex}}} % ordre lexicographique
\newcommand{\leL}{<_{\mathbb L}} % ordre sur la classe L
\newcommand{\ordbis}{\lhd} % ordre utilisé pour la démonstration de Hessenberg, et globalement symbole de relation d'ordre au cas où


%----- PROBLEMES ALGORITHMIQUES -----%

\newcommand{\problemStatement}[3]{
	\begin{center}
		\begin{tabularx}{.8\linewidth}{lX}
			\toprule
			\multicolumn{2}{c}{#1}\\
			\midrule
			\bfseries Instance:  & #2\\
			\bfseries Sortie: & #3\\
			\bottomrule
		\end{tabularx}
	\end{center}
}



%%%%%%%%%%%%%%%%%%%%%%%%%%%%%%%%%%%%%%
%----- OPERATEURS MATHÉMATIQUES -----%
%%%%%%%%%%%%%%%%%%%%%%%%%%%%%%%%%%%%%%

%----- FONCTIONS -----%

\DeclareMathOperator{\id}{id} % fonction identité
\DeclareMathOperator{\im}{Im} % ensemble image d'une fonction
\DeclareMathOperator{\dom}{Dom} % domaine d'une fonction
\DeclareMathOperator{\fix}{fix} % ensemble des points fixes d'une fonction
\DeclareMathOperator{\prefix}{prefix} % ensemble des pré-points fixes d'une fonction
\DeclareMathOperator{\postfix}{postfix} % ensemble des post-points fixes d'une fonction
\newcommand{\restrict}[1]{\left.#1\right\vert} % restriction d'une fonction
\newcommand{\makeFun}[5]{\begin{array}{ccccc} #1 & : & #2 & \longrightarrow & #3 \\ & & #4 & \longmapsto & #5\end{array}}
\newcommand{\makeFunName}[3]{#1 : #2 \longrightarrow #3}
\newcommand{\makeFunNamePartial}[3]{#1 : #2 \xrightharpoonup{\qquad} #3}
\DeclareMathOperator{\Supp}{Supp}

%----- PROPOSITIONS ET TERMES -----%

\DeclareMathOperator{\Prop}{Prop} % ensemble des propositions
\DeclareMathOperator{\Var}{Var} % ensemble des variables
\DeclareMathOperator{\VL}{VL} % ensemble des variables libres
\DeclareMathOperator{\VLie}{Vliées} % ensemble des variables liées
\DeclareMathOperator{\Term}{Term} % ensemble des termes
\DeclareMathOperator{\Formula}{Form} % ensemble des formules
\newcommand{\Formqf}{\Formula_{\mathrm{qf}}} % ensembles des formules sans quantificateurs
\newcommand{\FormMod}{\Formula_{\mathrm{mod}}} % ensemble des formules de la logique modale
\DeclareMathOperator{\Atom}{Atom} % ensemble des atomes
\newcommand{\vclose}[1]{\overline{#1}^{\vDash}} % clôture par conséquence syntaxique d'un ensemble de formules
\newcommand{\vdclose}[1]{\overline{#1}^{\vdash}} % clôture par conséquence sémantique d'un ensemble de formules
\newcommand{\hclose}[1]{\overline{#1}^{\mathrm H}} % clôture par ajout des témoins de Henkin d'une théorie
\DeclareMathOperator{\Th}{Th}

%----- ENSEMBLES CLASSIQUES -----%

\newcommand{\bN}{\mathbb N} % ensemble des entiers naturels
\newcommand{\bZ}{\mathbb Z} % ensemble des entiers relatifs
\newcommand{\bQ}{\mathbb Q} % ensemble des nombres rationnels
\newcommand{\bR}{\mathbb R} % ensemble des nombres réels
\newcommand{\bV}{\mathbb V} % hiérarchie cumulative de Von Neumann
\newcommand{\bL}{\mathbb L} % classe des constructibles de Gödel
\newcommand{\vide}{\varnothing} % ensemble vide

%----- OPERATIONS ENSEMBLISTES -----%

\newcommand{\compr}[2]{\{ #1 \mid #2 \}} % écriture d'un ensemble en compréhension
\newcommand{\restr}[1]{\!\upharpoonright_{#1}} % restriction d'une fonction à un domaine donné
\newcommand{\diagcap}{\bigtriangleup} % intersection diagonale

%----- THÉORIE DES MODÈLES -----%

\DeclareMathOperator{\Const}{Const} % ensemble des constantes
\DeclareMathOperator{\Fun}{Fun} % ensemble des symboles de fonctions
\DeclareMathOperator{\Rel}{Rel} % ensemble des symboles de relation
\DeclareMathOperator{\ar}{ar} % arité d'un symbole
\DeclareMathOperator{\Struct}{Struct} % classe des structures
\DeclareMathOperator{\Val}{Val} % valeur de vérité d'une formule dans un modèle

%----- EGALITES -----%

\newcommand{\defeq}{\triangleq} % égalité définitionnelle
\newcommand{\inddefeq}{\Coloneq} % définition inductive, ::=

%----- FILTRES -----%

\DeclareMathOperator{\Spec}{Spec} % spectre
\newcommand{\filclose}[1]{\overline{#1}^{\mathcal F}} % filtre engendré par un ensemble

%----- CONSTRUCTION DE R -----%

\newcommand{\Dedekind}{\mathfrak{D}} % ensemble des coupures de Dedking
\newcommand{\subfil}{\subseteq_{\mathrm{fil}}} % partie filtrante
\newcommand{\ideal}{\mathfrak{I}} % ensemble des idéaux

%----- THÉORIE DES ENSEMBLES -----%

\DeclareMathOperator{\ZFC}{ZFC} % théorie des ensembles ZFC
\DeclareMathOperator{\ZF}{ZF} % théorie des ensembles ZF
\DeclareMathOperator{\Zermelo}{Z} % théorie des ensembles de Zermelo
\DeclareMathOperator{\AxC}{AC} % axiome du choix
\DeclareMathOperator{\AxF}{AF} % axiome de fondation
\DeclareMathOperator{\HC}{HC} % hypothèse du continu
\DeclareMathOperator{\HCG}{HCG} % hypothèse du continu généralisée
\DeclareMathOperator{\Funct}{Funct} % ensemble des fonctions entre deux ensembles
\DeclareMathOperator{\Hom}{Hom} % ensemble des homomorphismes entre deux structures
\DeclareMathOperator{\trans}{trans} % transitivité
\DeclareMathOperator{\trcl}{trcl} % clôture transitive d'un ensemble
\DeclareMathOperator{\type}{type} % type d'un bon ordre
\DeclareMathOperator{\cof}{cof} % cofinalité d'un ordinal
\DeclareMathOperator{\Club}{Club} % ensemble des clubs d'un ordinal
\DeclareMathOperator{\Lim}{Lim} % classe des ordinaux limites, ensemble des limites d'un ordinal
\DeclareMathOperator{\rk}{rk} % rang
\newcommand{\VeqL}{\bV = \bL} % axiome V = L

%----- ORDINAUX -----%

\DeclareMathOperator{\Card}{Card} % classe des cardinaux, fonction cardinal
\DeclareMathOperator{\Ord}{Ord} % classe des ordinaux
\DeclareMathOperator{\Clos}{Clos} % ensemble des points de clôture

%----- THEORIE DESCRIPTIVE DES ENSEMBLES -----%

\newcommand{\Borel}{\mathcal{B}} % ensemble des boérliens d'un ensemble topologique

% hiérarchie de Borel
\newcommand{\bSigma}{\boldsymbol{\Sigma}}
\newcommand{\bPi}{\boldsymbol{\Pi}}
\newcommand{\bDelta}{\boldsymbol{\Delta}}

%----- LISTES -----%

\DeclareMathOperator{\List}{List} % ensemble des listes sur un ensemble
\DeclareMathOperator{\cons}{cons} % ajouter un élément en tête d'une liste
\DeclareMathOperator{\nil}{nil} % liste vide
\newcommand{\varlist}{\ell} % variable pour désigner une liste

%----- ARBRES -----%

\DeclareMathOperator{\BinTree}{BinTree} % ensemble des arbres binaires
\DeclareMathOperator{\node}{node} % noeud d'un arbre

%----- THÉORIE DE LA DÉMONSTRATION -----%

\DeclareMathOperator{\EDN}{EDN} % élimination de la double négation
\DeclareMathOperator{\TE}{TE} % tiers exclu
\DeclareMathOperator{\LK}{LK} % calcul des séquents classique
\DeclareMathOperator{\LJ}{LJ} % calcul des séquents intuitionniste
\DeclareMathOperator{\LS}{LS} % calcul des séquents générique
\DeclareMathOperator{\LL}{LL} % logique linéaire
\DeclareMathOperator{\NK}{NK} % déduction naturelle classique
\DeclareMathOperator{\NJ}{NJ} % déduction naturelle intuitionniste
\DeclareMathOperator{\hypo}{hypotheses}
\newcommand{\concl}{\rhd}
\newcommand{\vdLK}{\vdash_{\LK}} % relation de prouvabilité pour LK
\newcommand{\vdLJ}{\vdash_{\LJ}} % relation de prouvabilité pour LJ
\newcommand{\nvdLJ}{\nvdash_{\LJ}}
\newcommand{\vdLS}{\vdash_{\LS}} % relation de prouvabilité pour LS
\newcommand{\vdNK}{\vdash_{\NK}} % relation de prouvabilité pour NK
\newcommand{\vdNJ}{\vdash_{\NJ}} % relation de prouvabilité pour NJ
\newcommand{\vdLL}{\vdash_{\LL}} % relation de prouvabilité pour LL
\newcommand{\rOne}{\raisebox{.5pt}{\textcircled{\raisebox{-.9pt} {1}}}}
\newcommand{\rTwo}{\raisebox{.5pt}{\textcircled{\raisebox{-.9pt} {2}}}}
\newcommand{\rTwoB}{\raisebox{.5pt}{\textcircled{\raisebox{-.9pt} {2'}}}}
\newcommand{\reecr}[1]{\overset{#1}{\resizebox{1.5 \width}{!}{$\rightsquigarrow$}}}
\newcommand{\quantif}{\texttt{Q}}
\newcommand{\ProofLK}{\Proof{\LK}}
\newcommand{\ProofLS}{\Proof{\LS}}
\newcommand{\ProofLJ}{\Proof{\LJ}}
\newcommand{\ProofNK}{\Proof{\NK}}
\newcommand{\ProofNJ}{\Proof{\NJ}}
\newcommand{\tradNJ}[1]{\llbracket #1 \rrbracket}
\newcommand{\LKPA}{\LK_{\ArithPeano}}
\newcommand{\vdPA}{\vdash_{\ArithPeano}}
\newcommand{\ProofPA}{\Proof{\ArithPeano}}
\DeclareMathOperator{\HA}{HA} % arithmétique de Heyting

%----- REALISABILITE -----%

\newcommand{\real}{\Vdash}

%----- ARS -----%

\newcommand{\WN}{\mathcal{WN}}
\newcommand{\SN}{\mathcal{SN}}

%----- LAMBDA CALCUL -----%

\newcommand{\satur}{\mathcal{SAT}}
\newcommand{\lamSet}{\Lambda}
\newcommand{\lamVar}{\Var_{\lambda}}
\newcommand{\bred}{\rhd_{\beta}}
\newcommand{\bered}{\rhd_{\beta\eta}}
\newcommand{\Sred}{\rhd_{\beta\mathrm{s}}^\star}
\newcommand{\Lred}{\rhd_{\beta\mathrm{l}}}
\newcommand{\Beq}{=_{\beta}}
\newcommand{\BEeq}{=_{\beta\eta}}
\newcommand{\redPar}{\rhd_{\beta||}}
\newcommand{\Hred}{\rhd_{\beta\mathrm{h}}}
\newcommand{\Ired}{\rhd_{\beta\mathrm{i}}}
\newcommand{\WHred}{\rhd_{\mathrm{wh}}}
\newcommand{\IredPar}{\rhd_{\beta\mathrm{i}{||}}}
\newcommand{\HredPar}{\rhd_{\beta\mathrm{h}{||}}}
\newcommand{\lift}[3]{\uparrow_{#2}^{#3}#1}
\newcommand{\eqZ}{\mathrm{eq}_0}
\DeclareMathOperator{\HALT}{HALT}
\newcommand{\hnf}{\lamSet_{\mathrm{hnf}}}
\newcommand{\Elim}{\mathrm{Elim}}
\newcommand{\ElimSN}{\Elim_{\SN}}
\newcommand{\satSN}{\satur_{\SN}}

%----- TYPES -----%

\newcommand{\intT}{\texttt{int}}
\DeclareMathOperator{\tyT}{Types}
\newcommand{\vdNJto}{\vdash_{\NJ_\to}}
\newcommand{\STLCSet}{\Lambda_{\mathrm{STLC}}}
\newcommand{\tySTLC}{\tyT_{\mathrm{STLC}}}
\newcommand{\unitT}{\texttt{Unit}}
\newcommand{\emptyT}{\texttt{Empty}}
\newcommand{\deltaSTLC}[5]{\delta\;#1\;\{\kappa_1\;#2\mapsto #3|
  \kappa_2\;#4\mapsto #5\}}
\newcommand{\STLCred}{\rhd}
\newcommand{\FOLSet}{\Lambda_{\mathrm{FOL}}}
\newcommand{\tyFOL}{\tyT_{\mathrm{FOL}}}
\newcommand{\blam}{\boldsymbol{\lambda}}
\newcommand{\bvarx}{\mathbf{x}}
\newcommand{\bvary}{\mathbf{y}}
\newcommand{\btermt}{\mathbf{t}}
\newcommand{\btermu}{\mathbf{u}}
\newcommand{\letin}[4]{\texttt{let}\;\langle #1,#2\rangle = #3\;\texttt{in}\;#4}
\DeclareMathOperator{\refl}{refl}
\DeclareMathOperator{\transp}{transport}
\newcommand{\FOLred}{\rhd_{\mathrm{FOL}}}
\newcommand{\vdFOL}{\vdash_{\mathrm{FOL}}}
\newcommand{\OptionT}{\texttt{Option}}
\newcommand{\semT}[1]{\llbracket#1\rrbracket}
\newcommand{\satTO}{\to}
\newcommand{\satAND}{\times}
\newcommand{\satOR}{+}

%----- SYSTEM T -----%

\newcommand{\tyST}{\tyT_{\mathrm{FOL}}}
\newcommand{\STSet}{\Lambda_{\mathrm{T}}}
\newcommand{\STred}{\rhd_{\mathrm{T}}}

%----- LANGAGES RATIONNELS -----%

\DeclareMathOperator{\Regex}{Regex} % ensemble des expressions régulières
\DeclareMathOperator{\Ratio}{Ratio} % ensemble des langages rationnels
\DeclareMathOperator{\Reco}{Reco} % ensemble des langages reconnaissables
\DeclareMathOperator{\tr}{tr} % trace

%----- ARITHMETIQUE -----%

\DeclareMathOperator{\pair}{pair} % prédicat sur un entier pour qu'il soit pair
\DeclareMathOperator{\double}{double} % fonction doublant un entier
\newcommand{\append}{+\!\!+} % concaténation de listes

%----- CALCULABILITÉ -----%

\DeclareMathOperator{\Coher}{Coh} % cohérence d'une théorie
\newcommand{\convcal}{\!\downarrow} % calcul convergent
\newcommand{\conveq}{\convcal=} % convergence + égalité
\newcommand{\divcal}{\!\uparrow} % calcul divergent
\newcommand{\ceil}[1]{\lceil#1\rceil}
\newcommand{\btwo}{\boldsymbol{2}} % ensemble des booléens
\newcommand{\godcod}[1]{\ulcorner#1\urcorner} % codage de Gödel
\DeclareMathOperator{\MultExp}{MultiplicativeExpr} % expression multiplicative
\DeclareMathOperator{\RecP}{RP} % fonctions récursives primitives
\DeclareMathOperator{\Rec}{Rec} % fonctions récursives
\DeclareMathOperator{\RPG}{RPG} % fonctions récursives primitives généralisées
\DeclareMathOperator{\RecG}{RecG} % fonctions récursives généralisées
\DeclareMathOperator{\sucs}{succ} % fonction successeur
\DeclareMathOperator{\rec}{rec} % récurrence
\DeclareMathOperator{\ifrm}{if} % si
\DeclareMathOperator{\thenrm}{then} % then
\DeclareMathOperator{\elserm}{else}
\DeclareMathOperator{\km}{k} % k
\DeclareMathOperator{\TM}{TM} % machines de Turing
\DeclareMathOperator{\Calc}{Calc} % fonctions calculables
\DeclareMathOperator{\Enum}{\Phi} % Enumération des fonctions calculables
\newcommand{\manyOne}{\leq_{\mathrm{m}}}
\newcommand{\manyOneS}{<_{\mathrm{m}}}
\newcommand{\oneOne}{\leq_{1}}
\newcommand{\manyEq}{\equiv_{\mathrm{m}}}
\newcommand{\ManyDegrees}{\mathscr{D}_{\mathrm{m}}}
\newcommand{\degD}{\mathfrak{d}}
\newcommand{\TuringDeg}{\mathscr{D}}
\newcommand{\TuringRed}{\leq_{\mathrm{T}}}
\newcommand{\TuringRedS}{<_{\mathrm{T}}}
\newcommand{\TuringEq}{\equiv_{\mathrm{T}}}

%----- ARITHMETIQUE INCOMPLETUDE -----%

\newcommand{\HierArithS}[1]{\Sigma^0_{#1}}
\newcommand{\HierArithP}[1]{\Pi^0_{#1}}
\newcommand{\HierArithD}[1]{\Delta^0_{#1}}
\newcommand{\ArithRobin}{\mathrm{Q}}
\newcommand{\ArithPeano}{\mathrm{PA}}
\newcommand{\encode}[1]{\underline{#1}}
\DeclareMathOperator{\Subst}{Subst}
\DeclareMathOperator{\Not}{Not}
\DeclareMathOperator{\Dem}{Dem}
\DeclareMathOperator{\Imp}{Imp}
\newcommand{\DemT}[1]{\Dem_{#1}}
\newcommand{\DemBisT}[1]{\Dem'_{#1}}
\newcommand{\ThT}[1]{\Th_{#1}}
\newcommand{\Proof}[1]{\mathrm{PT}_{#1}}
\DeclareMathOperator{\PTO}{PTO}

%----- TOPOLOGIE -----%

\DeclareMathOperator{\Cylind}{Cylind} % cylindre d'une partie dans un produit
\DeclareMathOperator{\ext}{ext} % extension d'un élément dans le spectre
\DeclareMathOperator{\Adh}{Adh} % ensemble des valeurs d'adhérences
\newcommand{\inter}[1]{\overset{\circ}{#1}} % intérieur d'une partie
\newcommand{\adher}[1]{\overline{#1}^{\mathrm c}} % adhérence d'une partie
\newcommand{\cylin}[1]{\llbracket #1\rrbracket} % cylindre en une partie
\newcommand{\clB}{\overline{B}} % boule fermée
\newcommand{\fillim}{\lim^{\mathcal F}} % limite au sens d'un filtre

%%%%%%%%%%%%%%%%%%%%%%%%%
%----- LaTeX HACKS -----%
%%%%%%%%%%%%%%%%%%%%%%%%%

%----- REDIMENSIONNE LES PARENTHÈSES AUTOURS UNDERBRACE/UNDERBRACKET/UNDERARROW -----%

%% good formating for underbrace/overbrace when used inside parenthesis

\newcommand{\underbracein}[2]{\vphantom{#1}\smash{\underbrace{#1}_{#2}}}
\newcommand{\underbraceinclap}[2]{\vphantom{#1}\smash{\underbrace{#1}_{\mathclap{#2}}}}
\newcommand{\underbraceout}[2]{\vphantom{\underbrace{#1}_{#2}}}

\newcommand{\overbracein}[2]{\vphantom{#1}\smash{\overbrace{#1}_{#2}}}
\newcommand{\overbraceinclap}[2]{\vphantom{#1}\smash{\overbrace{#1}_{\mathclap{#2}}}}
\newcommand{\overbraceout}[2]{\vphantom{\overbrace{#1}_{#2}}}

%% good formating for underbracket/overbracket when used inside parenthesis

\newcommand{\underbracketin}[2]{\vphantom{#1}\smash{\underbracket[.6pt][3pt]{#1}_{#2}}}
\newcommand{\underbracketinclap}[2]{\vphantom{#1}\smash{\underbracket[.6pt][3pt]{#1}_{\mathclap{#2}}}}
\newcommand{\underbracketout}[2]{\vphantom{\underbracket[.6pt][3pt]{#1}_{#2}}}

\newcommand{\overbracketin}[2]{\vphantom{#1}\smash{\overbracket[.6pt][3pt]{#1}_{#2}}}
\newcommand{\overbracketinclap}[2]{\vphantom{#1}\smash{\overbracket[.6pt][3pt]{#1}_{\mathclap{#2}}}}
\newcommand{\overbracketout}[2]{\vphantom{\overbracket[.6pt][3pt]{#1}_{#2}}}

%% arrows above and under equation pointing a particular index in a list
%% see: https://tex.stackexchange.com/questions/513452/up-arrow-under-a-sequence

\newcommand{\underarrowin}[2]{\vphantom{#1}\smash{\underset{\substack{\uparrow\\#2}}{#1}}}
\newcommand{\underarrowinclap}[2]{\vphantom{#1}\smash{\underset{\mathclap{\substack{\uparrow\\#2}}}{#1}}}
\newcommand{\underarrowout}[2]{\vphantom{\underset{\substack{\uparrow\\#2}}{#1}}}

\newcommand{\overarrowin}[2]{\vphantom{#1}\smash{\underset{\substack{#2\\\downarrow}}{#1}}}
\newcommand{\overarrowinclap}[2]{\vphantom{#1}\smash{\underset{\mathclap{\substack{#2\\\downarrow}}}{#1}}}
\newcommand{\overarrowout}[2]{\vphantom{\underset{\substack{#2\\\downarrow}}{#1}}}



%%%%%%%%%%%%%%%%%%%%%%%%%%%%%%%%%%%%%
%----- FORMATAGE DES ÉQUATIONS -----%
%%%%%%%%%%%%%%%%%%%%%%%%%%%%%%%%%%%%%

%----- ÉTIQUETTES -----%

\makeatletter
\newcommand{\leqnomode}{\tagsleft@true\let\veqno\@@leqno}
\newcommand{\reqnomode}{\tagsleft@false\let\veqno\@@eqno}
\makeatother



%%%%%%%%%%%%%%%%%%%%%%%%%%
%----- ENVIRONMENTS -----%
%%%%%%%%%%%%%%%%%%%%%%%%%%

%----- HACKS -----%

\NewEnviron{centerwide}{
	\par
	\noindent
	\makebox[\textwidth]{\parbox{1.3\textwidth}{\BODY}}
}
\NewEnviron{centerwideequation}{\[\makebox[\displaywidth]{$\displaystyle\BODY$}\]}
