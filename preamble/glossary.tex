% file: glossary-expressions.tex



\newglossaryentry{enculer-les-mouches}
{
	name=Enculer les mouches,
	description={(apparition : fin du XXe siècle, variation possible : sodomiser les drosophiles) Composé de enculer et de mouche. Permet probablement d'imager une grande difficulté ou déployer de grands efforts pour un but dérisoire. Décrit également une démarche consistant à s'attarder inutilement sur des points de détail, en faisant preuve d'une méticulosité extrême, voire excessive, au détriment de l'essentiel.}
}

\newglossaryentry{honnete-et-frequentable}
{
	name=Honnête et fréquentable,
	description={Se dit d'une fonction manipulée par un physicien, c'est-à-dire continues et dérivables autant que les nécessités de calcul l'exigeront.}
}

\newglossaryentry{poussage-de-symboles}
{
	name=Poussage de symboles,
	description={Activité passionnante à laquelle s'attèlent certains logiciens, qui consiste à dérouler les définitions, les notations et les abréviations des symboles jusqu'à arriver au résultat, et ce, sans avoir besoin d'ajouter aucune conjonction de coordination de la langue française.}
}

\newglossaryentry{quanteur}
{
	name=Quanteur,
	description={(terme vieilli, usité seulement en logique) Version plus rare de quantificateur. Dérivé savant du latin \latinexpr{quantus} (qui signifie \guiexpr{combien}) et \emphexpr{-eur}. Utilisé notamment dans l'expression (uniquement française) \guiexpr{élimination des quanteurs}. Exemple d'utilisation : \begin{quote}
		Le quanteur universel I.6 est un signe logique : une paire de parenthèses avec une variable à l'intérieur ; la partie de formule qui le suit et qui généralement contient cette variable sera délimitée par des parenthèses ad hoc, qui serviront à faire reconnaître l’étendue (le scope) de ce quanteur.\\
		--- Jean Largeault, Intuitionisme et théorie de la démonstration, 1992
	\end{quote}}
}