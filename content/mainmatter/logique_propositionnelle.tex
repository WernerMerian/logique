\chapter{Logique propositionnelle}
\label{chp.logprop}

\minitoc

\lettrine{P}{our} commencer l'étude de la logique, nous allons étudier sa forme
la plus élémentaire : la logique propositionnelle. Lorsque l'on écrit une phrase
mathématique, disons par exemple
\[\forall n \in \mathbb N, (\exists m \in \mathbb N, n = 2\times m) \text{ or }
(\exists m \in \mathbb N, n = 2 \times m + 1)\]
on peut séparer plusieurs parties :
\begin{itemize}
\item les quantificateurs.
\item les connecteurs logiques, comme \og et\fg{}, \og ou\fg{} ou \og non\fg{}.
\item les propositions atomiques, comme $n = 2 \times m$ ci-dessus.
\end{itemize}

La logique propositionnelle est une simplification de cette grammaire, dans
laquelle les propositions atomiques sont remplacées par de simple variables,
pouvant prendre la valeur $\mathrm{Vrai}$ ou $\mathrm{Faux}$, et où l'on
supprime les quantificateurs. L'étude de la logique propositionnelle, beaucoup
plus simple que le calcul des prédicats, permet de donner une première idée des
propriétés qui nous intéressent dans la logique.

\section{Définitions}

Fixons tout d'abord un ensemble $\Var$ dénombrable de variables
propositionnelles. Nous allons construire l'ensemble des propositions comme un
ensemble inductif, en ajoutant les connecteurs logiques usuels.

\begin{definition}[Propositions]
  On définit l'ensemble $\mathrm{Prop}$ des propositions du calcul
  propositionnel par la grammaire suivante :
  \[P,Q ::= x \mid \top \mid \bot \mid \lnot P \mid P \lor Q \mid P \land Q
  \mid P \to Q\]
  où $x\in \Var$ est une variable propositionnelle.
\end{definition}

Le sens des différents symboles est le suivant :
\begin{itemize}
\item $\top$ représente la proposition vraie.
\item $\bot$ représente la proposition fausse.
\item $\lnot$ représente la négation logique.
\item $\lor$ représente la disjonction logique.
\item $\land$ représente la conjonction logique.
\item $\to$ représente l'implication logique.
\end{itemize}

Par convention, nous donnons l'ordre de priorité (du plus prioritaire au moins
prioritaire) suivant : $\lnot > \land > \lor > \to$. Cette convention évite ainsi
d'écrire certaines parenthèses. De plus, comme $\land$ et $\lor$ seront montrés
associatifs, nous ne parenthèserons pas $P\lor Q\lor R$ par exemple. Pour $\to$,
nous associons à droite, c'est-à-dire que $P\to Q\to R$ signifie
$P\to (Q \to R)$. Par exemple la proposition
\[(P \to Q)\land (R \to Q) \to P\lor R \to Q\] doit se lire
\[((P \to Q) \land (R \to Q)) \to ((P \lor R) \to Q)\]

L'objectif d'une proposition est bien sûr de lui attribuer une valeur de vérité.
Evidemment, la valeur de vérite d'une proposition dépend de celle des variables.
Par exemple, $(x\lor \lnot y) \land z$ n'aura pas la même valeur de vérité
suivant si $z$ est vrai ou faux.

Pour manipuler les variables d'une propositions, il est important de définir
l'ensemble des variables impliquées dans la construction d'une proposition.

\begin{definition}[Variables libres]
  Par induction sur la structure inductive de $\Prop$, on définit pour
  $P$ l'ensemble $\VL(P)$ des variables libres de $P$ :
  \begin{itemize}
  \item si $P = x$ et $x\in\Var$, alors $\VL(P) = \{x\}$.
  \item si $P = \lnot Q$ alors $\VL(P) = \VL(Q)$.
  \item si $P = Q \lor R$ ou $P = Q \land R$ ou $P = Q \to R$,
    alors $\VL(P) = \VL(Q)\cup\VL(R)$.
  \end{itemize}
\end{definition}

\begin{remark}
  La dénomination de variable \og libre\fg{} prendra son sens dans le prochain
  chapitre. L'intérêt de définir $\VL$ et non juste l'ensemble des variables est
  d'ignorer les variables muettes, mais dans le calcul propositionnel aucune
  variable n'est muette.
\end{remark}

Une attribution de valeurs de vérité aux variables propositionnelles est appelée
un environnement. On utilisera l'ensemble $\{0,1\}$ pour signifier
$\{\mathrm{Faux},\mathrm{Vrai}\}$.

\begin{definition}[Environnement]
  Un environnement est une fonction partielle $\rho : \Var \to \{0,1\}$
  dont le domaine (l'ensemble des valeurs $x\in \Var$ sur lesquelles
  $\rho$ est défini) est fini.
\end{definition}

Un environnement permet ensuite de définir la notion de valuation, qui est la
valeur de vérité d'une proposition dans un environnement donné. On considère

\begin{definition}[Valuation]
  Soit $\rho$ un environnement. On définit par induction sur $\mathrm{Prop}$
  la fonction partielle $\Val_\rho : \Prop \to \{0,1\}$ :
  \begin{itemize}
  \item pour $x\in \Var$, si $x\in \dom(\rho)$ alors
    $\Val_\rho (x)= \rho(x)$, si $x\notin \dom(\rho)$ alors
    $\Val_\rho (x)$ n'est pas défini.
  \item soit $P$ une proposition, alors
    $\Val_\rho(\lnot P) = 1 - \Val_\rho(P)$.
  \item soient $P$ et $Q$ deux propositions, alors
    $\Val_\rho(P\lor Q)=\max(\Val_\rho(P),\Val_\rho(Q))$
  \item soient $P$ et $Q$ deux propositions, alors
    $\Val_\rho(P\land Q)=\min(\Val_\rho(P),\Val_\rho(Q))$
  \item soient $P$ et $Q$ deux propositions, alors
    $\Val_\rho(P\to Q) = \max(1 - \Val_\rho(P),\Val_\rho(Q))$
  \end{itemize}
\end{definition}

\begin{exercise}
  Soit $\rho$ un environnement et $P$ une proposition, montrer que
  $\Val_\rho(P)$ est définie si et seulement si
  $\VL(P)\subseteq \dom(\rho)$.
\end{exercise}

\begin{exercise}\label{exo:val_VL}
  Soit $P$ une proposition. Soient $\rho$ et $\rho'$ deux environnements tels
  que $\rho_{|\;\VL(P)}=\rho'_{|\;\VL(P)}$. Montrer que
  $\Val_\rho(P)=\Val_{\rho'}(P)$.
\end{exercise}

On peut donc définir la relation de vérité, dans un environnement donné, pour
une proposition.

\begin{definition}[Satisfaction]
  Soit $\rho$ un environnement et $P\in\Prop$. On définir la relation
  $\rho\models P$ par $$(\rho\models P) \defeq (\Val_\rho(P) = 1)$$
  On dit alors que $\rho$ satisfait $P$.

  Si pour tout environnement $\rho$ tel que $\VL(P)\subseteq\dom(\rho)$,
  $\rho\models P$, alors on dit que $P$ est une tautologie, et on le note
  $\models P$. Si aucun environnement ne satisfait $P$, on dit alors que $P$ est
  une antilogie.
\end{definition}

\begin{exercise}
  Montrer que $P$ est une antilogie si et seulement si $\lnot P$ est une
  tautologie. En déduire que $P$ est une tautologie si et seulement si $\lnot P$
  est une antilogie.
\end{exercise}

Donnons un premier outil pour traiter des vérités du calcul des propositions :
les tables de vérité.

\begin{definition}[Table de vérité]
  Une table de vérité pour une proposition $P$ de variables libres
  $x_1,\ldots,x_n$ est un tableau contenant $2^n$ lignes et $n+1$ colonnes, où
  chaque ligne énumère un environnement différent, où la
  \ordinalnumeralfeminin{$i$} colonne représente la valeur des environnement en
  $x_i$ et où la dernière colonne représente la valuation de $P$ pour
  l'environnement donné.
\end{definition}

\begin{example}
  Voici une table de vérité pour l'expression
  $\lnot (x_0 \lor x_1)\to \lnot x_0 \land \lnot x_1$ :
\end{example}

Une table de vérité permet de vérifier qu'une proposition est bien une tautologie
puisque, grâce à l'exercice \ref{exo:val_VL}, nous savons que tout environnement
$\rho$ contenant les variable libres d'une proposition $P$ donnée ne définit
$\Val_\rho(P)$ que sur les variables libres listées dans la table de vérité.
Ainsi, une proposition est une tautologie si et seulement si toute la dernière
colonne est remplie de $1$.

\begin{exercise}
  Montrer que $\lnot x_0\land \lnot x_1 \implies \lnot (x_0\lor x_1)$ est une
  tautologie.
\end{exercise}

La notion de tautologie peut être considérée comme la bonne notion de vérité
dans le cadre du calcul propositionnel : une proposition vraie est une
proposition qui s'évalue toujours en une formule vraie. Comme $\to$ sert à
signifier l'implication logique, dire que $A \to B$ est une tautologie revient à
dire que chaque fois que $A$ est vraie, $B$ l'est aussi. Ainsi, en notant
$A \leftrightarrow B$ pour $(A \to B) \land (B \to A)$, dire que
$A \leftrightarrow B$ est une tautologie revient à dire que $A$ et $B$ prennent
toujours la même valeur de vérité. Cette relation est l'équivalence logique :
elle traduit que deux propositions ont la même valeur.

\begin{definition}[\'Equivalence logique]
  Pour tous $P,Q\in\Prop$, on dit que $P$ et $Q$ sont logiquement équivalents,
  ce que l'on écrit $P\equiv Q$, si $\models P \leftrightarrow Q$.
\end{definition}

Un affaiblissement est la relation de conséquence logique.

\begin{definition}[Conséquence logique]
  Pour tous $P,Q\in\Prop$, on dit que $Q$ est conséquence logique de $P$, ce que
  l'on note $P\vDash Q$, si $\models P\to Q$.
\end{definition}

\begin{exercise}
  Vérifier à l'aide de tables de vérités les équivalences suivantes (appelées
  lois de De Morgan) :
  \begin{align*}
    \lnot\lnot x &\equiv x\\
    \lnot (x \land y) &\equiv \lnot x \lor \lnot y\\
    \lnot (x \lor y) &\equiv \lnot x \land \lnot y
  \end{align*}
\end{exercise}

\section[Séquents propositionnels]{Calcul des séquents propositionnels et
  complétude}

Maintenant que nous avons défini une notion satisfaisante de vérité pour une
proposition, il convient de se demander quels outils permettent de l'établir.
Pour l'instant, pour prouver qu'une proposition est une tautologie, la seule
façon de procéder est d'en construire la table de vérité. C'est une façon
largement inefficace, puisqu'elle prend une taille exponentielle en le nombre de
variables libres d'une proposition, et la preuve qu'une proposition est une
tautologie est assez vide de sens : dans l'exercice précédent, il n'a été
question que de calcul et pas de considérations logiques.

\subsection{Calcul des séquents}

Ce qui manque à notre système, c'est une syntaxe : un système simple qui va nous
permettre de justifier des tautologies. Cette syntaxe se base sur la notion de
séquent : un séquent est une paire de listes de propositions, que l'on écrira
$\Gamma\vdash \Delta$, exprimant que $\bigwedge \Gamma \to \bigvee \Delta$ est
une tautologie. L'intérêt de la relation $\vdash$ à définir est de donner un
système facile de preuve : pour prouver que $P$ est une tautologie, il suffit
de prouver $\nil\vdash \cons(P,\nil)$ en suivant les règles de base définissant
$\vdash$.

\begin{notation}
  Dorénavant, pour utiliser des listes, nous écrirons $a, \ell$ pour signifie
  $\cons(a,\ell)$ et nous n'écrirons pas $\nil$ lorsque le contexte permet
  clairement de comprendre qu'il s'agit d'une liste. Par exemple, on confondra
  la simple proposition $P$ et la liste $\cons(\nil,P)$. La notation
  $\Gamma,\Delta$ correspond à la concaténation, qui serait notée avec les
  conventions précédentes $\Gamma\oplus\Delta$.
\end{notation}

\begin{definition}[Calcul des séquents]
  On définit la relation $\vdash \subseteq \List(\Prop)^2$ par les règles
  suivantes:
  \begin{center}
    \AxiomC{$P \in \Gamma$}
    \RightLabel{ax}
    \UnaryInfC{$\Gamma\vdash P$}
    \DisplayProof
    \qquad
    \AxiomC{$\Gamma,P\vdash \Delta$}
    \AxiomC{$\Theta\vdash \Xi,P$}
    \RightLabel{cut}
    \BinaryInfC{$\Gamma,\Theta\vdash\Delta,\Xi$}
    \DisplayProof

    \vspace{0.5cm}
    \AxiomC{$\Gamma,P,Q,\Gamma'\vdash \Delta$}
    \RightLabel{le}
    \UnaryInfC{$\Gamma,Q,P,\Gamma'\vdash \Delta$}
    \DisplayProof
    \qquad
    \AxiomC{$\Gamma\vdash \Delta,P,Q\Delta'$}
    \RightLabel{re}
    \UnaryInfC{$\Gamma\vdash \Delta,Q,P,\Delta'$}
    \DisplayProof

    \vspace{0.5cm}
    \AxiomC{$\Gamma,P,P\vdash \Delta$}
    \RightLabel{lc}
    \UnaryInfC{$\Gamma,P\vdash \Delta$}
    \DisplayProof
    \qquad
    \AxiomC{$\Gamma\vdash \Delta,P,P$}
    \RightLabel{rc}
    \UnaryInfC{$\Gamma\vdash \Delta,P$}
    \DisplayProof

    \vspace{0.5cm}
    \AxiomC{$\Gamma\vdash \Delta$}
    \RightLabel{lw}
    \UnaryInfC{$\Gamma,P\vdash \Delta$}
    \DisplayProof
    \qquad
    \AxiomC{$\Gamma\vdash \Delta$}
    \RightLabel{rw}
    \UnaryInfC{$\Gamma\vdash \Delta,P$}
    \DisplayProof

    \vspace{0.5cm}
    \AxiomC{$\Gamma\vdash\Delta$}
    \RightLabel{$\bot$}
    \UnaryInfC{$\Gamma\vdash\Delta,\bot$}
    \DisplayProof
    \qquad
    \AxiomC{$\Gamma\vdash\Delta$}
    \RightLabel{$\top$}
    \UnaryInfC{$\Gamma,\top\vdash\Delta$}
    \DisplayProof

    \vspace{0.5cm}
    \AxiomC{$\Gamma\vdash \Delta,P$}
    \RightLabel{l$\lnot$}
    \UnaryInfC{$\Gamma,\lnot P\vdash \Delta$}
    \DisplayProof
    \qquad
    \AxiomC{$\Gamma,P\vdash \Delta$}
    \RightLabel{r$\lnot$}
    \UnaryInfC{$\Gamma\vdash \Delta,\lnot P$}
    \DisplayProof

    \vspace{0.5cm}
    \AxiomC{$\Gamma,P\vdash \Delta$}
    \AxiomC{$\Theta,Q\vdash \Xi$}
    \RightLabel{l$\lor$}
    \BinaryInfC{$\Gamma,\Theta,P\lor Q\vdash \Delta,\Xi$}
    \DisplayProof
    \qquad
    \AxiomC{$\Gamma\vdash \Delta,P$}
    \RightLabel{r$\lor_1$}
    \UnaryInfC{$\Gamma\vdash \Delta,P\lor Q$}
    \DisplayProof
    \quad
    \AxiomC{$\Gamma\vdash \Delta,Q$}
    \RightLabel{r$\lor_2$}
    \UnaryInfC{$\Gamma\vdash \Delta,P\lor Q$}
    \DisplayProof

    \vspace{0.5cm}
    \AxiomC{$\Gamma,P\vdash \Delta$}
    \RightLabel{l$\land_1$}
    \UnaryInfC{$\Gamma,P\land Q\vdash \Delta$}
    \DisplayProof
    \quad
    \AxiomC{$\Gamma,Q\vdash \Delta$}
    \RightLabel{l$\land_2$}
    \UnaryInfC{$\Gamma,P\land Q\vdash \Delta$}
    \DisplayProof
    \qquad
    \AxiomC{$\Gamma\vdash \Delta,P$}
    \AxiomC{$\Theta\vdash \Xi,Q$}
    \RightLabel{r$\land$}
    \BinaryInfC{$\Gamma,\Theta\vdash \Delta,\Xi,P\land Q$}
    \DisplayProof

    \vspace{0.5cm}
    \AxiomC{$\Gamma,Q\vdash \Delta$}
    \AxiomC{$\Theta\vdash \Xi,P$}
    \RightLabel{l$\to$}
    \BinaryInfC{$\Gamma,\Theta,P\to Q\vdash \Delta,\Xi$}
    \DisplayProof
    \qquad
    \AxiomC{$\Gamma,P\vdash \Delta$}
    \RightLabel{r$\to_1$}
    \UnaryInfC{$\Gamma\vdash \Delta,P\to Q$}
    \DisplayProof
    \quad
    \AxiomC{$\Gamma\vdash \Delta,Q$}
    \RightLabel{r$\to_2$}
    \UnaryInfC{$\Gamma\vdash \Delta,P\to Q$}
    \DisplayProof
  \end{center}

  On dit qu'une proposition $P$ est prouvable si $\vdash P$ est dérivable.
\end{definition}

L'objectif de la suite de cette section est de prouver que $\vdash$ définit en
fait exactement les tautologies, au sens suivant : pour toutes propositions
$P$ et $Q$, $P\vDash Q$ si et seulement si $P\vdash Q$ (ainsi, en prenant par
exemple $P = \top$, on peut montrer que $\vDash P$ si et seulement si
$\vdash P$).

\subsection{Correction du calcul des séquents}

La première étape, qui est la plus simple, est de montrer que le calcul des
séquents est correct, c'est-à-dire que si l'on arrive à dériver
$\Gamma\vdash\Delta$, alors $\bigwedge \Gamma\to\bigvee \Delta$ est une
tautologie. La preuve est simplement une induction sur la relation $\vdash$.
Comme c'est la première longue preuve par induction, nous allons la rédiger
complètement, mais il est d'usage pour prouver une induction dont les cas se
ressemblent de ne prouver que quelques cas les plus significatifs.

\begin{theorem}[Correction du calcul des séquents]
  Soient $\Gamma,\Delta\in\List(\Prop)$. Si $\Gamma\vdash\Delta$, alors
  $\models \bigwedge \Gamma\to \bigvee \Delta$.
\end{theorem}

\begin{proof}
  Il suffit de prouver le résultat par induction sur $\vdash$ :
  \begin{itemize}
  \item Si $P\in\Gamma$, alors pour tout environnement $\rho$ tel que
    $\Val_\rho(\bigwedge\Gamma\to P)$ est défini, l'égalité devient
    \begin{align*}
      \Val_\rho(\bigwedge\Gamma\to P) &= \max(1 - \Val_\rho(\bigwedge\Gamma),
      \Val_\rho(P))\\
      &=\max(1 - \min_{P\in\Gamma}(\Val_\rho(P),\Val_\rho(P)))
    \end{align*}
    Si $\Val_\rho(P) = 0$ alors $\min_{P\in\Gamma}(\Val_\rho(P)) = 0$ puisque
    $P\in\Gamma$, donc $\Val_\rho(\bigwedge\Gamma\to P) = 1$. Si $\Val_\rho(P)=1$
    alors le maximum vaut $1$.
  \item Supposons que $\models (\bigwedge\Gamma)\land P \to \bigvee\Delta$ et
    que $\models \bigwedge \Theta \to (\bigvee \Xi) \lor P$. Soit $\rho$ un
    environnement pour lequel les valuations sont bien définies. Par le calcul
    suivant :
    \begin{align*}
      \Val_\rho(\bigwedge (\Gamma,\Theta)\to\bigvee(\Delta,\Xi)) =&
      \max(1 - \Val_\rho(\bigwedge(\Gamma,\Theta)),
      \Val_\rho(\bigvee(\Delta,\Xi)))\\
      =& \max(1 - \min(\min_{P \in \Gamma}(\Val_\rho(P)),
      \min_{P\in\Theta}(\Val_\rho(P))),\\
      &\quad\max(\max_{P\in\Delta}(\Val_\rho(P)),
      \max_{P\in\Xi}(\Val_\rho(P))))\\
      =& \max(\max_{P\in \Gamma}(1-\Val_\rho(P)),\max_{P\in\Theta}(1-\Val_\rho(P)),\\
      &\quad \max_{P\in\Delta}(\Val_\rho(P)),\max_{P\in\Xi}(\Val_\rho(P)))\\
      =&\max(\max(\max_{P\in \Gamma}(1-\Val_\rho(P)),\max_{P\in\Delta}(\Val_\rho(P))),
      \\
      &\quad \max(\max_{P\in\Theta}(1-\Val_\rho(P)),\max_{P\in\Xi}(\Val_\rho(P))))\\
      =&\max(\Val_\rho(\bigwedge\Gamma\to\bigvee\Delta),\Val_\rho(\bigwedge\Theta
      \to\bigvee\Xi))
    \end{align*}
    On raisonne maintenant par cas suivant la valeur de $\Var_\rho(P)$. Si cette
    valuation vaut $0$, alors comme
    $\models \bigwedge \Theta\to (\bigvee\Xi) \lor P$ on en déduit que
    $\Val_\rho(\bigwedge\Theta\to\bigvee\Xi) = 1$, auquel cas l'identité
    précédente nous donne le résultat. Si la valuation vaut $1$, alors on
    utilise le fait que $\models(\bigwedge\Gamma)\land P\to\bigvee \Delta$ pour
    en déduire que $\Val_\rho(\bigwedge\Gamma\to\bigvee\Delta) = 1$.
  \item Remarquons que les opérations $\min$ et $\max$ sont commutatives et
    idempotentes, ce qui implique directement la validité des règles d'échange
    (le et re) et de contraction (lc et rc).
  \item Supposons que $\models\bigwedge\Gamma\to\bigvee\Delta$, alors pour tout
    environnement $\rho$, comme
    $\Val_\rho((\bigwedge\Gamma)\land P) \leq \Val_\rho(\bigwedge\Gamma)$, on
    en déduit que
    $1 - \Val_\rho(\bigwedge\Gamma) \leq 1 - \Val_\rho((\bigwedge\Gamma)\land P)$
    d'où $\Val_\rho((\bigwedge\Gamma)\land P\to\bigvee\Delta) = 1$. Un
    raisonnement similaire donne le résultat pour la règle rw.
  \item Le résultat découle directement du fait que
    $\Val_\rho(\bigvee\Delta) = \Val_\rho((\bigvee\Delta)\lor\bot)$.
  \item De même, le résultat vient de l'égalité
    $\Val_\rho(\bigwedge\Gamma) = \Val_\rho((\bigwedge\Gamma)\land\top)$.
  \end{itemize}
\end{proof}

On en déduit donc, comme $\bigwedge P = P$ et $\bigvee Q = Q$, que
$P\vdash Q \implies P \vDash Q$.
