\chapter[Topologie]{Introduction à la topologie}
\label{chp.topo}

\minitoc

\lettrine{U}{n} des principes les plus fréquents, en mathématiques, est de se
ramener à de la topologie. En algèbre, en analyse ou en géométrie, on peut
décrire nos problèmes par de la topologie. En logique, des liens existent aussi
avec la topologie, et nous avons donc besoin de bases en topologie pour les
explorer.

Cependant, cet ouvrage n'est pas un cours de topologie : nous n'allons donc pas
entrer dans les détails de ce domaine. Nous conseillons donc au lecteur étranger
à la topologie de chercher un cours plus complet. Nous allons par exemple
considérer en premier lieu la topologie générale et les constructions à partir
de filtres, pour ensuite traiter le cas des espaces métriques.

Nous aurons trois objectifs principaux.
Le premier est de définir un fragment suffisamment large de la topologie
générale pour prouver le théorème de Tychonov grâce aux ultrafiltres.
Le deuxième, qui peut être vu comme la suite du premier, est de montrer le
théorème de dualité de Stone (dans une version n'utilisant pas le vocabulaire
catégorique), qui permet de donner une lecture topologique du \cref{chp.ordres}.
Le troisième nous fera quitter le monde de la topologie générale, pour préparer
le terrain pour la théorie descriptive des ensembles, en définissant les notions
élémentaires de topologie métrique.

\section{Topologie générale et filtres}

Pour commencer, donnons une présentation générique de la topologie~: nous
étudions des espaces topologiques, que l'ont peut décrire comme des ensembles
(dont les éléments sont en général appelés \og points\fg{}) muni d'une notion de
proximité. La notion la plus générale de proximité peut être donnée au choix
par les ouverts, les fermés ou les voisinages, mais la convention est de définir
une topologie par ses ouverts. Donnons donc ces définitions et les objets les
plus élémentaires qui y sont liés.

\subsection{Définitions de topologie générale}

La définition de topologie par les ouverts peut sembler \latinexpr{ad hoc}, nous
allons donc essayer de la motiver. L'objectif d'une topologie est de définir
quels éléments sont proches et quels éléments sont éloignés les uns des autres.
Si la notion naturelle pour décrire cela est celle de distance, on peut se
demander si celle-ci est importante pour une étude topologique. En effet, le
point important de la topologie est de pouvoir décrire ce qui est plus ou moins
local, c'est-à-dire ce qui arrive \emphexpr{suffisamment proche}, et non proche
de façon absolue.

L'idée d'un ouvert dans une topologie est alors de donner une mesure fixée de
proximité~ : deux points sont proches s'ils appartiennent à un ouvert, et ils
sont d'autant plus proches qu'ils le restent pour des ouverts pris de plus en
plus petits. Ainsi un phénomène local est un phénomène qui survient dans un
certain ouvert.

On voit déjà pourquoi la notion de filtre sera intéressante : on veut qu'une
description soit de plus en plus précise à mesure qu'on affine nos ouverts,
ce se rapproche du fait qu'on puisse affiner des filtres.

Donnons donc la définition formelle d'une topologie grâce à ses ouverts.

\begin{definition}[Espace topologique]
  On appelle espace topologique une paire $(X,\Omega_X)$ où
  $\Omega_X\subseteq\powerset(X)$ est une partie vérifiant :
  \begin{itemize}
  \item $\Omega_X$ est stable par intersection finie : si $F\subfin \Omega_X$
    alors $\bigcap F \in \Omega_X$.
  \item $\Omega_X$ est stable par union quelconque : si $Y\subseteq \Omega_X$
    alors $\bigcup Y \in \Omega_X$.
  \end{itemize}

  En particulier, $X\in \Omega_X$ et $\varnothing\in \Omega_X$. On appelle
  ouvert de $X$ un élément $O\in\Omega_X$.
\end{definition}

Si un ouvert représente une mesure de localité, le complémentaire d'un ouvert a
un autre sens : il est une partie stable, mais nous en aurons une compréhension
plus précise plus tard. Ces parties, complémentaires d'ouverts, sont appelées
les fermés.

\begin{definition}[Fermé]
  Soit $(X,\Omega_X)$ un espace topologique. On appelle fermé de $X$ une partie
  $F\subseteq X$ telle que $X\setminus F \in \Omega_X$.
\end{definition}

\begin{remark}
  La structure ordonnée $(\Omega_X,\subseteq)$ est en fait une algèbre de
  Heyting complète. Remarquons que la borne inférieure d'une partie
  $Y\subseteq\Omega_X$ n'est pas forcément l'intersection : elle est le plus
  grand ouvert contenu dans $\bigcap Y$.

  On peut établir une version plus générale de la topologie en étudiant les
  algèbres de Heyting complètes, c'est ce que l'on appelle la topologie sans
  points (\foreignexpr{pointfree topology} ou \foreignexpr{pointless topology}).
\end{remark}

On voit que décrire une topologie par ses ouverts ou par ses fermés est
équivalent, puisqu'on peut retrouver l'un à partir de l'autre par l'opération
$\mathcal T \mapsto \{X\setminus T \mid T \in \mathcal T\}$. Voyons enfin
l'autre façon possible de décrire une topologie : par les voisinages.

\begin{definition}[Voisinage]
  Soit $(X,\Omega_X)$ un espace topologique et $x\in X$. On appelle ensemble des
  voisinages de $x$ l'ensemble des parties de $X$ contenant un ouvert contenant
  $x$~ :
  \[\vois_x \defeq \{V \subseteq X \mid \exists U \in \Omega_X, x\in U \land
  U\subseteq V\}\]
\end{definition}

L'ensemble des voisinages d'un point est, en particulier, un filtre.

\begin{property}
  Soit $(X,\Omega_X)$ un espace topologique, alors pour tout $x\in X$,
  l'ensemble $\vois_x$ est un filtre propre sur $(\powerset(X),\subseteq)$.
\end{property}

\begin{proof}
  Il nous suffit de vérifier les hypothèses d'un filtre propre :
  \begin{itemize}
  \item puisque $X\in\Omega_X$ et $x\in X$, $X\in \vois_x$.
  \item si $V\in \vois_x$ et $W$ est tel que $V\subseteq W$, alors on trouve
    $U\in \Omega_X$ tel que $x\in U \subseteq V$, et par transitivité de
    l'inclusion, $x\in U \subseteq W$, donc $W\in \vois_x$.
  \item si $V\in \vois_x$ et $W\in \vois_x$ alors on trouve $U$ et $U'$ des
    ouverts tels que $x\in U \subseteq V$ et $x\in U' \subseteq W$. Comme
    $\Omega_X$ est stable par intersection finie, $U\cap U' \subseteq W$ donc
    $x\in U\cap U' \subseteq V\cap W$, donc $V\cap W \in \vois_x$.
  \item l'ensemble $\varnothing$ ne contient pas $x$, donc
    $\varnothing\notin\vois_x$.
  \end{itemize}

  Ainsi $\vois_x$ est un filtre propre de $\powerset(X)$.
\end{proof}

On peut donc définir la notion de voisinage à partir des ouverts.
Réciproquement, il est possible de retrouver l'ensemble des ouverts à partir des
voisinages.

\begin{proposition}
  Soit $(X,\Omega_X)$ un espace topologique. Alors $U\in \Omega_X$ si et
  seulement si pour tout $x\in U$, $U\in\vois_x$.
\end{proposition}

\begin{proof}
  Si $U\in \Omega_X$, il est évident que pour tout $x\in U$ il existe un ouvert
  ($U$ lui-même) inclus dans $U$ et contenant $x$. Réciproquement, supposons que
  $U$ est un voisinage de tous ses points. Alors pour tout $x\in U$, on trouve
  un ouvert $U_x\subseteq U$ contenant $x$. Comme les ouverts sont stables par
  union quelconque, $\displaystyle\bigcup_{x\in U} U_x$ est encore un ouvert.
  Mais comme $U_x\subseteq U$, on en déduit que $\bigcup U_x \subseteq U$, et
  comme pour tout $x\in U, x\in U_x$, on en déduit que $U\subseteq\bigcup U_x$.
  Il en vient que $U$ est un ouvert en tant qu'union d'ouverts.
\end{proof}

\begin{remark}
  Dans cette preuve comme dans d'autres qui suivront, nous utilisons l'axiome du
  choix pour, à partir du fait que pour tout $x$ il existe un ouvert $U_x$
  ayant une propriété souhaitée, en extraire une fonction $x \mapsto U_x$.
\end{remark}

Pour définir plus facilement une topologie, il nous faut parler de la notion de
base d'une topologie. Par défaut, une famille $\mathcal T\subseteq\powerset(X)$
n'a pas de raison d'être une topologie, puisqu'il faut des propriétés de clôture
à notre famille. Cependant, la clôture par union quelconque peut s'obtenir en
considérant directement les unions d'éléments de la famille. Il nous reste à
traiter le cas des intersections finies, mais on préfère exiger directement des
famille de parties d'être stables par intersection finie.

\begin{definition}[Base d'une topologie, topologie engendrée]
  Soit un ensemble $X$. On dit qu'une partie $\mathcal B\subseteq\powerset(X)$
  est une base de topologie si elle contient $X$ et si, pour toutes parties
  $U,V\in \mathcal B$, il existe une famille $\mathcal A \subseteq \mathcal B$
  telle que $\displaystyle U \cap V = \bigcup \mathcal A$.

  Si $\mathcal B\subseteq\powerset(X)$ est une base de topologie, alors on
  appelle topologie engendrée par $\mathcal B$, que l'on note
  $\Omega_{\mathcal B}$, l'ensemble
  \[\Omega_{\mathcal B} \defeq \Bigg\{\bigcup_{B \in F} B\;\Bigg|\;
  F\subseteq \mathcal B\Bigg\}\]
\end{definition}

\begin{exercise}
  Montrer que $\Omega_{\mathcal B}$ est bien une topologie.
\end{exercise}

\begin{remark}
  Si la famille $\mathcal B$ est stable par intersections finies, alors c'est
  une base de topologie puisque dans ce cas $U \cap V = \bigcup \{U \cap V\}$.
\end{remark}

\begin{example}
  Donnons des exemples de topologies :
  \begin{itemize}
  \item on dispose sur tout ensemble $X$ de la topologie discrète donnée par
    $\powerset(X)$ tout entier.
  \item on dispose sur tout ensemble $X$ de la topologie grossière donnée
    par $\{\varnothing,X\}$.
  \item sur $\mathbb R$, on dispose de la topologie engendrée par les
    intervalles ouverts, c'est-à-dire de la forme
    $\{x \in \mathbb R \mid a < x < b\}$ où
    $a,b\in\mathbb R\cup\{-\infty,+\infty\}$. C'est la topologie naturelle que
    l'on utilise sur $\mathbb R$.
  \item plus généralement, sur un ensemble ordonné $(X,\leq)$, on peut définir
    la topologie de l'ordre comme la topologie engendrée par les intervalles
    ouverts, de la forme $\{x\in X\mid a < x < b\}$ pour
    $a,b\in X\cup\{-\infty,+\infty\}$. Remarquons que par exemple sur
    $\mathbb N$, la topologie de l'ordre coïncide avec la topologie discrète.
  \end{itemize}
\end{example}

Nous verrons plus tard une famille d'espaces topologiques, les espace métriques.

\'Etant donnée une partie $Y\subseteq X$, qui n'est ni un ouvert ni un fermé,
disons par exemple $\{x\in\mathbb R\mid 1 \leq x < 2\}$, peut-on lui associer
un ouvert (respectivement un fermé) naturellement ? La réponse est oui, en
considérant l'intérieur (respectivement l'adhérence).

\begin{definition}[Intérieur]
  Soit $(X,\Omega_X)$ un espace topologique et $Y\subseteq X$. On appelle
  intérieur de $Y$, que l'on note $\inter Y$, le plus grand ouvert contenu dans
  $Y$.
\end{definition}

\begin{definition}[Adhérence]
  Soit $(X,\Omega_X)$ un espace topologique et $Y\subseteq X$. On appelle
  adhérence de $X$, que l'on note $\adher Y$, le plus petit fermé contenant
  $Y$.
\end{definition}

\begin{exercise}
  On considère $\mathbb R$ muni de la topologie décrite plus haut. Montrer que
  $\inter{\mathbb Q} = \varnothing$ et que $\adher{\mathbb Q} = \mathbb R$.
\end{exercise}

On peut aussi définir la notion de point intérieur et de point adhérent,
correspondant naturellement au fait d'appartenir à l'intérieur (respectivement à
l'adhérence), mais sous une forme plus intuitive.

\begin{definition}[Point intérieur]
  Soit $(X,\Omega_X)$ un espace topologique et $Y\subseteq X$. On dit qu'un
  point $x$ est intérieur à $Y$ si $Y\in\vois_x$.
\end{definition}

\begin{definition}[Point adhérent]
  Soit $(X,\Omega_X)$ un espace topologique et $Y\subseteq X$. On dit qu'un
  point $x$ est adhérent à $Y$ si pour tout ouvert $U$ contenant $x$,
  $U\cap Y\neq \varnothing$.
\end{definition}

\begin{exercise}
  Montrer qu'un point est intérieur à une partie si et seulement s'il est dans
  son intérieur.
\end{exercise}

\begin{exercise}\label{exo.carac.adhe}
  Montrer qu'un point est adhérent à une partie si et seulement s'il est dans
  l'adhérence de cette partie.
\end{exercise}

Enfin, définissons la notion de continuité. On la présente pour l'instant comme
une simple notion de morphisme associée à notre structure d'espace topologique.
Nous en verrons une interprétation grâce aux filtres.

\begin{definition}[Fonction continue]
  Soient $(X,\Omega_X)$ et $(Y,\Omega_Y)$ deux espaces topologiques. Une
  fonction $f : X \to Y$ est dite continue si pour tout $U\in\Omega_Y$,
  l'ensemble $f^{-1}(U)$ est un ouvert de $X$.
\end{definition}

\begin{remark}
  Si l'on considère un espace topologique comme l'algèbre de Heyting complète de
  son ensemble d'ouverts, alors la définition que nous venons de donner signifie
  que $f$ induit grâce à $f^{-1}$ un morphisme d'algèbres de Heyting complètes.
  Cependant, il est important de voir que là où $f : X \to Y$,
  $f^{-1} : \powerset(Y)\to\powerset(X)$, on a donc une inversion du sens entre
  le point de vue algébrique par un ensemble ordonné et le point de vue
  analytique comme un ensemble de points muni d'une topologie.
\end{remark}

La notion duale, en quelque sorte, de fonction continue, est celle de fonction
ouverte. Celle-ci est largement moins utile, mais nous servira parfois.

\begin{definition}[Fonction ouverte]
  Soient $(X,\Omega_X)$ et $(Y,\Omega_Y)$ deux espaces topologiques. Une
  fonction $f : X \to Y$ est dite ouverte si pour tout $U \in \Omega_X$,
  l'ensemble $f(U)$ est un ouvert de $Y$.
\end{definition}

On définit alors la notion d'isomorphisme, dans le cas des espaces topologiques,
comme une fonction continue admettant un inverse continu.

\begin{definition}[Homéomorphisme]
  Soient $(X,\Omega_X)$ et $(Y,\Omega_Y)$ deux espaces topologiques, une
  fonction $f : X \to Y$ est appelée un homéomorphisme entre $X$ et $Y$ s'il
  existe une fonction continue $g : Y \to X$ telle que $g\circ f = \id_X$ et
  $f\circ g = \id_Y$.
\end{definition}

\begin{exercise}
  Montrer qu'une fonction est un homéomorphisme si et seulement si elle est
  une bijection continue dont la réciproque (en tant que fonction ensembliste)
  est elle aussi continue.
\end{exercise}

\begin{exercise}
  Montrer qu'une fonction est un homéomorphisme si et seulement si elle
  est une bijection continue et ouverte.
\end{exercise}

\subsection{Filtres, limites, séparation}

On aborde en général la notion de limite de façon métrique. La définition la
plus souvent vue d'une limite est la suivante, pour une fonction $f$ admettant
une limite en $a$ valant $b$ :
\[\forall \varepsilon > 0, \exists \delta > 0, \forall x \in \mathbb R,
|x - a| < \delta \implies |f(x) - b| < \varepsilon\]
Si nous étudions cette formule en essayant de traduire ce qu'elle énonce au
niveau topologique, on peut commencer par remplacer les inégalités en valeur
absolue par l'appartenance à certains ouverts : au lieu de
$|f(x) - b| < \varepsilon$, on peut dire que $f(x)$ appartient à un
ouvert, qui est $(b-\varepsilon,b+\varepsilon)$. De même, plutôt que de dire
que $|x - a| < \delta$, on peut dire que $x$ appartient à un certain ouvert.

Puisqu'on quantifie sur n'importe quel $\varepsilon$, on peut de la même façon
considérer qu'on quantifie sur tous les ouverts contenant $b$, sans perte de
généralité sur les voisinages de $b$ : on voit alors apparaître un filtre.
En effet, plutôt que de dire que $f(x)$ est dans tel ouvert autour de $b$,
on peut dire que $x$ est dans la préimage de l'ouvert par $f$. On trouve donc,
plutôt qu'une suite de quantificateurs quelconque, une considération de
finesse. En fait, on peut reformuler cette formule par la formule équivalente
suivante~ :
\[f\vois_a \supseteq \vois_b\]
c'est-à-dire que le filtre image de $\vois_a$ par $f$ est plus fin que le filtre
$\vois_b$.

Ainsi, la convergence est une notion particulièrement adaptée à une définition
par les filtres. On va donc pouvoir tout d'abord définir la convergence d'un
filtre.

\begin{definition}[Convergence d'un filtre]
  Soit un espace topologique $(X,\Omega_X)$ et un filtre $\mathcal F$, on
  définit le fait que $\mathcal F$ converge vers $a\in X$ par~ :
  \[\lim \mathcal F = a \defeq \mathcal F \supseteq \vois_a\]
\end{definition}

Le sens de cette définition est qu'un filtre converge vers $a$ lorsque les
parties qu'il décrit permettent de se rapprocher autant que l'on veut de
$a$.

On peut maintenant facilement définir ce que signifie la convergence pour une
fonction en un point.

\begin{definition}[Convergence d'une fonction]
  Soient $(X,\Omega_X),(Y,\Omega_Y)$ deux espaces topologiques et $f : X \to Y$.
  On dit que $f$ converge vers $b\in Y$ en $a\in X$ (au sens des filtres) si
  \[\fillim_{x\to a}f(x) = b\defeq f\vois_a \supseteq\vois_b\]
\end{definition}

Remarquons que comme tout voisinage de $a$ contient $a$, il est nécessaire que
$\displaystyle f(a)\in\bigcap\vois_b$. On a donc envie de dire que $f(a)=b$,
mais on n'a pas de raison de dire \latinexpr{a priori} que les filtres d'un
voisinage ne contiennent qu'un élément dans leur intersection. Prenons par
exemple sur $\{a,b,c\}$ la topologie \latinexpr{ad hoc} suivante :
\[\{\{a,b\},\{c\},\{a,b,c\},\varnothing\}\]
Les deux éléments $a$ et $b$ ne peuvent pas être distingués.

Cela motive donc une nouvelle notion topologique : la séparation. Un axiome de
séparation est un axiome assurant que pour $a\neq b$, on a bien
$\vois_a\neq\vois_b$. De nombreux axiomes de séparation existent, mais nous ne
verrons que le plus classique, qu'on appelle axiome de séparation $T_2$. Il est
à la fois suffisamment puissant pour assurer un bon comportement à nos espaces,
et suffisamment faible pour être vérifié par de nombreux espaces. La
terminologie anglaise donne le nom \foreignexpr{Hausdorff} aux espaces vérifiant
l'axiome $T_2$, mais comme nous n'utiliserons que cet axiome de séparation, nous
parlerons simplement d'espace séparé.

\begin{definition}[Espace séparé]
  Soit $(X,\Omega_X)$ un espace topologique. On dit que cet espace est séparé
  si pour tous $x,y\in X$, il existe une paire $(U_x,U_y)\in (\Omega_X)^2$ telle
  que
  \[\begin{cases}
  x\in U_x\\
  y\in U_y\\
  U_x\cap U_y = \varnothing
  \end{cases}\]
\end{definition}

Montrons qu'on a alors l'unicité de la limite.

\begin{proposition}[Unicité de la limite]
  Soit $\mathcal F$ un filtre propre sur un espace $(X,\Omega_X)$ séparé, alors
  si $\mathcal F$ converge, il ne peut converger que vers un seul élément de
  $X$.
\end{proposition}

\begin{proof}
  Supposons que $a$ et $b$ sont deux limites possibles de $\mathcal F$. Par
  séparation, on trouve $U_a$ et $U_b$ deux ouverts disjoints, contenant
  respectivement $a$ et $b$. Comme $U_a$ et $U_b$ sont des ouverts, on en déduit
  que $U_a\in\vois_a$ et $U_b\in\vois_b$. De plus, par convergence de
  $\mathcal F$ vers $a$ et $b$, on en déduit que $U_a\in\mathcal F$ et
  $U_b\in\mathcal F$, donc $U_a\cap U_b\in\mathcal F$, c'est-à-dire
  $\varnothing\in\mathcal F$, ce qui contredit le fait que $\mathcal F$ est
  un filtre propre.
\end{proof}

En particulier, si $f$ converge en un point $a$ et que son espace d'arrivée est
un espace séparé, alors on a unicité du point vers lequel $f$ converge. Cela
signifie aussi que si $f$ converge en un point $a$, ce point ne peut être que
$f(a)$, puisque celui-ci est le seul à appartenir à $\bigcap f\vois_a$.

La notion de limite permet aussi de caractériser la continuité : une fonction
est continue quand la limite commute avec l'application par $f$.

\begin{property}\label{prop.carac.cont.fil}
  Soient deux espaces $(X,\Omega_X)$ et $(Y,\Omega_Y)$ séparés, et une fonction
  $f : X \to Y$. Alors $f$ est continue si et seulement si
  \[\forall x \in X, \lim_{y \to x}^\mathcal F f(y) = f(x)\]
\end{property}

\begin{proof}
  Soit $x \in X$. Tout d'abord, supposons que $f$ est continue montrons que
  $f\vois_x \supseteq \vois_{f(x)}$.

  Soit $V\in\vois_{f(a)}$ et $U$ un ouvert contenant $f(x)$ contenu dans $V$
  (donné par la définition du fait d'être un voisinage de $f(x)$).
  Comme $f$ est continue, cela signifie que $f^{-1}(U)$ est aussi un ouvert. Or
  $f^{-1}(U)\subseteq f^{-1}(V)$ et $x\in f^{-1}(U)$ puisque $f(x)\in U$, donc
  $f^{-1}(V)\in\vois_x$, ce qui signifie exactement que $V\in f\vois_x$, d'où
  l'inclusion.

  Supposons maintenant que $f\vois_x\supseteq \vois_{f(x)}$, et montrons que
  $f$ est continue.

  Soit $U$ un ouvert de $Y$, montrons que $f^{-1}(U)$ est un ouvert de $X$.
  Comme $U$ est un ouvert, pour tout $x\in U$ on a $U\in\vois_x$. On en
  déduit que pour tout $y\in f^{-1}(U)$, $U\in\vois_{f(y)}$ donc, par inclusion,
  $U\in f\vois_y$, c'est-à-dire que $f^{-1}(U)\in\vois_y$ pour tout
  $y\in f^{-1}(U)$. On vient donc de montrer que $f^{-1}(U)$ est un voisinage de
  tous ses éléments, ce qui par une caractérisation précédente signifie qu'il
  est un ouvert. Donc la préimage d'un ouvert est un ouvert~: $f$ est donc
  continue.
\end{proof}

\begin{exercise}\label{exo.conv.suite}
  On dit qu'une suite $u : \mathbb N \to X$ converge lorsque le filtre image du
  filtre de Fréchet sur $\mathbb N$ converge vers un certain élément $x\in X$.
  Montrer que cela revient à vérifier la formule usuelle~:
  \[\exists x\in X,\forall V\in\vois_x, \exists n \in \mathbb N,
  \forall m \geq n, u_m \in V\]
\end{exercise}

\subsection{Valeur d'adhérence, compacité}

La notion de séparation permet de s'assurer qu'au plus une limite est possible,
pour un filtre donné. Il est cependant possible qu'aucune limite n'existe~ : par
exemple, si on considère dans $\mathbb R$ le filtre des parties contenant
$\{1,2\}$, on ne peut trouver aucun filtre de voisinage dont il est plus fin.
Cependant, on voudrait pouvoir dire de ce filtre qu'il converge partiellement
vers $1$ et vers $2$, qui sont des limites plus faibles.

C'est cette notion de limite plus faible qui est donnée par la définition de
valeur d'adhérence. Une valeur d'adhérence d'un filtre est une valeur vers
laquelle le filtre pourrait converger, s'il était plus fin. Dans notre exemple
précédent, le filtre pourrait converger si l'on décidait de le rendre plus fin
pour contenir $\{1\}$ ou $\{2\}$.

\begin{definition}[Valeur d'adhérence d'un filtre]
  Soit un espace topologique $(X,\Omega_X)$ et un filtre propre $\mathcal F$. On
  dit que $x$ est une valeur d'adhérence de $\mathcal F$ s'il existe un filtre
  propre $\mathcal H$ plus fin que $\mathcal F$ qui converge vers $x$.
\end{definition}

Même si l'espace est séparé, un filtre peut avoir plusieurs valeurs d'adhérence.
Cependant, il est aussi possible qu'un filtre n'en possède aucune. Si on prend
par exemple $\mathbb N$ avec la topologie discrète, le filtre de Fréchet n'a pas
de valeur d'adhérence, puisqu'un filtre de voisinage serait un ultrafiltre
principal, et que nous avons déjà prouvé que le filtre de Fréchet n'était
contenu dans aucun ultrafiltre principal.

Donnons une caractérisation simple des valeurs d'adhérence d'un filtre.

\begin{proposition}\label{prop.carac.adh}
  Soit un espace topologique $(X,\Omega_X)$ séparé et un filtre propre
  $\mathcal F$. Un point $x$ est une valeur d'adhérence de $\mathcal F$ si et
  seulement si $\displaystyle x\in \bigcap_{F\in\mathcal F} \adher F$.
\end{proposition}

\begin{proof}
  Supposons que $x$ est une valeur d'adhérence de $\mathcal F$. On trouve donc
  un filtre propre $\mathcal H$ contenant $\mathcal F$ et $\vois_x$. Soit
  $F\in\mathcal F$, montrons que $x\in \adher F$. Pour cela, en utilisant
  l'\cref{exo.carac.adhe}, montrons que
  tout ouvert $U$ contenant $x$, $U\cap F\neq\varnothing$. Soit donc
  $U\in \Omega_X$ contenant $x$, c'est donc un voisinage de $x$, donc en
  particulier $U\in\mathcal H$, et $F\in\mathcal H$, donc par compatibilité
  $U\cap F \in\mathcal H$. Comme $\mathcal H$ est un filtre propre,
  $U\cap F \neq \varnothing$, donc $x\in\adher F$.

  Réciproquement, supposons que $x$ est adhérent à tous les éléments de
  $\mathcal F$, montrons qu'alors il existe un filtre propre $\mathcal H$
  contenant $\mathcal F$ et $\vois_x$. Pour cela, il nous suffit de montrer que
  $\mathcal F\cup\vois_x$ a la propriété de l'intersection finie (le filtre
  engendré sera alors un filtre propre vérifiant les conditions plus haut).

  Comme $\mathcal F$ et $\vois_x$ sont des filtres, quitte à prendre
  l'intersection des éléments, on peut considérer qu'on a un élément
  $F\in\mathcal F$ et un élément $U\in\vois_x$, et il nous suffit alors de
  montrer que $F\cap U \neq\varnothing$. Comme $x\in\adher F$, si
  $F\cap U = \varnothing$, alors $F\subseteq (X\setminus U)$ et $X\setminus U$
  est un fermé, donc $\adher F \subseteq (X\setminus U)$, donc
  $x\in X\setminus U$, sauf que $x\in U$ par hypothèse, ce qui est
  contradictoire. On en déduit donc que $F\cap U \neq\varnothing$, donc que
  $\mathcal F\cup\vois_x$ a la propriété de l'intersection finie, d'où le
  résultat.
\end{proof}

On peut alors se demander quand cet ensemble des valeurs d'adhérence est vide.
C'est le cas par exemple du filtre de Fréchet sur $\mathbb N$ muni de la
topologie discrète, et cela s'interprète par le fait que les parties du filtre
vont à l'infini. D'une certaine manière, celles-ci sont des voisinages d'un
point se situant lui-même à l'infini (mais qui n'est pas dans l'espace, menant
à cette absence de valeur d'adhérence).

On motive donc la notion d'espace compact comme un espace dans lequel un tel cas
ne peut pas arriver. En reprenant le cas du filtre de Fréchet, on voit en
particulier qu'on peut prendre l'ensemble des points
$(\{n\})_{n\in\mathbb N}$ et en combiner une infinité à chaque fois pour
pouvoir construire une suite de parties, toutes contenues les unes dans
les autres, telle qu'il faut une infinité d'étapes pour vider les parties
ainsi construites, mais qu'un nombre fini d'étapes laisse un nombre infini
d'éléments.

En inversant le processus et en cherchant à faire croître les segments
initiaux $([0,n])_{n\in\mathbb N}$, on voit qu'il faut un nombre infini d'étapes
pour pouvoir recouvrir tout $\mathbb N$~ : la compacité est précisément la
négation de ce fait.

\begin{definition}[Compacité]
  Soit $(X,\Omega_X)$ un espace topologique. On dit que cet espace est compact
  si pour toute partie $R\subseteq \Omega_X$, si $\bigcup R = X$ alors il existe
  une partie finie $F\subfin R$ telle que $\bigcup F = X$.
\end{definition}

Cette propriété exprime que de toute recouvrement de $X$ par des ouverts, on
peut extraire un recouvrement fini. En prenant, au lieu des ouverts, des fermés,
cela nous donne qu'une famille infinie dont toute intersection finie est non
vide est une famille dont l'intersection est non vide.

\begin{property}
  Un espace topologique $(X,\Omega_X)$ est compact si et seulement si pour tout
  ensemble $S$ de fermés tel que pour tout $F\subfin S$,
  $\bigcap F \neq\varnothing$,
  on a $\bigcap S \neq\varnothing$.
\end{property}

\begin{proof}
  A $S$ on associe la famille $R$ par
  \[R \defeq \{X\setminus A\mid A \in S\}\]
  en écrivant la contraposée de la proposition ci-dessus, on obtient~:
  \[\bigcap S =\varnothing\implies \exists F\subfin S, \bigcap F = \varnothing\]
  En passant au complémentaire~:
  \[\bigcup_{A \in S}(X\setminus A) = X\implies
  \exists F\subfin S, \bigcup_{A\in F}(X\setminus A) = X\]
  ce qui correspond exactement au fait que si $X = \bigcup R$ alors il existe
  $F\subfin R$ tel que $\bigcup F = X$, où $R$ est un ensemble d'ouvert
  (puisqu'un ensemble de complémentaires de fermés).
\end{proof}

Le fait que tout filtre admette une valeur d'adhérence est maintenant une
conséquence directe.

\begin{proposition}
  Soit $(X,\Omega_X)$ un espace topologique, alors c'est un espace compact si
  et seulement si tout filtre propre admet une valeur d'adhérence.
\end{proposition}

\begin{proof}
  Supposons que $(X,\Omega_X)$ est un espace compact.
  
  En utilisant la \cref{prop.carac.adh}, on sait que
  $\displaystyle\bigcap_{F\in\mathcal F}\adher F$ est l'ensemble des valeurs
  d'adhérence. Chaque $\adher F$ est un fermé, et on sait par définition d'un
  filtre propre que pour tout $A\subfin\mathcal F$,
  $\displaystyle\bigcap_{F\in A}\adher F$ est non vide, puisqu'il contient au
  moins $\bigcap A$ qui est non vide. Par compacité, on en déduit que l'ensemble
  des valeur d'adhérence de $\mathcal F$ est non vide.

  Supposons que tout filtre propre admette une valeur d'adhérence.

  On veut montrer que toute famille de fermés ayant la propriété de
  l'intersection finie possède une intersection non vide. Soit $\mathcal C$ une
  telle famille, on définit
  \[\mathcal F \defeq \filclose{\mathcal C}\]
  qui est un filtre propre car $\mathcal C$ vérifie la propriété de
  l'intersection finie. Ainsi $\mathcal F$ a une valeur d'adhérence, disons
  $x$. Par la \cref{prop.carac.adh},
  $\displaystyle x\in\bigcap_{F\in \mathcal F}\adher F$ donc
  $\displaystyle x\in\bigcap_{F\in \mathcal C}\adher F$, mais comme les $F$ sont
  des fermés, $\adher F = F$, donc $x\in \bigcap \mathcal C$. Ainsi
  $\bigcap\mathcal C$ est non vide.
\end{proof}

Dans le cas particulier d'un ultrafiltre, une valeur d'adhérence est un point
vers lequel le filtre converge. On a donc la proposition suivante.

\begin{proposition}
  Soit $(X,\Omega_X)$ un espace topologique compact. Alors tout
  ultrafiltre $\mathcal U$ sur $X$ converge vers un point. Ce point est unique
  si l'espace est séparé.
\end{proposition}

\begin{proof}
  Par compacité, on sait que $\mathcal U$ a une valeur d'adhérence, disons
  $a$. On sait donc qu'il existe un filtre propre $\mathcal F$ contenant
  $\mathcal U$ et tel que $\vois_a\subseteq \mathcal F$. Mais comme
  $\mathcal U$ est un ultrafiltre, cela signifie que $\mathcal F = \mathcal U$,
  donc que $\vois_a\subseteq\mathcal U$, c'est-à-dire que $\mathcal U$
  converge vers $a$. Par unicité de la limite, on sait que ce $a$ est unique si
  l'espace est séparé.
\end{proof}

\subsection{Topologie induite et topologie produit}

Nous allons maintenant nous intéresser à des moyens de construire de nouvelles
topologies à partir d'anciennes topologies. Les deux moyens que nous allons
voir ici sont la topologie induite et la topologie produit. Un autre outil
classique de topologie est la topologie quotient, mais comme nous n'aurons pas à
l'utiliser (et qu'elle demande plus de technicité), nous l'occultons ici.

La topologie induite est naturelle à considérer : étant donné un espace
topologique $(X,\Omega)$ et une partie $Y\subseteq X$, on veut pouvoir munir $Y$
d'une topologie, en partant de la topologie $\Omega$. Par exemple, si l'on
travaille dans $\mathbb R$ avec sa topologie habituelle, on peut parler assez
facilement d'être au voisinage d'un point lorsqu'on se restreint à
$\mathbb R_+$. Nous allons généraliser ce procédé.

\begin{definition}[Topologie induite]
  Soit $(X,\Omega_X)$ un espace topologique, et $Y\subseteq X$. On appelle la
  topologie induite par $Y$ l'espace topologique $(Y,\Omega_Y)$ défini par
  \[\Omega_Y \defeq \{U \cap Y \mid U \in \Omega_X\}\]
\end{definition}

\begin{proof}
  On vérifie que $\Omega_Y$ est bien une topologie, en utilisant le fait que
  pour la stabilité par intersections finies, il suffit de vérifier que $Y$ est
  un ouvert et que l'intersection de deux ouverts est un ouvert~ :
  \begin{itemize}
  \item soit $A\subseteq \Omega_Y$, pour chaque $U\in A$, par définition, on
    trouve un ouvert $V_U\in \Omega_X$ tel que $U = V_U \cap Y$. Alors
    \[\bigcup A = \bigcup_{U\in A} (V_U\cap Y) =
    \Bigg(\bigcup_{U\in A} V_U\Bigg)\cap Y\]
    donc $\bigcup A \in \Omega_Y$.
  \item $Y = X \cap Y$ donc $Y\in \Omega_Y$.
  \item si $U,V\in\Omega_Y$, alors on trouve $U',V'\in\Omega_X$ tels que
    $U = U'\cap Y$ et $V = V'\cap Y$, et $U\cap V = U'\cap Y \cap V'\cap Y$
    soit $U\cap V = (U'\cap V')\cap Y$, donc $U\cap V \in \Omega_Y$.
  \end{itemize}
\end{proof}

\begin{exercise}
  Montrer que si $(X,\Omega_X)$ est un espace topologique séparé et $Y$ une
  partie de $X$, alors $Y$ munie de la topologie induite est un espace séparé.
\end{exercise}

La topologie induite permet de définir une partie compacte dans un espace
quelconque.

\begin{definition}[Partie compacte]
  Soit $(X,\Omega_X)$ un espace topologique. On dit que $K\subseteq X$ est une
  partie compacte de $X$ si $K$ muni de la topologie induite est un espace
  compact.
\end{definition}

\begin{remark}
  De façon équivalente, une partie $K$ est compacte si pour toute famille
  d'ouverts $A$, si $\bigcup A \supseteq K$ alors on a $F\subfin A$ tel que
  $\bigcup A \supseteq K$.
\end{remark}

\begin{property}
  Soit $(X,\Omega_X)$ un espace topologique séparé et $K$ une partie compacte de
  $X$. Alors $K$ est fermé.
\end{property}

\begin{proof}
  Pour montrer que $K$ est fermé, on va chercher à exprimer $K$ comme une
  intersection de fermés, en montrant que pour chaque $y\in X\setminus K$ il
  existe un fermé $C_y$ contenant $K$ et ne contenant pas $y$ (l'intersection
  de tous les $C_y$ sera alors $K$).

  Soit donc $y\in X\setminus K$. Pour chaque $x\in K$, on trouve deux ouverts
  $U_x$ et $V_x$ disjoints, $U_x$ contenant $y$ et $V_x$ contenant $x$. Par
  compacité de $K$, comme $\{V_x\mid x \in K\}$ forme un recouvrement de $K$,
  on extrait $F\subfin K$ tel que $\{V_x\mid x \in F\}$ recouvre $K$.
  On en déduit que $\displaystyle\bigcap_{x\in F}U_x$ ne contient aucun des
  $V_x$, et est donc disjoint de $K$. Comme c'est une intersection finie
  d'ouverts, c'est aussi un ouvert~: on a donc un ouvert $U$ contenant
  $y$ et disjoint de $K$, donc $X\setminus U$ est un fermé contenant $K$ et
  ne contenant pas $y$.
\end{proof}

Dans un espace compact séparé, on a une caractérisation forte des parties
compactes~: ce sont les fermés de l'espace.

\begin{proposition}\label{prop.compac.equiv.ferme}
  Soit $(X,\Omega_X)$ un espace compact séparé, alors une partie $K\subseteq X$
  est compacte si et seulement si elle est fermée.
\end{proposition}

\begin{proof}
  Montrons qu'un fermé est compact. Soit un recouvrement
  $A\subseteq\Omega_K$ de $K$ par des ouverts de $K$. On trouve donc, pour
  chaque $U\in A$, un ouvert $V_U\in\Omega_X$ tel que $U = V_U\cap K$. Par
  hypothèse, $\displaystyle\bigcup_{U \in A} V_U \supseteq K$, et comme
  $K$ est un fermé, $X\setminus K$ est un ouvert, donc
  $\{V_U\mid U \in A\}\cup\{X\setminus K\}$ est un recouvrement de $X$ :
  on en déduit par compacité de $X$ qu'on peut trouver un sous-recouvrement
  fini $F\subfin \{V_U\mid U \in A\}\cup\{\{X\setminus K\}\}$. On voit alors
  que $(\bigcup F)\setminus (X\setminus K) \supseteq K$, donc en prenant
  \[F' \defeq F\setminus\{X\setminus K\}\]
  on trouve un recouvrement de $K$. De plus, tous les éléments de $F'$ sont
  des éléments de $A$, donc on a un sous recouvrement fini de $A$.

  Réciproquement, comme l'espace est séparé, une partie compacte est fermée.
\end{proof}

Intéressons-nous maintenant à la topologie produit. Celle-ci, au lieu de
munir d'une topologie un espace plus petit, permet à partir d'une famille
d'espaces, d'obtenir un espace topologique sur le produit de cette famille.

Plutôt que de simplement imaginer l'ensemble $\prod X_i$ et de chercher à le
munir d'une topologie, présentons la construction de la topologie produit par
une approche plus catégorique.

On a vu que la notion de morphisme, pour des espaces topologiques, est celle de
fonction continue, pour laquelle l'image réciproque d'un ouvert est encore un
ouvert. De plus, si $\prod X_i$ est muni d'une topologie, alors il est légitime
d'exiger de cette topologie qu'elle rende les projections
$\pi_j : \prod X_i \to X_j$ continues (les projections sont en quelque sorte
l'élément le plus primitif pour un produit). On veut donc une topologie qui
rende ces fonctions continues, mais beaucoup de choix de topologies sont
possibles. En particulier, on demande que les $\pi_j^{-1}(U)$ pour chaque
$U$ ouvert de $X_i$ soient ouverts, qui sont de la forme $\prod X_i \times U$.

A partir du moment où ces images réciproques appartiennent à la topologie, on
peut ajouter autant d'ouverts que l'on veut et garder la continuité des
projections : c'est donc la possibilité de rajouter des ouverts qui empêche
d'avoir un choix canonique. La réponse est donc d'imposer que la topologie
produit est la plus petite topologie vérifiant cette continuité des projections.

On donnera une définition plus explicite de la topologie produit, mais celle-ci
nous servira avant tout à montrer la propriété définie ci-dessus.

\begin{definition}[Cylindre]
  Soit $(X_i)_{i\in I}$ une famille d'ensembles. Soit $F\subfin I$ et
  $(Y_i)_{i\in F} \in \prod_{i \in F} \powerset(X_i)$, on appelle cylindre en $Y$
  l'ensemble
  \[\cylin Y \defeq \Bigg\{(x_i) \in \prod_{i \in I} X_i \;\Bigg|\;
  \forall j \in F, x_j \in Y_j\Bigg\}\]
  que l'on peut aussi écrire
  $\displaystyle\left(\prod_{j\notin F} X_j\right)\times
  \left(\prod_{i\in F} Y_i\right)$.
\end{definition}

\begin{property}
  Soit une famille $(X_i,\Omega_{X_i})_{i\in I}$ d'espaces topologiques. La
  famille
  \[\Cylind((X_i,\Omega_{X_i})_{i\in I})\defeq\Bigg\{\cylin Y \;\Bigg|\;
  F\subfin I, (Y_i)_{i\in F}\in \prod_{i\in F}\Omega_{X_i}\Bigg\}\]
  est une base de topologie sur $\displaystyle\prod_{i \in I} X_i$.
\end{property}

\begin{proof}
  On montre donc que cette famille est stable par intersection finie (donc que
  $\prod_{i\in I} X_i$ est dans cette famille, et qu'elle est stable par
  intersections binaires)~:
  \begin{itemize}
  \item En prenant la famille vide, on a directement
    $\cylin\varnothing = \prod_{i\in I} X_i$.
  \item Soient $\cylin Y$ et $\cylin Z$ deux éléments de la famille. Soient
    $F$ et $F'$ les parties finies de $I$ apparaissant dans la définition de
    $Y$ et de $Z$. Remarquons d'abord que
    \[\cylin Y \cap \cylin Z = \Bigg\{(x_i)\in \prod_{i \in I} X_i \;\Bigg|\;
    \forall j \in F, x_j \in Y_j, \forall k \in F', x_k \in Z_j\Bigg\}\]

    On définit la famille $A$ sur $F\cup F'$ par
    \[A_i =
    \begin{cases}
      Y_i\cap Z_i\text{ si } i\in F\cap F'\\
      Y_i\text{ si } i \in F \setminus F'\\
      Z_i\text{ si } i \in F'\setminus F
    \end{cases}\]
    alors $\cylin A$ appartient à la famille ci-dessus, car $Y_i\cap Z_i$ est
    une intersection d'ouvert, donc un ouvert, et $Y_i$ (respectivement $Z_i$)
    est un ouvert, donc on a bien une famille finie d'ouverts. De plus,
    $\cylin A = \cylin Y \cap \cylin Z$ en déroulant les définitions.
  \end{itemize}

  Ainsi $\Cylind((X_i,\Omega_{X_i})_{i\in I})$ est bien une base de topologie.
\end{proof}

La topologie produit est donc la topologie engendrée par les cylindre d'ouverts.

\begin{definition}[Topologie produit]
  Soit $(X_i,\Omega_{X_i})_{i\in I}$ une famille d'espaces topologiques. Alors
  l'espace topologique produit est l'espace
  \[\Bigg(\prod_{i\in I} X_i, \Cylind((X_i,\Omega_{X_i})_{i\in I})\Bigg)\]
\end{definition}

On peut maintenant se demander comment nos deux propriétés, la séparation et la
compacité, sont préservées par produit. Dans le cas de la séparation, le
résultat est assez facile.

\begin{proposition}
  Soit $(X_i,\Omega_{X_i})_{i\in I}$ une famille d'espaces séparés, alors l'espace
  produit de cette famille est aussi séparé.
\end{proposition}

\begin{proof}
  Soient $(x_i)_{i\in I}\neq (y_i)_{i\in I}$ deux éléments différents du produit.
  Puisqu'ils sont différents, on peut trouver $i\in I$ tel que $x_i \neq y_i$.
  Comme $X_i$ est séparé, on trouve $U_x$ et $U_y$ deux ouverts de $X_i$
  disjoints contenant respectivement $x_i$ et $y_i$. Par définition de notre
  topologie produit, $\cylin{U_x}$ et $\cylin{U_y}$ sont des ouverts, et ils
  sont disjoints puisqu'un élément du produit ne peut avoir qu'une valeur en $i$
  (et $U_x\cap U_y = \varnothing$). De plus, $(x_i)\in \cylin{U_x}$ et
  $(y_i)\in\cylin{U_y}$, donc on a trouvé deux ouverts disjoints séparant nos
  deux éléments.
\end{proof}

Le produit préserve aussi la compacité, mais ce fait est plus technique à
montrer, et constitue le théorème de Tychonov, dont on peut montrer (dans cette
version) qu'il est équivalent à l'axiome du choix.

\begin{theorem}[Tychonov \cite{TychonovThm}]
  Soit $(X_i,\Omega_{X_i})_{i\in I}$ une famille d'espaces topologiques compacts.
  Alors l'espace produit $\displaystyle\prod_{i\in I} X_i$ muni de la topologie
  produit est aussi un espace compact.
\end{theorem}

\begin{proof}
  Pour cela, il nous suffit de montrer que tout filtre propre sur $\prod X_i$
  admet une valeur d'adhérence.

  Soit $\mathcal F$ un filtre propre sur $\prod X_i$. Pour tout $i\in I$, on
  écrit
  \[\mathcal F_i \defeq (\pi_i)\mathcal F\]
  où $\pi_i$ est la projection de $\prod X_i$ sur $X_i$. Comme $\mathcal F$ est
  un filtre propre, $\mathcal F_i$ est aussi un filtre propre. Comme
  chaque $X_i$ est compact, on en déduit que chaque $\mathcal F_i$ admet une
  valeur d'adhérence~: soit $x_i$ une valeur d'adhérence de $\mathcal F_i$
  (choisir un $x_i$ pour chaque $i\in I$ demande l'axiome du choix), on
  peut donc définir l'élément
  $\displaystyle (x_i)_{i\in I}\in \prod_{i \in I}X_i$
  dont on veut montrer qu'il est une valeur d'adhérence de $\mathcal F$.

  Soit $F\in \mathcal F$ (on notera $F_i \defeq \pi_i(F)$), on veut montrer que
  $(x_i)\in \adher F$. Soit donc $U$ un ouvert contenant $(x_i)$. Sans perte de
  généralité, on suppose que $U$ est un cylindre car tout ouvert est un union
  de cylindre, donc si $(x_i)$ appartient à une union de cylindre, on peut
  trouver un cylindre en particulier auquel il appartient. On note donc
  $U = \cylin{(U_i)_{i\in J}}$ avec $J\subfin I$. On veut donc montrer que pour
  tout $i\in J$, $U_i\cap F_i\neq\varnothing$ (pour $j\notin J$, la condition
  est simplement que $X_i \cap F_i\neq\varnothing$, ce qui signifie que
  $F_i\neq\varnothing$, mais $F_i\in\mathcal F_i$ est non vide par hypothèse).
  On sait que $x_i \in \adher{F_i}$ puisque $x_i$ est une valeur d'adhérence de
  $\mathcal F_i$, et $U_i$ est par construction un ouvert de $X_i$ contenant
  $x_i$, donc $U_i\cap F_i \neq\varnothing$. Donc pour tout $i\in I$,
  $x_i\in \adher{F_i}$.

  En remarquant que
  \[\bigcap_{G \in \mathcal F_i}\adher G =
  \pi_i\left(\bigcap_{G \in \mathcal F} \adher G\right)\]
  grâce à la continuité de $\pi_i$, il vient donc que
  $\displaystyle (x_i) \in \bigcap_{G \in\mathcal F}\adher G$, donc $\mathcal F$
  admet une valeur d'adhérence, donc $\displaystyle\prod_{i\in I} X_i$ est
  compact.
\end{proof}

\section{Dualité de Stone}

Nous profitons de cette introduction à la topologie pour aborder la dualité de
Stone, remontant aux travaux du mathématicien éponyme (en particulier à
\cite{StoneDuality}), qui relie fortement la logique booléenne et la topologie de
certains espaces. Dans son énoncé le plus théorique, il établit une équivalence
de catégories, mais comme nous ne présentons pas les éléments suffisants de
théorie des catégories pour l'énoncer, nous nous contenterons de montrer la
construction des espaces à partir d'algèbres de Boole et réciproquement, et le
fait qu'elles sont presque inverses l'une de l'autre.

Nous étudierons dans un premier lieu les ouverts-fermés et les espaces de Stone,
et donnerons la construction seulement ensuite.

\subsection{Ouverts-fermés et espace de Stone}

Notre premier objectif est, étant donné un espace topologique, de trouver une
algèbre de Boole qui lui est naturellement associée. Les ouverts sont stables
par union et par intersection, donnant un bon candidat, mais ils ne le sont pas
par complémentaire. De même, les fermés ne sont pas stables par complémentaires.
Cependant, les ensembles qui sont à la fois ouverts et fermés, eux, sont bien
stables par complémentaires.

\begin{definition}[Ouvert-fermé]
  Soit $(X,\Omega_X)$ un espace topologique. On dit que $C\subseteq X$ est un
  ouvert-fermé de $X$ si $C\in \Omega_X$ et $X\setminus C \in \Omega_X$.

  On notera $K\Omega_X$ l'ensemble des ouverts-fermés de $X$.
\end{definition}

\begin{remark}
  En réalité, $K\Omega_X$ désigne l'ensemble des ouverts compacts de
  $(X,\Omega)$, mais nous verrons que dans le cas qui nous intéresse, cette
  notion coïncide avec celle des ouverts-fermés.
\end{remark}

\begin{property}
  Soit $(X,\Omega_X)$ un espace topologique. L'ensemble ordonné
  $(K\Omega_X,\subseteq)$ est une algèbre de Boole.
\end{property}

\begin{proof}
  Puisqu'on travaille sur une partie de $\powerset(X)$ et qu'on prend comme
  candidats pour les constructions de l'algèbre de Boole les constructions
  induites par $\powerset(X)$, on vérifie seulement que celles-ci sont bien
  des ouverts-fermés :
  \begin{itemize}
  \item $X\in\Omega_X$ et $\varnothing\in\Omega_X$, donc $X$ et $\varnothing$
    sont des ouverts-fermés.
  \item si $C,C'\in K\Omega_X$, alors $C\cup C'$ est un ouvert, et $C\cap C'$
    est un ouvert (parce que $C$ et $C'$ sont des ouverts), et de même
    $C\cup C'$ est un fermé et $C\cap C'$ est un fermé.
  \item si $C\in K\Omega_X$, alors comme $C$ est un ouvert, $X\setminus C$ est
    un fermé, et comme $C$ est un fermé, $X\setminus C$ est un ouvert, donc
    $X\setminus C \in K\Omega_X$.
  \end{itemize}
\end{proof}

\begin{exercise}
  Soit $f : X \to Y$ une fonction continue entre deux espaces topologiques
  $(X,\Omega_X)$ et $(Y,\Omega_Y)$. Montrer que $f^{-1}$ induit un morphisme
  d'algèbres de Booles entre $(K\Omega_Y,\subseteq)$ et
  $(K\Omega_X,\subseteq)$.
\end{exercise}

On veut ainsi considérer des espaces pour lesquels l'information de leur algèbre
de Boole des ouverts-fermés suffit à la construction. Cela signifie donc qu'on
se limite à des espaces dont les ouverts \og importants\fg sont des
ouverts-fermés, ce qui signifie en fait qu'on veut que notre espace admette une
base d'ouverts-fermés. Ces espaces sont dits de dimension zéro
(ou zéro-dimensionnels).

\begin{definition}[Espace de dimension zéro]
  Un espace topologique $(X,\Omega)$ est dit de dimension zéro s'il existe une
  base de $\Omega$ constituée d'ouverts-fermés de $\Omega$.
\end{definition}

Nous verrons que par la construction que nous utiliserons pour avoir un espace
topologique à partir d'une algèbre de Boole, il est nécessaire d'assurer que
tout ultrafiltre converge vers un unique point, ce qui nous pousse à vouloir
étudier des espaces compacts séparés. Nous avons maintenant les éléments
suffisants pour définir ce qu'est un espace de Stone.

\begin{definition}[Espace de Stone]
  On dit qu'un espace topologique $(X,\Omega_X)$ est un espace de Stone s'il
  est compact, séparé, et admet une base d'ouverts-fermés.
\end{definition}

\subsection{Spectre d'une algèbre de Boole}

En algèbre, le spectre d'un anneau est l'ensemble de ses idéaux premiers. Dans
le cas d'une algèbre de Boole, qui est un cas particulier d'anneau, on peut donc
définir la notion de spectre, qui coïncide avec l'ensemble des idéaux maximaux
(puisque dans une algèbre de Boole, premier et maximal sont confondus). On
préfère ici traiter du cas des ultrafiltres, donc on définit le spectre comme
l'ensemble des ultrafiltres, qui par dualité est équivalent.

Le spectre d'un anneau peut être muni d'une topologie : la topologie de Zariski.
Celle-ci est engendrée (en engendrant la topologie par une famille de fermés
plutôt que par une famille d'ouverts) par les images des idéaux par la fonction
\[I \mapsto \{P\mid P\text{ est un idéal premier de } A, I\subseteq P\}\]
pour $A$ l'anneau considéré. Dans notre cas, on veut générer une topologie
sur les ultrafiltres à partir des éléments de notre algèbre de Boole. Pour cela,
on a une façon naturelle d'associer à un élément un filtre : pour un élément
$x\in B$, on considère le filtre $\{ y \in B \mid x \leq y\}$ qui est le filtre
engendré par $\{x\}$.

La topologie du spectre d'une algèbre de Boole sera donc la topologie engendrée
par les ensembles d'ultrafiltres contenant $x$. On commence par introduire la
fonction construisant notre base de topologie, et on montre qu'elle se comporte
correctement.

\begin{definition}[Extension d'un élément]
  Soit une algèbre de Boole $(B,\leq)$. On définit la fonction
  \[\begin{array}{ccccc}
  \ext & : & B & \longrightarrow & \powerset(\Spec(B))\\
  & & x & \longmapsto & \{\mathcal U \in \Spec(B)\mid x \in \mathcal U\}
  \end{array}\]
\end{definition}

\begin{property}
  Soit une algèbre de Boole $(B,\leq)$. Pour tous $x,y\in B$, les propriétés
  suivantes sont vérifiées :
  \begin{itemize}
  \item $\ext(\top) = \Spec(B)$
  \item $\ext(\bot) = \varnothing$
  \item $\ext(x\lor y) = \ext(x)\cup\ext(y)$
  \item $\ext(x\land y) = \ext(x)\cap\ext(y)$
  \item $\ext(\lnot x) = \Spec(B)\setminus\ext(x)$
  \end{itemize}
\end{property}

\begin{proof}
  Ces propriétés découlent directement du fait que $\ext$ est un morphisme
  d'algèbres de Boole, ce que l'on va prouver :
  \begin{itemize}
  \item tout filtre contient $\top$, donc $\ext(\top) = \Spec(B)$.
  \item tout filtre propre (ce qu'est un filtre premier) ne contient pas $\bot$,
    donc $\ext(\bot) = \varnothing$.
  \item on montre que la condition $x\lor y \in \mathcal U$, pour un filtre
    premier $\mathcal U$, est équivalente à $x\in\mathcal U$ ou
    $y\in\mathcal U$ : un sens se déduit de la définition même d'être un filtre
    premier, et l'autre sens se déduit du fait que $\mathcal U$ est clos par le
    haut. Ainsi
    \[\{\mathcal U\mid x\lor y \in \mathcal U\} =
    \{\mathcal U \mid x\in\mathcal U \text{ ou } y\in\mathcal U\}\]
  \item de même, on sait que pour un filtre, $x\land y \in \mathcal U$ est
    équivalent à $x\in\mathcal U$ et $y\in\mathcal U$, d'où l'égalité.
  \end{itemize}
\end{proof}

On sait donc, en particulier, que la famille $\{\ext(x)\mid x\in B\}$ est une
base de topologie (puisque $B$ est un inf demi-treillis, on a la stabilité par
intersection). On peut donc définir le spectre d'une algèbre de Boole.

\begin{definition}[Spectre d'une algèbre de Boole]
  Soit $(B,\leq)$ une algèbre de Boole. On définit le spectre de $B$ par
  l'ensemble $\Spec(B)$ muni de la topologie engendrée par
  $\{\ext(x)\mid x \in B\}$.
\end{definition}

Notre définition d'espace de Stone se justifie maintenant par le fait qu'un
spectre est un tel espace.

\begin{property}
  Soit $(B,\leq)$ une algèbre de Boole. Alors le spectre de $B$ est un espace de
  Stone.
\end{property}

\begin{proof}
  On vérifie les axiomes d'un espace de Stone :
  \begin{itemize}
  \item $\{\ext(x)\mid x \in B\}$ est une base d'ouverts-fermés. Ce sont des
    ouverts par définition de notre topologie, et comme
    $\ext(\lnot x)=\Spec(B)\setminus\ext(x)$, ce sont aussi des fermés, d'où
    $\ext(x)\in K\Omega_{\Spec(X)}$.
  \item L'espace est séparé : soient $\mathcal U$ et $\mathcal V$ deux filtres
    premiers sur $B$, différents. Comme ils sont maximaux, on peut trouver
    $x\in \mathcal U\setminus \mathcal V$, mais alors $\ext(x)$ et
    $\ext(\lnot x)$ forment deux ouverts disjoints tels que
    $\mathcal U \in\ext(x)$ et $\mathcal V \in\ext(\lnot x)$
    (car $x\notin\mathcal V$ donc $\lnot x \in \mathcal V$).
  \item L'espace est compact~: on veut montrer qu'une famille de fermés
    $\mathcal C$ vérifiant la propriété d'intersection finie a une
    intersection non vide. Comme la topologie est engendré par les $\ext(x)$,
    on peut dire sans perte de généralité que
    $\mathcal C = \{\ext(x_i)\mid i\in I\}$ avec $x_i \in B$ (quitte à
    avoir plus d'éléments et à ce que le $\mathcal C$ original soit une
    intersection d'éléments du nouveau $\mathcal C$). On définit maintenant le
    filtre
    \[\mathcal A_\mathcal C = \filclose{\{x_i\mid i \in I\}}\]
    et on veut montrer que ce filtre est un filtre propre. Il nous suffit de
    montrer que $\{x_i\mid i \in I\}$ a la propriété de l'intersection finie~:
    soit $F\subfin I$, on a alors l'identité
    $\displaystyle\ext(\bigwedge_{i \in F} x_i) = \bigcap_{i \in F}\ext(x_i)$,
    mais l'expression de droite est une intersection finie de fermés dans
    $\mathcal C$~: c'est donc par hypothèse un ensemble non vide. On trouve
    donc $\mathcal V \in \displaystyle\ext(\bigwedge_{i\in F} x_i)$, qui est un
    ultrafiltre et ne contient donc pas $\bot$. On en déduit donc que
    $\displaystyle\bigwedge_{i \in F} x_i \neq \bot$. Ainsi
    $\mathcal A_\mathcal C$ est bien un filtre propre~: on peut donc par le
    \cref{thm.ultrafilter.lemma} l'étendre en un ultrafiltre
    $\mathcal U_\mathcal C$ qui, comme $\mathcal A_\mathcal C$ contient tous les
    $x_i$, contient aussi tous les $x_i$, c'est donc un élément de
    $\bigcap \ext(x_i)$, d'où
    \[\bigcap \mathcal C \neq \varnothing\]
    donc $\Spec(B)$ est compact.
  \end{itemize}
  On a donc prouvé que le spectre de $B$ est un espace de Stone.
\end{proof}

On dispose maintenant d'une fonction
$(B,\leq) \mapsto (\Spec(B),\Omega_{\Spec(B)})$ et d'une fonction
$(X,\Omega_X)\mapsto (K\Omega_X,\subseteq)$. On peut voir assez directement
que ces fonctions ne sont pas inverses, puisque le spectre de l'algèbre des
ouverts-fermés de $X$ est une partie de $\powerset(X)$. On veut donc que ces
fonctions soient des inverses \og faible\fg~: appliquer une fonction puis
l'autre nous donne deux structures isomorphes, même si potentiellement
différentes.

\subsection{Dualité de Stone}

Il nous reste à vérifier que ces espaces sont isomorphes.

\begin{lemma}
  Soit $(B,\leq)$ une algèbre de Boole. On a un isomorphisme d'algèbres de Boole
  \[(B,\leq) \cong (K\Omega_{\Spec(B)},\subseteq)\]
\end{lemma}

\begin{proof}
  On a montré que la fonction $\ext$ était un morphisme d'algèbres de Boole.
  On montre donc que c'est une bijection~:
  \begin{itemize}
  \item $\ext$ est injective~: si $x\neq y$ alors soit $x < y$, soit
    $y < x$, soit $x$ et $y$ sont incomparables. Dans les trois cas,
    quitte à inverser $x$ et $y$, on voit que $\lnot x \land y \neq \bot$. On
    sait donc que $\filclose{\{\lnot x,y\}}$ peut être étendu en un ultrafiltre,
    qui est donc dans $\ext(y)$ et pas dans $\ext(x)$, donc
    $\ext(x)\neq\ext(y)$.
  \item $\ext$ est surjective~: soit $K$ un ouvert-fermé de $\Spec(B)$. Par
    définition du fait que $K$ est un ouvert, on peut l'écrire comme une union~:
    \[K = \bigcup_{i \in I} \ext(x_i)\]
    où $x_i \in B$ et $I$ est un ensemble quelconque. On sait que $\Spec(B)$
    est compact, donc comme $K$ est fermé, il est aussi compact. On en déduit
    qu'on peut trouver $F\subfin I$ tel que
    \[K = \bigcup_{i \in F} \ext(x_i)\]
    Mais alors $\displaystyle K = \ext\left(\bigvee_{i \in F} x_i\right)$,
    donc $K$ est dans l'image de $\ext$.
  \end{itemize}

  On a donc un morphisme d'algèbres de Boole bijectif, c'est donc un
  isomorphisme d'algèbres de Boole (en tant qu'isomorphisme d'anneaux).
\end{proof}

\begin{lemma}
  Soit $(X,\Omega_X)$ un espace de Stone, on a un homéomorphisme
  \[(X,\Omega_X)\cong (\Spec(K\Omega_X),\Omega_{\Spec(K\Omega_X)})\]
\end{lemma}

\begin{proof}
  Le candidat pour cet homéomorphisme est la fonction
  \[\begin{array}{ccccc}
  \varphi & : & X & \longrightarrow & \Spec(K\Omega_X)\\
  & & x & \longmapsto & \{K \in K\Omega_X\mid x \in K\}
  \end{array}\]

  On vérifie que cette fonction retourne bien un élément de l'espace d'arrivée,
  pour tout $x\in X$~:
  \begin{itemize}
  \item $x\in X$ donc $X\in \varphi(x)$, de même $x\notin\varnothing$ donc
    $\varnothing\notin\varphi(x)$.
  \item si $K\subseteq K'$ et $x\in K$ alors $x\in K'$, donc $\varphi(x)$ est
    clos par le haut.
  \item si $K\cup K' \in \varphi(x)$, alors $x\in K$ ou $x\in K'$, par
    définition de l'union, donc $K\in\varphi(x)$ ou $K'\in\varphi(x)$.
  \item si $K\in\varphi(x)$ et $K'\in\varphi(x)$ alors $x\in K\cap K'$, donc
    $K\cap K'\in\varphi(x)$.
  \end{itemize}
  Ainsi $\varphi(x)$ est bien un filtre premier sur $K\Omega_X$.

  Vérifions maintenant que $\varphi$ est bijective~:
  \begin{itemize}
  \item si $x\neq y$, on peut trouver un ouvert-fermé contenant $x$ et pas $y$
    (ce qui suffit à prouver que $\varphi(x)\neq\varphi(y)$). En effet, par
    séparation, on trouve $U_x$ et $U_y$ deux ouverts disjoints, le premier
    contenant $x$ et le second $y$. Comme $\Omega_X$ est un espace de Stone on
    peut écrire $U_x$ comme union d'ouverts-fermés~: on trouve donc un
    ouvert-fermé $K$ tel que $x\in K$ et, comme $U_x\cap U_y=\varnothing$, on
    en déduit que $y\notin K$, d'où le résultat.
  \item $\varphi$ est surjective. Soit $\mathcal F$ un filtre premier sur
    $K\Omega_X$. Comme l'espace est compact et séparé, on trouve un unique
    $x$ dans l'adhérence de $\mathcal F$~: on en déduit que
    \[x\in\bigcap_{K\in \mathcal F}\adher K\]
    mais chaque $K$ est un fermé, donc $\adher K = K$, donc tous les éléments
    de $\mathcal F$ contiennent $x$, donc $\mathcal F\subseteq \varphi(x)$.
    Comme $\mathcal F$ est un ultrafiltre, on déduit de l'inclusion que
    $\mathcal F = \varphi(x)$. Donc $\varphi$ est surjective.
  \end{itemize}

  Comme on sait que $\varphi$ est une fonction bijective, il nous suffit de
  montrer que c'est une fonction continue et ouverte, c'est-à-dire telle que
  l'image d'un ouvert est un ouvert. En effet, l'image réciproque de sa fonction
  réciproque est exactement son image directe. Vérifions donc ces deux
  conditions~:
  \begin{itemize}
  \item Soit $U$ un ouvert de $\Spec(K\Omega_X)$. Par hypothèse, on peut
    l'écrire
    \[U = \bigcup_{i\in I} \ext(K_i)\]
    où $K_i\in K\Omega_X$ et $I$ est un ensemble quelconque. Il nous suffit
    de montrer que $\varphi^{-1}(\ext(K_i))$ est un ouvert-fermé de $X$,
    puisqu'on sait qu'une telle union est alors un ouvert.

    Soit donc $K\in K\Omega_X$, montrons que $\varphi^{-1}(\ext(K))$ est un
    ouvert-fermé. Par définition, $\ext(K)$ se réécrit
    \[\ext(K) = \{\mathcal U \in \Spec(K\Omega_X)\mid K \in \mathcal U\}\]
    mais comme $\varphi$ est surjectif, on peut trouver un ensemble
    $K'\subseteq X$ vérifiant l'équation $\ext(K) = \varphi(K')$, c'est-à-dire
    tel que $\varphi^{-1}(\ext(K)) = K'$. L'ensemble $K'$ est
    \[K' \defeq \{x \mid K \in \varphi(x)\}\]
    c'est-à-dire $\{x\in X\mid x\in K\}$, soit $K$, qui est par hypothèse
    un ouvert-fermé.
  \item Comme précédemment, on peut se ramener à montrer que l'image d'un
    ouvert-fermé est un ouvert-fermé. Soit donc $K\in\Omega_X$, montrons que
    $\varphi(K)$ est un ouvert-fermé de $\Spec(K\Omega_X)$. Par définition~:
    \[\varphi(K) \defeq \{\{K' \in K\Omega_X\mid x \in K'\}\mid x \in K\}\]
    Montrons que $\varphi(K) = \ext(K)$. Soit $\mathcal U \in \varphi(K)$,
    par définition on trouve $x\in K$ tel que $\mathcal U = \varphi(x)$.
    On a alors, comme $x\in K$, que $K\in \varphi(x)$, donc
    $\mathcal U \in \ext(K)$. Réciproquement, soit $\mathcal U \in \ext(K)$.
    On sait donc que $K\in \mathcal U$, et $\mathcal U$ converge vers un
    unique élément $x$, donc (car $K$ est fermé) $x\in K$. On en déduit donc
    que $\mathcal U \in\varphi(x)$, donc que $\mathcal U \in \varphi(K)$.

    On en déduit donc que $\varphi(K) = \ext(K)$, ce qui signifie en particulier
    que $\varphi(K)$ est un ouvert-fermé.
  \end{itemize}

  Ainsi $\varphi$ est un homéomorphisme.
\end{proof}

On peut donc énoncer une version faible du théorème de dualité de Stone.

\begin{theorem}[Dualité de Stone \cite{StoneDuality}]
  On a une correspondance, bijective à isomorphisme près, entre les algèbres
  de Boole et les espaces de Stone.
\end{theorem}

Le vrai théorème de dualité de Stone s'énonce sur les catégories, et ajoute le
fait que ce quasi-isomorphisme (que l'on appelle équivalence dans le contexte
catégorique) fait aussi correspondre les morphismes, mais dans les directions
différentes : une fonction continue $X \to Y$ induit un morphisme d'algèbres de
Boole $K\Omega_Y\to K\Omega_X$ (et réciproquement). Il est un important
représentant du phénomène de dualité géométrie / algèbre, qui établit des
correspondances entre d'un côté le point de vue géométrique (ou encore
topologique) et de l'autre le point de vue algébrique. Le nom de dualité vient
du fait qu'une action géométrique se traduit par une action algébrique dans
l'autre sens (et réciproquement).

\section{Topologie métrique}

Cette dernière section s'attardera sur la topologie métrique, qui est un cas
particulier mais très fréquent de topologie (et ayant le bon goût de beaucoup
mieux se comporter).

Nous aborderons ici les notions d'espace métrique, de partie dense, d'espace
séparable et d'espace complet, pour pouvoir définir les espaces polonais qui
seront étudiés en théorie descriptive des ensembles.

\subsection{Espace métrique}

Dans un espace métrique, on s'éloigne de l'idée de la topologie générale de
considérer des parties pour décrire la proximité, et on utilise une fonction
donnant explicitement la distance entre deux points. On appelle une telle
fonction, très naturellement, une distance.

\begin{definition}[Distance]
  Soit $X$ un ensemble. On appelle distance sur $X$ une fonction
  $d : X \times X \to \mathbb R_+$ vérifiant les axiomes suivants :
  \begin{itemize}
  \item séparation : $\forall x,y\in X, d(x,y) =0 \implies x = y$
  \item symétrie : $\forall x,y\in X, d(x,y)=d(y,x)$
  \item inégalité triangulaire :
    $\forall x,y,z\in X, d(x,z) \leq d(x,y)+d(y,z)$
  \end{itemize}

  On utilise aussi le terme \og métrique\fg pour désigner une distance.
\end{definition}

\begin{definition}[Espace métrique]
  Un espace métrique est un couple $(X,d)$ où $X$ est un ensemble et $d$ est une
  distance sur $X$.
\end{definition}

Un espace métrique a naturellement une topologie, ce qui est heureux vu le
thème de notre chapitre.

\begin{definition}[Boule, topologie métrique]
  Soit $(X,d)$ un espace métrique. Pour $x\in X$ et $\varepsilon \geq 0$, on
  définit la boule ouverte centrée en $x$ de rayon $\varepsilon$ par
  \[B_\varepsilon(x) \defeq \{y\in X \mid d(x,y) < \varepsilon\}\]
  On définit de façon analogue la boule fermée~:
  \[\clB_\varepsilon(x)\defeq\{y\in X \mid d(x,y)\leq\varepsilon\}\]

  On appelle topologie métrique de $(X,d)$ la topologie engendrée par la
  famille
  \[\{B_\varepsilon(x) \mid \varepsilon > 0, x \in X \}\]
\end{definition}

\begin{remark}
  Dans ce cas-là, notre famille n'est pas stable par intersection finie, même
  si elle reste une base de topologie. En effet, l'intersection de deux boules
  ouvertes n'a pas de raison d'être une boule ouverte, mais on peut quand même
  écrire cette intersection comme une union quelconque de boules ouvertes.
  Si $x \in B_\varepsilon(a) \cap B_{\varepsilon'}(b)$, alors on peut trouver
  un $\delta$ tel que $B_\delta(x)$ est inclus dans cette intersection, et ainsi
  considérer l'union de toutes les boules $B_\delta(x)$ pour $x$ dans
  l'intersection.
\end{remark}

\begin{exercise}
  Soit un ensemble $X$, montrer que la distance discrète
  \[\begin{array}{ccccc}
  d & : & X \times X & \longrightarrow & \mathbb R\\
  & & (x,y) & \longmapsto &
  \begin{cases}
    0\text{ si } x = y\\
    1\text{ sinon}
  \end{cases}
  \end{array}\]
  est bien une distance. Montrer que la topologie métrique associée est la
  topologie discrète sur $X$.
\end{exercise}

\begin{exercise}[Sur les espaces ultra-métriques]
  Soit un ensemble $X$. On dit qu'une distance $d$ est ultramétrique lorsqu'elle
  vérifie les hypothèses d'une distance à l'exception de la propriété suivante,
  remplaçant l'inégalité triangulaire~:
  \[\forall x,y,z\in X, d(x,z)\leq \max(d(x,y),d(y,z))\]

  Montrer qu'une distance ultramétrique est bien une distance. Montrer que pour
  tous $x,y\in X$ et $\varepsilon > 0$, si
  $B_\varepsilon(x)\cap B_\varepsilon(y)\neq\varnothing$ alors
  $B_\varepsilon(x) = B_\varepsilon(y)$. Ainsi, dans un espace ultramétrique, tous
  les points d'une boule ouverte en sont centre.
\end{exercise}

\begin{exercise}
  Montrer qu'un espace métrique est séparé (on pourra prendre, pour deux points
  $x\neq y$, deux boules de rayon $d(x,y)/3$).
\end{exercise}

\begin{remark}
  Avec l'exemple de la distance discrète, on voit que dans le cas général,
  $\clB_\varepsilon(x)$ n'est pas toujours l'adhérence de $B_\varepsilon(x)$~: en
  prenant $\varepsilon = 1$, la boule ouverte est un singleton fermé mais la
  boule fermée est l'espace tout entier.
\end{remark}

On peut généraliser la notion de distance pour établir la distance entre un
élément et une partie (ou entre deux parties).

\begin{definition}[Distance entre parties]
  Soit $(X,d)$ un espace métrique, $Y\subseteq X$ et $x\in X$. On définit la
  distance de $x$ à $Y$ par
  \[d(x,Y)\defeq \inf_{y\in Y} d(x,y)\]
  et la distance d'une partie $Y\subseteq X$ à une partie $Z\subseteq X$ par
  \[d(Y,Z)\defeq \inf_{y\in Y, z \in Z} d(y,z)\]
\end{definition}

On en déduit une caractérisation de l'adhérence par les distances.

\begin{proposition}
  Soit $(X,d)$ un espace métrique et $Y\subseteq X$. Alors pour tout $x\in X$,
  on a l'équivalence suivante :
  \[x\in \adher Y \iff d(x,Y) = 0\]
\end{proposition}

\begin{proof}
  Supposons que $x\in \adher{Y}$ et montrons que $d(x,Y) = 0$. Soit
  $\varepsilon > 0$, on va montrer qu'il existe $y\in Y$ tel que
  $d(x,y)< \varepsilon$ : par définition de $x\in \adher{Y}$, on peut trouver
  un élément dans $B_\varepsilon(x)\cap Y$ (puisque l'intersection avec un
  ouvert contenant $x$ est non vide), mais cet élément $y$ vérifie justement
  $d(x,y) < \varepsilon$, d'où le résultat.

  Réciproquement, supposons que $d(x,Y) = 0$. Soit $\varepsilon > 0$, on veut
  montrer qu'il existe $y\in B_\varepsilon(x)\cap Y$. Comme on sait que
  $\inf d(x,Y) = 0$, on peut trouver $y\in Y$ tel que $d(x,y) <\varepsilon$,
  et cet élément vérifie $y\in B_\varepsilon(x)\cap Y$, d'où le fait que
  $x\in \adher Y$.
\end{proof}

\begin{exercise}
  Trouver deux parties de $\mathbb R$ qui sont à une distance $0$ l'une de
  l'autre et qui ont une intersection vide.
\end{exercise}

\subsection{Caractérisations séquentielles}

Un intérêt des espaces métriques, outre le fait qu'ils sont souvent bien plus
faciles à visualiser que des espaces topologiques quelconques, est que leur
filtre de voisinages peut être engendré par un ensemble dénombrable. Pour le
voir, il suffit de remarquer qu'on peut engendrer les voisinages d'un point
$x$ par la famille des $B_{1/n}(x)$ pour $n \in \mathbb N^*$.

La conséquence principale que nous voulons en tirer est la suivante : pour
caractériser des comportements topologiques dans un espace métrique, il suffit
de l'étudier pour des suites. Cette conséquence vient du fait que, pour
caractériser une convergence, il suffit de rencontrer tous les voisinages d'un
point. Pour ce faire, dans le cas général, on peut avoir besoin d'un nombre
excessivement grand (et indénombrable) de points, ce qui empêche une suite
d'atteindre la limite. Par contre, avec un filtre dont une base est
dénombrable, il suffit de rencontrer chaque ensemble de cette base, ce qui peut
se faire avec un nombre dénombrable de points.

On a vu dans l'\cref{exo.conv.suite} qu'on pouvait définir la convergence
d'une suite à l'aide de filtres. \'Evidemment, on utilisera dans le cas
métrique la définition plus pragmatique avec les quantificateurs :
\[\lim_{n \to \infty} u_n = \ell \defeq \forall \varepsilon > 0,
\exists n \in \mathbb N, \forall m > n, d(u_m,\ell) < \varepsilon\]

On va donc montrer, dans un premier temps, que les notions déjà vues peuvent
être caractérisées par des suites. Commençons par un exercice pour nous mettre
en jambe (qui découle directement de ce la proposition précédente).

\begin{exercise}
  Soit un espace métrique $(X,d)$, $Y\subseteq X$ et $x\in X$. Montrer que
  $x\in \adher Y$ si et seulement s'il existe une suite
  $(x_n)\in Y^\mathbb N$ telle que $\lim x_n = x$.
\end{exercise}

Pour pouvoir généraliser la notion de valeur d'adhérence aux suites, on définit
d'abord la notion de sous-suite.

\begin{definition}[Sous-suite]
  Soit un espace métrique $(X,d)$ et une suite $(x_n)\in X^\mathbb N$. On dit
  que $(y_n)$ est une sous-suite, ou suite extraite, de $(x_n)$, s'il existe
  $\varphi : \mathbb N \to \mathbb N$ une fonction strictement croissante
  telle que
  \[\forall n \in \mathbb N, y_n = x_{\varphi(n)}\]

  On appelle la fonction $\varphi$ de cette définition l'extractrice de
  $(y_n)$.
\end{definition}

On a vu qu'on pouvait faire converger un filtre vers une de ses valeurs
d'adhérence en ajoutant des parties au filtre pour en faire un ultrafiltre.
Le procédé d'extraction de sous-suite est assez analogue, en ce qu'il permet
de choisir seulement quelques valeurs d'une suite pour se rapprocher d'une
valeur plus qu'une autre. Ainsi, on peut faire converger une suite vers une
valeur d'adhérence en en prenant une sous-suite bien choisie. C'est la
définition que nous prendrons pour la valeur d'adhérence d'une suite.

\begin{definition}[Valeur d'adhérence d'une suite]
  Soit $(X,d)$ un espace métrique et $(x_n)\in X^\mathbb N$ une suite à
  valeurs dans $\mathbb N$. On dit que $y$ est une valeur d'adhérence de
  $(x_n)$ s'il existe une sous-suite $(y_n)$ telle que $\lim y_n = y$. On
  note $\Adh(x_n)$ l'ensemble des valeurs d'adhérence de $x_n$.
\end{definition}

\begin{exercise}\label{exo.adh.fil.metr}
  Montrer que l'ensemble des valeurs d'adhérence de $(x_n)$ peut s'écrire
  comme~:
  \[\Adh(x_n) = \bigcap_{n \in \mathbb N} \adher{\{u_k \mid k > n\}}\]
\end{exercise}

\begin{remark}
  On voit une forme très proche de celle des valeurs d'adhérence d'un filtre.
  L'ensemble $\{u_k \mid k > n\}$ joue le rôle d'un ensemble du filtre. En
  fait, on peut associer à une suite le filtre des parties contenant un ouvert
  contenant une infinité de valeurs de la suite. Ce point relie directement
  les définitions avec les filtres et les définitions séquentielles.
\end{remark}

On veut maintenant caractériser la compacité dans un espace métrique grâce à des
suites. Notre analogie entre les suites et le filtres nous motive à donner une
définition séquentielle des compacts utilisant la notion de valeur d'adhérence,
c'est ce que nous faisons.

\begin{definition}[Compact séquentiel]
  Soit un espace métrique $(X,d)$. On dit qu'une partie $Y\subseteq X$ est une
  partie séquentiellement compacte si toute suite $(x_n)\in Y^\mathbb N$ admet
  une valeur d'adhérence.
\end{definition}

Un espace séquentiellement compact est alors, naturellement, un espace métrique
$(X,d)$ tel que $X$ en est une partie séquentiellement compacte.

Pour prouver cette généralisation, un sens demande plus de technique~: c'est
celui de montrer qu'un espace compact au sens séquentiel l'est au sens général.
Nous introduisons deux notions importantes pour prouver ce sens plus
naturellement~: le nombre de Lebesgue et la précompacité.

\begin{definition}[Nombre de Lebesgue]
  Soit $(X,d)$ un espace métrique et $(U_i)_{i\in I}$ un recouvrement de $X$. On
  dit que $(U_i)$ admet un nombre de Lebesgue si la propriété suivante est
  vérifiée~:
  \[\exists \rho > 0,\forall x \in X, \exists i \in I, B_\rho(x) \subseteq U_i \]
  Si $(U_i)$ admet un nombre de Lebesgue, on appelle nombre de Lebesgue de $(U_i)$ un
  $\rho$ tel que dans la formule précédente.
\end{definition}

\begin{lemma}\label{lem.nb.lebesgue}
  Si $(X,d)$ est séquentiellement compact, alors tout recouvrement
  $(U_i)_{i\in I}$ de $X$ admet un nombre de Lebesgue.
\end{lemma}

\begin{proof}
  On raisonne par l'absurde. Supposons donc que pour tout $\rho > 0$, on puisse
  trouver $x \in X$ tel que pour tout
  $i \in I, B_\rho(x) \setminus U_i \neq \varnothing$. On décide donc, pour
  chaque $\rho > 0$, de noter $x_\rho$ un élément de $B_\rho(x) \setminus U_i$.
  Cela forme une suite $(x_{1/n})_{n\in \mathbb N}$, dont on peut par compacité
  séquentielle extraire une suite convergente, dont on note $\varphi$
  l'extractrice, $(y_n)$ la suite extraite et $x$ la limite.

  Comme $(U_i)$ est un recouvrement, on trouve $i\in I$ tel que $x \in U_i$.
  Comme $U_i$ est ouvert et $x\in U_i$, on trouve $\varepsilon > 0$ tel que
  $B_\varepsilon(x) \subseteq U_i$. Cependant, comme $(y_n)$ converge vers $x$,
  on trouve $n\in \mathbb N$ tel que pour tout
  $m > n, y_m\in B_{\varepsilon/2}(x)$. On sait alors que pour tout $m > n$ et
  tout $y\in B_{\varepsilon/2}(y_m)$, $d(y,x) < \varepsilon$ par inégalité
  triangulaire. Cela signifie donc que
  $B_{\varepsilon/2}(y_m)\subseteq B_\varepsilon(x)\subseteq U_i$, mais cela
  est impossible pour $m$ suffisamment grand (puisqu'alors $B_{1/m}(y_m)$ serait
  inclus dans $U_i$), ce qui est absurde.

  Le recouvrement $(U_i)_{i\in I}$ possède donc un nombre de Lebesgue.
\end{proof}

L'idée du nombre de Lebesgue est de montrer qu'un recouvrement donné contient
des ouverts suffisamment gros~: il montre que tout point est contenu dans au
moins un des ouverts avec assez d'espace autour de lui. Cet élément est
important car on pourrait s'attendre à un échec de l'extraction d'un
recouvrement fini justement à cause d'ouverts devenant de plus en plus petits.
Ici, on a montré que s'il est possible d'avoir des ouverts devenant de plus en
plus petits, ces ouverts trop petits ne sont pas important à considérer~: tous
les points sont déjà recouverts par d'autres suffisamment gros.

La deuxième étape est donc de prouver que ces ouverts suffisamment grands
suffisent à avoir notre compacité. Cela se réduit à la propriété de
précompacité, dont nous donnons la définition.

\begin{definition}[Précompacité]
  Soit $(X,d)$ un espace métrique. On dit que $X$ est précompact si pour tout
  $\varepsilon > 0$, il existe un recouvrement fini de $X$ constitué
  uniquement de boules de rayon $\varepsilon$.
\end{definition}

\begin{lemma}\label{lem.precomp}
  Soit $(X,d)$ un espace métrique et séquentiellement compact, alors il est
  précompact.
\end{lemma}

\begin{proof}
  Par l'absurde, supposons qu'il existe $\varepsilon > 0$ tel qu'il n'existe
  aucun recouvrement fini de $X$ constitué de boules de rayon $\varepsilon$.
  On peut alors construire une suite $(x_n)$ de la façon suivante~:
  \begin{itemize}
  \item $x_0 \in X$
  \item si $(x_n)_{n < N}$ est construite, alors on considère
    $\displaystyle x_{n+1}\in X\setminus \bigcup_{n < N} B_\varepsilon(x_n)$, qui
    est non vide car sinon on aurait un recouvrement fini de $X$ par des boules
    ouvertes de rayon $\varepsilon$.
  \end{itemize}
  On voit donc que $(x_n)$ vérifie la propriété que pour tout
  $i < j, d(x_i,x_j) \geq \varepsilon$ (sinon on a une contradiction vis à vis
  de la définition de $x_j$). Comme $(X,d)$ est séquentiellement compact, on
  peut extraire une suite convergente de $(x_n)$~: notons-la $(y_n)$ et $x$ sa
  limite. On voit alors que par définition de la limite, on peut trouver
  $n_0\in\mathbb N$ tel que pour tout
  $n > n_0, d(y_n,x) < \dfrac{\varepsilon}{2}$, mais alors pour $p,q > n_0$ on
  trouve
  \begin{align*}
    d(y_p,y_q) &\leq d(y_p,x) + d(x,y_q)\\
    &< \frac\varepsilon 2 + \frac\varepsilon 2\\
    &< \varepsilon
  \end{align*}
  mais on sait que $d(y_p,y_q) \geq \varepsilon$~: c'est absurde.

  Ainsi, pour tout $\varepsilon > 0$, il doit exister un recouvrement de $(X,d)$
  par des boules ouvertes de rayon $\varepsilon$.
\end{proof}

On peut maintenant prouver l'équivalence entre la compacité séquentielle et la
compacité générale.

\begin{theorem}[Bolzano-Weierstrass généralisé]
  Soit $(X,d)$ un espace métrique. Alors une partie $K\subseteq X$ est compacte
  si et seulement si elle est séquentiellement compacte.
\end{theorem}

\begin{proof}
  Supposons que $K$ est compacte. Alors toute famille de fermés de $K$ ayant
  la propriété d'intersection finie a une intersection non vide. Soit
  $(x_n)\in K^{\mathbb N}$. On peut définir
  $\Adh(x_n)$ par une intersection de fermés par l'\cref{exo.adh.fil.metr}, et
  la famille ainsi définie vérifie la propriété d'intersection finie~: si
  $F\subfin \mathbb N$, alors on peut trouver $p > \max(F)$, et alors
  \[\bigcap_{n \in F}\adher{\{u_k\mid k > n\}} \supseteq
  \adher{\{u_k\mid k > p\}}\]
  On déduit de la compacité que $\Adh(x_n)$ est non vide.
  
  Supposons que $K$ est séquentiellement compacte. Soit $(U_i)_{i\in I}$ un
  recouvrement ouvert de $K$. Par le \cref{lem.nb.lebesgue}, on trouve $\rho$ un
  nombre de Lebesgue de $(U_i)$. De plus, par le \cref{lem.precomp}, on sait
  qu'on peut trouver un recouvrement fini de $K$ par des boules ouvertes de
  rayon $\rho$, notons $Y \subfin X$ l'ensemble des points tel que
  $\{B_\rho(x)\mid x \in Y\}$ soit le recouvrement fini obtenu par notre lemme.
  Par définition du nombre de Lebesgue, on sait que pour chaque $x \in Y$, il
  existe $i_x \in I$ tel que $B_\rho(x) \subseteq U_{i_x}$~: notre recouvrement
  fini est alors $\{U_{i_x}\mid x \in Y\}$, puisque
  \[K\subseteq \bigcup_{x \in Y} B_\rho(x) \subseteq \bigcup_{x \in Y} U_{i_x}\]
  On en déduit que $K$ est compacte.
\end{proof}

On introduit maintenant une nouvelle notion, qui a du sens en topologie générale
mais que l'on préfère introduire dans le cas métrique~: la densité.

\begin{definition}[Densité]
  Soit $(X,\Omega_X)$ un espace topologique. On dit qu'une partie
  $Y\subseteq X$ est dense si $\adher Y = X$.
\end{definition}

La caractérisation séquentielle est directe, mais on l'écrit tout de même.

\begin{property}
  Soit $(X,d)$ un espace métrique. Une partie $Y\subseteq X$ est dense si et
  seulement si pour tout $x\in X$ il existe une suite $(y_n)\in Y^\mathbb N$
  telle que $\lim y_n = x$.
\end{property}

\begin{example}
  L'exemple le plus évident de densité est le fait que $\mathbb Q$ est dense
  dans $\mathbb R$ (en munissant $\mathbb R$ de la topologie usuelle, dont on
  peut montrer qu'elle dérive de la métrique $(x,y)\mapsto |x-y|$).
\end{example}

Nous donnons en exercice une caractérisation de la continuité, analogue à la
\cref{prop.carac.cont.fil}, mais utilisant la limite de suites.

\begin{exercise}\label{exo.carac.cont.suit}
  Soient $(X,d)$ et $(Y,d)$ deux espaces métriques. Montrer que toute fonction
  $f : X \to Y$ est continue si et seulement si, en notant $\ell(X)$ l'ensemble
  des suites convergentes à valeurs dans $X$, on a
  \[\forall (x_n) \in \ell(X),
  \lim_{n \to \infty} f(x_n) = f(\lim_{n\to\infty} x_n)\]
\end{exercise}

On peut donc définir la notion de séparabilité, qui exprime qu'un espace est,
d'une certaine manière, l'adhérence d'un espace dénombrable.

\begin{definition}[Espace séparable]
  Soit $(X,\Omega_X)$ un espace topologique. On dit que cet espace est
  séparable s'il existe une partie $Y\subseteq X$ dense et dénombrable.
\end{definition}

Le fait d'être dénombrable se prête particulièrement bien au cas métrique,
d'où le fait de l'introduire ici.

\subsection{Complétude}

Nous finissons ce chapitre sur l'introduction de la notion de complétude. Cette
notion est purement métrique, et non topologique. On peut la motiver par une
observation à propos de $\mathbb R$ et $\mathbb Q$~: dans le premier, une
suite qui \og a l'air\fg de converger, converge bien~; dans le deuxième,
une suite qui \og a l'air\fg de converger, peut n'avoir aucune limite car sa
limite est un irrationnel.

A partir de cette observation, on voit que $\mathbb R$ possède une propriété que
$\mathbb Q$ ne vérifie pas. Cependant, cette observation ne suffit pas~: il faut
encore décrire concrètement ce qu'est une suite ayant l'air de converger. Cette
notion est celle des suites de Cauchy, dont l'écart entre les différentes
valeurs tend vers $0$.

\begin{definition}[Suite de Cauchy]
  Soit $(X,d)$ un espace métrique. On dit qu'une suite $(x_n)\in X^\mathbb N$
  est de Cauchy si elle vérifie la propriété suivante~:
  \[\forall \varepsilon > 0, \exists n \in \mathbb N, \forall p,q > n,
  d(x_p,x_q) < \varepsilon\]
\end{definition}

\begin{remark}
  La notion de suite de Cauchy dépend de la métrique et pas seulement de la
  topologie. On peut vérifier que la topologie induite par $\mathbb R$ sur
  $(-1,1)$ et la topologie obtenue par un homéomorphisme tel que
  $\mathrm{argth}$ sont les mêmes, mais les distances induites sont très
  différentes. En particulier, une suite qui tend vers $-1$ est de Cauchy pour
  la première métrique mais pas pour la deuxième.
\end{remark}

Pour se convaincre qu'une telle suite a l'air de converger, donnons une
propriété importante des suites de Cauchy~: une valeur d'adhérence est une
limite pour ces suites.

\begin{property}\label{prop.cauchy.adh}
  Soit $(X,d)$ un espace métrique et $(x_n)\in X^\mathbb N$ une suite de
  Cauchy. On a l'équivalence suivante, pour tout $x \in X$~:
  \[\lim_{n\to\infty} x_n = x \iff x \in \Adh(x_n)\]
\end{property}

\begin{proof}
  Si $x$ est la limite de $(x_n)$ il est clair que $x$ en est une valeur
  d'adhérence~: il suffit de considérer $\id$ comme extractrice.

  Réciproquement, supposons que $x$ est une valeur d'adhérence de $(x_n)$. On
  trouve donc une fonction $\varphi : \mathbb N \to \mathbb N$ strictement
  croissante telle que $\lim x_{\varphi(n)} = x$. Montrons que $(x_n)$ converge
  vers $x$~: soit $\varepsilon > 0$, par définition de la limite, on peut
  trouver $n_0$ tel que
  \[\forall n \geq n_0, d(x_{\varphi(n)},x) < \frac{\varepsilon}{2}\]
  De plus, comme $(x_n)$ est de Cauchy, on trouve $n_1$ tel que
  \[\forall p,q \geq n_1, d(x_p,x_q) < \frac{\varepsilon}{2}\]
  On sait donc que pour $m \geq \max(n_0,n_1)$, on a
  \begin{align*}
    d(x_m,x) &\leq d(x_m,x_{\varphi(m)}) + d(x_{\varphi(m)},x) \\
    &< \frac{\varepsilon}{2} + \frac{\varepsilon}{2}\\
    &< \varepsilon
  \end{align*}
  Où l'inégalité $d(x_m,x_{\varphi(m)}) < \dfrac{\varepsilon}{2}$ vient du
  fait que $\varphi(m) \geq m \geq n_1$. On en déduit que $(x_n)$ converge vers
  $x$.
\end{proof}

On peut alors introduire la notion d'espace complet.

\begin{definition}[Espace complet]
  Soit $(X,d)$ un espace métrique. On dit que c'est un espace complet si toute
  suite de Cauchy converge dans $X$.
\end{definition}

On peut aussi définir la notion d'être métrisable (respectivement complètement
métrisable), qui n'impose pas la métrique mais seulement son existence.

\begin{definition}[Espace complètement métrisable]
  Soit $(X,\Omega_X)$ un espace topologique. On dit qu'il est un espace
  (complètement) métrisable s'il existe une distance $d$ engendrant la topologie
  $\Omega_X$ (et telle que l'espace métrique est complet).
\end{definition}

Il existe une construction permettant de plonger un espace métrique dans un
espace métrique complet, on l'appelle la complétion.

L'intérêt de cette construction est de donner un espace semblant le plus proche
possible de notre espace initial. Pour un espace $X$, son complété $\hat X$ est
un espace contenant $X$ comme partie dense, donc tous les points ajoutés pour
construire $\hat X$ ne sont que des limites d'éléments de $X$~: on n'ajoute en
quelque sorte que les limites manquantes aux suites de Cauchy non convergentes.
De plus, on peut montrer que la distance sur $\hat X$ est induite par la
distance sur $X$ de telle sorte que deux éléments de $X$ sont à la même
distance, qu'ils soient considérés comme des éléments de $X$ ou comme des
éléments de $\hat X$. Pour formaliser cette idée, nous donnons d'abord la
définition d'isométrie.

\begin{definition}[Isométrie]
  Soient $(X,d)$ et $(Y,d')$ deux espaces métriques. On dit qu'une fonction
  $X \to Y$ est une isométrie si elle vérifie la propriété suivante~:
  \[\forall x,y \in X, d(x,y) = d'(f(x),f(y))\]
\end{definition}

\begin{property}
  Une isométrie entre $(X,d)$ et $(Y,d')$ est une fonction continue.
\end{property}

\begin{proof}
  En utilisant l'\cref{exo.carac.cont.suit},
  il nous suffit de montrer que pour une suite convergente $(x_n)$, dont on
  notera $x$ sa limite, on a $\displaystyle\lim_{n \to \infty} f(x_n) = f(x)$.

  Soit $\varepsilon > 0$, par définition de $\lim x_n = x$ on trouve $n_0$ tel
  que
  \[\forall n > n_0, d(x_n,x) < \varepsilon\]
  alors, pour tout $n > n_0$, on a
  \begin{align*}
    d(f(x_n),f(x)) &= d(x_n,x)\\
    &< \varepsilon
  \end{align*}
  d'où le fait que $\displaystyle\lim_{n\to\infty} f(x_n) = f(x)$.
\end{proof}

\begin{proposition}
  Soit $(X,d)$ un espace métrique. Il existe un espace métrique complet
  $(\hat X,\hat d)$ pour lequel $(X,d)$ se plonge isométriquement dans une
  partie dense.
\end{proposition}

\begin{proof}
  On considère l'espace
  \[Y \defeq \{(x_n)\in X^{\mathbb N}\mid (x_n)\text{ est de Cauchy}\}\]
  et la distance
  \[d'((x_n),(y_n)) \defeq \lim_{n \to \infty} d(x_n,y_n)\]

  Cet espace n'est pas métrique car il ne vérifie pas la séparabilité~: on
  quotiente donc cet espace par la relation
  \[(x_n)\sim (y_n) \defeq \lim_{n\to\infty} d(x_n,y_n) = 0\]

  On vérifie qu'on a alors une distance et que l'espace engendré est complet~:
  \begin{itemize}
  \item la fonction $d'$ est bien définie~: si $(x_n)$ et $(y_n)$ sont de
    Cauchy, alors pour tout $\varepsilon > 0$ on trouve $n_0$ tel que
    \[\forall p,q > n_0, d(x_p,x_q) \leq \dfrac{\varepsilon}{2}
    \land d(y_p,y_q) \leq \dfrac{\varepsilon}{2}\]
    Alors, pour tous $p,q > n_0$, on voit que
    \begin{align*}
      |d(x_p,y_p) - d(x_q,y_q)| &\leq |d(x_p,x_q) + d(x_q,y_p) - d(x_q,y_q)|\\
      &\leq |d(x_p,x_q) + d(x_q,y_q) + d(y_p,y_q) - d(x_q,y_q)|\\
      &\leq \left|\dfrac{\varepsilon}{2} + \dfrac{\varepsilon}{2}
      + d(x_q,y_q) - d(x_q,y_q)\right|\\
      &\leq \varepsilon
    \end{align*}
    La suite $d(x_n,y_n)$ est donc de Cauchy, ce qui signifie qu'elle converge
    (puisque la suite est à valeurs dans $\mathbb R_+$).
  \item la fonction $\hat d$ induite par $d'$ est bien définie. En effet, si
    $(x_n)\sim (y_n)$ et $(z_n)\sim (a_n)$, cela signifie que
    $\displaystyle\lim_{n\to\infty} d(x_n,y_n) = 0$ et
    $\displaystyle\lim_{n\to\infty}d(z_n,a_n) = 0$, d'où
    \begin{align*}
      d'((x_n),(z_n)) &= \displaystyle\lim_{n \to \infty} d(x_n,z_n)\\
      &= \lim_{n\to \infty} d(x_n,z_n) + \lim_{n\to\infty} d(x_n,y_n) +
      \lim_{n\to \infty} d(z_n,a_n)\\
      &= \lim_{n\to\infty} d(y_n,x_n) + d(x_n,z_n) + d(z_n,a_n)\\
      &\geq \lim_{n\to \infty} d(y_n,a_n)
    \end{align*}
    De la même façon, on prouve l'autre inégalité, donc
    $d'((x_n),(z_n)) = d'((y_n),(a_n))$, donc $d'$ est constante sur les
    classes d'équivalences.
  \item pour la séparation, soient $\overline{(x_n)},\overline{(y_n)}$ deux
    éléments de l'espace quotient. Si l'on suppose $\hat d((x_n),(y_n)) = 0$,
    alors par définition $\displaystyle\lim_{n\to\infty} d(x_n,y_n) = 0$, donc
    $(x_n)\sim (y_n)$, ce qui
    signifie que les deux classes sont égales.
  \item pour la symétrie, cela découle directement de la symétrie de $d$ puisque
    les définitions utilisent $d$.
  \item pour l'inégalité triangulaire, soient
    $\overline{(x_n)},\overline{(y_n)},\overline{(z_n)}$ trois éléments de
    l'espace quotient. On veut prouver l'inégalité
    \[\lim_{n\to\infty} d(x_n,z_n) \leq \lim_{n\to\infty} d(x_n,y_n) +
    \lim_{n\to\infty} d(y_n,z_n)\]
    mais cette inégalité passe directement à la limite puisque, pour tout $n$,
    le fait que $d$ est une distance implique
    \[d(x_n,z_n) \leq d(x_n,y_n) + d(y_n,z_n)\]
  \item si $x_n$ est une suite de Cauchy dans $\hat X$, alors par définition on
    peut réécrire cette suite
    $(\overline{(a_{i,j})_{i \in \mathbb N}})_{j \in \mathbb N}$. Intuitivement, une
    limite à cette suite est la suite des limites sur $j$ des $a_{i,j}$, mais
    cette limite n'existe pas. Cependant, on peut considérer un procédé
    diagonal, en prenant la suite $a_{i,i}$~: à mesure que $i$ grandit, on
    s'approche de la limite de $a_i$, et $a_i$. Il reste malheureusement un
    dernier souci, qui est que la suite $a_{i,i}$ n'est pas une suite de
    Cauchy à première vue. Pour contourner ce problème, on commence par
    définir une fonction $i \mapsto m_i$ de la façon suivante~: pour tout
    $i \in \mathbb N$, on sait que $x_i$ est de Cauchy, donc on peut trouver
    $m_i$ tel que
    \[\forall p,q \geq m_i, d(a_{p,i},a_{q,i}) < \frac{1}{i + 1}\]
    On définit la limite de $x_n$ comme
    \[x \defeq \overline{(a_{m_i,i})_{i\in \mathbb N}}\]

    Montrons que $(a_{m_i,i})$ est bien de Cauchy. Soit $\varepsilon > 0$, on
    trouve $n_0$ tel que $\dfrac{1}{n_0 + 1} < \dfrac{\varepsilon}{4}$ et $n_1$
    tel que
    \[\forall p,q \geq n_1, \hat d(x_p,x_q) < \frac{\varepsilon}{4}\]
    On peut réécrire $\hat d(x_p,x_q)$ en
    $\displaystyle\lim_{n \to \infty} d(a_{n,p}, a_{n,q})$ donc, par définition de
    la limite, on peut trouver $n_2(p,q)$ tel que pour tout $m > n_2(p,q)$, on a
    \[d(a_{m,p},a_{m,q}) < \hat d(x_p,x_q) + \frac{\varepsilon}{4}\]
    soit encore $d(a_{m,p},a_{m,q}) < \dfrac{\varepsilon}{2}$ par inégalité
    trianglulaire, si $p,q > n_1$. Alors, on pose $n_3 = \max(n_0,n_1)$, et on
    sait que pour $p,q > n_3$ et $m > \max(n_2(p,q), m_p,m_q)$, on a
    \begin{align*}
      d(a_{m_p,p},a_{m_q,q}) &\leq d(a_{m_p,p},a_{m,p}) + d(a_{m,p}, a_{m,q}) +
      d(a_{m,q},a_{m_q,q}) \\
      &\leq \frac{1}{p + 1} + \frac{\varepsilon}{2} + \frac{1}{q + 1}\\
      &\leq \frac{\varepsilon}{4} + \frac{\varepsilon}{2} +
      \frac{\varepsilon}{4}\\
      &\leq \varepsilon
    \end{align*}
    Ainsi, on a prouvé qu'il existait $n_3$ tel que
    \[\forall p,q > n_3, d(a_{m_p,p}, a_{m_q,q}) < \varepsilon\]
    donc notre suite est de Cauchy.

    Montrons que la suite $(x_n)$ converge vers $x$. Par la
    \cref{prop.cauchy.adh}, il nous suffit de montrer que $x$ est une valeur
    d'adhérence de $(x_n)$, ce que l'on prouve donc. Soit $\varepsilon > 0$.
    Avec les arguments précédents, on sait qu'on peut trouver $n_0$ tel que
    \[\forall p,q \geq n_0, d(a_{m_p,p},a_{m_q,q}) < \frac{\varepsilon}{2}\]
    On trouve aussi $n_1$ tel que
    $\dfrac{1}{n_1 + 1} < \dfrac{\varepsilon}{4}$. Soit $n_2 = \max(n_0,n_1)$.
    Par définition de la limite, et en réécrivant $\hat d(x,x_{n_2})$ en
    $\displaystyle\lim_{n\to\infty} d(a_{m_n,n},a_{n,n_2})$, on trouve $n_3$ tel que
    pour tout $i \geq n_3$, on a
    \[\hat d(x,x_{n_2}) < d(a_{m_i,i}, a_{i,n_2}) + \frac{\varepsilon}{4}\]
    Soit $i \geq \max(n_2,n_3)$, on montre alors l'inégalité voulue~:
    \begin{align*}
      \hat d(x,x_{n_2}) &< d(a_{m_i,i}, a_{i,n_2}) + \frac{\varepsilon}{4}\\
      &< d(a_{m_i,i}, a_{m_{n_2},n_2}) + d(a_{m_{n_2},n_2}, a_{i,n_2})
      + \frac{\varepsilon}{4}\\
      &< \frac{\varepsilon}{2} + \frac{1}{n_2 + 1} + \frac{\varepsilon}{4}\\
      &< \varepsilon
    \end{align*}
    Donc pour tout $\varepsilon > 0$, il existe $n_2$ tel que
    $\hat d(x,x_{n_2}) < \varepsilon$, d'où
    $\displaystyle\lim_{n \to \infty} x_n = x$.
  \end{itemize}
  L'espace $(\hat X,\hat d)$ est donc bien un espace métrique complet.
  
  De plus, on veut montrer que $(X,d)$ s'injecte isométriquement dans
  $(\hat X,\hat d)$. Pour cela, on considère l'application
  \[\begin{array}{ccccc}
  \iota & : & X & \longrightarrow & \hat X\\
  & & x & \longmapsto & \overline{(n \mapsto x)_{n\in \mathbb N}}
  \end{array}\]
  Dans ce cas, on voit que $\hat d(\iota(x),\iota(y)) = d(x,y)$, donc $\iota$
  est bien isométrique (et donc injective).

  Enfin, montrons que l'image de $\iota$ est dense dans $\hat X$. Soit un
  élément $\overline{(x_n)}\in\hat X$ et $\varepsilon > 0$, montrons qu'il
  existe $x\in X$ tel que $\hat d(\overline{(x_n)},\iota(x)) \leq\varepsilon$,
  ce qui se traduit par
  \[\lim_{n\to\infty} d(x_n,x) \leq \varepsilon\]
  mais on sait, par définition, que $(x_n)$ est une suite de Cauchy~: on peut
  donc trouver $n_0$ tel que
  \[\forall p,q \geq n_0, d(x_p,x_q) \leq\varepsilon\]
  donc, en prenant $x = x_{n_0}$, la limite précédente est majorée à partir
  d'un certain rang par $\varepsilon$, d'où l'inégalité. Ainsi $X$ est dense
  dans $\hat X$.
\end{proof}

Pour clore ce chapitre, on peut maintenant définir ce qu'est un espace polonais,
et comment celui-ci se comporte vis à vis de la complétion.

\begin{definition}[Espace polonais]
  Un espace polonais $(X,\Omega_X)$ est un espace topologique complètement
  métrisable séparable.
\end{definition}

\begin{proposition}
  Soit $(X,d)$ un espace métrique séparable. Alors la complétion
  $(\hat X,\hat d)$ de cet espace, comme construite précédemment, est un espace
  polonais.
\end{proposition}

\begin{proof}
  On sait qu'il existe une partie dense dénombrable $Y\subseteq X$, et que
  $X$ est dense dans $\hat X$, donc $Y$ est dense (et dénombrable) dans
  $\hat X$. Comme $\hat X$ est par construction un espace complet, on en déduit
  qu'il est un espace séparable complet~: c'est un espace polonais.
\end{proof}
