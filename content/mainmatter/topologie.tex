\chapter[Topologie]{Introduction à la topologie}
\label{chp.topo}

\minitoc

Un des principes les plus fréquents, en mathématiques, est de se ramener à de la
topologie. En algèbre, en analyse ou en géométrie, on peut décrire nos problèmes
par de la topologie. En logique, des liens existent aussi avec la topologie, et
nous avons donc besoin de bases en topologie pour les explorer.

Cependant, cet ouvrage n'est pas un cours de topologie : nous n'allons donc pas
entrer dans les détails de ce domaine. Nous conseillons donc au lecteur étranger
à la topologie de chercher un cours plus complet. Nous allons par exemple
considérer en premier lieu la topologie générale et les constructions à partir de
filtres, pour ensuite traiter le cas des espaces métriques.

Nous aurons trois objectifs principaux.
Le premier est de définir un fragment suffisamment large de la topologie générale
pour prouver le théorème de Tychonov grâce aux ultrafiltres.
Le deuxième, qui peut être vu comme la suite du premier, est de montrer le
théorème de dualité de Stone (dans une version n'utilisant pas le vocabulaire
catégorique), qui permet de donner une lecture topologique du \cref{chp.ordres}.
Le troisième nous fera quitter le monde de la topologie générale, pour préparer
le terrain pour la théorie descriptive des ensembles, en définissant les notions
élémentaires de topologie métrique.

\section{Topologie générale et filtres}

Pour commencer, donnons une présentation générique de la topologie : nous
étudions des espaces topologiques, que l'ont peut décrire comme des ensembles
(dont les éléments sont en général appelés \og points\fg{}) muni d'une notion de
proximité. La notion la plus générale de proximité peut être donnée au choix
par les ouverts, les fermés ou les voisinages, mais la convention est de définir
une topologie par ses ouvertes. Donnons donc ces définitions et les objets les
plus élémentaires qui y sont liés.

\subsection{Définitions de topologie générale}

La définition de topologie par les ouverts peut sembler \textit{ad hoc}, nous
allons donc essayer de la motiver. L'objectif d'une topologie est de définir
quels éléments sont proches et quels éléments sont éloignés les uns des autres.
Si la notion naturelle pour décrire cela est celle de distance, on peut se
demander si celle-ci est importante pour une étude topologique. En effet, le
point important de la topologie est de pouvoir décrire ce qui est plus ou moins
local, c'est-à-dire ce qui arrive \textbf{suffisamment proche}, et non proche de
façon absolue.

L'idée d'un ouvert dans une topologie est alors de donner une mesure fixée de
proximité~ : deux points sont proches s'ils appartiennent à un ouvert, et ils
sont d'autant plus proches qu'ils le restent pour des ouverts pris de plus en
plus petits. Ainsi un phénomène local est un phénomène qui survient dans un
certain ouvert.

On voit déjà pourquoi la notion de filtre sera intéressante : on veut qu'une
description soit de plus en plus précise à mesure qu'on affine nos ouverts,
ce se rapproche du fait qu'on puisse affiner des filtres.

Donnons donc la définition formelle d'une topologie grâce à ses ouverts.

\begin{definition}[Espace topologique]
  On appelle espace topologique une paire $(X,\Omega_X)$ où
  $\Omega_X\subseteq\powerset(X)$ est une partie vérifiant :
  \begin{itemize}
  \item $\Omega_X$ est stable par intersection finie : si $F\subfin \Omega_X$
    alors $\bigcap F \in \Omega_X$.
  \item $\Omega_X$ est stable par union quelconque : si $Y\subseteq \Omega_X$
    alors $\bigcup Y \in \Omega_X$.
  \end{itemize}

  En particulier, $X\in \Omega_X$ et $\varnothing\in \Omega_X$. On appelle ouvert
  de $X$ un élément $O\in\Omega_X$.
\end{definition}

Si un ouvert représente une mesure de localité, le complémentaire d'un ouvert a
un autre sens : il est une partie stable, mais nous en aurons une compréhension
plus précise plus tard. Ces parties, complémentaires, d'ouverts, sont appelées
les fermés.

\begin{definition}[Fermé]
  Soit $(X,\Omega_X)$ un espace topologique. On appelle fermé de $X$ une partie
  $F\subseteq X$ telle que $X\setminus F \in \Omega_X$.
\end{definition}

\begin{remark}
  La structure ordonnée $(\Omega_X,\subseteq)$ est en fait une algèbre de
  Heyting complète. Remarquons que la borne inférieure d'une partie
  $Y\subseteq\Omega_X$ n'est pas forcément l'intersection : elle est le plus
  grand ouvert contenu dans $\bigcap Y$.

  On peut établir une version plus générale de la topologie en étudiant les
  algèbres de Heyting complètes, c'est ce que l'on appelle la topologie sans
  points (\textit{pointfree topology} ou \textit{pointless topology}).
\end{remark}

On voit que décrire une topologie par ses ouverts ou par ses fermés est
équivalent, puisqu'on peut retrouver l'un à partir de l'autre par l'opération
$\mathcal T \mapsto \{X\setminus T \mid T \in \mathcal T\}$. Voyons enfin l'autre
façon possible de décrire une topologie : par les voisinages.

\begin{definition}[Voisinage]
  Soit $(X,\Omega_X)$ un espace topologique et $x\in X$. On appelle ensemble des
  voisinages de $x$ l'ensemble des parties de $X$ contenant un ouvert contenant
  $x$~ :
  \[\vois_x \defeq \{V \subseteq X \mid \exists U \in \Omega_X, x\in U \land
  U\subseteq V\}\]
\end{definition}

Un voisinage est en particulier un filtre.

\begin{property}
  Soit $(X,\Omega_X)$ un espace topologique, alors pour tout $x\in X$, l'ensemble
  $\vois_x$ est un filtre propre sur $(\powerset(X),\subseteq)$.
\end{property}

\begin{proof}
  Il nous suffit de vérifier les hypothèses d'un filtre propre :
  \begin{itemize}
  \item puisque $X\in\Omega_X$ et $x\in X$, $X\in \vois_x$.
  \item si $V\in \vois_x$ et $W$ est tel que $V\subseteq W$, alors on trouve
    $U\in \Omega_X$ tel que $x\in U \subseteq V$, et par transitivité de
    l'inclusion, $x\in U \subseteq W$, donc $W\in \vois_x$.
  \item si $V\in \vois_x$ et $W\in \vois_x$ alors on trouve $U$ et $U'$ des
    ouverts tels que $x\in U \subseteq V$ et $x\in U' \subseteq W$. Comme
    $\Omega_X$ est stable par intersection finie, $U\cap U' \subseteq W$ donc
    $x\in U\cap U' \subseteq V\cap W$, donc $V\cap W \in \vois_x$.
  \item l'ensemble $\varnothing$ ne contient pas $x$, donc
    $\varnothing\notin\vois_x$.
  \end{itemize}

  Ainsi $\vois_x$ est un filtre propre de $\powerset(X)$.
\end{proof}

On peut donc définir la notion de voisinage à partir des ouverts. Réciproquement,
il est possible de retrouver l'ensemble des ouverts à partir des voisinages.

\begin{proposition}
  Soit $(X,\Omega_X)$ un espace topologique. Alors $U\in \Omega_X$ si et
  seulement si pour tout $x\in U$, $U\in\vois_x$.
\end{proposition}

\begin{proof}
  Si $U\in \Omega_X$, il est évident que pour tout $x\in U$ il existe un ouvert
  ($U$ lui-même) inclus dans $U$ et contenant $x$. Réciproquement, supposons que
  $U$ est un voisinage de tous ses points. Alors pour tout $x\in U$, on trouve
  un ouvert $U_x\subseteq U$ contenant $x$. Comme les ouverts sont stables par
  union quelconque, $\displaystyle\bigcup_{x\in U} U_x$ est encore un ouvert.
  Mais comme $U_x\subseteq U$, on en déduit que $\bigcup U_x \subseteq U$, et
  comme pour tout $x\in U, x\in U_x$, on en déduit que $U\subseteq\bigcup U_x$.
  Il en vient que $U$ est un ouvert en tant qu'union d'ouverts.
\end{proof}

Pour définir plus facilement une topologie, il nous faut parler de la notion de
base d'une topologie. Par défaut, une famille $\mathcal T\subseteq\powerset(X)$
n'a pas de raison d'être une topologie, puisqu'il faut des propriétés de clôture
à notre famille. Cependant, la clôture par union quelconque peut s'obtenir en
considérant directement les unions d'éléments de la famille. Il nous reste à
traiter le cas des intersections finies, mais on préfère exiger directement des
famille de parties d'être stables par intersection finie.

\begin{definition}[Base d'une topologie, topologie engendrée]
  Soit un ensemble $X$. On dit qu'une partie $\mathcal B\subseteq\powerset(X)$
  est une base de topologie si elle est close par intersection finie,
  c'est-à-dire si pour toute partie $F\subfin\mathcal B$,
  $\bigcap F \in\mathcal B$.

  Si $\mathcal B\subseteq\powerset(X)$ est une base de topologie, alors on
  appelle topologie engendrée par $\mathcal B$, que l'on note
  $\Omega_{\mathcal B}$, l'ensemble
  \[\Omega_{\mathcal B} \defeq \Bigg\{\bigcup_{B \in F} B\;\Bigg|\;
  F\subseteq \mathcal B\Bigg\}\]
\end{definition}

\begin{exercise}
  Montrer que $\Omega_{\mathcal B}$ est bien une topologie.
\end{exercise}

\begin{example}
  Donnons des exemples de topologies :
  \begin{itemize}
  \item on dispose sur tout ensemble $X$ de la topologie discrète donnée par
    $\powerset(X)$ tout entier.
  \item on dispose sur tout ensemble $X$ de la topologie grossière donnée
    par $\{\varnothing,X\}$.
  \item sur $\mathbb R$, on dispose de la topologie engendrée par les intervalles
    ouverts, c'est-à-dire de la forme $\{x \in \mathbb R \mid a < x < b\}$ où
    $a,b\in\mathbb R$. C'est la topologie naturelle que l'on utilise sur
    $\mathbb R$.
  \item plus généralement, sur un ensemble ordonné $(X,\leq)$, on peut définir
    la topologie de l'ordre comme la topologie engendrée par les intervalles
    ouverts, de la forme $\{x\in X\mid a < x < b\}$ pour $a,b\in X$.
    Remarquons que par exemple sur $\mathbb N$, la topologie de l'ordre coïncide
    avec la topologie discrète.
  \end{itemize}
\end{example}

Remarquons que tout fermé de $\mathbb R$ peut s'écrire comme intersection
d'intervalles fermés, de la forme $\{x\in \mathbb R \mid a\leq  x \leq b\}$.

Nous verrons plus tard une famille d'espaces topologiques, les espace métriques.

\'Etant donnée une partie $Y\subseteq X$, qui n'est ni un ouvert ni un fermé,
disons par exemple $\{x\in\mathbb R\mid 1 \leq x < 2\}$, peut-on lui associer
un ouvert (respectivement un fermé) naturellement ? La réponse est oui, en
considérant l'intérieur (respectivement l'adhérence).

\begin{definition}[Intérieur]
  Soit $(X,\Omega_X)$ un espace topologique et $Y\subseteq X$. On appelle
  intérieur de $Y$, que l'on note $\inter Y$, le plus grand ouvert contenu dans
  $Y$.
\end{definition}

\begin{definition}[Adhérence]
  Soit $(X,\Omega_X)$ un espace topologique et $Y\subseteq X$. On appelle
  adhérence de $X$, que l'on note $\adher Y$, le plus petit fermé contenant
  $Y$.
\end{definition}

\begin{exercise}
  On considère $\mathbb R$ muni de la topologie décrite plus haut. Montrer que
  $\inter{\mathbb Q} = \varnothing$ et que $\adher{\mathbb Q} = \mathbb R$.
\end{exercise}

On peut aussi définir la notion de point intérieur et de point adhérent,
correspondant naturellement au fait d'appartenir à l'intérieur (respectivement à
l'adhérence), mais sous une forme plus intuitive.

\begin{definition}[Point intérieur]
  Soit $(X,\Omega_X)$ un espace topologique et $Y\subseteq X$. On dit qu'un
  point $x$ est intérieur à $Y$ si $Y\in\vois_x$.
\end{definition}

\begin{definition}[Point adhérent]
  Soit $(X,\Omega_X)$ un espace topologique et $Y\subseteq X$. On dit qu'un
  point $x$ est adhérent à $Y$ si pour tout ouvert $U\in\Omega_X$, si
  $Y\subseteq U$ alors $x\in U$.
\end{definition}

\begin{exercise}
  Montrer qu'un point est intérieur à une partie si et seulement s'il est dans
  son intérieur. Montrer qu'il est adhérent si et seulement s'il est dans son
  adhérence.
\end{exercise}

Enfin, définissons la notion de continuité. On la présente pour l'instant comme
une simple notion de morphisme associée à notre structure d'espace topologoque.
Nous en verrons une interprétation grâce aux filtres.

\begin{definition}[Fonction continue]
  Soient $(X,\Omega_X)$ et $(Y,\Omega_Y)$ deux espaces topologiques. Une fonction
  $f : X \to Y$ est dite continue sur pour tout $U\in\Omega_Y$, l'ensemble
  $f^{-1}(U)$ est un ouvert de $X$.
\end{definition}

\begin{remark}
  Si l'on considère un espace topologique comme l'algèbre de Heyting complète de
  son ensemble d'ouverts, alors la définition que nous venons de donner signifie
  que $f$ induit grâce à $f^{-1}$ un morphisme d'algèbres de Heyting complètes.
  Cependant, il est important de voir que là où $f : X \to Y$,
  $f^{-1} : \powerset(Y)\to\powerset(X)$, on a donc une inversion du sens entre
  le point de vue algébrique par un ensemble ordonné et le point de vue
  analytique comme un ensemble de points muni d'une topologie.
\end{remark}

\subsection{Filtres, limites, séparation}

On aborde en général la notion de limite de façon métrique. La définition la plus
souvent vue d'une limite est la suivante, pour une fonction $f$ admettant une
limite en $a$ valant $b$ :
\[\forall \varepsilon > 0, \exists \delta > 0, \forall x \in \mathbb R,
|x - a| < \delta \implies |f(x) - b| < \varepsilon\]
Si nous étudions cette formule en essayant de traduire ce qu'elle énonce au
niveau topologique, on peut commencer par remplacer les inégalités en valeur
absolue par l'appartenance à certains ouverts : au lieu de
$|f(x) - b| < \varepsilon$, on peut dire que $f(x)$ appartient à un
ouvert, qui est $(b-\varepsilon,b+\varepsilon)$. De même, plutôt que de dire
que $|x - a| < \delta$, on peut dire que $x$ appartient à un certain ouvert.

Puisqu'on quantifie sur n'importe quel $\varepsilon$, on peut de la même façon
considérer qu'on quantifie sur tous les ouverts contenant $b$, sans perte de
généralité sur les voisinages de $b$ : on voit alors apparaître un filtre.
En effet, plutôt que de dire que $f(x)$ est dans tel ouvert autour de $b$,
on peut dire que $x$ est dans la préimage de l'ouvert par $f$. On trouve donc,
plutôt qu'une suite de quantificateurs quelconque, une considération de finnesse.
En fait, on peut reformuler cette formule par la formule équivalente suivante~ :
\[(\vois_a)_f \supseteq \vois_b\]
c'est-à-dire que le filtre image de $\vois_a$ par $f$ est plus fin que le filtre
$\vois_b$.

Ainsi, la convergence est une notion particulièrement adaptée à une définition
par les filtres. On va donc pouvoir tout d'abord définir la convergence d'un
filtre.

\begin{definition}[Convergence d'un filtre]
  Soit un espace topologique $(X,\Omega_X)$ et un filtre $\mathcal F$, on
  définit le fait que $\mathcal F$ converge vers $a\in X$ par~ :
  \[\lim \mathcal F = a \defeq \mathcal F \supseteq \vois_a\]
\end{definition}

Le sens de cette définition est qu'un filtre converge vers $a$ lorsque les
parties qu'il décrit permettent de se rapprocher autant que l'on veut de
$a$.

On peut maintenant facilement définir ce que signifie la convergence pour une
fonction en un point.

\begin{definition}[Convergence d'une fonction]
  Soient $(X,\Omega_X),(Y,\Omega_Y)$ deux espaces topologiques et $f : X \to Y$.
  On dit que $f$ converge vers $b\in Y$ en $a\in X$ (au sens des filtres) si
  \[\lim_{\mathcal F \to a}f(\mathcal F) = b\defeq (\vois_a)_f \supseteq\vois_b\]
\end{definition}

Remarquons que comme tout voisinage de $a$ contient $a$, il est nécessaire que
$f(a)\in\cap\vois_b$. On a donc envie de dire que $f(a)=b$, mais on n'a pas de
raison de dire \textit{a priori} que les filtres d'un voisinage ne contiennent
qu'un élément dans leur intersection. Prenons par exemple sur $\{a,b,c\}$ la
topologie \textit{ad hoc} suivante :
\[\{\{a,b\},\{c\},\{a,b,c\},\varnothing\}\]
Les deux éléments $a$ et $b$ ne peuvent pas être distingués.

Cela motive donc une nouvelle notion topologique : la séparation. Un axiome de
séparation est un axiome assurant que pour $a\neq b$, on a bien
$\vois_a\neq\vois_b$. De nombreux axiomes de séparation existent, mais nous ne
verrons que le plus classique, qu'on appelle axiome de séparation $T_2$. Il est
à la fois suffisamment puissant pour assurer un bon comportement à nos espaces,
et suffisamment faible pour être vérifié par de nombreux espaces. La terminologie
anglaise donne le nom \textit{Hausdorff} aux espaces vérifiant l'axiome $T_2$,
mais comme nous n'utiliserons que cet axiome de séparation, nous parlerons
simplement d'espace séparé.

\begin{definition}[Espace séparé]
  Soit $(X,\Omega_X)$ un espace topologique. On dit que cet espace est séparé
  si pour tous $x,y\in X$, il existe une paire $(U_x,U_y)\in (\Omega_X)^2$ telle
  que
  \[\begin{cases}
  x\in U_x\\
  y\in U_y\\
  U_x\cap U_y = \varnothing
  \end{cases}\]
\end{definition}

Montrons qu'on a alors l'unicité de la limite.

\begin{proposition}[Unicité de la limite]
  Soit $\mathcal F$ un filtre propre sur un espace $(X,\Omega_X)$ séparé, alors
  si $\mathcal F$ converge, il ne peut converger que vers un seul élément de $X$.
\end{proposition}

\begin{proof}
  Supposons que $a$ et $b$ sont deux limites possibles de $\mathcal F$. Par
  séparation, on trouve $U_a$ et $U_b$ deux ouverts disjoints, contenant
  respectivement $a$ et $b$. Comme $U_a$ et $U_b$ sont des ouverts, on en déduit
  que $U_a\in\vois_a$ et $U_b\in\vois_b$. De plus, par convergence de
  $\mathcal F$ vers $a$ et $b$, on en déduit que $U_a\in\mathcal F$ et
  $U_b\in\mathcal F$, donc $U_a\cap U_b\in\mathcal F$, c'est-à-dire
  $\varnothing\in\mathcal F$, ce qui contredit le fait que $\mathcal F$ est
  un filtre propre.
\end{proof}

En particulier, si $f$ converge en un point $a$ et que son espace d'arrivée est
un espace séparé, alors on a unicité du point vers lequel $f$ converge. Cela
signifie aussi que si $f$ converge en un point $a$, ce point ne peut être que
$f(a)$, puisque celui-ci est le seul à appartenir à $\bigcap (\vois_a)_f$.

La notion de limite permet aussi de caractériser la continuité : une fonction
est continue quand la limite commute avec l'application par $f$.

\begin{property}
  Soient deux espaces $(X,\Omega_X)$ et $(Y,\Omega_Y)$ sépérés, et une fonction
  $f : X \to Y$. Alors $f$ est continue si et seulement si
  \[\forall x \in X, \lim_{\mathcal F \to a} f(\mathcal F) = f(a)\]
\end{property}

\begin{proof}
  Tout d'abord, supposons que $f$ est continue, montrons alors que
  $(\vois_a)_f \supseteq \vois_{f(a)}$.

  Soit $V\in\vois_{f(a)}$ et $U$ un ouvert contenant $f(a)$ contenu dans $V$
  (donné par la définition du fait d'être un voisinage de $f(a)$).
  Comme $f$ est continue, cela signifie que $f^{-1}(U)$ est aussi un ouvert. Or
  $f^{-1}(U)\subseteq f^{-1}(V)$ et $a\in f^{-1}(U)$ puisque $a\in U$, donc
  $f^{-1}(V)\in\vois_a$, ce qui signifie exactement que $V\in(\vois_a)_f$, d'où
  l'inclusion.

  Supposons maintenant que $(\vois_a)_f\supseteq \vois_{f(a)}$, et montrons que
  $f$ est continue.

  Soit $U$ un ouvert de $Y$, montrons que $f^{-1}(U)$ est un ouvert de $X$.
  Comme $U$ est un ouvert, pour tout $x\in U$ on a $U\in\vois_x$. On en
  déduit que pour tout $y\in f^{-1}(x)$, $U\in\vois_{f(y)}$ donc, par inclusion,
  $U\in(\vois_y)_f$, c'est-à-dire que $f^{-1}(U)\in\vois_y$ pour tout
  $y\in f^{-1}(U)$ (puisque c'est le cas pour tout $y\in f^{-1}(x)$ pour chaque
  $x\in U$). On vient donc de montrer que $f^{-1}(U)$ est un voisinage de tous
  ses éléments, ce qui par une caractérisation précédente signifie qu'il est un
  ouvert. Donc la préimage d'un ouvert est un ouvert : $f$ est donc continue.
\end{proof}

\begin{exercise}
  On dit qu'une suite $u : \mathbb N \to X$ converge lorsque le filtre image du
  filtre de Fréchet sur $\mathbb N$ converge vers $X$. Montrer que cela revient à
  vérifier la formule usuelle :
  \[\exists x\in X,\forall V\in\vois_x, \exists n \in \mathbb N,
  \forall m \geq n, u_m \in V\]
\end{exercise}

\subsection{Valeur d'adhérence, compacité}

La notion de séparation permet de s'assurer qu'au plus une limite est possible,
pour un filtre donné. Il est cependant possible qu'aucune limite n'existe~ : par
exemple, si on considère dans $\mathbb R$ le filtre des parties contenant
$\{1,2\}$, on ne peut trouver aucun filtre de voisinage dont il est plus fin.
Cependant, on voudrait pouvoir dire de ce filtre qu'il converge partiellement
vers $1$ et vers $2$, qui sont des limites plus faibles.

C'est cette notion de limite plus faible qui est donnée par la définition de
valeur d'adhérence. Une valeur d'adhérence d'un filtre est une valeur vers
laquelle le filtre pourrait converger, s'il était plus fin. Dans notre exemple
précédent, le filtre pourrait converger si l'on décidait de le rendre plus fin
pour contenir $\{1\}$ ou $\{2\}$.

\begin{definition}[Valeur d'adhérence d'un filtre]
  Soit un espace topologique $(X,\Omega_X)$ et un filtre propre $\mathcal F$. On
  dit que $x$ est une valeur d'adhérence de $\mathcal F$ s'il existe un filtre
  propre $\mathcal H$ plus fin que $\mathcal F$ qui converge vers $x$.
\end{definition}

Même si l'espace est séparé, un filtre peut avoir plusieurs valeurs d'adhérence.
Cependant, il est aussi possible qu'un filtre n'en possède aucune. Si on prend
par exemple $\mathbb N$ avec la topologie discrète, le filtre de Fréchet n'a pas
de valeur d'adhérence, puisqu'un filtre de voisinage serait un ultrafiltre
principal, et que nous avons déjà prouvé que le filtre de Fréchet n'était contenu
dans aucun ultrafiltre principal.

Donnons une caractérisation simple des valeurs d'adhérence d'un filtre.

\begin{proposition}\label{prop.carac.adh}
  Soit un espace topologique $(X,\Omega_X)$ séparé et un filtre propre
  $\mathcal F$. Un point $x$ est une valeur d'adhérence de $\mathcal F$ si et
  seulement si $\displaystyle x\in \bigcap_{F\in\mathcal F} \adher F$.
\end{proposition}

\begin{proof}
  Supposons que $x$ est une valeur d'adhérence de $\mathcal F$. Alors on trouve
  un filtre propre $\mathcal H$ plus fin que $\mathcal F$ contenant $\vois_x$.
  On veut montrer que pour tout $F\in \mathcal F$, $x$ est adhérent à $F$. Soit
  $F\in\mathcal F$ et $U$ un ouvert contenant $F$, montrons que $x\in U$. Soit
  $V\in\mathcal H$ tel que $x\in V$, existant car $\vois_x\subseteq\mathcal H$.
  Par inclusion, $F\in\mathcal H$ et donc $U\in\mathcal H$, donc
  $U\cap V\in\mathcal H$. Comme $H$ est un filtre propre, $U\cap V$ est non vide.
  De plus, ce procédé fonctionne pour tout $V\in \vois_x$, donc on en déduit que
  $U\cap\bigcap \vois_x$ est non vide, mais $\bigcap \vois_x = \{x\}$ car
  l'espace est séparé, donc $x\in U$. Ainsi $x$ est un point adhérent à $F$, donc
  $x\in\adher F$.

  Réciproquement, supposons que $x$ est adhérent à tous les éléments de
  $\mathcal F$, montrons qu'alors il existe un filtre propre $\mathcal H$
  contenant $\mathcal F$ et $\vois_x$. Pour cela, il nous suffit de montrer que
  $\mathcal F\cup\vois_x$ a la propriété de l'intersection finie (le filtre
  engendré sera alors un filtre propre vérifiant les conditions plus haut).

  Comme $\mathcal F$ et $\vois_x$ sont des filtres, quitte à prendre
  l'intersection des éléments, on peut considérer qu'on a un élément
  $F\in\mathcal F$ et un élément $U\in\vois_x$, et il nous suffit alors de
  montrer que $F\cap U \neq\varnothing$. Comme $x\in\adher F$, si
  $F\cap U = \varnothing$, alors $F\subseteq (X\setminus U)$ et $X\setminus U$
  est un fermé, donc $\adher F \subseteq (X\setminus U)$, donc
  $x\in X\setminus U$, sauf que $x\in U$ par hypothèse, ce qui est
  contradictoire. On en déduit donc que $F\cap U \neq\varnothing$, donc que
  $\mathcal F\cup\vois_x$ a la propriété de l'intersection finie, d'où le
  résultat.
\end{proof}

On peut alors se demander quand cet ensemble des valeurs d'adhérence est vide.
C'est le cas par exemple du filtre de Fréchet sur $\mathbb N$ muni de la
topologie discrète, et cela s'interprète par le fait que les parties du filtres
vont à l'infini. D'une certaine manière, celles-ci sont des voisinages d'un
point se situant lui-même à l'infini (mais qui n'est pas dans l'espace, menant
à cette absence de valeur d'adhérence).

On motive donc la notion d'espace compact comme un espace dans lequel un tel cas
ne peut pas arriver. En reprenant le cas du filtre de Fréchet, on voit en
particulier qu'on peut prendre l'ensemble des points
$(\{n\})_{n\in\mathbb N}$ et en combiner une infinité à chaque fois pour
pouvoir construire une suite de parties, toutes contenues les unes dans
les autres, telle qu'il faut une infinité d'étapes pour vider les parties
ainsi construites, mais qu'un nombre fini d'étapes laisse un nombre infini
d'éléments.

En inversant le processus et en cherchant à faire croître les segments
initiaux $([0,n])_{n\in\mathbb N}$, on voit qu'il faut un nombre infini d'étapes
pour pouvoir recouvrir tout $\mathbb N$~ : la compacité est précisément la
négation de ce fait.

\begin{definition}[Compacité]
  Soit $(X,\Omega_X)$ un espace topologique. On dit que cet espace est compact si
  pour toute partie $R\subseteq \Omega_X$, si $\bigcup R = X$ alors il existe
  une partie finie $F\subfin R$ telle que $\bigcup F = X$.
\end{definition}

Cette propriété exprime que de toute recouvrement de $X$ par des ouverts, on peut
extraire un recouvrement fini. En prenant, au lieu des ouverts, des fermés, cela
nous donne qu'une famille infinie dont toute intersection finie est non vide est
une famille dont l'intersection est non vide.

\begin{property}
  Un espace topologique $(X,\Omega_X)$ est compact si et seulement si pour tout
  ensemble $S$ de fermés tel que pour tout $F\subfin R$,
  $\bigcap F \neq\varnothing$,
  on a $\bigcap S \neq\varnothing$.
\end{property}

\begin{proof}
  A $S$ on associe la famille $R$ par
  \[R \defeq \{X\setminus A\mid A \in S\}\]
  en écrivant la contraposée de la proposition ci-dessus, on obtient :
  \[\bigcap S =\varnothing\implies \exists F\subfin S, \bigcap F = \varnothing\]
  d'où, en appliquant la fonction $A \mapsto X\setminus A$ :
  \[\bigcup_{A \in S}(X\setminus A) = X\implies
  \exists F\subfin S, \bigcup_{A\in F}(X\setminus A) = X\]
  ce qui correspond exactement au fait que si $X = \bigcup R$ alors il existe
  $F\subfin R$ tel que $\bigcup F = X$, où $R$ est un ensemble d'ouvert
  (puisqu'un ensemble de complémentaires de fermés).
\end{proof}

Le fait que tout filtre admette une valeur d'adhérence est maintenant une
conséquence directe.

\begin{proposition}
  Soit $(X,\Omega_X)$ un espace topologique compact séparé, alors tout filtre
  $\mathcal F$ propre admet au moins une valeur d'adhérence.
\end{proposition}

\begin{proof}
  En utilisant la \cref{prop.carac.adh}, on sait que
  $\displaystyle\bigcap_{F\in\mathcal F}\adher F$ est
  l'ensemble des valeurs d'adhérence. Chaque $\adher F$ est un fermé, et on
  sait par définition d'un filtre propre que pour tout $A\subfin\mathcal F$,
  $\displaystyle\bigcap_{F\in A}\adher F$ est non vide, puisqu'il contient au
  moins $\bigcap A$ qui est non vide. Par compacité, on en déduit que l'ensemble
  des valeur d'adhérence de $\mathcal F$ est non vide.
\end{proof}
