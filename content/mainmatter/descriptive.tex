\chapter{Théorie descriptive des ensembles}
\label{chp.desc}

\minitoc

\lettrine{C}{e} dernier chapitre de théorie des ensembles aborde la théorie
descriptive des ensembles. Nous n'en donnerons qu'une courte introduction, en
présentant les propriétés essentielles des espace polonais ainsi qu'une étude
des espaces de Cantor et de Baire, espaces polonais incontournables.

Tout d'abord, rappelons les définitions~: un espace complet est un espace
métrique $(X,d)$ tel que toute suite de Cauchy dans $X$ converge. Un espace
séparable est un espace qui contient une partie dense dénombrable. Un espace
polonais est un espace complètement métrisable séparable, c'est-à-dire un espace
séparable tel qu'il existe une distance complète qui induit la topologie de cet
espace.

Remarquons qu'un espace polonais n'est pas la donnée d'un espace métrique~:
c'est un espace topologique, mais qui peut être induit par une métrique. La
différence est subtile, mais elle signifie que la métrique n'entre pas en compte
dans la notion d'espace polonais, simplement son existence. Par exemple, comme
on l'a dit, on peut définir la même topologie sur $(0,1)$ de deux façons~: en
considérant la topologie induite par $\mathbb R$ et en considérant la topologie
induite par $\tanh$. Les deux nous donnent des métriques, mais seulement celle
induite par $\tanh$ est complète~: cela ne change rien dans notre définition
d'espace polonais, où les deux cas donnent le même espace.

A FAIRE
