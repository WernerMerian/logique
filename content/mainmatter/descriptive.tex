\chapter{Théorie descriptive des ensembles}
\label{chp.desc}

\minitoc

\lettrine{C}{e} dernier chapitre de théorie des ensembles aborde la théorie
descriptive des ensembles. Nous n'en donnerons qu'une courte introduction, en
présentant les propriétés essentielles des espace polonais ainsi qu'une étude
des espaces de Cantor et de Baire, espaces polonais incontournables.

Tout d'abord, rappelons les définitions~: un espace complet est un espace
métrique $(X,d)$ tel que toute suite de Cauchy dans $X$ converge. Un espace
séparable est un espace qui contient une partie dense dénombrable. Un espace
polonais est un espace complètement métrisable séparable, c'est-à-dire un espace
séparable tel qu'il existe une distance complète qui induit la topologie de cet
espace.

Remarquons qu'un espace polonais n'est pas la donnée d'un espace métrique~:
c'est un espace topologique, mais qui peut être induit par une métrique. La
différence est subtile, mais elle signifie que la métrique n'entre pas en compte
dans la notion d'espace polonais, simplement son existence. Par exemple, comme
on l'a dit, on peut définir la même topologie sur $(0,1)$ de deux façons~: en
considérant la topologie induite par $\mathbb R$ et en considérant la topologie
induite par $\tanh$. Les deux nous donnent des métriques, mais seulement celle
induite par $\tanh$ est complète~: cela ne change rien dans notre définition
d'espace polonais, où les deux cas donnent le même espace.

\section{Stabilité des espaces polonais}

Pour commencer, nous allons voir que la structure d'espace polonais est stable
par de nombreuses opérations. Nous allons distinguer plusieurs opérations~: les
opérations élémentaires, qui sont des opérations topologiques classiques, les
opérations liées à la hiérarchie de Borel, et la compactification d'Alexandrov.

\subsection{Opérations élémentaires}

La première opération est celle de produit dénombrable. On a vu que la compacité
était stable par des produits quelconques, donc cette stabilité est légèrement
plus faible, mais il faut garder à l'esprit que la structure d'espace polonais
est largement plus restrictive que celle d'espace compact, donc garder une
stabilité par produits dénombrables (et non juste par produits finis) est déjà
remarquable.

\begin{proposition}
  Soit $(X_n)_{n\in \mathbb N}$ une suite dénombrable d'espaces polonais (on garde
  la topologie sous-entendue). Alors l'espace produit de ces espaces polonais
  est encore un espace polonais.
\end{proposition}

\begin{proof}
  Tout d'abord, on remarque que si $(X,d)$ est un espace métrique, alors la
  métrique suivante engendre la même topologie~:
  \[d'(x,y) \defeq \frac{d(x,y)}{1+d(x,y)}\]
  Le fait que $d'$ est une métrique se vérifie directement, sauf pour
  l'inégalité triangulaire. Soient $x,y,z \in X$, alors
  \begin{align*}
    d'(x,z) &= \frac{d(x,z)}{1 + d(x,z)}\\
    &\leq \frac{d(x,y) + d(y,z)}{1 + d(x,y) + d(y,z)}\text{ par croissance
      de la fonction } x \mapsto \frac{x}{1 + x}\\
    &\leq \frac{d(x,y)}{1 + d(x,y)} + \frac{d(y,z)}{1 + d(y,z)}\\
    &\leq d'(x,y) + d'(y,z)
  \end{align*}

  On montre donc que $d$ et $d'$ engendrent la même topologie. Pour cela,
  comme on sait que $d' \leq d$, on sait que dans toute boule ouverte $U$ pour
  $d$ il existe une boule ouverte $V$ pour $d'$ (puisque si
  $d(x,y) < \varepsilon$, alors $d'(x,y) < \varepsilon$) contenue dans $U$. On
  en déduit que tout voisinage au sens de $d$ est un voisinage au sens de $d'$,
  montrant que la topologie engendrée par $d'$ est plus fine (contient plus
  d'ouverts) que la topologie engendrée par $d$. Dans le sens réciproque, pour
  une boule
  \[U \defeq \{ y \in X \mid d'(x,y) \leq \varepsilon\}\]
  Quitte à prendre $\varepsilon$ plus petit, on sait que la fonction
  $x\mapsto \dfrac{x}{1 + x}$ est surjective sur $[0,1]$ donc sur
  $[0,\varepsilon]$ (si $\varepsilon > 1$, alors trouver une boule inclue dans
  $B_1(x)$ nous donne encore le résultat souhaité). On peut donc trouver
  $\rho$ tel que $\dfrac{\rho}{1 + \rho} = \varepsilon$. Par stricte croissance
  de $x\mapsto \dfrac{x}{1 + x}$, on sait aussi que pour tout $z \in [0,1]$,
  \[\frac{z}{1 + z} < \varepsilon \iff z < \rho\]
  On sait donc que pour tout $y \in X$,
  \[\frac{d(x,y)}{1 + d(x,y)}< \varepsilon \iff d(x,y) < \rho\]
  donc la boule pour $d$ de centre $x$ et de rayon $\rho$ est incluse dans $U$.
  Ainsi tout voisinage au sens de $d'$ est un voisinage au sens de $d$, donc
  les deux topologies coïncident.
  
  On voit, de plus, que $d'$ est bornée (si $d$ tend vers $\infty$, alors $d'$
  tend vers $1$). $d'$ est complète si et seulement si $d$ est complète~:
  une suite $(u_n)$ est de Cauchy pour $d$ si et seulement si elle est de
  Cauchy pour $d'$. Dans un sens, il suffit d'utiliser le fait que
  $d \leq d'$. Dans l'autre sens, si $(u_n)$ est de Cauchy pour
  $d$, alors soit $\varepsilon > 0$, qu'on choisit sans perte de généralité
  inférieur strict à $1$. Par l'argument de surjectivité précédent, on trouve
  $\rho$ tel que
  \[\frac{\rho}{1 + \rho} = \varepsilon\]
  Comme $(u_n)$ est de Cauchy, on trouve $n_0$ tel que
  \[\forall p,q \geq n_0, d(u_p,u_q) < \rho\]
  donc on en déduit, par stricte croissance de $x\mapsto \dfrac{x}{1 + x}$,
  que pour tout $p,q \geq n_0$, on a l'inégalité $d'(u_p,u_q) < \varepsilon$.
  Ainsi les deux distances définissent les mêmes suites de Cauchy. Comme la
  convergence est une notion purement topologique et que $d$ et $d'$
  définissent les mêmes topologies, $d$ est complète si et seulement si $d'$
  l'est.

  Ainsi on peut considérer que pour chaque espace $X_i$, on possède une
  distance complète $d_i$ bornée induisant la topologie de $X_i$. On définit
  alors la $d$ distance sur $\displaystyle\prod_{i\in\mathbb N} X_i$ par
  \[d((x_n),(y_n)) \defeq \sum_{n \in \mathbb N} \frac{d_n(x_n,y_n)}{2^n}\]
  Cette distance définit bien la topologie produit sur
  $\displaystyle\prod_{i\in\mathbb N} X_n$.

  Soit une boule $B_\varepsilon((x_n))$ pour notre distance $d$, alors à partir
  d'un certain rang $n_0$~:
  \[\forall m \geq n, \forall x,y \in X_m,
  \frac{d_m(x,y)}{2} < \frac{\varepsilon}{2}\]
  On veut prouver qu'on peut trouver un ouvert contenant $(x_n)$ pour la
  topologie produit (montrant ainsi qu'un voisinage pour $d$ est un voisinage
  pour la topologie produit). On sait qu'un ouvert de la topologie produit
  est de la forme $\displaystyle\prod_{i \in F} X_i$ pour une certaine partie
  finie $F\subfin \mathbb N$~: notre candidat pour $F$ est ici
  $\{0,\ldots,n_0\}$ puisqu'on a déjà majoré la distance des éléments après
  $n_0$. On peut donc déjà écrire que pour tout $(y_n)$~:
  \[d((x_n),(y_n)) < \frac{\varepsilon}{2} +
  \sum_{i = 0}^{n_0 - 1} \frac{d_i(x_i,y_i)}{2^i}\]
  Il nous reste donc à trouver pour chaque $i$ une boule centrée en
  $x_i$ de rayon suffisamment petit pour que la somme finie soit inférieure à
  $\dfrac{\varepsilon}{2}$, ce qui se fait directement en prenant
  \[\varepsilon'_i \defeq \frac{2^i\times \varepsilon}{2 \times n_0}\]
  Ainsi si on considère pour chaque $i$ la boule $B_{\varepsilon'_i}(x_i)$, on
  sait que pour tout $y$ dans le cylindre engendré par cette famille~:
  \[d((x_n),(y_n)) < \varepsilon\]
  Donc tout voisinage au sens de $d$ est un voisinage au sens de la topologie
  produit.

  Réciproquement, soit une famille finie $(U_i)_{i \in F}$ où
  $F \subfin \mathbb N$ où $U_i$ est une boule ouverte de $X_i$, disons
  \[U_i \defeq B_{\varepsilon_i}(x_i)\]
  et soient $(x_j)_{j \notin F}$ une famille d'éléments complétant
  $(x_i)_{i \in I}$ en un élément de
  $\displaystyle\prod_{n \in \mathbb N} X_n$. Montrons qu'on peut trouver une boule
  ouverte centrée en $(x_i)_{i \in \mathbb N}$ incluse dans le cylindre engendré
  par $(U_i)$. En prenant
  \[\varepsilon \defeq \min_{i \in F} \varepsilon_i\]
  et en posant
  \[U \defeq B_\varepsilon ((x_i))\]
  on sait que pour tout
  $\displaystyle (y_i)_{i \in \mathbb N} \in \prod_{n \in \mathbb N} X_n$,
  si $(y_i) \in U$ alors
  \[\sum_{n \in \mathbb N} d_i(x_i,y_i) < \varepsilon\]
  donc, puisque $d_i(x_i,y_i)$ est inférieur à $d((x_n),(y_n))$, on sait que
  $d_i(x_i,y_i) < \varepsilon_i$, donc $y_i \in U_i$. Ainsi $(y_i)$ est dans
  le cylindre engendré par $(U_i)$.

  En conclusion, $d$ engendre la topologie produit sur
  $\displaystyle\prod_{n \in \mathbb N} X_n$.

  Enfin, cette distance est complète~: soit
  \[(a_i)_{i \in \mathbb N} \defeq ((x_{i,j})_{j\in\mathbb N})_{i \in\mathbb N}\]
  une suite de Cauchy dans cet espace. On définit la limite
  $(b_j)_{j \in \mathbb N}$ par
  \[b_j \defeq \lim_{i\to\infty} x_{i,j}\]
  Pour montrer que cette suite est bien définie, il nous suffit de montrer que
  pour tout $j\in\mathbb N$, la suite $(x_{i,j})_{i\in\mathbb N}$ est de Cauchy.
  Soit $\varepsilon > 0$, comme $(a_i)$ est de Cauchy, on trouve $n_0$ tel que
  \[\forall p,q > n_0, d(a_p,a_q) < \varepsilon\]
  or on sait que $d_j(x_{p,j},x_{q,j}) < d(a_p,a_q)$, donc
  $(x_{i,j})_{i \in \mathbb N}$ est de Cauchy, donc $(b_j)$ est bien définie.

  On vérifie maintenant que $(a_i)$ tend vers $b$. Soit $\varepsilon > 0$.
  On sait (par définition de la convergence), que pour tout $j$ il existe
  $i_j$ tel que
  \[\forall m \geq i_j, d_j(x_{m,j},b_j) < \frac{\varepsilon}{4}\]
  De plus, comme $(a_i)$ est de Cauchy, on trouve $n_0$ tel que
  \[\forall p,q \geq n_0, d(a_p,a_q) < \frac{\varepsilon}{4}\]
  d'où on déduit que pour tout $j \in \mathbb N$,
  \[\forall p,q \geq n_0, d_j(x_{p,j},x_{q,j}) < \frac{\varepsilon}{4}\]
  Ainsi, pour tout $j \in \mathbb N$ et $m > n_0$, on a
  \begin{align*}
    d_j(x_{m,j},b_j) &\leq d_j(x_{m,j},x_{m + i_j,j}) + d_j(x_{m + i_j,j}, b_j)\\
    &< \frac{\varepsilon}{4} + \frac{\varepsilon}{4}\\
    &< \frac{\varepsilon}{2}
  \end{align*}
  D'où, en sommant ces inégalités pour tout $j \in \mathbb N$~:
  \[d(a_m,b) < \sum_{n = 0}^\infty \frac{\varepsilon}{2}\]
  d'où $d(a_m,b) < \varepsilon$. Donc $\lim_{i \to \infty} a_i = b$.
  
  Enfin, l'espace produit est séparable. Pour tout $n \in \mathbb N$, on
  choisit $Y_n$ une partie dense dénombrable de $X_n$ et $x_n$ un élément
  quelconque de $X_n$. On définit alors, pour chaque $F\subfin \mathbb N$~:
  \[Z_F \defeq \left\{ (y_i)_{i \in \mathbb N} \;\Bigg|\; \begin{cases}
    \forall i \in F, y_i \in Y_i\\
    \forall j \in \mathbb N \setminus F, y_j = x_j
  \end{cases}\right\}\]
  Le cardinal de $Z_F$ est donc dénombrable, puisqu'il y a autant d'éléments
  dans $Z_F$ que de tuples dans le produit fini des $Y_i$ pour $i \in F$. On
  en déduit, comme l'ensemble des parties finies de $\mathbb N$ est dénombrable,
  que
  \[Z \defeq \bigcup_{F\subfin \mathbb N} Z_F\]
  est dénombrable. Montrons maintenant que $Z$ est dense dans le produit
  dénombrable des $X_n$. Un ouvert de
  $\displaystyle\prod_{n \in \mathbb N} X_n$
  correspond à un ensemble fini $F\subfin \mathbb N$ et, pour chaque
  $n \in F$, un ouvert $U_n \subseteq X_n$. Par chaque $n$, comme $Y_n$ est
  dense dans $X_n$, on peut donc trouver $y_n \in Y_n \cap Y_n$. Alors
  la suite
  \[u_n \defeq \begin{cases} y_n\text{ si } n \in F \\
    x_n \text{ sinon}\end{cases}\]
  appartient au cylindre en $(U_n)_{n \in F}$, et à $Y$, donc $Y$ intersecte
  tout ouvert, donc $Y$ est dense dans
  $\displaystyle \bigcup_{n \in \mathbb N} X_n$.
\end{proof}

La seconde propriété de stabilité, là encore dénombrable, est celle par union
disjointe. Nous n'avons pas précisé ce qu'est l'union disjointe dans le cas des
espaces topologiques, nous donnons donc la définition ici~:

\begin{definition}[Union disjointe d'espaces topologiques]
  Soit $(X_i,\Omega_i)_{i\in I}$ une famille d'espaces topologiques. L'union
  disjointe de cette famille est l'espace donné par
  \[\coprod_{i\in I} X_i \defeq \{(i,x)\mid x \in X_i\}\]
  et où la topologie est engendrée par la famille de parties
  \[\Omega\left(\coprod_{i\in I} X_i\right)\defeq
  \bigcup_{i \in I}\left\{\{(i,x)\mid x\in U\}\mid U\in\Omega_i\right\}\]
\end{definition}

Intuitivement, l'union disjointe des $(X_i)$ est l'espace global dans lequel
on considère tous les $X_i$ en même temps, mais en ne considérant aucune
proximité entre eux.

\begin{proposition}
  Soit $(X_i)_{i\in \mathbb N}$ une suite d'espaces polonais. Alors l'espace
  $\coprod_i X_i$ est un espace polonais.
\end{proposition}

\begin{proof}
  Pour chaque $X_n$, on fixe une distance complète $d_n$ (qu'on choisit bornée
  par $1$ grâce à la construction précédente) qui induit la topologie de $X_n$.
  On définit sur le coproduit la distance
  \[d((i,x),(j,y)) \defeq
  \left\{\begin{array}{cl}
  d_i(x,y) & \text{si } i = j\\
  1 & \text{sinon}
  \end{array}\right.\]
  On veut montrer que les deux bases s'engendrent l'une l'autre~:
  \begin{itemize}
  \item soit $i\in I, U \in \Omega_i$. Comme la topologie de $X_i$ est
    engendrée par $d_i$, on peut écrire
    \[U = \bigcup_{j \in J} B_i(x_j,\varepsilon_j)\]
    mais dans ce cas,
    \[\bigcup_{j \in J}B((i,x_j),\varepsilon) = \{(i,x)\mid x \in U\}\]
  \item soit $x\in \coprod_i X_i$ et $\varepsilon > 0$. Si
    $\varepsilon \geq 1$, alors $B(x,\varepsilon) = \coprod_i X_i$ qui est un
    ouvert par définition. Sinon, on peut écrire $x = (i,y)$, et dans ce cas
    $B(x,\varepsilon) = \{(i,z) \mid d_i(y,z) < \varepsilon\}$, qui est par
    définition un élément de la topologie sur $\coprod_i X_i$.
  \end{itemize}
  donc $d$ définit la bonne topologie sur $\coprod_i X_i$.

  Cette distance est complète~: soit $(x_n)$ une suite de Cauchy sur notre
  espace. Comme la suite est de Cauchy, on peut trouver $n\in\mathbb N$ tel que
  \[\forall p,q > n, d(x_p,x_q) < \frac{1}{2}\]
  donc à partir du rang $n$, $(x_n)$ est dans la boule $B(x_n,\dfrac{1}{2})$.
  Cela signifie donc qu'à partir du rang $n$, tous les éléments de la suite
  sont de la forme $(i,y_n)$ pour un même $i$. Naturellement, cette suite
  correspond à la suite $(y_n) \in X_i$ qui est de Cauchy dans l'espace complet
  $(X_i,d_i)$, donc converge vers un certain point $y$. Au final, $(x_i)$
  converge vers $(i,y)$. Donc $d$ est une distance complète.

  L'espace est séparable~: pour chaque $i\in \mathbb N$, on trouve une partie
  $D_i\subseteq X_i$ dénombrable et dense dans $X_i$. Alors la partie
  \[\bigcup_{i\in \mathbb N}\{(i,x)\mid x \in D_i\}\]
  est une partie dénombrable et dense dans $\coprod_{i\in \mathbb N} X_i$.

  L'espace $\displaystyle\coprod_{i\in \mathbb N} X_i$ est donc un espace
  polonais.
\end{proof}

Enfin, on a stabilité par partie fermée.

\begin{proposition}\label{desc.ferme.polonais}
  Soit $(X,\Omega)$ un espace polonais et $F\subseteq X$ un fermé de $X$. Alors
  l'espace induit par $F$ sur $X$ est un espace polonais.
\end{proposition}

\begin{proof}
  On sait qu'une partie d'un espace complet est un espace complet pour la
  distance induite si et seulement si la partie est fermée, donc le fait d'être
  complètement métrisable passe directement à $F$.
  
  Soit $D$ une partie dénombrable dense dans $X$, notons-la
  $\{x_n\}_{n\in\mathbb N}$. On construit alors un ensemble $D'\subseteq Y$~: pour
  chaque $i\in \mathbb N$ et chaque $j\in \mathbb N^*$, si
  $B\left(x_n,\dfrac{1}{j}\right)\neq\varnothing$ alors on ajoute à $D'$ un
  élément $y_{i,j}$ de cet ensemble. Soit $y\in Y$ et $\varepsilon > 0$, on veut
  trouver un élément $y_{i,j}$ tel que $d(y,y_{i,j}) < \varepsilon$. Pour cela,
  on commence par fixer $j$ tel que $\dfrac{1}{j} < \dfrac{\varepsilon}{2}$.
  Par définition de la densité de $x_n$ dans $X$, on trouve $n$ tel que
  \[\forall m > n, d(y,x_n) < \frac{1}{j}\]
  et, par construction de $D'$, on trouve $i$ tel que
  $d(x_n,y_{i,j})<\dfrac{1}{j}$. Ainsi
  \begin{align*}
    d(y,y_{i,j}) &\leq d(y,x_n) + d(x_n,y_{i,j})\\
    &\leq \frac{1}{j} + \frac{1}{j}\\
    &\leq \varepsilon
  \end{align*}
  $D'$ est donc dense dans $Y$. $Y$ est donc un espace polonais.
\end{proof}

On donne l'exemple paradigmatique d'espace polonais~: l'ensemble des réels.

\begin{exercise}
  Montrer que $\mathbb R$ muni de la topologie usuelle est un espace polonais.
\end{exercise}

Avec ces quelques opérations élémentaires, on peut déjà donner plusieurs
exemples d'espaces polonais.

\begin{example}
  Les espaces suivants sont des espaces polonais (avec leur topologie usuelle)~:
  \begin{itemize}
  \item l'intervale $[0,1]$, les hypercubes $[0,1]^n$ pour tout $n\in\mathbb N$
    et le cube de Hilbert $[0,1]^{\mathbb N}$.
  \item de même avec $\mathbb R$ et $\mathbb C$ à la place de $[0,1]$.
  \item le cercle unité \[\mathbb U \defeq \{z\in \mathbb C\mid |z| = 1\}\]
  \item tous les espaces de Banach séparables.
  \item pour tout espace $X$ dénombrable et discret, l'espace $X^\mathbb N$.
  \end{itemize}
\end{example}

\begin{exercise}
  Prouver les exemples précédents. Prouver aussi que le tore de dimension $n$,
  défini par
  \[\mathbb T^n \defeq \prod_{i = 0}^{n-1} \mathbb U\]
  est un espace polonais pour tout $n \in \mathbb N$.
\end{exercise}

\subsection{La hiérarchie de Borel}

Un outil important de théorie descriptive des ensembles est la hiérarchie de
Borel. Elle permet de hiérarchiser les parties d'un espace topologique suivant
la complexité à les construire. Plus précisément, cette hiérarchie s'applique
aux boréliens, notion importante entre autre en théorie de la mesure. Nous
rappelons la définition d'ensemble des boréliens, qui peut être vu comme la plus
petite $\sigma$-algèbre contenant $\Omega$.

\begin{definition}[Ensemble des boréliens]
  Soit un espace topologique $(X,\Omega)$. On appelle ensemble des boréliens de
  $(X,\Omega)$, que l'on note $\Borel(X,\Omega)$, le plus petit ensemble
  $\Borel\subseteq \powerset(X)$ tel que~:
  \begin{itemize}
  \item $\Omega\subseteq \Borel$
  \item si $(Y_i)_{i\in\mathbb N}$ est une suite d'éléments de $\Borel$, alors
    $\displaystyle\bigcap_{i\in\mathbb N} Y_i\in \Borel$ et
    $\displaystyle\bigcup_{i\in\mathbb N} Y_i\in\Borel$.
  \item si $Y\in\Borel$, alors $X\setminus Y \in \Borel$.
  \end{itemize}
\end{definition}

On peut alors décomposer $\Borel(X,\Omega)$ en une hiérarchie indicée par
$\Ord$.

\begin{definition}[Hiérarchie de Borel]
  Soit $(X,\Omega)$ un espace topologique. On définit par induction transfinie
  les ensembles $\bSigma^0_\alpha,\bPi^0_\alpha$ et $\bDelta^0_\alpha$ pour
  $\alpha \in \Ord$~:
  \begin{itemize}
  \item $\bSigma_1^0 = \Omega$
  \item pour une partie $Y\subseteq X$, $Y\in\bSigma_\alpha^0$ si et seulement
    s'il existe une suite $(Y_i)_{i\in\mathbb N}$ telle que
    \[\left\{\begin{array}{c}
    \forall i \in \mathbb N, \exists \beta < \alpha, Y_i \in \bPi_\beta^0\\
    \displaystyle\bigcup_{i\in\mathbb N} Y_i = Y
    \end{array}\right.\]
  \item pour tout $\alpha \in \Ord$, on définit
    \[\bPi_\alpha^0 \defeq \{X\setminus Y\mid Y \in \bSigma_0^\alpha\}\]
  \item pour tout $\alpha \in \Ord$, on définit
    \[\bDelta_\alpha^0\defeq \bSigma_\alpha^0\cap \bPi_\alpha^0\]
  \end{itemize}

  En particulier, on dit qu'une partie $Y\subseteq X$ est un $F_\sigma$ si
  $Y\in\bSigma_2^0$ et un $G_\delta$ si $Y\in\bPi_2^0$, ce sont respectivement les
  unions dénombrables de fermés et les intersections dénombrables d'ouverts.
\end{definition}

\begin{exercise}
  Montrer que pour tout espace topologique $(X,\Omega)$, on a l'égalité
  \[\Borel(X,\Omega) = \bigcup_{\alpha < \omega_1} \bSigma_\alpha^0\]
\end{exercise}

On peut désormais montrer la stabilité des espaces polonais pour les parties
$G_\delta$.

\begin{proposition}
  Soit $(X,\Omega)$ un espace polonais et $Y\subseteq X$ un $G_\delta$. Alors
  $Y$ muni de la topologie induite est un espace polonais.
\end{proposition}

\begin{proof}
  On se donne une distance complète $d$ sur $X$ engendrant $\Omega$. Soit
  $(U_n)_{n\in\mathbb N}$ une suite d'ouverts telle que $\bigcap U_n = Y$. On
  définit
  \[C_n \defeq X\setminus U_n\]
  et une nouvelle distance $d'$ sur $Y$~:
  \[d'(x,y) \defeq d(x,y) + \sum_{n\in \mathbb N} \min\left(2^{-n-1},
  \left|\frac{1}{d(x,C_n)} - \frac{1}{d(y,C_n)}\right|
  \right)\]

  La topologie induite par $d'$ est la même que celle induite par $d$
  (vérification laissée dans l'\cref{exo.desc.verif1}).

  $d'$ est une distance complète~: soit $(x_n)_{n\in\mathbb N}$ une suite de
  Cauchy de $(Y,d')$. $(x_n)$ est aussi une suite de Cauchy dans $(X,d)$. En
  effet, pour $\varepsilon > 0$, on trouve $n_0$ tel que
  \[\forall p,q > n_0, d'(x_p,x_q) \leq \varepsilon\]
  mais $d \leq d'$, donc $d(x_p,x_q) \leq \varepsilon$ dans l'expression
  ci-dessus, donc la fonction est de Cauchy pour $d$. Comme $(X,d)$ est complet,
  on trouve une limite $x$ pour $(x_n)_{n\in\mathbb N}$. Soit $n\in \mathbb N$, on
  veut montrer que $x\notin C_n$, ce que l'on peut prouver en montrant que
  $d(x,C_n) \neq 0$. Pour cela, on veut montrer que
  $\dfrac{1}{d(x,C_n)} \neq 0$ en utilisant la définition de $d'$.

  Soit $\varepsilon > 0$, comme $(x_n)$ est de Cauchy, on peut trouver $n_0$ tel
  que
  \[\forall p,q > n_0, d'(x_p,x_q) \leq \min(2^{-n-2}, \varepsilon)\]
  Soient $p,q > n_0$. Comme $d'(x_p,x_q) \leq 2^{-n-2}$, il n'est pas possible
  d'avoir
  \[\left|\frac{1}{d(x_p,C_n)}-\frac{1}{d(x_q,C_n)}\right| \leq 2^{-n-1}\]
  d'où l'inégalité dans le sens inverse, d'où
  \begin{align*}
    \left|\frac{1}{d(x_p,C_n)} - \frac{1}{d(x_q,C_n)}\right| & \leq
    d(x_p,x_q) + \min\left(2^{-n-1},\left|\frac{1}{d(x_p,C_n)}
    - \frac{1}{d(x_q,C_n)}\right|\right)\\
    &\leq d'(x_p,x_q)\\
    &\leq \varepsilon
  \end{align*}

  On en déduit que
  $\displaystyle\left(\frac{1}{d(x_p,C_n)}\right)_{p\in\mathbb N}$ est une
  suite de Cauchy, donc converge (dans $\mathbb R$), ce qui signifie que
  pour tout $p$ suffisamment grand, $d(x_p,C_n) \neq 0$, donc que
  $d(x,C_n) \neq 0$, d'où $x\notin C_n$. Comme cela est vrai pour tout $n$, on
  en déduit que $x\in \bigcap U_n$, donc que $x\in Y$. Ainsi $(Y,d')$ est
  complet.

  Cet espace est séparable par le même argument que pour la preuve de la
  \cref{desc.ferme.polonais}.
\end{proof}

\begin{exercise}\label{exo.desc.verif1}
  Vérifier, dans la preuve précédente, que $d'$ est une distance et qu'elle
  induit la même topologie que $d$.
\end{exercise}

\subsection{Compactification d'Alexandrov}

La dernière opération élémentaire que nous abordons est celle de la
compactification d'Alexandrov. L'idée de cette opération est de transformer un
espace se comportant suffisamment bien en un espace compact. Cette construction
s'applique aux espace localement compact, que nous devons donc introduire.

\begin{definition}[Espace localement compact]
  Un espace $(X,\Omega)$ est localement compact si pour tout $x\in X$, il existe
  un ouvert $U\in \Omega$ et un compact $K$ de $X$ tels que $x\in U\subseteq K$.
\end{definition}

L'idée d'un espace localement compact, dans le cas métrique, est qu'une boule
suffisamment petite est contenue dans un compact. Remarquons que dans le cas
d'un espace vectoriel normé de dimension infini par exemple, n'importe quel
ouvert aussi petit soit-il n'est pas inclus dans un compact, donc cette
condition n'est pas anodine.

Une version équivalente de cette définition est que tout filtre de voisinage
peut se définir par des compacts. Cette définition est largement plus
manipulable, puisqu'elle permet de directement considérer des compacts aussi
proches d'un points que l'on veut.

\begin{proposition}\label{prop.carac.compac.loc}
  Soit un espace $(X,\Omega)$ séparé. Cet espace est localement compact si et
  seulement si, pour tout $x$, il existe une famille $(K_i)_{i \in I}$ engendrant
  le filtre des voisinages de $x$.
\end{proposition}

\begin{proof}
  On procède par double implication~:
  \begin{itemize}
  \item si tout voisinage est engendré par un compact, alors pour tout $x$, on
    trouve un voisinage $V$ de $x$~: celui-ci contient un compact $K$ de notre
    famille. Mais comme $K$ fait partie d'une famille qui engendre le
    filtre des voisinages de $x$, $K$ est lui-même un voisinage de $x$, donc
    on trouve un ouvert $U$ tel que $x \in U \subseteq K$.
  \item réciproquement, si l'espace est localement compact, montrons que pour
    tout $x \in X$, il existe une famille $(K_i)$ de compacts engendrant
    le filtre des voisinages de $x$. On sait qu'il existe un compact $K$ tel que
    $x \in K$, et qu'une partie d'un compact est compacte si et seulement si
    elle est fermée (par la \cref{prop.compac.equiv.ferme}), donc il suffit pour
    tout voisinage $U$ de $x$ de montrer qu'il existe $F$ fermé contenant $x$
    et contenu dans $U$. On considère la frontière de $K \cap U$, c'est-à-dire
    l'ensemble
    \[\partial(K \cap U) \defeq \adher{K \cap U} \setminus \inter{K \cap U}\]
    qui est fermée puisqu'intersection de fermés, donc compacte (puisque partie
    de $K$). Pour chaque $y \in \partial(K \cap U)$, on trouve par séparation
    un voisinage ouvert $V_y$ de $y$ et un voisinage ouvert $W_y$ de $x$,
    disjoints. Quitte à intersecter $W_y$ par $K\cap U$, on peut considérer que
    $W_y$ est une partie de $K \cap U$. Comme $(V_y)$ est un recouvrement de
    $\partial(K\cap U)$, on peut par compacité en extraire un recouvrement
    fini $(V_y)_{y \in J}$.En prenant alors
    \[W \defeq \bigcap_{y \in J} W_y\]
    on a alors un voisinage de $x$ disjoint de tous les $V_y$, supposés
    ouverts, donc l'adhérence de $W$ est disjointe de $\partial(K \cap U)$.
    On a donc trouvé un voisinage fermé $F$ de $x$ contenu dans $U$.
  \end{itemize}
  L'équivalence est donc prouvée.
\end{proof}

\begin{definition}[Compactification d'Alexandrov]
  Soit $(X,\Omega)$ un espace localement compact. On appelle compactifié
  d'Alexandrov de $(X,\Omega)$ l'espace $(\tilde X,\tilde\Omega)$ défini par
  \[\tilde X \defeq X\cup \{\infty\}\qquad
  \tilde\Omega\defeq \Omega \cup
  \{(X\setminus K) \cup \{\infty\}\mid K \text{ compact de } X\}\]
  où $\infty\notin X$.
\end{definition}

Le compactifié d'Alexandrov d'un espace est donc l'espace auquel on ajoute un
point (considéré à l'infini), dont les voisinages ont comme base les
complémentaires de compacts dans $X$. La propriété essentielle du compactifié
d'Alexandrov, comme son nom l'indique, est de produire un espace compact.

\begin{proposition}
  Soit $(X,\Omega)$ un espace localement compact, alors
  $(\tilde X,\tilde\Omega)$ est un espace compact.
\end{proposition}

\begin{proof}
  Soit $(U_i)_{i\in I}$ un recouvrement de $\tilde X$ par des ouverts de
  $\tilde\Omega$, alors il existe $i\in I$ tel que $\infty \in U_i$. Par
  définition de $\tilde \Omega$, on trouve $K$ compact tel que
  $U_i = (X\setminus K)\cup\{\infty\}$. On en déduit que
  $(U_j)_{j\in I\setminus\{i\}}$ est un recouvrement de $K$, mais comme
  on a un recouvrement d'un compact, on peut en extraire un recouvrement
  fini $(U_j)_{j\in J}$ où $J\subfin I\setminus\{i\}$. Ainsi
  $(U_j)_{j\in J\cup \{i\}}$ est un recouvrement fini de $\tilde X$, donc
  $(\tilde X,\tilde\Omega)$ est compact.
\end{proof}

Cette notion de compactification va nous servir pour considérer de nouveaux
espaces. Cependant, il nous faut aussi d'autres notions et résultats. Ces
résultats sont importants en topologie, mais cet ouvrage n'est pas un cours de
topologie~: nous décidons donc d'admettre le \cref{desc.thm.urysohn}.

\begin{definition}[Espace à base dénombrable]
  Un espace $(X,\Omega)$ est dit à base dénombrable s'il existe une famille
  $(U_i)_{i\in\mathbb N}$ engendrant $\Omega$.
\end{definition}

\begin{proposition}
  Un espace polonais est à base dénombrable.
\end{proposition}

\begin{proof}
  Soit $(X,\Omega)$ un espace polonais. On trouve un ensemble dense
  $D\subseteq X$, on notera $D = \{x_n\}_{n\in\mathbb N}$. Pour tout
  $n\in\mathbb N$, on considère la base de voisinages
  \[(V_{n,j})_{j\in \mathbb N} \defeq B\left(x_n,\frac{1}{n}\right)\]

  La famille $(V_{n,j})_{(n,j)\in\mathbb N\times\mathbb N}$ est une base de $\Omega$.
  En effet, soit $U\in\Omega$. Pour tout $x\in U$, on peut trouver
  $\varepsilon$ tel que $B(x,\varepsilon)\subseteq U$. Comme $D$ est dense,
  on peut trouver un certain $x_n$ tel que
  $x_n \in B\left(x,\dfrac{\varepsilon}{3}\right)$. On peut alors, en
  choisissant $i$ tel que
  $\dfrac{\varepsilon}{3} < \dfrac{1}{i} < \dfrac{\varepsilon}{2}$, remarquer
  que $x\in V_{n,i}\subseteq B(x,\varepsilon)\subseteq U$. On peut donc écrire,
  en notant $n(x)$ et $i(x)$ respectivement les indices trouvés,
  \[U = \bigcup_{x\in U} V_{n(x),i(x)}\]
  donc $(V_{n,j})$ est bien une base dénombrable de $\Omega$.
\end{proof}

\begin{definition}[Espace régulier]
  Un espace $(X,\Omega)$ est dit régulier si pour tout fermé $C\subseteq X$ et
  point $x\notin C$ il existe deux ouverts $U_1,U_2\in\Omega$ tels que
  \[\left\{
  \begin{array}{l}
    x\in U_1\\
    C\subseteq U_2\\
    U_1\cap U_2 = \varnothing
  \end{array}
  \right.\]
\end{definition}

\begin{theorem}[Métrisation d'Urysohn]\label{desc.thm.urysohn}
  Tout espace régulier à base dénombrable est métrisable.
\end{theorem}

On peut alors énoncer nos derniers résultats de stabilité.

\begin{proposition}\label{desc.prop.second}
  Un espace compact séparé est métrisable si et seulement s'il est à base
  dénombrable. Un espace vérifiant ces propriétés équivalentes est un espace
  polonais.
\end{proposition}

\begin{proof}
  Supposons qu'on possède une métrique sur $(X,\Omega)$ supposé compact séparé.
  Soit $n\in \mathbb N$. Il est clair que la famille
  $\{B(x,2^{-n})\mid x \in X\}$ recouvre $X$, donc on peut en extraire un
  recouvrement fini $\{B_{2^{-n}}(x)\mid x \in I_n\}$ où $I_n\subfin X$. Notre
  base dénombrable est
  \[\bigcup_{i\in \mathbb N} \{B_{2^{-n}}(x)\mid x \in I_n, q \in \mathbb Q\}\]
  On veut donc montrer que pour tout $x\in X, \varepsilon > 0$, il existe un
  élément de notre base qui contient $x$ et est contenu dans $B_\varepsilon(x)$
  (cela montrera alors que la topologie engendrée par notre base est plus
  fine que la topologie $\Omega$, ce qui suffit puisque notre base est
  extraite de $\Omega$). On sait que la famille des $B_\varepsilon(y)$ pour
  $y \in (X \setminus B_\varepsilon(x))\cup \{x\}$ recouvre $X$, et on sait que
  $X$ est compact. On en déduit donc, par le \cref{lem.nb.lebesgue}, qu'il
  existe un nombre de Lebesgue $\rho$ pour ce recouvrement. Ainsi, pour $n$ tel
  que $2^{-n} < \rho$, comme la famille des $B_{2^{-n}}(x_i)$ pour $x_i \in I_n$
  recouvre $X$, on sait qu'il existe $x_0$ tel que $B_{2^{-n}}(x_0) \ni x$. Par la
  propriété du nombre de Lebesgue, on trouve $a$ dans notre recouvrement tel que
  $B_{2^{-n}}(x_0)\subseteq B_\varepsilon(a)$, mais cela n'est possible que pour
  $a = x$, étant donné qu'aucun autre élément du recouvrement ne contient $x$.
  On en déduit donc que $B_{2^{-n}}(x_0)\subseteq B_\varepsilon(x)$, d'où le
  résultat.

  Pour le sens réciproque, si $(X,\Omega)$ est compact séparé à base
  dénombrable, on veut appliquer le \cref{desc.thm.urysohn}~: il nous suffit
  donc de vérifier que cet espace est régulier. Soit $C\subseteq X$ fermé et
  $x\in X \setminus C$. Notons $U = X \setminus C$. Comme $(X,\Omega)$ est
  séparé, on sait que pour tout $y \in C$, il existe $(U_y,v_y)$ deux ouverts
  tels que
  \[\left\{\begin{array}{c}
  x \in U_y\\
  y \in v_y\\
  U_y \cap V_y = \varnothing
  \end{array}\right.\]
  La famille $\{V_y\}_{y \in C}$ est alors un recouvrement de $C$, qui est
  compact car c'est un fermé dans un espace compact~: on en déduit un ensemble
  $I\subfin C$ tel que $\bigcup_I V_y \supseteq C$, notons cet ensemble $V$.
  On voit alors que
  \[U \defeq \bigcap_{y\in I} U_y\]
  est un ouvert comme intersection finie d'ouverts, et $U\cap V = \varnothing$
  et $x \in U$ puisque chaque $U_y$ contient $x$ par construction. On a donc
  trouvé les deux ouverts souhaités.

  Enfin, supposons que $(X,\Omega)$ est un espace métrique compact à base
  dénombrable, montrons que c'est un espace polonais. Pour la séparabilité,
  on voit que la famille $\bigcup_{i \in \mathbb N} I_n$ définie au début de la
  preuve est dense dans $X$. On sait que l'espace est métrique, il nous reste
  donc à montrer qu'on a une métrique complète. Soit $(x_n)_{n\in\mathbb N}$
  une suite de Cauchy. Comme l'espace est compact, on sait qu'il existe une
  sous-suite $(x_{\varphi(n)})_{n\in\mathbb N}$ convergente~: notons $x$ cette
  limite. Soit $\varepsilon > 0$, comme $(x_n)$ est de Cauchy on peut trouver
  $n_0$ tel que
  \[\forall p,q > n_0, d(x_p,x_q) \leq \frac{\varepsilon}{2}\]
  de plus, puisque $(x_{\varphi(n)})$ converge, on peut trouver $n_1$ tel que
  \[\forall n > n_1, d(x_{\varphi(n)},x) \leq \frac{\varepsilon}{2}\]
  On voit donc que, pour $n > \max(n_0,n_1)$~:
  \begin{align*}
    d(x_n,x) &\leq d(x_n,x_{\varphi(n)}) + d(x_{\varphi(n)},x)\\
    &\leq \frac{\varepsilon}{2} + \frac{\varepsilon}{2}\\
    &\leq\varepsilon
  \end{align*}
  donc $(x_n)$ converge vers $x$. Ainsi $d$ est complète, achevant la preuve
  que $(X,\Omega)$ est un espace polonais.
\end{proof}

L'exerice suivant cherche à montrer la stabilité pour les fonctions continues à
valeurs dans $\mathbb C$.

\begin{exercise}
  Soit $(X,\Omega)$ un espace compact métrisable. Montrer que l'espace
  $\mathcal C(X,\mathbb C)$ muni de la distance
  \[d(f,g) \defeq \sup_{x \in X}|f(x) - g(x)|\]
  est un espace polonais.
\end{exercise}

Enfin, en réutilisant le résultat sur les $G_\delta$, on peut affaiblir les
conditions de la \cref{desc.prop.second} à de la compacité locale.

\begin{proposition}
  Soit $(X,\Omega)$ un espace localement compact à base dénombrable, c'est un
  espace polonais.
\end{proposition}

\begin{proof}
  On voit d'abord que $X$ est un ouvert de $\tilde X$, et un ouvert est en
  particulier un $G_\delta$, donc il suffit de montrer que $\tilde X$ est un
  espace polonais.

  On veut donc montrer que l'espace est compact et à base dénombrable. La
  compacité est directe, il nous reste donc à trouver une base dénombrable. Un
  ouvert de $\tilde X$ est soit un ouvert de $X$ (auquel cas, $X$ étant à base
  dénombrable, on peut directement reconstruire cet ouvert grâce à la base
  dénombrable de $X$), soit un voisinage de $\infty$. Il nous reste donc à
  trouver une base de $\mathcal V_\infty$. Pour trouver cette base, on commence
  par remarquer que, comme $X$ est à base dénombrable, c'est un espace
  sépérable~: on peut prendre un élément de chaque ouvert de la base dénombrable
  pour construire un espace dense dénombrable. On considère maintenant une
  partie dense dénombrable $D \subseteq X$. Pour chaque $x \in D$ et chaque
  $n \in \mathbb N$, comme l'espace est localement compact, par la
  \cref{prop.compac.equiv.ferme}, on trouve un voisinage compact $K_x$ de
  $x$. On définit maintenant notre base de $\mathcal V_\infty$ par le filtre
  $\mathcal W$, engendré par les complémentaires des $K_x$~:
  \[\mathcal W \defeq \filclose{\bigcup_{x \in D} X\setminus K_{x}}\]
  On sait déjà, comme cette famille est constituée d'ouverts, que le filtre
  $\mathcal V_\infty$ est plus fin que $\mathcal W$. Dans le sens réciproque,
  soit $K$ un compact, on veut montrer que l'ouvert $X\setminus K$ contient un
  ouvert de $\mathcal W$. Cela revient à dire que pour tout compact $K$, il
  existe une union finie de $K_x$ recouvrant $K$. Or, on sait que
  $(K_x)_{x \in K}$ recouvre $K$, et que chaque $K_x$ contient un voisinage
  ouvert $U_x$ de $x$. Ainsi, par compacité de $K$, on extrait une partie finie
  $F\subseteq K$ telle que $(U_x)_{x \in F}$ recouvre $K$, donc $(K_x)_{x \in F}$
  recouvre $K$, d'où le résultat.
\end{proof}

\section{Espace de Cantor et de Baire}

Dans cette dernière section, on s'intéresse aux espaces de Cantor et de Baire,
qui sont des exemples essentiels d'espaces polonais. L'espace de Cantor a été
d'une certaine manière introduit dans le \cref{chp.topo} puisqu'il est un espace
de Stone, mais nous allons étudier plus en détail ses propriétés topologiques.
Nous allons voir les notions de schéma de Cantor et de schéma de Lusin,
permettant de construire efficacement des fonctions depuis les espace
respectifs. Nous verrons ensuite le théorème de Cantor-Bendixon, qui est un des
théorèmes importants de la théorie descriptive des ensembles, puis nous verrons
la caractérisation des espaces de Cantor et de Baire.

\subsection{Schémas de Cantor et de Baire}

On sait que la topologie de $2^{\mathbb N}$ peut être engendrée par la famille de
parties
$\{\alpha \in 2^{\mathbb N}\mid \forall i \in \{1,\ldots,n\}, \alpha(i) = x_i\}$
pour $(x_1,\ldots,x_n)$ des éléments de $\{0,1\}$. On voit ainsi qu'un élément
de l'espace de Cantor peut être vu comme la limite de suites finies sur $2$.
On veut donc associer à des suites finies sur $2$ des ouverts, en voulant
associer ces ouverts de base de l'espace de Cantor.

Ainsi, un schéma de Cantor sera une association des suites finies sur un espace,
de sorte que l'image des suites infinies est obtenue par passage à la limite.
Pour une suite finie $s \in 2^{<\omega}$ de taille $n$ et $a \in \{0,1\}$, on
note $s\star a$ la suite finie obtenue en associant $n$ à $a$ et les termes
de $0$ à $n-1$ à ceux correspondant dans $s$.

\begin{definition}[Schéma de Cantor]
  Soit $(X,\Omega)$ un espace polonais. On appelle schéma de Cantor sur $X$ une
  famille $\mathcal X = \{X_s\mid s \in 2^{<\omega}\}$ vérifiant les propriétés
  suivantes~:
  \begin{itemize}
  \item pour tout $s \in 2^{<\omega}$, $X_s \neq \varnothing$.
  \item pour tout $s\in 2^{<\omega}$, $X_{s\star 0} \subseteq X_s$ et
    $X_{s\star 1} \subseteq X_s$.
  \item pour tout $s\in 2^{<\omega}$,
    $X_{s\star 0}\cap X_{s\star 1} = \varnothing$.
  \end{itemize}

  On dit qu'un schéma de Cantor est de diamètre évanescant si pour toute suite
  $u\in 2^{\mathbb N}$, on a
  \[\lim_{n\in\mathbb N} \sup_{x,y\in X_{u\restr n}}d(x,y) = 0\]
\end{definition}

\begin{proposition}\label{desc.prop.schem.C}
  Soit $(X,\Omega)$ un espace polonais et $\mathcal X$ un schéma de Cantor dont
  tous les éléments sont des ouverts. Alors une fonction
  $f : 2^{\mathbb N} \to X$ vérifiant
  \[\forall u \in 2^{\mathbb N}, f(u) \in \bigcap_{n \in \mathbb N}X_{u\restr n}\]
  est une injection continue.
\end{proposition}

\begin{proof}
  Pour l'injectivité, supposons que $u \neq v$ pour deux suites
  $u,v\in 2^{\mathbb N}$. On trouve alors un indice minimal $i\in\mathbb N$ tel
  que $u(i) \neq v(i)$. Ainsi, comme $f(u) \in X_{u\restr i}$ et
  $f(v)\in X_{v\restr i}$, on en déduit que $f(u) \neq f(v)$.

  Il nous reste à vérifier la continuité. Soit $x\in X$ et $\varepsilon > 0$.
  Si $f^{-1}(B(x,\varepsilon)) = \varnothing$, la prémiage est effectivement un
  ouvert. Sinon, soit $u$ tel que $d(x,f(u)) < \varepsilon$. Comme $\mathcal X$
  est de diamètre évanescent, on trouve $n$ tel que
  \[\forall m \geq n, \sup_{x,y \in X_{u\restr m}} d(x,y) <
  \varepsilon - d(x,f(u))\]
  Alors, pour tout $v \in \ext(u\restr n)$~:
  \[d(f(v),x) \leq d(f(v),f(u)) + d(\] A FAIRE
\end{proof}

\begin{remark}
  Si $\mathcal X$ est de diamètre évanescent, alors il existe une et une seule
  fonction vérifiant la condition précédente. En effet, l'intersection à
  laquelle appartient $f(u)$ contient au plus un élément, et elle contient
  bien un élément car, en prenant un élément dans chaque $X_{u\restr n}$ supposé
  non vide, on obtient une suite de Cauchy dont la limite doit appartenir à
  l'intersection décrite plus haut.
\end{remark}

L'analogue d'un schéma de Cantor pour l'espace de Baire $\mathbb N^\mathbb N$
est appelé schéma de Lusin. On impose ici directement que le diamète est
évanescent (en un sens fort), puisque dans tous les cas on n'utilisera un schéma
de Lusin que dans ce cas.

\begin{definition}[Schéma de Lusin]
  Soit $(X,\Omega)$ un espace polonais. On appelle schéma de Lusin une famille
  $\{X_s\}_{s\in \mathbb N^{<\omega}}$ telle que~:
  \begin{itemize}
  \item $X = U_{\varnothing}$
  \item pour tout $s\in \mathbb N^{<\omega}$, on a
    $\displaystyle \bigcup_{n\in\mathbb N} X_{s\star n} = X_s$
  \item pour tout $s\in\mathbb N^{<\omega}$, $X_s$ est un ouvert-fermé non vide
    tel que $\displaystyle\sup_{x,y\in X_s} d(x,y) < 2^{-n}$.
  \end{itemize}
\end{definition}

\begin{proposition}
  Soit $(X,\Omega)$ un espace polonais et $\mathcal X$ un schéma de Lusin sur
  $X$. Alors il existe une unique fonction $f : \mathbb N^\mathbb N \to X$
  vérifiant
  \[\forall u \in \mathbb N^\mathbb N, f(u) \in
  \bigcap_{n \in \mathbb N} X_{u\restr n}\]
  et cette fonction est une injection continue.
\end{proposition}

\begin{proof}
  A FAIRE
\end{proof}

Revenons maintenant aux schémas de Cantor. Une façon naturelle de l'appliquer
est de considérer un espace $X$ et de le couper en deux ouverts disjoints
$X_0$ et $X_1$, puis ceux-ci récursivement en deux ouverts disjoints, jusqu'à
obtenir un schéma de Cantor. Si on considère un ouvert $U\in \Omega$, le
couper en deux ouverts disjoints est particulièrement facile~: étant donnés deux
points dans $U$, la séparation de $X$ (métrisable) nous donne deux ouverts
disjoints, chacun contenant l'un des deux points. Le souci étant que $U$ n'a pas
de raison de contenir deux points.

Cela nous mène à la notion de point isolé et d'espace parfait.

\begin{definition}[Point isolé]
  Soit $(X,\Omega)$ un espace topologique. On dit qu'un point $x\in X$ est isolé
  si $\{x\}\in\Omega$.
\end{definition}

\begin{remark}
  Un espace est discret si et seulement si tous ses points sont isolés.
\end{remark}

Le contraire d'un point isolé est appelé un point d'accumulation. On le définit
habituellement relativement à un ensemble, ce que nous allons donc faire ici,
mais ce qui nous importe est de considérer un point d'accumulation de l'espace
entier.

\begin{definition}[Point d'accumulation]
  Soit $(X,\Omega)$ un espace topologique, $Y\subseteq X$ une partie et
  $x\in X$. On dit que $x$ est un point d'accumulation de $Y$ si pour tout
  ouvert $U\in\Omega$ contenant $x$, il existe un point $y\in Y$ différent de
  $x$ tel que $y\in U$. On dira que $x$ est un point d'accumulation sans
  autre précision si $x$ est un point d'accumulation de $X$.
\end{definition}

On voit donc qu'un point est soit un point isolé, soit un point d'accumulation.
Pour construire le schéma de Cantor énoncé plus haut, il nous suffit donc
que l'espace ne contienne aucun point isolé~: ces espaces sont nommés des
espaces parfaits.

\begin{definition}[Espace parfait]
  Un espace parfait est un espace topologique dont tous les points sont des
  points d'accumulation. Si $(X,\Omega)$ est un espace topologique et
  $Y\subseteq X$ est un espace parfait (pour la topologie induite), on dit alors
  que $Y$ est parfait dans $X$.
\end{definition}

On peut maintenant construire le schéma de Cantor que l'on évoquait.

\begin{proposition}
  Soit $(X,\Omega)$ un espace polonais parfait. Il existe une injection continue
  $2^{\mathbb N} \to X$.
\end{proposition}

\begin{proof}
  On construit pour cela un schéma de Cantor de diamètre évanescent constitué
  d'ouverts de $X$~:
  \begin{itemize}
  \item on définit $X_\varnothing = X$.
  \item pour toute suite finie $s\in 2^{<\omega}$, on considère deux éléments
    $x,y\in X_s$ (qui existent car $X_s$ est un ouvert et $X$ est parfait). Par
    séparation, on trouve $U_x,U_y$ disjoints contenant respectivement $x$ et
    $y$. On pose alors
    \[X_{s\star 0} = X_s\cap U_x \cap B(x,2^{-|s|+1})\qquad
    X_{s\star 1} = X_s\cap U_y \cap B(y,2^{-|s|+1})\]
  \end{itemize}
  Comme $X_s$ a un diamètre inférieur à $2^{-|s|}$, on en déduit que le diamètre
  du schéma de Cantor est bien évanescent. Il nous reste donc à vérifier que
  l'assertion suivante~:
  \[\forall u\in 2^\mathbb N,\bigcap_{n \in \mathbb N}X_{u\restr n}
  \neq\varnothing\]
  On sait que chaque $X_s$ est non vide, par construction, donc on peut trouver
  une suite $(x_n)_{n\in\mathbb N}$ telle que $x_i\in X_{u\restr n}$~: celle-ci
  est de Cauchy. En effet, comme à partir du rang $n$ tous les points de
  $(x_n)$ appartiennent à $X_{u\restr n}$, on en déduit que la distance entre
  deux éléments est inférieure à $2^{-n}$, nous donnant la propriété d'être de
  Cauchy puisque $2^{-n} \xrightarrow[n\to\infty]{} 0$. Comme $X$ est complet,
  on en déduit que $x$ converge et donc que
  $\displaystyle x\in\bigcap_{n\in\mathbb N} X_{u\restr n}$.

  On a ainsi, grâce à la \cref{desc.prop.schem.C}, on en déduit l'existence
  d'une injection continue de $2^\mathbb N$ dans $X$.
\end{proof}

\begin{corollary}
  Un espace polonais parfait est de cardinal $2^{\aleph_0}$.
\end{corollary}

\begin{proof}
  On sait que $2^{\aleph_0}$ minore de le cardinal d'un espace polonais parfait
  par la proposition précédente. De plus, comme tout point peut être écrit
  comme limite de points dans un ensemble dénombrable, on peut injecter l'espace
  dans $\mathbb N^\mathbb N$, de cardinal $2^{\aleph_0}$, d'où le résultat.
\end{proof}

En fait, on peut être plus précis sur les espaces polonais~: ceux-ci sont
presque entièrement parfait, au sens où l'ensemble des points non parfaits d'un
espace polonais est dénombrable. L'énoncé plus précis est celui du théorème de
Cantor-Bendixson.

\begin{theorem}[Cantor-Bendixson]
  Soit $(X,\Omega)$ un espace polonais. Il existe alors un ouvert dénombrable
  $U\in \Omega$ et une partie parfaite $P\subseteq X$ tels que
  $X = U \cup P$.
\end{theorem}

\begin{proof}
  Une idée intuitive de preuve est que depuis l'ensemble $X$, on peut décider
  de retirer les points isolés. Cependant, on peut imaginer que de nouveaux
  points deviennent isolés à cause de ce retrait, obligeant à itérer ce
  processus. Comme l'ensemble $U$ souhaité se trouve être dénombrable, le
  processus n'a à être itéré qu'un nombre dénombrable de fois pour converger,
  ce qui signifie qu'un point qui ne sera jamais retiré est un point dont les
  voisinages sont indénombrables.

  Le point de départ de la preuve à proprement parler va à rebours de cette
  ébauche. On appelle un point de condensation de $X$ un point $x$ tel que
  tous les voisinages de $x$ sont indénombrables. On définit alors
  \[P \defeq \{x \in X \mid x\text{ est un point de condensation}\}\]
  On fixe désormais une base dénombrable de $\Omega$, qu'on notera
  $(U_i)_{i\in I}$. Pour tout $i\in I$, si $U_i$ est dénombrable alors tous ses
  éléments sont hors de $P$ (par définition d'être un point de condensation).
  Réciproquement, si un point $x$ n'est pas de condensation, alors on trouve un
  ouvert $U$ dénombrable tel que $x\in U$, mais alors comme $(U_i)$ est une
  base, on peut trouver un ouvert $U_i\subseteq U$ tel que $x\in U_i$, qui est
  nécessairement dénombrable par inclusion~: $x$ appartient à un certain $U_i$
  dénombrable.

  On en déduit donc que
  \[X\setminus P = \bigcup\left\{U_i\mid i \in I, |U_i| \leq \aleph_0\right\}\]
  cet ensemble, qu'on notera désormais $U$, est dénombrable (comme union
  dénombrable d'ensembles dénombrable). Il nous reste donc à montrer que $P$ est
  parfait dans $X$. Soit $x\in P$ et $U$ un ouvert contenant $x$. Cet ouvert
  est indénombrable puisqu'il est un voisinage de $x$~: il a donc une
  intersection indénombrable avec $P$ (puisque $X = U \cup P$ et $U$ est
  dénombrable), donc il contient en particulier d'autres points que $x$ dans
  $P$, donc $x$ est un point d'accumulation.
\end{proof}

\begin{remark}
  La décomposition donnée ici est de plus unique. Si on trouve une autre
  décomposition $X = U' \cup P'$, alors $U = U'$ et $P = P'$. Pour le prouver,
  on peut utiliser le fait que l'ensemble des points de condensation d'un
  espace parfait est lui-même.
\end{remark}

On peut donc en déduire qu'il y a une forme d'hypothèse du continu dans les
espaces polonais.

\begin{corollary}
  Un espace polonais indénombrable est de cardinal $2^{\aleph_0}$.
\end{corollary}

\begin{proof}
  Un espace polonais contient une partie parfaite, qui est non vide si l'espace
  est indénombrable~: on peut donc injecter l'espace de Cantor dans cet espace.
\end{proof}

\subsection{Caractérisation des espaces de Cantor et de Baire}

Nous montrons maintenant que quelques propriétés topologiques suffisent à
caractériser, à homéomorphisme près, les espaces de Cantor et de Baire.

On peut voir que l'espace de Cantor est un espace de Stone. A ce titre, il est
compact et possède une base d'ouverts-fermés. De plus, on a vu qu'il est un
espace polonais parfait. En réalité, ces propriétés suffisent à caractériser
l'espace de Cantor.

Pour faire construire un homéomorphisme entre un espace quelconque vérifiant ces
hypothèses et l'espace de Cantor, on s'attend à devoir construire un schéma de
Cantor. En utilisant la compacité de l'espace (qu'on nommera $X$), on peut
trouver un nombre fini de parties $X_1,\ldots,X_n$ disjointes, qui seront
de taille $2^{-x}$ à mesure qu'on effectue les séparations. Malheureusement,
on ne peut pas dire que $n = 2$, donc notre construction de schéma de Cantor
demande à être adaptée. Pour cela, en considérant un arbre à branchements
finis dont la racine est $X$ et les fils de chaque n\oe ud $X_0$ sont les
parties disjointes de diamètre moitié (par rapport à $X_0$) décomposant $X_0$,
que l'on peut représenter par le schéma ci-dessous~:
\begin{center}
  \begin{tikzpicture}
    \node (A) at (0,0) {$X_0$};
    \node (B) at (2,2.25) {$X_1$};
    \node (C) at (2,0.75) {$X_2$};
    \node (D) at (2,-0.75) {$\vdots$};
    \node (E) at (2,-2.25) {$X_n$};
    \draw[->, >=latex] (A) -- (B);
    \draw[->, >=latex] (A) -- (C);
    \draw[->, >=latex] (A) -- (E);
  \end{tikzpicture}
\end{center}
on effectue la transformation en un arbre binaire, en transformant
la suite de fils de $X_0$ (notons-les $X_1,\ldots,X_n$) en une branche partant
de $X_0$, où les n\oe uds intermédiaires sont les unions partielles des
$X_i\cup\cdots \cup X_n$~:
\begin{center}
  \begin{tikzpicture}
    \node (A) at (0,0) {$X_0$};
    \node (B) at (2,1) {$X_1$};
    \node (C) at (2,-1) {$X_2\cup \cdots \cup X_n$};
    \node (D) at (5,-0.5) {$X_2$};
    \node (E) at (5,-1.5) {$X_3\cup\cdots\cup X_n$};
    \draw[->, >=latex] (A) -- (B);
    \draw[->, >=latex] (A) -- (C);
    \draw[->, >=latex] (C) -- (D);
    \draw[->, >=latex] (C) -- (E);
  \end{tikzpicture}
\end{center}

\begin{theorem}[Caractérisation de l'espace de Cantor]
  Soit $(X,\Omega)$ un espace polonais parfait compact, non vide, possédant une
  base d'ouverts-fermés. Alors $X$ est homéomorphe à l'espace de Cantor.
\end{theorem}

\begin{proof}
  On construit donc notre schéma de Cantor sur $X$, dont on s'assure que
  chaque $X_s$ est un ouvert-fermé de $X$. Supposons construit notre schéma pour
  un mot $s$ de taille $n$ et tel qu'aucun $X_t$ n'est défini, pour $t$ dont
  $s$ est un préfixe. On considère alors $X_s$ et on note $\rho$ son diamètre.
  On sait que $X_s$ est un compact (puisque fermé dans l'espace de Cantor,
  compact). De plus, comme $X_s$ peut être recouvert par des boules de taille
  $\dfrac{\rho}{2}$, et qu'on a une base d'ouverts-fermés, on peut recouvrir
  $X_s$ par des ouverts-fermés de diamètres inférieur à $\dfrac{\rho}{2}$~: par
  compacité, on en déduit donc l'existence d'un recouvrement fini
  $Y_1,\ldots,Y_n$ de $X_s$ par des ouverts-fermés de diamètre inférieur à
  $\dfrac{\rho}{2}$, comme on a un ensemble fini d'ouverts-fermés, on peut
  conserver toutes ces propriétés en considérant
  \[X_i = Y_i \setminus \bigcup_{ j < i} Y_i\]
  pour obtenir une nouvelle famille $(X_i)_{i = 1,\ldots, n}$ d'ouverts-fermés
  disjoints de diamètre suffisamment petit. On ajoute alors la suite des
  parties $X_i$ comme décrit plus tôt. On remarque que, dans cette
  construction~:
  \begin{itemize}
  \item lorsqu'on ajoute une branche pour une famille de $X_i$ donnée, on
    a toujours au moins deux $X_i$, car sinon cela voudrait dire que
    $\rho = \dfrac{\rho}{2}$ et donc que $X_s$, supposé ouvert, est un
    singleton~: c'est impossible dans un espace parfait.
  \item après l'ajout d'une branche, on ne peut donc pas avoir un n\oe ud avec
    une seule branche~: soit on a défini ses deux fils, soit on n'en a
    défini aucun. Cela signifie que sur les n\oe ud non encore définis, on peut
    appliquer notre hypothèse de récurrence.
  \end{itemize}
  On a ainsi construit un schéma de Cantor $\mathcal X$, dont on veut vérifier
  que son diamètre est évanescent. Or, on sait que si $X_s$ est de diamètre
  $\rho$, alors pour un certain $n$ fini et tout mot $u$ de taille $n$,
  $X_{s\star u}$ est de diamètre inférieur à $\rho/2$, donc le diamètre de
  $u\restr n$ tend vers $0$ quand $n$ tend vers $\infty$.

  On a donc construit une injection continue de l'espace de Cantor dans $X$.
  Comme à chaque étape, $X_s$ est partitionné, on sait que l'injection est de
  plus surjective~: si on considère $x \in X$, alors on peut trouver une suite
  $u$ telle que $f(u) = x$. Si $x \in X_0$, alors on définit $u_0 = 0$, sinon
  on définit $u_0 = 1$, et on procède ainsi par induction sur $n$. Si la suite
  n'atteint pas $x$, cela signifie que $x$ est dans un ensemble $X_s$ (car il
  est au moins dans $X_\varepsilon = X$) mais dans aucun élément de la partition
  de $X_s$ en $X_1,\ldots,X_n$, ce qui est absurde. Il nous reste enfin à
  vérifier que l'image d'un ouvert de l'espace de Cantor est ouvert. Cela se
  voit directement, puisque l'image de $\ext(u)$ pour $u\in 2^{<\omega}$ est
  choisie comme un ouvert-fermé.

  On en déduit que $X \cong 2^\mathbb N$.
\end{proof}

Enfin, l'espace de Baire possède aussi une caractérisation.

A FAIRE
