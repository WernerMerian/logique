\chapter{Calcul des prédicats}
\label{chp.logpred}

\minitoc

\lettrine{L}{e} premier but de la logique mathématique est de rendre compte du
langage mathématique. A ce titre, la logique propositionnelle est clairement
insuffisante, et nous l'avons déjà présentée comme une simplification du langage
mathématique habituel. Dans ce chapitre, nous allons nous intéresser à la
formalisation de ce langage mathématique habituel, qui est le calcul des
prédicats de la logique du premier ordre. Le terme prédicat désigne le fait que
nos propositions dépendent de termes, qui représentent des objets mathématiques.
L'expression \og logique du premier ordre\fg{} désigne la capacité d'expression
de nos propositions : celles-ci ne peuvent parler que des objets mathématiques
désignés préalablement par l'univers de discours. Par contraste, la logique du
deuxième ordre permet de parler, en plus de ces objets, des propositions
elles-mêmes : on peut écrire par exemple $\forall P\in\Prop, P\to P$.

Nous nous attarderons d'abord sur la définition, à partir d'une signature du
premier ordre, des termes et des formules, ainsi que les notions syntaxiques de
variables libres, liées et de substitution. Ensuite, nous introduirons les
notions les plus élémentaires de la théorie des modèles, et la relation de
satisfaction $\models$. Enfin, comme dans le chapitre précédent, nous allons
définir une syntaxe pour le calcul des prédicats. La différence, cependant,
est que nous prouverons le théorème de compacité à partir de la complétude.

Ce chapitre peut être vu comme la base commune de la théorie de la démonstration
et de la théorie des modèles. A ce titre, nous donnerons avant tout les
définitions des concepts importants, mais n'allons pas nous attarder sur ceux-ci,
puisque nous les reverrons dans des chapitres dédiés. En particulier nous allons
donner un formalisme pour la syntaxe du calcul des prédicats et un seul, alors
que la partie dédiée à la théorie de la démonstration donnera un résultat plus
général sur toute une famille de systèmes de preuves, et présentera plusieurs
formalismes.

\section{Signatures, termes et formules}

Reprenons l'exemple que nous avions donné au début du \cref{chp.logprop} :
\[\forall n \in \mathbb N, (\exists m \in \mathbb N, n = 2\times m) \lor
(\exists m \in \mathbb N, n = 2 \times m + 1)\]
Remarquons tout d'abord que l'on remplace \og ou\fg{} par le symboles $\lor$,
maintenant que nous connaissons le formalisme de la logique propositionnelle.
Il nous reste cependant plusieurs points à définir formellement : tout d'abord,
la phrase précédente contient des termes, comme $2$ ou $n$. Ceux-ci sont d'une
nature différente d'une variable propositionnelle par exemple, puisque dans le
premier cas, les formules ne relient pas directement des termes, mais des
relations entre termes. Nous devons donc tout d'abord construire un ensemble de
termes, qui représenterons les objets dont les formules parleront. Cependant,
comme nous cherchons en premier lieu à élaborer des phrases finies, nous
cherchons aussi à limiter les symboles que nous utiliserons. Cela s'explique par
le fait que pour lire une phrase, il est nécessaire de savoir à l'avance quels
sont les symboles constitutifs de ce langage. En particulier, nous devons savoir
ce que signifie chaque symbole.

\begin{definition}[Signature]
  Une signature, ou langage, du premier ordre, est un quadruplet $\mathcal L =
  (\mathcal F,\mathcal R, \alpha,\beta)$ où $\alpha : \mathcal F \to \mathbb N$
  et $\beta : \mathcal R \to \mathbb N$. On appelle les éléments de $\mathcal F$
  les symboles de fonction et les éléments de $\mathcal R$ les symboles de
  relation. Pour un symboles de fonction $f\in\mathcal F$, $\alpha(f)$ est appelé
  l'arité de $f$, et de même $\beta(r)$ est l'arité de $r$ pour $r\in\mathcal R$.
  Si $f\in\mathcal F$ est d'arité $0$, on dit que c'est une constante.
\end{definition}

\begin{example}
  Un premier exemple de langage est le langage des groupes, qui est
  \[\mathcal L_{\mathrm{Grp}} \defeq \{e^0,\times^2,((-)^{-1})^1\}\]
  où l'on indique par un exposant l'arité d'un symbole, et où tous les symboles
  sont des symboles de fonction. De même, comme on préfère la notation additive
  pour les groupes abéliens, on peut aussi définir
  \[\mathcal L_{\mathrm{Ab}} \defeq \{0^0,+^2,-^1\}\]
\end{example}

\begin{example}
  Un autre exemple est le langage des anneaux :
  \[\mathcal L_{\mathrm{Ring}}\defeq \{0^0,1^0,+^2,\times^2,-^1\}\]
\end{example}

