\chapter{Calcul des prédicats}
\label{chp.logpred}

\minitoc

\lettrine{L}{e} premier but de la logique mathématique est de rendre compte du
langage mathématique. A ce titre, la logique propositionnelle est clairement
insuffisante, et nous l'avons déjà présentée comme une simplification du langage
mathématique habituel. Dans ce chapitre, nous allons nous intéresser à la
formalisation de ce langage mathématique habituel, qui est le calcul des
prédicats de la logique du premier ordre. Le terme prédicat désigne le fait que
nos propositions dépendent de termes, qui représentent des objets mathématiques.
L'expression \og logique du premier ordre\fg{} désigne la capacité d'expression
de nos propositions : celles-ci ne peuvent parler que des objets mathématiques
désignés préalablement par l'univers de discours. Par contraste, la logique du
deuxième ordre permet de parler, en plus de ces objets, des propositions
elles-mêmes : on peut écrire par exemple $\forall P\in\Prop, P\to P$.

Nous nous attarderons d'abord sur la définition, à partir d'une signature du
premier ordre, des termes et des formules, ainsi que les notions syntaxiques de
variables libres, liées et de substitution. Ensuite, nous introduirons les
notions les plus élémentaires de la théorie des modèles, et la relation de
satisfaction $\models$. Enfin, comme dans le chapitre précédent, nous allons
définir une syntaxe pour le calcul des prédicats. La différence, cependant,
est que nous prouverons le théorème de compacité à partir de la complétude.

Ce chapitre peut être vu comme la base commune de la théorie de la démonstration
et de la théorie des modèles. A ce titre, nous donnerons avant tout les
définitions des concepts importants, mais n'allons pas nous attarder sur ceux-ci,
puisque nous les reverrons dans des chapitres dédiés. En particulier nous allons
donner un formalisme pour la syntaxe du calcul des prédicats et un seul, alors
que la partie dédiée à la théorie de la démonstration donnera un résultat plus
général sur toute une famille de systèmes de preuves, et présentera plusieurs
formalismes.

On se fixe pour tout ce chapitre un ensemble $\Var$ dénombrable de variables.

\section{Signatures, termes et formules}

\subsection{Définition d'une signature}

Reprenons l'exemple que nous avions donné au début du \cref{chp.logprop} :
\[\forall n \in \mathbb N, (\exists m \in \mathbb N, n = 2\times m) \lor
(\exists m \in \mathbb N, n = 2 \times m + 1)\]
Remarquons tout d'abord que l'on remplace \og ou\fg{} par le symboles $\lor$,
maintenant que nous connaissons le formalisme de la logique propositionnelle.
Il nous reste cependant plusieurs points à définir formellement : tout d'abord,
la phrase précédente contient des termes, comme $2$ ou $n$. Ceux-ci sont d'une
nature différente d'une variable propositionnelle par exemple, puisque dans le
premier cas, les formules ne relient pas directement des termes, mais des
relations entre termes. Nous devons donc tout d'abord construire un ensemble de
termes, qui représenterons les objets dont les formules parleront. Cependant,
comme nous cherchons en premier lieu à élaborer des phrases finies, nous
cherchons aussi à limiter les symboles que nous utiliserons. Cela s'explique par
le fait que pour lire une phrase, il est nécessaire de savoir à l'avance quels
sont les symboles constitutifs de ce langage. En particulier, nous devons savoir
ce que signifie chaque symbole.

\begin{definition}[Signature]
  Une signature, ou langage, du premier ordre, est un quadruplet $\mathcal L =
  (\mathcal F,\mathcal R, \alpha,\beta)$ où $\alpha : \mathcal F \to \mathbb N$
  et $\beta : \mathcal R \to \mathbb N$. On appelle les éléments de $\mathcal F$
  les symboles de fonction et les éléments de $\mathcal R$ les symboles de
  relation. Pour un symboles de fonction $f\in\mathcal F$, $\alpha(f)$ est
  appelée l'arité de $f$, et de même $\beta(r)$ est l'arité de $r$ pour
  $r\in\mathcal R$. Si $f\in\mathcal F$ est d'arité $0$, on dit que c'est une
  constante.
\end{definition}

\begin{example}
  Un premier exemple de langage est le langage des groupes, qui est
  \[\mathcal L_{\mathrm{Grp}} \defeq \{e^0,\times^2,((-)^{-1})^1\}\]
  où l'on indique par un exposant l'arité d'un symbole, et où tous les symboles
  sont des symboles de fonction. De même, comme on préfère la notation additive
  pour les groupes abéliens, on peut aussi définir
  \[\mathcal L_{\mathrm{Ab}} \defeq \{0^0,+^2,-^1\}\]
\end{example}

\begin{example}
  Un autre exemple est le langage des anneaux :
  \[\mathcal L_{\mathrm{Ring}}\defeq \{0^0,1^0,+^2,\times^2,-^1\}\]
\end{example}

\begin{example}
  Un autre exemple classique de langage est celui de l'arithmétique :
  \[\mathcal L_{\mathrm{Arith}}\defeq \{0^0, S^1, +^2,\times^2,\leq^2\}\]
  où $\leq$ est un symbole de relation, et les autres symboles sont des symboles
  de fonction.
\end{example}

L'exemple du langage de l'arithmétique permet de voir ce que nous entendons par
termes : avec ce langage, nous avons envie de pouvoir écrire $0$ (qui est une
constante) mais aussi $1$ défini par $S\;0$ ou $S\;S\;0$. De plus, il doit être
possible d'écrire $(S\;S\;0) + (S\;0)$ par exemple : l'écriture est donc
naturellement donnée comme un ensemble inductif, où les arités des symboles de
fonction nous donnent les arités des constructeurs de l'ensemble.

\subsection{Termes et formules}

\begin{definition}[Termes]
  Soit une signature
  $\Sigma = (\mathcal F_\Sigma,\mathcal R_\Sigma, \alpha_\Sigma,\beta_\Sigma)$,
  on définit $\Term(\Sigma)$ comme l'ensemble inductif engendré par
  $\mathcal F_\Sigma\cup\Var$ où l'arité de $f\in\mathcal F_\Sigma$ est
  $\alpha_\Sigma(f)$ et où l'arité de $x\in \Var$ est $0$. On peut représenter
  cet ensemble par la grammaire suivante :
  \[t,u ::= x \mid f(t_1,\ldots,t_{\alpha(f)})\]
  où $x\in \Var$ et $f\in \mathcal F_\Sigma$.
\end{definition}

Ainsi, les termes écrits dans notre langage vont représenter les objets
mathématiques sur lesquels porteront nos formules. Ces formules sont définies
par induction, d'une façon analogue aux propositions de la logique
propositionnelle. En l'absence de variables propositionnelles, les éléments
atomiques des formules seront construits à partir des termes.

\begin{definition}[Proposition atomique]
  Soit une signature $\Sigma$. On définit l'ensemble $\Atom(\Sigma)$ des
  propositions atomiques par
  \begin{multline*}
    \Atom(\Sigma) \defeq \{(r,t_1,\ldots,t_k)\mid r\in\mathcal R_\Sigma,
    (t_1,\ldots,t_k) \in (\Term(\Sigma))^k, k = \beta_\Sigma(r)\}\\
    \cup\{(``=",t,u)\mid t,u\in \Term(\Sigma)\}
  \end{multline*}
  où $=$ est un symbole n'appartenant pas à $\mathcal R_\Sigma$.
\end{definition}

\begin{remark}
  L'égalité est une relation, mais celle-ci n'appartient pas formellement au
  langage, car son comportement est donné par les règles logiques, de la même
  façon que $\lor$ et $\land$ ont leur sens imposés. Certains auteurs
  considèrent au contraire que $=$ doit être ajouté au langage, en tant que
  symbole de relation binaire, et d'autres font la différence entre un langage
  égalitaire (incluant le symbole $=$) et un langage non égalitaire. Notre choix
  est motivé à la fois par la simplicité et par l'expressivité : l'égalité est
  clairement utile pour formaliser les mathématiques et raisonner dessus, mais
  chercher à préciser quand nous l'utilisons ne l'est pas, étant donné qu'elle
  sera toujours présente.
\end{remark}

Nous pouvons maintenant définir l'ensemble des formules sur une signature
donnée.

\begin{definition}[Formules]
  Soit une signature $\Sigma$. On définit l'ensemble $\Formula(\Sigma)$ par
  la grammaire suivante :
  \[\varphi,\psi ::= a\mid \top\mid\bot\mid\lnot\varphi\mid\varphi\lor\psi\mid
  \varphi\land\psi\mid \varphi\to\psi\mid\forall x,\varphi\mid
  \exists x,\varphi\]
  où $a\in \Atom(\Sigma)$ et $x\in \Var$.
\end{definition}

\subsection{Variables et substitution}

Maintenant que les formules sont définies, nous voulons définir les opérations
basiques sur celles-ci. Tout d'abord, nous devons introduire les notions
élémentaires liées aux variables.

\begin{definition}[Variable libre, formule close]
  On définit la fonction qui étant donné un terme (respectivement une formule),
  retourne l'ensemble des variables libres y apparaissant. La fonction $\VL$ est
  définie par induction sur $\Term(\Sigma)$ (respectivement $\Formula(\Sigma)$)
  par les équations suivantes :
  \begin{itemize}
  \item si $t = x\in Var$, alors $\VL(t) = \{x\}$.
  \item si $t = f(t_1,\ldots,t_n)$ où $f\in \mathcal F_\Sigma$,
    $t_1,\ldots,t_n\in\Term(\Sigma)$, alors
    $\displaystyle\VL(t) =\bigcup_{k = 1}^n \VL(t_k)$.
  \item si $\varphi = r(t_1,\ldots,t_n)$ est une proposition atomique où
    $r\in\mathcal R_\Sigma, t_1,\ldots,t_n\in\Term(\Sigma)$, alors
    $\displaystyle\VL(\varphi) = \bigcup_{k = 1}^n \VL(t_k)$.
  \item si $\varphi = \top$, alors $\VL(\varphi) = \varnothing$.
  \item si $\varphi = \bot$, alors $\VL(\varphi) = \varnothing$.
  \item si $\varphi = \lnot \psi$, alors $\VL(\varphi) = \VL(\psi)$.
  \item si $\varphi = \psi\lor\chi$, $\varphi = \psi\land \chi$ ou
    $\varphi = \psi\to\chi$, alors $\VL(\varphi) = \VL(\psi)\cup\VL(\chi)$.
  \item si $\varphi = \forall x, \psi$ ou $\varphi = \exists x, \psi$, alors
    $\VL(\varphi) = \VL(\psi)\backslash\{x\}$.
  \end{itemize}
  On dit que $\varphi$ est une formule close si $\VL(\varphi) =\varnothing$.
  On note par $\Clos(\Sigma)$ l'ensemble des formules closes sur la signature
  $\Sigma$. De même, un terme $t$ est appelé un terme clos s'il n'a pas de
  variables libres (c'est-à-dire s'il n'a pas de variable).
\end{definition}

\begin{remark}\label{rmk.alpha}
  Une variable non libre est dite liée : les variables liées sont muettes, elles
  n'ont pas d'importance en elle-même et seulement sur le quantificateur qui les
  lie. On considère implicitement que si $\varphi$ et $\psi$ diffèrent seulement
  en remplaçant la variable liée par un quantificateur et les variables que ce
  quantificateur lie, alors $\varphi = \psi$. Par exemple, $\forall x, x = x$ et
  $\forall y, y = y$ sont identifiées.
\end{remark}

Comme pour la logique propositionnelle, la valeur de vérité d'une formule va
dépendre de la valeur associée aux variables libres. Cependant, pour l'instant,
nous n'avons pas de système clair d'évaluation : nous verrons comment évaluer
une formule quand nous aborderons la notion de modèle. Au niveau syntaxique,
cependant, nous pouvons déjà introduire la substitution, que l'on peut voir
comme une évaluation syntaxique : on remplace une variable libre par un terme.

\begin{definition}[Substitution]
  Soient une signature $\Sigma$, un terme $t\in \Term(\Sigma)$ et une variable
  $x\in \Var$. On définit les deux fonctions
  \[\begin{array}{rcccc}
  -[t/x] &: & \Term(\Sigma) & \longrightarrow & \Term(\Sigma)\\
  & & u & \longmapsto & u[t/x] \\
  \\
  -[t/x] &: & \Formula(\Sigma) & \longrightarrow & \Formula(\Sigma)\\
  & & \varphi & \longmapsto & \varphi[t/x]
  \end{array}\]
  par induction sur la structure de $\Term(\Sigma)$ (respectivement sur la
  structure de $\Formula(\Sigma)$) :
  \begin{itemize}
  \item si $u = x$, alors $u[t/x] = t$.
  \item si $u = y\in Var$ avec $y \neq x$, alors $u[t/x] = y$.
  \item si $u = f(u_1,\ldots,u_n)$, alors
    $u[t/x] = f(u_1[t/x],\ldots,u_n[t/x])$.
  \item si $\varphi = r(u_1,\ldots,u_n)$, alors
    $\varphi[t/x] = r(u_1[t/x],\ldots,u_n[t/x])$.
  \item si $\varphi = \top$ alors $\varphi[t/x] = \top$.
  \item si $\varphi = \bot$ alors $\varphi[t/x] = \bot$.
  \item si $\varphi = \lnot \psi$ alors $\varphi[t/x] = \lnot \psi[t/x]$.
  \item si $\varphi = \psi \lor \chi$ alors
    $\varphi[t/x] = \psi[t/x]\lor\chi[t/x]$.
  \item si $\varphi = \psi \land \chi$ alors
    $\varphi[t/x] = \psi[t/x]\land\chi[t/x]$.
  \item si $\varphi = \psi \to \chi$ alors
    $\varphi[t/x] = \psi[t/x]\to\chi[t/x]$.
  \item si $\varphi = \forall z, \psi$ où $z\notin \VL(t)$, alors
    $\varphi[t/x] = \forall z, \psi[t/x]$.
  \item si $\varphi = \exists z, \psi$ où $z\notin \VL(t)$, alors
    $\varphi[t/x] = \exists z, \psi[t/x]$.
  \end{itemize}
\end{definition}

\begin{remark}
  La condition de $z\notin\VL(t)$ dans les derniers cas peut toujours être
  réalisée quitte à renommer la variable liée $z$ : puisque $\Var$ est infini
  et que $\VL(t)$ est fini, on peut toujours trouver $a\notin\VL(t)$ et
  remplacer $\forall z, \psi$ par $\forall a, \psi[a/z]$ en utilisant
  l'identification de la \cref{rmk.alpha}.

  Si l'on veut être parfaitement formel, il conviendrait de procéder dans
  l'autre sens : on définit d'abord la substitution comme donnée précédemment,
  puis on définit la relation $\varphi \equiv \psi$ engendrée par
  $\forall x, \psi \equiv \forall y, \psi[y/x]$ et
  $\exists x,\psi \equiv\exists y, \psi[y/x]$ dont on prouve qu'elle est une
  relation d'équivalence, puis on définit le \og vrai\fg{} ensemble
  $\Formula(\Sigma)$ par $\Formula(\Sigma)/ \equiv$ (cela n'est pas nécessaire
  pour $\Term(\Sigma)$ puisque toute variable est libre, dans un terme), et que
  la fonction $-[t/x]$ est bien définie sur ce quotient. Ce processus est
  évidemment plus laborieux et n'apporte rien à la compréhension, c'est pourquoi
  nous ne le détaillons pas ici.
\end{remark}

\begin{exercise}
  Soit $x\in \Var$, $t\in\Term(\Sigma)$ et $\varphi\in\Clos(\Sigma)$ pour une
  signature $\Sigma$ quelconque. Montrer que $\varphi[t/x] = \varphi$.
\end{exercise}

\begin{exercise}
  Soient $t,u,v\in\Term(\Sigma)$ et $x,y\in\Var$, montrer que
  \[t[u/x][v/y] = (t[v/y])[u[v/y]/x]\]
\end{exercise}

\section{Bases de théorie des modèles}

Maintenant que nous avons défini la syntaxe élémentaire, nous allons lui donner
un sens : une sémantique. Dans le cas de la logique propositionnelle, le sens
d'une proposition était simple à définir, puisqu'il s'agissait d'une valeur de
vérité en fonction des variables propositionnelles. Dans le cas de la logique du
premier ordre, les propositions parlent d'objets, et il faut donc fixer un
univers ambiant sur lequel porte le discours donné par les formules. Par
exemple, en prenant le langage de l'arithmétique, le terme $S\;S\;0 + S\;S\;0$
et en considérant $0$ comme l'entier naturel $0$, $S$ comme la fonction
$n \mapsto n + 1$ et $+$ comme l'addition usuelle sur les entiers, le terme
devient le terme $4$, mais on peut imaginer une autre interprétation de ce terme
donnant par exemple $5$ ou tout autre nombre.

\subsection{Structure et interprétation}

Nous travaillons donc sur l'interprétation d'une formule en deux parties : tout
d'abord, nous introduisons la notion de structure, qui offre une interprétation
claire du langage dans lequel la formule est écrite, et c'est seulement à partir
de cette interprétation que l'on peut évaluer une formule. Cela modifie notre
notion de vérité : les formules closes prennent une plus grande importance que
le reste des formules, mais il faut quantifier sur des structures en
contrepartie.

\begin{definition}[Structure]
  Soit une signature $\Sigma$. On appelle $\Sigma$-structure (ou simplement
  structure) $\mathcal M$ un triplet
  $(|\mathcal M|,-^{\mathcal M}_\mathcal F,-^{\mathcal M}_\mathcal R)$
  (on notera les deux $-^{\mathcal M}$, sans indice) où :
  \begin{itemize}
  \item $|\mathcal M|$ est un ensemble.
  \item pour chaque $f\in \mathcal F_\Sigma$ d'arité $n$,
    $f^{\mathcal M} : |\mathcal M|^n \to |\mathcal M|$.
  \item pour chaque $r\in\mathcal R_\Sigma$ d'arité $n$,
    $r^{\mathcal M} \subseteq |\mathcal M|^n$.
  \end{itemize}

  On identifie $|\mathcal M|^0 \to |\mathcal M|$ à $|\mathcal M|$ : un symbole
  de constante est associé directement à un élément.
\end{definition}

\begin{example}
  En reprenant les différents langages définis précédemments, on peut voir que,
  par exemple, $(\mathbb Z,0,+,-)$ est une structure sur le langage des groupes.
  C'est même, en incluant $1$ et $\times$, une structure sur le langage des
  anneaux. De même, $(\mathbb N,0,n\mapsto n + 1,+,\times,\leq)$ est une
  structure sur le langage de l'arithmétique.
\end{example}

Avec ces nouveaux exemples, on voit qu'il devient naturel d'interpréter dans la
structure $(\mathbb N,0,n\mapsto n+1,+,\times,\leq)$ le terme
$S\;S\;0 + S\;S\;0$
par l'élément $4\in\mathbb N$. Nous pouvons donc généraliser ce résultat. Pour
cela, on définit d'abord la notion d'environnement, puis de valuation étant
donné un environnement.

\begin{definition}[Environnement]
  Soit une signature $\Sigma$ et une structure $\mathcal M$. Un environnement
  $\rho$ est une fonction partielle $\rho : \Var\partialto |\mathcal M|$. On
  note $\mathcal E$ l'ensemble des environnements. Etant donnés un élément
  $m\in|\mathcal M$, une variable $x$ et un environnement $\rho$, on note
  $\rho[x \mapsto m]$ l'environnement coïncidant avec $\rho$ sur
  $\Var\backslash\{x\}$ et valant $m$ en $x$.
\end{definition}

\begin{definition}[Interprétation, valuation]
  Soit une signature $\Sigma$, une structure $\mathcal M$ et un environnement
  $\rho$. On définit par induction sur $t$ (respectivement $\varphi$), où
  $\VL(t)\subseteq\dom(\rho)$ (respectivement $\VL(\varphi)\subseteq\dom(\rho)$)
  les fonctions suivantes :
  \[\begin{array}{rcccc}
  -_\rho^\mathcal M & : & \Term(\Sigma) & \longrightarrow & |\mathcal M|\\
  & & t & \longmapsto & t^{\mathcal M}_\rho\\
  \\
  \Val_\rho & : & \Formula(\Sigma) & \longrightarrow & \{0,1\}\\
  & & \varphi & \longmapsto & \Val_\rho(\varphi)
  \end{array}\]

  \begin{itemize}
  \item si $t = x \in \Var$, alors $t^\mathcal M_\rho = \rho(x)$.
  \item si $t = f(t_1,\ldots,t_n)$, alors
    $t_\rho^\mathcal M =
    f^\mathcal M((t_1)^\mathcal M_\rho,\ldots,(t_n)^\mathcal M_\rho)$.
  \item si $\varphi = r(t_1,\ldots,t_n)$ alors
    $\Val_\rho(\varphi) =
    \chi_{r^\mathcal M}((t_1)^\mathcal M_\rho,\ldots,(t_n)^\mathcal M_\rho)$.
    On interprète le symbole $=$ par la partie
    $\{(m,m)\mid m\in |\mathcal M|\}$.
  \item si $\varphi = \top$, alors $\Val_\rho(\varphi) = 1$.
  \item si $\varphi = \bot$, alors $\Val_\rho(\varphi) = 0$.
  \item si $\varphi = \lnot \psi$, alors
    $\Val_\rho(\varphi) = 1 - \Val_\rho(\psi)$.
  \item si $\varphi = \psi \lor \chi$, alors
    $\Val_\rho(\varphi) = \max(\Val_\rho(\psi),\Val_\rho(\chi))$
  \item si $\varphi = \psi \land \chi$, alors
    $\Val_\rho(\varphi) = \min(\Val_\rho(\psi),\Val_\rho(\chi))$
  \item si $\varphi = \psi \to \chi$, alors
    $\Val_\rho(\varphi) = \max(1 - \Val_\rho(\psi),\Val_\rho(\chi))$
  \item si $\varphi = \forall x, \psi$, alors
    $\displaystyle\Val_\rho(\varphi) =
    \min_{m \in |\mathcal M|}(\Val_{\rho[x\mapsto m]}(\psi))$
  \item si $\varphi = \exists x, \psi$, alors
    $\displaystyle\Val_\rho(\varphi) =
    \max_{m \in |\mathcal M|}(\Val_{\rho[x\mapsto m]}(\psi))$
  \end{itemize}
  
  Si $t$ est un terme clos, alors $t^{\mathcal M}$ est un élément de
  $|\mathcal M|$. Si $\varphi$ est une formule close, alors $\Val(\varphi)$ est
  un élément de $\{0,1\}$. Ces deux éléments sont obtenus en interprétant le
  terme (respectivement la formule) dans le contexte vide, ou de façon
  équivalente dans n'importe quel contexte.
\end{definition}

Enfin, donnons un résultat statuant que la substitution et l'interprétation
commutent.

\begin{proposition}\label{prop.comm.subst.env}
  Soit une signature $\Sigma$, une structure $\mathcal M$, un environnement
  $\rho$, une variable $x$, deux termes $t,u$ et une formule $\varphi$. Alors on
  a les deux égalités suivantes :
  \[(t[u/x])_\rho^\mathcal M = t^\mathcal M_{\rho[x \mapsto u^\mathcal M_\rho]} \qquad
  \Val_\rho(\varphi[u/x]) = \Val_{\rho[x\mapsto u^\mathcal M_\rho]}(\varphi)\]
\end{proposition}

\begin{proof}
  On montre la première égalité par induction sur $t$, et l'autre par induction
  sur $\varphi$ :
  \begin{itemize}
  \item si $t = x$ alors $(t[u/x])_\rho^\mathcal M = u^\mathcal M_\rho$ d'où
    l'égalité.
  \item si $t = y$ où $y\in\Var$ et $y\neq x$, alors
    $(t[u/x])_\rho^\mathcal M = \bot$ si $y\notin\dom(\rho)$, ce qui est aussi
    la valeur de $y^\mathcal M_{\rho[x \mapsto u^\mathcal M_\rho]}$, ou bien
    $\rho(y)$, qui est là encore la même valeur pour l'autre partie de
    l'équation.
  \item si $t = f(t_1,\ldots,t_n)$ où pour tout $i\in\{1,\ldots,n\}$,
    $(t_i[u/x])^\mathcal M_\rho = (t_i)^\mathcal M_{\rho[x\mapsto u^\mathcal M_\rho]}$, alors
    \begin{align*}
      (t[u/x])^\mathcal M_\rho &= f((t_1[u/x])^\mathcal M_\rho,\ldots,
      (t_n[u/x])^\mathcal M_\rho)\\
      &= f((t_1)^\mathcal M_{\rho[x\mapsto u^\mathcal M_\rho]},\ldots,
      (t_n)^\mathcal M_{\rho[x\mapsto u^\mathcal M_\rho]})\\
      &= t^\mathcal M_{\rho[x\mapsto u^\mathcal M_\rho]}
    \end{align*}
  \item si $\varphi = \top$ ou $\varphi = \bot$, l'égalité est directe.
  \item si $\varphi = r(t_1,\ldots,t_n)$ est une proposition atomique, alors
    \begin{align*}
      \Val_\rho(r(t_1,\ldots,t_n)[u/x])
      &= \Val_\rho(r(t_1[u/x],\ldots,t_n[u/x]))\\
      &= \chi_r((t_1[u/x])^\mathcal M_\rho,\ldots,(t_n[u/x])^\mathcal M_\rho)\\
      &= \chi_r((t_1)^\mathcal M_{\rho[x\mapsto u^\mathcal M_\rho]},\ldots,
      (t_n)^\mathcal M_{\rho[x\mapsto u^\mathcal M_\rho]})\\
      &= \Val_{\rho[x\mapsto u^\mathcal M_\rho]}(r(t_1,\ldots,t_n))
    \end{align*}
  \item si $\varphi = \lnot \psi$, alors
    \begin{align*}
      \Val_\rho((\lnot \psi)[u/x]) &= \Val_\rho(\lnot (\psi[u/x]))\\
      &= 1 - \Val_\rho(\psi[u/x])\\
      &= 1 - \Val_{\rho[x\mapsto u^\mathcal M_\rho]}(\psi)\\
      &= \Val_{\rho[x\mapsto u^\mathcal M_\rho]}(\lnot\psi)\\
    \end{align*}
  \item si $\varphi = \psi \lor \chi$, alors
    \begin{align*}
      \Val_\rho((\psi \lor \chi)[u/x]) &= \Val_\rho(\psi[u/x] \lor \chi[u/x])\\
      &= \max (\Val_\rho(\psi[u/x]),\Val_\rho(\chi[u/x]))\\
      &= \max (\Val_{\rho[x\mapsto u^\mathcal M_\rho]}(\psi),
      \Val_{\rho[x \mapsto u^\mathcal M_\rho]}(\chi))\\
      &= \Val_{\rho[x\mapsto u^\mathcal M_\rho]}(\psi\lor \chi)
    \end{align*}
  \item si $\varphi = \psi \land \chi$, alors
    \begin{align*}
      \Val_\rho((\psi \land \chi)[u/x]) &= \Val_\rho(\psi[u/x] \land \chi[u/x])\\
      &= \min (\Val_\rho(\psi[u/x]),\Val_\rho(\chi[u/x]))\\
      &= \min (\Val_{\rho[x\mapsto u^\mathcal M_\rho]}(\psi),
      \Val_{\rho[x \mapsto u^\mathcal M_\rho]}(\chi))\\
      &= \Val_{\rho[x\mapsto u^\mathcal M_\rho]}(\psi\land \chi)
    \end{align*}
  \item si $\varphi = \forall y, \psi$ (sans perte de généralité,
    $y\notin \VL(u)$), alors
    \begin{align*}
      \Val_\rho((\forall y,\psi)[u/x]) &= \Val_\rho(\forall y,\psi[u/x])\\
      &= \min_{m\in|\mathcal M|}(\Val_{\rho[y \mapsto m]}(\psi[u/x]))\\
      &= \min_{m\in|\mathcal M|}(\Val_{\rho[y\mapsto m]
        [x\mapsto u^\mathcal M_{\rho[y\mapsto m]}]}(\psi))\\
      &= \min_{m\in|\mathcal M|}(\Val_{\rho[y\mapsto m][x\mapsto u^\mathcal M_\rho]}(\psi))\\
      &= \min_{m\in|\mathcal M|}(\Val_{\rho[x\mapsto u^\mathcal M_\rho][y\mapsto m]}(\psi))\\
      &= \Val_{\rho[x\mapsto u^\mathcal M_\rho]}(\forall y, \psi)
    \end{align*}
  \item si $\varphi = \exists y, \psi$ (sans perte de généralité,
    $y\notin \VL(u)$), alors
    \begin{align*}
      \Val_\rho((\exists y,\psi)[u/x]) &= \Val_\rho(\exists y,\psi[u/x])\\
      &= \max_{m\in|\mathcal M|}(\Val_{\rho[y \mapsto m]}(\psi[u/x]))\\
      &= \max_{m\in|\mathcal M|}(\Val_{\rho[y\mapsto m]
        [x\mapsto u^\mathcal M_{\rho[y\mapsto m]}]}(\psi))\\
      &= \max_{m\in|\mathcal M|}(\Val_{\rho[y\mapsto m][x\mapsto u^\mathcal M_\rho]}(\psi))\\
      &= \max_{m\in|\mathcal M|}(\Val_{\rho[x\mapsto u^\mathcal M_\rho][y\mapsto m]}(\psi))\\
      &= \Val_{\rho[x\mapsto u^\mathcal M_\rho]}(\exists y, \psi)
    \end{align*}
  \end{itemize}
  D'où le résultat par induction.
\end{proof}

\subsection{Satisfaction, modèle}

La notion de valuation permet de définir la relation de satisfaction, $\models$,
d'une façon analogue à ce que nous avons fait pour la logique propositionnelle.

\begin{definition}[Satisfaction]
  Soit une signature $\Sigma$, une structure $\mathcal M$ et, une formule
  $\varphi$ et un environnement $\rho$ tel que
  $\VL(\varphi)\subseteq\dom(\rho)$.
  On définit $\mathcal M,\rho\models \varphi$ par
  \[\mathcal M,\rho\models \varphi \defeq \Val_\rho(\varphi) = 1\]

  Soit un ensemble $\mathcal F\subseteq\Formula(\Sigma)$ et un environnement
  $\rho$ tel que
  $\forall \varphi\in\mathcal F, \VL(\varphi)\subseteq\dom(\rho)$.
  On dit que $\mathcal M,\rho$ satisfont $\mathcal F$, ce que l'on écrit
  $\mathcal M,\rho\models\mathcal F$, lorsque pour toute formule
  $\varphi\in\mathcal F$, il est vrai que $\mathcal M,\rho\models \varphi$.

  Dans le cas de formules closes, on écrira directement
  $\mathcal M\models\varphi$ et $\mathcal M\models \mathcal F$.
\end{definition}

Cela permet alors de définir ce qu'est un modèle.

\begin{definition}[Modèle]
  Soit une signature $\Sigma$ et $\mathcal C\subseteq\Clos(\Sigma)$. On dit que
  $\mathcal M$ est un modèle de $\mathcal C$ si $\mathcal M\models \mathcal C$.
\end{definition}

Un modèle ne se définit qu'avec un ensemble de formules closes. On pourrait
imaginer une définition analogue avec une formule non close, mais faire cela
signifie qu'au lieu de donner une structure, il faudrait donner une structure et
un environnement en même temps. Le but des modèles est plutôt, ici, de
construire une classe particulière de structure vérifiant certaines conditions
que l'on peut exprimer au premier ordre.

Un exemple simple est la formule
\[\forall x,\forall y, x + y = y + x\]
Les modèles de cette formule, sur le langage $\{+^2\}$, sont les magmas
commutatifs (ensembles munis d'une loi de composition interne commutative) :
l'intérêt ici est de pouvoir décrire parmi tous les magmas possibles ceux qui
ont une lci commutative, et donc de le faire en quelque sorte uniformément parmi
les structures, ce qui n'est pas le cas si les formules n'étaient pas closes.

\subsection{Théories et conséquence logique}

Cela mène naturellement à deux notions connexes : celle de théorie, et celle
de conséquence logique. Une théorie permet de décrire des classes de modèles, et
la conséquence logique permet de créer des liens entre les formules, de la même
façon que nous avions $\vDash$ pour le calcul propositionnel.

\begin{definition}[Théorie]
  Une théorie axiomatique, ou simplement théorie, sur une signature $\Sigma$,
  est une partie $\mathcal T\subseteq\Clos(\Sigma)$.
\end{definition}

\begin{definition}[Conséquence logique, équivalence]
  Soit un ensemble $\mathcal F \subseteq \Formula(\Sigma)$ et une formule
  $F\in\Formula(\Sigma)$. On dit que $F$ est conséquence logique de
  $\mathcal F$, ce que l'on écrit $\mathcal F \vDash F$, lorsque pour toute
  structure $\mathcal M$, si $\mathcal M\models \mathcal F$ alors
  $\mathcal M\models F$. Si deux formules $F$ et $G$ sont telles que
  $F\vDash G$ et $G\vDash F$, alors $F$ et $G$ sont dites logiquement
  équivalentes, ce que l'on note $F\equiv G$.
\end{definition}

On peut relier ces deux notions par celle de théorie saturée.

\begin{definition}[Théorie saturée, clôture par conséquence]
  Soit une théorie $\mathcal T$ sur une signature $\Sigma$. On dit que
  $\mathcal T$ est saturée si pour toute formule $A\in\Formula(\Sigma)$,
  si $\mathcal T\vDash A$ alors $A\in \mathcal T$.

  Pour une théorie $\mathcal T$, on définit sa clôture par conséquence, notée
  $\vclose{\mathcal T}$, par
  \[\vclose{\mathcal T} \defeq \{ A \in \Formula(\Sigma)
  \mid \mathcal T \vDash A\}\]
\end{definition}

De plus, la conséquence logique permet directement de décrire une théorie
\og fausse\fg{} : une telle théorie est une théorie dans laquelle la proposition
fausse est considérée comme vraie. Au niveau des modèles, cela se traduit par le
fait qu'il n'existe pas de modèle de la théorie, puisqu'un tel modèle
associerait automatiquement à $\bot$ la valeur de vérité $0$.

\begin{proposition}
  Soit une signature $\Sigma$ et une théorie $\mathcal T$ sur $\Sigma$. Alors
  $\mathcal T$ admet un modèle si et seulement si $\mathcal T \not\vDash \bot$.
\end{proposition}

\begin{proof}
  Supposons que $\mathcal T$ admette un modèle $\mathcal M$. Alors par
  définition de $\Val$, $\Val(\bot) = 0$ donc $\mathcal M\not\models \bot$. On
  en déduit que $\mathcal T\not\vDash\bot$.

  Dans le sens réciproque et par contraposée, supposons que $\mathcal T$ n'admet
  pas de modèle. Alors pour tout $\mathcal M$ tel que
  $\mathcal M\models \mathcal T$, $\mathcal M\models \bot$, par vacuité de la
  condition : on en déduit donc que $\mathcal T \vDash \bot$, donc que si
  $\mathcal T \not\vDash\bot$, alors $\mathcal T$ a un modèle.
\end{proof}

\begin{definition}[Théorie cohérente, contradictoire]
  Soit une signature $\Sigma$ et une théorie $\mathcal T$ sur $\Sigma$. On dit
  que $\mathcal T$ est cohérente quand elle admet un modèle, et qu'elle est
  contradictoire si elle n'admet pas de modèle. De façon équivalente,
  $\mathcal T$ est cohérente si et seulement si elle n'est pas contradictoire,
  et si et seulement si $\bot\notin\vclose{\mathcal T}$.
\end{definition}

On voit donc qu'une théorie ne doit pas pouvoir prouver trop de choses.
Néanmoins, on peut vouloir une théorie la plus forte possible, qui reste malgré
tout cohérente. Une telle théorie est une théorie complète : elle est une
théorie dans laquelle si un énoncé est faux, alors son contraire est vrai. Elle
peut donc décider tout énoncé, et puisqu'elle est complète elle ne peut pas
décider plus (sinon il serait possible de vérifier $A$ et $\lnot A$, ce qui est
impossible).

\begin{definition}[Théorie complète]
  Une théorie $\mathcal T$ sur une signature $\Sigma$ est dite complète si pour
  toute formule $A\in\Formula(\Sigma)$, soit $A\in \vclose{\mathcal T}$ soit
  $\lnot A \in \vclose{\mathcal T}$, mais pas les deux.
\end{definition}

On va voir plus tard que toute théorie cohérente, c'est-à-dire dont $\bot$ n'est
pas une conséquence logique, peut être étendue en une théorie complète.

\subsection{Morphismes de modèles}

Pour manipuler efficacement les modèles, il est utile de prendre du recul sur ce
qui est manipulé, et d'adopter un point de vue plus algébrique. Pour cela, nous
allons donner quelques propriétés basiques liées aux morphismes de modèles.

En mathématiques, un morphisme est en toute généralité une application qui
préserve la structure. C'est l'idée qui est formalisée dans la notion de
morphisme entre structures : un morphisme commute avec les symboles de relation
et de fonction.

\begin{definition}[Morphisme de structures]
  Soit une signature $\Sigma$ et deux structures $\mathcal M,\mathcal N$, un
  morphisme $\varphi$ de $\mathcal M$ vers $\mathcal N$ est une application
  \[\varphi : |\mathcal M|\longrightarrow |\mathcal N|\]
  vérifiant les propositions suivantes :
  \begin{itemize}
  \item pour tout symbole de fonction $f$ d'arité $n$ et tout tuple
    $(m_1,\ldots,m_n)\in|\mathcal M|^n$ :
    \[\varphi(f^{\mathcal M}(m_1,\ldots,m_n))
    = f^{\mathcal N}(\varphi(m_1),\ldots,\varphi(m_n))\]
  \item pour tout symbole de relation $r$ d'arité $n$ et tout tuple
    $(m_1,\ldots,m_n)\in|\mathcal M|^n$ :
    \[r^{\mathcal M}(m_1,\ldots,m_n) \implies
    r^{\mathcal N}(\varphi(m_1),\ldots,\varphi(m_n)) \]
  \end{itemize}
\end{definition}

\begin{exercise}
  Montrer que pour toute signature $\Sigma$ et toute structure $\mathcal M$ sur
  $\Sigma$, la fonction $\id_{|\mathcal M|}$ induit un morphisme de $\mathcal M$
  vers elle-même. On notera $\id_\mathcal M$ ce morphisme.
\end{exercise}

\begin{exercise}
  Montre que l'opération $\circ$, de composition, s'étend en une opération sur
  les morphismes de structures, c'est-à-dire que si $f$ et $g$ sont deux
  morphismes tels que $\dom(g) = \im(f)$, alors $g\circ f$ est aussi un
  morphisme.
\end{exercise}

\begin{exercise}\label{pred.exo.morph.val}
  Montrer que pour tout morphisme de structure
  $\varphi : \mathcal M \to \mathcal N$, terme $t$ sur la même signature et
  environnement $\rho$ tel que $\VL(t) \subseteq \dom(\rho)$, on a l'égalité
  suivante~:
  \[t_\rho^\mathcal M = t_{\varphi\circ\rho}^\mathcal N\]
\end{exercise}

Un renforcement de la notion de morphisme est celle de plongement : un
plongement est un morphisme dont l'image est isomorphe au domaine, c'est-à-dire
que non seulement le morphisme est injectif, mais le comportement vis à vis
des propositions peut s'étudier dans l'image uniquement en étudiant le modèle de
départ.

\begin{definition}[Plongement]
  Un plongement d'une structure $\mathcal M$ vers une structure $\mathcal N$ est
  un morphisme pour lequel la deuxième condition n'est plus une implication mais
  une équivalence :
  \[\forall(m_1,\ldots,m_n)\in|\mathcal M|^n,r^{\mathcal M}(m_1,\ldots,m_n) \iff
  r^{\mathcal N}(\varphi(m_1),\ldots,\varphi(m_n))\]
\end{definition}

\begin{remark}
  Puisque nous avons toujours la relation $=$ dont l'interprétation est
  l'égalité dans son sens naturel, on en déduit qu'un plongement est injectif.
\end{remark}

Enfin, la notion d'isomorphisme est celle à laquelle on s'attend.

\begin{definition}[Isomorphisme]
  Un isomorphisme $\varphi : \mathcal M \cong \mathcal N$ est un morphisme de
  $\mathcal M$ vers $\mathcal N$ tel qu'il existe un morphisme $\psi$ de
  $\mathcal N$ vers $\mathcal M$ vérifiant
  \[\begin{cases}
  \varphi\circ \psi = \id_{\mathcal N}\\
  \psi\circ\varphi = \id_{\mathcal M}
  \end{cases}\]
\end{definition}

Assez naturellement, un isomorphisme préserve les formules. Cela se voit par
exemple en théorie des groupes, où on n'étudie les groupes qu'à isomorphisme
près puisque toute formule qu'on voudrait énoncer reste stable en passant à
travers un isomorphisme.

\begin{proposition}
  Si $\varphi : \mathcal M \cong \mathcal N$ alors pour toute formule close $A$,
  $\mathcal M\models A$ si et seulement si $\mathcal N\models A$.
\end{proposition}

\begin{proof}
  On montre par induction sur la structure de $A$ que pour tout environnement
  $\rho : \Var\partialto |\mathcal M|$ tel que $\VL(A) \subseteq\dom(\varphi)$,
  on a l'équivalence
  \[\mathcal M,\rho\models A \iff \mathcal N,\varphi\circ \rho\models A\]
  \begin{itemize}
  \item si $A$ est de la forme $r(t_1,\ldots,t_n)$ avec un symbole de relation
    $r$, alors comme $\varphi$ est un morphisme on sait que si
    $(t_{1,\rho}^\mathcal M,\ldots,t_{n,\rho}^\mathcal M)\in r^\mathcal M$, alors
    $(t_{1,\varphi\circ \rho}^\mathcal N,\ldots,
    t_{n,\varphi\circ\rho}^\mathcal N)\in r^{\mathcal N}$, comme on sait qu'on a
    un morphisme réciproque $\psi : \mathcal N \to \mathcal M$, on peut
    écrire $t_{i,\rho}^\mathcal M = t_{i,\psi\circ\varphi\circ\rho}^\mathcal M$, donc
    l'argument précédent appliqué à $\psi$ permet de déduire que si
    $\mathcal N,\varphi\circ \rho\models A$ alors $\mathcal M,\rho\models A$.
  \item si $A$ est $\top$ ou $\bot$, le résultat est direct puisqu'indépendant
    de la structure dans laquelle on l'évalue.
  \item si $A$ est de la forme $\lnot B$, alors
    \[\mathcal M, \rho\models \lnot B\iff \mathcal M, \rho\not\models B
    \iff \mathcal N,\varphi\circ\rho\not\models B
    \iff \mathcal N,\varphi\circ\rho\models \lnot B\]
  \item si $A$ est de la forme $B\lor C$, alors
    \begin{align*}
      \mathcal M, \rho \models B\lor C &\iff (\mathcal M,\rho\models B)\lor
      (\mathcal M,\rho\models C) \\
      &\iff (\mathcal N,\varphi\circ\rho\models B)\lor
      (\mathcal N,\varphi\circ\rho\models C)\\
      &\iff \mathcal N,\varphi\circ \rho \models B\lor C
      \end{align*}
  \item si $A$ est de la forme $B\land C$, alors
    \begin{align*}
      \mathcal M, \rho \models B\land C &\iff (\mathcal M,\rho\models B)\land
      (\mathcal M,\rho\models C) \\
      &\iff (\mathcal N,\varphi\circ\rho\models B)\land
      (\mathcal N,\varphi\circ\rho\models C)\\
      &\iff \mathcal N,\varphi\circ \rho \models B\land C
      \end{align*}
  \item si $A$ est de la forme $B\to C$, alors
    \begin{align*}
      \mathcal M,\rho\models B\to C &\iff
      \mathcal M,\rho\models (\lnot B)\lor C\\
      &\iff \mathcal N,\varphi\circ\rho\models (\lnot B) \lor C\\
      &\iff \mathcal N,\varphi\circ\rho\models B \to C
    \end{align*}
  \item si $A$ est de la forme $\forall x,B$, on montre le résultat par double
    implication. Si pour tout
    $m \in |\mathcal M|, \mathcal M,\rho[x\leftarrow m]\models B$, alors
    pour tout $n \in |\mathcal N|$, par surjectivité de $\varphi$, on trouve
    $m \in |\mathcal M|$ tel que $\varphi(m) = n$. On sait alors que
    $\mathcal M,\rho[x\leftarrow m]\models B$ donc, par hypothèse d'induction,
    que $\mathcal N,\varphi\circ(\rho[x\leftarrow m])\models B$, ce qui revient
    à $\mathcal N,(\varphi\circ\rho)[x\leftarrow \varphi(m)]\models B$, donc
    $\mathcal N,(\varphi\circ\rho)[x\leftarrow n]\models B$, d'où le résultat.
    Dans le sens réciproque, si pour tout
    $n\in|\mathcal N|, \mathcal N,(\varphi\circ\rho)[x\leftarrow n]\models B$,
    on peut faire le même raisonnement en écrivant tout $m \in |\mathcal N|$
    comme $\varphi(\varphi^{-1}(m))$ et utiliser l'hypothse d'induction pour en
    déduire que $\mathcal M,\rho\models \forall x,B$.
  \item le cas où $A$ est de la forme $\exists x, B$ est analogue au précédent.
  \end{itemize}
  D'où le résultat par induction. On a le résultat dans le cas particulier d'une
  formule close.
\end{proof}

On a montré qu'un isomorphisme était un morphisme bijectif~: il est faux en
toute généralité de considérer que cette propriété caractérise un isomorphisme,
comme le montre l'exercice suivant.

\begin{exercise}
  On considère les deux ensembles ordonnés suivants~:
  \begin{itemize}
  \item $\{0,1\}$ où $0$ et $1$ ne sont pas comparables
  \item $\{\bot,\top\}$ où $\bot \leq \top$
  \end{itemize}
  Construire un morphisme bijectif entre les deux ensembles ordonnés qui n'est
  pas un isomorphisme.
\end{exercise}

Cependant, pour un plongement, ce résultat devient valide.

\begin{proposition}
  Un morphisme $\varphi : \mathcal M \to \mathcal N$ est un isomorphisme si et
  seulement si c'est un plongement surjectif.
\end{proposition}

\begin{proof}
  Si $\varphi$ est un isomorphisme, alors c'est en particulier une bijection,
  le rendant surjectif. De plus, comme $\varphi$ préserve toutes les
  propositions dans les deux sens, ce morphisme préserve en particulier les
  propositions atomiques, c'est donc un plongement.

  Si $\varphi$ est un plongement surjectif, alors c'est une bijection~: on
  peut donc considérer la réciproque $\psi$ de $\varphi$. Il nous reste à
  prouver que $\psi$ est bien un morphisme~: la préservation des termes est
  automatique, et il nous faut encore montrer que pour tout symbole de relation
  $r$ d'arité $p$ et $n_1,\ldots,n_p\in|\mathcal N|$~:
  \[r^\mathcal N(n_1,\ldots,n_p) \implies r^\mathcal M
  (\psi(n_1),\ldots,\psi(n_p))\]
  mais chaque $n_1,\ldots,n_p$ peut se réécrire
  $\varphi(m_1),\ldots,\varphi(m_p)$, donnant
  \[r^\mathcal N(\varphi(m_1),\ldots,\varphi(m_p)) \implies r^\mathcal M
  (\psi(\varphi(m_1)),\ldots,\psi(\varphi(m_p)))\]
  Or on sait que $\psi\circ \varphi = \id$, donc la condition revient à la
  réciproque de celle disant que $\varphi$ est un morphisme~: c'est justement
  ce que nous donne le fait que $\varphi$ est un plongement. Donc $\varphi$ est
  un isomorphisme.
\end{proof}

\subsection{Agrandir des modèles}

Pour conclure cette partie introductive à propos des modèles, nous allons
définir la notion d'enrichissement.

\begin{definition}[Enrichissement]
  Soit deux signatures $\Sigma\subseteq\Sigma'$, c'est-à-dire telles que tout
  symbole de $\Sigma$ est un symbole de $\Sigma'$ de même arité. Soit une
  structure $\mathcal M'$ sur $\Sigma'$, alors on dit que $\mathcal M'$ est un
  enrichissement d'une structure $\mathcal M$ sur $\Sigma$ si :
  \begin{itemize}
  \item pour tout symbole de fonction $f\in \Sigma$,
    $f^{\mathcal M} = f^{\mathcal M'}$
  \item pour tout symbole de fonction $r\in \Sigma$,
    $r^{\mathcal M} = r^{\mathcal M'}$
  \end{itemize}

  On définit l'appauvrissement de $\mathcal M'$ sur $\Sigma$ comme l'unique
  structure sur $\Sigma$ dont $\mathcal M'$ est l'enrichissement.
\end{definition}

\begin{remark}
  On peut toujours appauvrir une strucutre, mais il n'est pas toujours possible
  d'enrichir (ou du moins naturellement) une structure sur une extension de sa
  signature.
\end{remark}

Par exemple, le langage des anneaux est un enrichissement du langage des
groupes. On remarque qu'un anneau est toujours, en particulier, un groupe
additif~: nous allons généraliser ce résultat pour une structure sur un
enrichissement.

\begin{proposition}
  Soit deux signatures $\Sigma\subseteq\Sigma'$, $\mathcal M'$ une structure
  sur $\Sigma'$ et $\mathcal M$ son appauvrissement sur $\Sigma$, alors pour
  toute formule $A\in\Formula(\Sigma)$ et tout environnement $\rho$, on a
  \[\mathcal M,\rho\models A \iff \mathcal M',\rho\models A\]
\end{proposition}

\begin{proof}
  On montre d'abord que pour tout terme $t\in\Term(\Sigma)$,
  $t_\rho^\mathcal M = t_\rho^{\mathcal M'}$. Cela se fait par induction~: pour une
  variable $x$, on a $\rho(x) = \rho(x)$ et si l'on suppose que pour tout
  $i\in\{1,\ldots,n\}$ on a $(t_i)_\rho^\mathcal M = (t_i)_\rho^{\mathcal M'}$ alors
  \[(f(t_1,\ldots,t_n))_\rho^\mathcal M = (f(t_1,\ldots,t_n))_\rho^\mathcal M\]
  d'où le résultat pour tout terme $t$.
  
  On procède maintenant par induction sur la structure de $A$~:
  \begin{itemize}
  \item si $A = r(t_1,\ldots,t_n)$ alors
    \begin{align*}
      \Val_\rho^\mathcal M(r(t_1,\ldots,t_n)) &= r^\mathcal M((t_1)_\rho^\mathcal M,
      \ldots,(t_n)_\rho^\mathcal M) \\
      &= r^{\mathcal M'}((t_1)_\rho^{\mathcal M'},\ldots,(t_n)_{\rho}^{\mathcal M'})
    \end{align*}
  \item si $A$ est un connecteur logique, alors le résultat est direct.
  \item si $A = \forall x, B$ ou $A = \exists x, B$ alors la quantification se
    fait sur le même ensemble pour $\mathcal M$ et $\mathcal M'$ donc le
    résultat est le même dans les deux cas (en utilisant l'hypothèse
    d'induction).
  \end{itemize}
  Donc par induction $\Val_\rho^\mathcal M(A) = \Val_\rho^{\mathcal M'}(A)$,
  ce qui est le résultat attendu.
\end{proof}

\begin{corollary}
  Soit deux signatures $\Sigma\subseteq\Sigma'$, une structure $\mathcal M'$
  et une théorie $\mathcal T$ sur $\Sigma$. Si $\mathcal M'\models \mathcal T$,
  alors en prenant l'appauvrissement $\mathcal M$ de $\mathcal M'$ sur $\Sigma$,
  $\mathcal M\models \mathcal T$.
\end{corollary}

La notion d'enrichissement est fondamentale en théorie des modèles~: elle
permet d'avoir un meilleur contrôle sur les modèles qu'on manipule en y ajoutant
des informations par enrichissement, pour ensuite les appauvrir de nouveau vers
le langage d'origine. C'est par exemple l'idée menant à la méthode des
diagrammes, développés dans la partie dédiée à la théorie des modèles.

\section[Syntaxe de preuves]{Système de démonstration du calcul des prédicats}

Nous avons défini la sémantique des formules à travers la notion de modèle. Pour
suivre ce que nous avons fait dans le chapitre précédent, nous allons maintenant
introduire un formalisme pour décrire mathématiquement des preuves en calcul des
prédicats.

Nous avons alors plusieurs choix, car plusieurs formalismes existent. Les trois
principaux sont les sytèmes à la Hilbert, le calcul des séquents et la déduction
naturelle. Nous avons eu un aperçu du calcul des séquents dans le chapitre
précédent, et il pourrait être pertinent de continuer à l'utiliser, mais le
choix fait ici est d'introduire un nouveau formalisme : celui de la déduction
naturelle. Ce choix se justifie par deux raisons principales :
\begin{itemize}
\item Tout d'abord, il permet d'aborder d'autres systèmes de preuves, puisque
  nous avons déjà exploré le calcul des séquents. Cet argument est en même temps
  un contre-argument, puisque cela signifie aussi que l'on peut trouver là une
  occasion de réexplorer le calcul des séquents pour mieux le comprendre, c'est
  donc seulement une raison mineure qui nous pousse à le choisir. De plus, le
  formalisme des systèmes à la Hilbert ne sera pas exploré du tout dans cet
  ouvrage, car s'il est très simple à définir, il possède peu de propriétés
  intéressantes à explorer contrairement aux deux autres formalismes et n'a rien
  de naturel à manipuler.
\item La deuxième raison, plus importante, est que la déduction naturelle est en
  quelque sorte le raisonnement le plus primitif d'un mathématicien. Chaque
  règle exprime une règle tout à fait évidente à l'intuition, particulièrement à
  celle du mathématicien, et on peut trouver une correspondance importante
  entre une preuve utilisant la déduction naturelle et une preuve en langage
  naturel (à la différence évidente que la première est illisible pour un
  profane quand la deuxième est\ldots illisible aussi pour un profane, mais il y
  a moins de profanes des mathématiques que de profanes de la déduction
  naturelle).
\end{itemize}

Il y a de nombreux avantages à définir une syntaxe pour nos preuves en calcul
des prédicats. Le premier est évident : avoir, comme dans le chapitre précédent,
un système efficace pour prouver des relations entre formules, ne passant pas
par une interprétation et une quantification sur toutes les valuations (et ici,
en plus~: tous les modèles). Le deuxième, moins évident, sera plus lourd de
conséquences : une syntaxe nous permet de voir l'ensemble de l'activité
mathématique comme un processus finitaire. Nous n'avons le droit que d'employer
des phrases finies en des textes finis pour étudier des objets potentiellement
infinis. Si ce fait semble en premier lieu purement philosophique (et il est
effectivement important, philosophiquement parlant) il mène aussi à une
conséquence importante~: le théorème de compacité. Ce théorème est un analogue
au \cref{thm.compac.prop} dans le cas du calcul des prédicats. Simplement, au
lieu de quantifier sur les valuations, nous quantifions sur les modèles, et
puisque nous nous occupons en priorité des formules closes cela nous donne un
énoncé parlant uniquement de modèles.

\subsection{Déduction naturelle}

Commençons par définir la relation $\vdash$ de conséquence syntaxique. Pour
cela, comme pour le calcul des séquents, on va définir un système travaillant
sur des listes (par nature finies) plutôt que sur des ensembles. Une différence
importante~: nous n'allons pas relier deux listes, mais une liste avec une
proposition. Ainsi un séquent $\Gamma\vdash A$ signifie directement que, sous
les hypothèses listées dans $\Gamma$, la proposition $A$ est vraie.

\begin{definition}[Déduction naturelle]
  Soit une signature $\Sigma$. On définit la relation
  $\vdash\subseteq \List(\Formula(\Sigma))\times \Formula(\Sigma)$ par induction
  par les règles suivantes :
  \begin{center}
    \AxiomC{$A \in \Gamma$}
    \RightLabel{Ax}
    \UnaryInfC{$\Gamma\vdash A$}
    \DisplayProof
    \qquad
    \AxiomC{}
    \RightLabel{$\top$}
    \UnaryInfC{$\Gamma\vdash \top$}
    \DisplayProof
    \qquad
    \AxiomC{$\Gamma,\lnot A\vdash \bot$}
    \RightLabel{$\bot_\mathrm c$}
    \UnaryInfC{$\Gamma\vdash A$}
    \DisplayProof

    \vspace{0.5cm}
    \AxiomC{$\Gamma,A\vdash \bot$}
    \RightLabel{$\lnot_\mathrm i$}
    \UnaryInfC{$\Gamma\vdash \lnot A$}
    \DisplayProof
    \qquad
    \AxiomC{$\Gamma\vdash \lnot A$}
    \AxiomC{$\Gamma\vdash A$}
    \RightLabel{$\lnot_\mathrm e$}
    \BinaryInfC{$\Gamma\vdash \bot$}
    \DisplayProof

    \vspace{0.5cm}
    \AxiomC{$\Gamma\vdash A$}
    \RightLabel{$\lor_\mathrm i^\mathrm g$}
    \UnaryInfC{$\Gamma\vdash A\lor B$}
    \DisplayProof
    \quad
    \AxiomC{$\Gamma\vdash B$}
    \RightLabel{$\lor_\mathrm i^\mathrm d$}
    \UnaryInfC{$\Gamma\vdash A\lor B$}
    \DisplayProof
    \qquad
    \AxiomC{$\Gamma\vdash A\lor B$}
    \AxiomC{$\Gamma,A\vdash C$}
    \AxiomC{$\Gamma,B\vdash C$}
    \RightLabel{$\lor_\mathrm e$}
    \TrinaryInfC{$\Gamma\vdash C$}
    \DisplayProof

    \vspace{0.5cm}
    \AxiomC{$\Gamma\vdash A$}
    \AxiomC{$\Gamma\vdash B$}
    \RightLabel{$\land_\mathrm i$}
    \BinaryInfC{$\Gamma\vdash A\land B$}
    \DisplayProof
    \qquad
    \AxiomC{$\Gamma\vdash A\land B$}
    \RightLabel{$\land_\mathrm e^\mathrm g$}
    \UnaryInfC{$\Gamma\vdash A$}
    \DisplayProof
    \quad
    \AxiomC{$\Gamma\vdash A\land B$}
    \RightLabel{$\land_\mathrm e^\mathrm d$}
    \UnaryInfC{$\Gamma\vdash B$}
    \DisplayProof

    \vspace{0.5cm}
    \AxiomC{$\Gamma,A\vdash B$}
    \RightLabel{$\to_\mathrm i$}
    \UnaryInfC{$\Gamma\vdash A\to B$}
    \DisplayProof
    \qquad
    \AxiomC{$\Gamma\vdash A\to B$}
    \AxiomC{$\Gamma\vdash A$}
    \RightLabel{$\to_\mathrm e$}
    \BinaryInfC{$\Gamma\vdash B$}
    \DisplayProof

    \vspace{0.5cm}
    \AxiomC{$\Gamma\vdash A[v/x]$}
    \RightLabel{$\forall_\mathrm i^\dagger$}
    \UnaryInfC{$\Gamma\vdash \forall x, A$}
    \DisplayProof
    \qquad
    \AxiomC{$\Gamma\vdash \forall x, A$}
    \RightLabel{$\forall_\mathrm e$}
    \UnaryInfC{$\Gamma\vdash A[t/x]$}
    \DisplayProof

    \vspace{0.5cm}
    \AxiomC{$\Gamma\vdash A[t/x]$}
    \RightLabel{$\exists_\mathrm i$}
    \UnaryInfC{$\Gamma\vdash \exists x, A$}
    \DisplayProof
    \qquad
    \AxiomC{$\Gamma\vdash \exists x, A$}
    \AxiomC{$\Gamma, A[v/x]\vdash B$}
    \RightLabel{$\exists_\mathrm e^\dagger$}
    \BinaryInfC{$\Gamma\vdash B$}
    \DisplayProof

    \vspace{0.5cm}
    \AxiomC{}
    \RightLabel{$=_\mathrm i$}
    \UnaryInfC{$\Gamma\vdash t = t$}
    \DisplayProof
    \qquad
    \AxiomC{$\Gamma\vdash A[t/x]$}
    \AxiomC{$\Gamma\vdash t = u$}
    \RightLabel{$=_\mathrm e$}
    \BinaryInfC{$\Gamma\vdash A[u/x]$}
    \DisplayProof
  \end{center}
  Où $x,v\in \Var$ et où $t,u\in \Term(\Sigma)$.
  
  Les règles avec $^\dagger$ signifient que $v\notin\VL(\Gamma)\cup\VL(B)$.

  De plus, on définit la relation
  $\vdash\subseteq\powerset(\Formula(\Sigma))\times\Formula(\Sigma)$ par
  $\mathcal F\vdash A$ si et seulement s'il existe $\Gamma\in\List(\mathcal F)$
  tel que $\Gamma\vdash A$.
\end{definition}

\begin{exercise}[Sur la négation]
  Montrer que l'on peut prouver les deux séquents suivants :
  \[\vdash \lnot A \to (A \to \bot) \qquad \vdash (A \to \bot) \to \lnot A\]

  Montrer de plus que $\lnot A \vDash A \to \bot$ et que
  $A\to \bot\vDash \lnot A$.
\end{exercise}

On peut donc, sans perdre d'expressivité, redéfinir $\lnot$ comme l'opération
$A \mapsto A \to \bot$. Cela nous permet alors de réduire le nombre de règles
dans notre système.

\begin{exercise}[Sur l'implication]
  Montrer que l'équivalence suivante est prouvable~ :
  \[A \to B \dashv\vdash \lnot A \lor B\]
  où $A \dashv\vdash B$ signifie que $A\vdash B$ et $B\vdash A$.
\end{exercise}

\begin{exercise}[De Morgan]
  Montrer que les lois de De Morgan sont dérivables :
  \begin{itemize}
  \item $\lnot (A \lor B) \dashv\vdash \lnot A \land \lnot B$
  \item $\lnot (A \land B) \dashv\vdash \lnot A \lor \lnot B$
  \item $\lnot (\exists x, A) \dashv\vdash \forall x, \lnot A$
  \item $\lnot (\forall x, A) \dashv\vdash \exists x, \lnot A$
  \item $\lnot\lnot A \dashv\vdash A$
  \item $\lnot (A\to B) \dashv\vdash A \land \lnot B$
  \end{itemize}
\end{exercise}

\begin{exercise}[Tiers exclu et non contradiction]
  Montrer les deux équivalences suivantes :
  \begin{itemize}
  \item $A \lor \lnot A \dashv\vdash \top$
  \item $A \land \lnot A \dashv\vdash \bot$
  \end{itemize}
\end{exercise}

\begin{remark}
  A partir des exercices précédents, on peut être tenté de réduire l'ensemble
  des formules et des règles à un fragment tel que les propositions atomiques,
  $\lnot$, $\lor$ et $\exists$. En effet, toute formule est équivalent à une
  formule écrite avec ce fragment : on peut donc imaginer que toute autre
  formule est en fait une simple écriture plus lisible de cette constituée
  uniquement du fragment restreint.

  Nous n'emploierons pas cette méthode de restriction car, structurellement,
  il n'est pas évident par exemple que $\lnot (\lnot A \lor \lnot B)$, qui
  est code $A \land B$, s'utilise de la même façon au niveau des règles. Si l'on
  sait que l'on peut se ramener en utilisant certaines règles de l'un à l'autre,
  il faudrait travailler à montrer en plus que les règles à propos de $A\land B$
  peut se dériver des règles de $\lnot$ et $\lor$ sur
  $\lnot (\lnot A \lor \lnot B)$. Mais cela motive aussi un déroulement plus
  lent des preuves par induction et des différents cas, qui s'ils sont
  laborieux peuvent pour autant être instructifs pour une première lecture.

  Le cas de $\lnot$ que l'on peut remplacer par $\to \bot$ est différent, car
  les règles à propos de $\lnot$ sont exactement celle de $\to \bot$.
\end{remark}

\paragraph{A propos de la vacuité}
Notre formalisme a un problème essentiel : on peut prouver la proposition
$\forall x, A \implies \exists x, A$, qui est fausse dans le modèle vide. Deux
façons permettent de régler cet écart entre la syntaxe et la sémantique :
changer la syntaxe, ou changer la sémantique. Dans notre cas, nous changeons
alors la sémantique en considérant que toute structure (et donc tout modèle)
est non vide. Cette restriction n'est pas très limitante, puisque le modèle vide
est inintéressant en général. Cependant, celle-ci parait particulièrement
artificielle. Pour contrer cela, on peut à la place considérer des séquents
enrichis de la forme $\Gamma\mid \Theta \vdash \varphi$ où $\Gamma$ va être un
contexte de variables, $\Theta$ un contexte logique et $\varphi$ la conclusion.
Dans ce formalisme, les règles avec $\dagger$ ont, plutôt qu'une restriction,
une action sur le contexte des variables, avec par exemple
\begin{prooftree}
  \AxiomC{$\Gamma, v\mid \Theta\vdash \varphi[v/x]$}
  \RightLabel{$\forall_\mathrm i$}
  \UnaryInfC{$\Gamma\mid \Theta\vdash \forall x, \varphi$}
\end{prooftree}
où $\Gamma\mid\Theta\vdash \forall x, \varphi$ doit être une proposition bien
typée, imposant ansi que $v\notin\VL(\Theta)$.

Le fait de gérer les variables est bien plus naturel, étant donné qu'une preuve
en langage naturel va toujours tenir compte des variables (en particulier, il
semblerait incongru de mentionner une variable non déjà introduite dans le
contexte), et dans un contexte avec plusieurs sortes, c'est-à-dire où les
variables du premier ordre peuvent appartenir à différents ensembles, ce
formalisme gagne en utilité. Dans notre cas, il parait trop lourd de devoir
gérer les variables pour simplement pouvoir inclure le cas du modèle vide,
c'est pourquoi nous préférons simplement modifier notre sémantique.

Plutôt que de prouver directement des résultats, nous allons nous attarder sur
le sens de chaque règle, pour montrer en quoi elles sont intuitives (et donc
robustes au niveau de l'évidence qu'elles énoncent) et permettent de
retranscrire n'importe quelle preuve (en particulier, toute preuve en langage
naturel est virtuellement équivalente à un arbre de preuve en déduction
naturelle).
\begin{itemize}
\item La règle Ax est surement la plus évidente : si $A$ est une hypothèse,
  alors on peut en déduire $A$.
\item La règle $\top$ dit simplement que $\top$ est toujours prouvable.
\item La règle $\bot_\mathrm c$ dit que pour prouver $A$, on peut supposer
  $\lnot A$ est aboutir à une contradiction : c'est le raisonnement par
  l'absurde, d'où l'indice \og c\fg{} exprimant que cette règle est propre à la
  logique classique (nous le verrons, remplacer cette règle par une autre plus
  faible a des conséquences particulièrement intéressantes).
\item La règle $\lnot_\mathrm i$ dit que pour prouver $\lnot A$, il suffit de
  prouver que $A$ aboutit à une absurdité.
\item La règle $\lnot_\mathrm e$ dit que prouver $A$ et $\lnot A$ en même temps
  est une absurdité.
\item Les règles $\lor_\mathrm i$ montrent que si on prouve $A$ (respectivement
  $B$) alors on prouve $A\lor B$.
\item La règle $\lor_\mathrm e$ montre que pour prouver $C$ à partir de
  $A\lor B$, il suffit de montrer que $A\to C$ et $B\to C$ ou, comme nous
  l'avons écrit, que l'on peut prouver $C$ à la fois sous l'hypothèse $A$ et
  sous l'hypothèse $B$. C'est un raisonnement par disjonction de cas.
\item La règle $\land_\mathrm i$ dit que pour prouver $A\land B$, il suffit de
  prouver $A$ d'une part, et $B$ d'autre part.
\item Les règles $\land_\mathrm e$ permettent d'affaiblir une preuve de
  $A\land B$ en une preuve de $A$ (respectivement de $B$).
\item La règle $\to_\mathrm i$ dit que prouver $A\to B$ signifie prouver $B$ en
  ajoutant l'hypothèse $A$.
\item La règle $\to_\mathrm e$ dit que si l'on a prouvé $A\to B$ et $A$, alors on
  peut en déduire $B$. C'est la règle du \textit{modus ponens}.
\item La règle $\forall_\mathrm i$ signifie que pour prouver $\forall x, A$, il
  suffit de prouver $A$ pour une variable $v$ quelconque à la place de $x$. La
  nécessité que $v\notin\VL(\Gamma)$ exprime que ce $v$ est quelconque : aucune
  hypothèse n'est faite sur celui-ci.
\item La règle $\forall_\mathrm e$ signifie qu'à partir d'une preuve de
  $\forall x, A$ on peut instancier $x$ à un terme $t$ quelconque pour obtenir
  une preuve de $A[t/x]$.
\item La règle $\exists_\mathrm i$ permet de déduire une preuve de $\exists x,A$
  à partir d'une preuve de $A[t/x]$, pour n'importe quel terme $t$.
\item La règle $\exists_\mathrm e$ dit qu'à partir d'une proposition de la forme
  $\exists x, A$, on peut déduire une proposition $B$ en ajoutant dans le
  contexte $A[v/x]$, où $v$ est quelconque (ce qui se traduit par la condition
  de $v\notin\VL(\Gamma)\cup\VL(B)$).
\item La règle $=_\mathrm i$ est la réflexivité de l'égalité : un terme est
  égal à lui-même.
\item La règle $=_\mathrm e$, parfois appelée principe de Leibniz, exprime que
  si deux termes $t$ et $u$ sont égaux, alors ils vérifient les mêmes formules.
  On appelle aussi ce principe \og indiscernabilité des identités\fg{}.
\end{itemize}

\subsection[Théorème de complétude]{Complétude de la déduction naturelle}

Nous allons maintenant montrer que $\vDash$ et $\vdash$ coïncident. Un sens est
élémentaire : montrer que $\vdash \subseteq\vDash$, nous allons donc le traiter
en premier. C'est la propriété qu'on appelle correction. Elle énonce que ce que
notre syntaxe dérive est valide. L'autre direction, disans que tout ce qui est
valide est dérivable, est très souvent plus techniques : ce fut le cas pour
la logique propositionnelle, c'est encore le cas pour le calcul des prédicats.

On appelle en général, par abus de langage, complétude du système la propriété
que $\vdash = \vDash$, plutôt que simplement la propriété
$\vDash\subseteq\vdash$. Il est en effet peu pratique de devoir citer deux
théorèmes différents lorsque l'on parle de la correspondance des deux relations,
c'est pourquoi on préfère englober les deux en un résultat, et comme le sens le
plus technique est celui de complétude, c'est celui qu'on utilise pour nommer le
théorème.

Comme $\vdash$ est définie par induction, la preuve de correction est directement
une induction sur sa structure. Remarquons cependant qu'il est nécessaire de
pouvoir prendre en compte les variables, et donc d'introduire des environnements
en plus. Nous allons donc utiliser la \cref{prop.comm.subst.env}.

\begin{theorem}[Correction]
  Soit une signature $\Sigma$, une liste $\Gamma\in\List(\Formula(\Sigma))$ et
  une formule $A\in\Formula(\Sigma)$. Si $\Gamma\vdash A$ alors, en notant
  $X_\Gamma$ l'ensemble des formules dans $\Gamma$, pour toute structure
  $\mathcal M$, tout environnement $\rho$ sur $\mathcal M$, si
  $\mathcal M,\rho\models X_\Gamma$ alors $\mathcal M,\rho\models A$.
\end{theorem}

\begin{proof}
  Nous allons prouver ce résultat par induction sur $\Gamma\vdash A$, en
  supposant introduits $\mathcal M$ et $\rho$ :
  \begin{itemize}
  \item Si $A\in \Gamma$, alors il est évident que $\mathcal M,\rho\models A$.
  \item Peu importe les prémisses, $\mathcal M,\rho\models \top$.
  \item Supposons qu'aucun modèle $\mathcal M$ et aucun environnement $\rho$ ne
    sont tels que $\mathcal M,\rho\models X_\Gamma\cup\{\lnot A\}$. Alors si
    $\mathcal M,\rho\models X_\gamma$, on ne peut pas avoir
    $\mathcal M,\rho\models \lnot A$, donc $\Val_\rho(\lnot A) = 0$, donc
    $\Val_\rho(A) = 1$, d'où $\mathcal M,\rho\models A$.
  \item Supposons qu'aucun $(\mathcal M,\rho)$ ne vérifie que
    $\mathcal M,\rho\models X_\Gamma\cup\{A\}$, alors si
    $\mathcal M,\rho\models X_\Gamma$, $\Val_\rho(A) = 0$ donc
    $\Val_\rho(\lnot A) = 1$, d'où $\mathcal M,\rho\models \lnot A$.
  \item Supposons qu'un modèle de $\Gamma$ est un modèle à la fois de $A$ et de
    $\lnot A$. Alors $\Val_\rho(A) = 1$ et $\Val_\rho(\lnot A) = 1$, mais
    $\Val_\rho(\lnot A) = 1 - \Val_\rho(A)$, donc $0 = 1$ : c'est absurde, donc
    il n'y a pas de modèle de $\Gamma$.
  \item Supposons que si $\mathcal M,\rho\models X_\Gamma$ alors
    $\mathcal M,\rho\models A$. Soit $\mathcal M,\rho\models X_\Gamma$, par
    hypothèse $\Val_\rho(A) = 1$, donc $\Val_\rho(A\lor B) = \max(1,\Val_\rho(B))$,
    donc $\mathcal M,\gamma\models A\lor B$.
  \item L'argument précédent fonctionne exactement de la même manière.
  \item A partir de maintenant, nous n'expliciterons plus les hypothèses
    d'induction, ni ce que l'on cherche à prouver, pour gagner de la place.
    Supposons que $\mathcal M,\rho\models X_\Gamma$. Alors par hypothèse
    d'induction, $\mathcal M,\rho\models A\lor B$, donc
    $\max(\Val_\rho(A),\Val_\rho(B)) = 1$ : on en déduit qu'au moins l'un des deux
    entre $A$ et $B$ est tel que $\Val_\rho = 1$. Sans perte de généralité,
    supposons que $\Val_\rho(A) = 1$. On sait donc que
    $\mathcal M,\rho\models X_\Gamma\cup\{A\}$, donc par hypothèse d'induction
    $\mathcal M,\rho\models C$.
  \item Supposons que $\mathcal M,\rho\models X_\Gamma$, alors par hypothèse
    d'induction $\Val_\rho(A) = 1$ et $\Val_\rho(B) = 1$, donc
    $\Val_\rho(A\land B) = \min(1,1) = 1$, donc
    $\mathcal M,\rho\models A\land B$.
  \item Supposons que $\mathcal M,\rho\models X_\Gamma$, alors par hypothèse
    d'induction $\Val_\rho(A\land B) = 1$, donc $\Val_\rho(A) = 1$ car cette
    valeur est supérieure à $\Val_\rho(A\land B)$ : donc
    $\mathcal M,\rho\models A$.
  \item L'argument précédent fonctionne de la même manière.
  \item Supposons que $\mathcal M,\rho\models X_\Gamma$. On travaille par
    disjonction de cas sur la valeur de $\Val_\rho(A)$ :
    \begin{itemize}
    \item si $\Val_\rho(A) = 0$ alors $\Val_\rho(A\to B) = 1$ donc
      $\mathcal M,\rho\models A \to B$.
    \item si $\Val_\rho(A) = 1$ alors $\mathcal M,\rho\models X_\Gamma\cup\{A\}$,
      donc par hypothèse d'induction $\mathcal M,\rho\models B$, donc
      $\Val_\rho(A\to B) = 1$, donc $\mathcal M,\rho\models A \to B$.
    \end{itemize}
  \item Supposons que $\mathcal M,\rho\models X_\Gamma$. On sait donc que
    $\Val_\rho(A\to B) = 1$ et $\Val_\rho(A) = 1$. Ainsi,
    $\min(0,\Val_\rho(B)) = 1$ : on en déduit que $\Val_\rho(B) = 1$, c'est-à-dire
    que $\mathcal M,\rho\models B$.
  \item Supposons que $\mathcal M,\rho\models X_\Gamma$. On veut montrer que
    $\Val_\rho(\forall x, A) = 1$. Pour cela, soit $m \in |\mathcal M|$ : on
    remarque que, par hypothèse d'induction, $\mathcal M,\rho\models A[v/x]$
    (ici on remarque un point important pour la question de la vacuité : on
    suppose que $\rho$ est bien définie en $v$, ce qui peut se faire sans
    perte de généralité si on a bien un élément dans le modèle), mais à ce
    moment-là comme $v$ est libre dans $\Gamma$, on remarque que
    $\mathcal M,\rho[v\mapsto m]\models X_\Gamma$, soit
    $\mathcal M,\rho[v\mapsto m]\models A[v/x]$, d'où
    $\Val_{\rho[v\mapsto m]}(A[v/x]) = 1$, mais l'expression de gauche vaut
    $\Val_{\rho[x\mapsto m]}(A)$. Comme cela fonctionne pour tout $m$, on en
    déduit que $\Val_\rho(\forall x, A) = 1$.
  \item Supposons que $\mathcal M, \rho\models X_\Gamma$, on sait donc que
    $\mathcal M,\rho\models \forall x, A$, ce qui signifie en particulier que
    pour tous, $m\in\mathcal M$, $\mathcal M,\rho[x\mapsto m]\models A$.
    Mais alors, en considérant $m = t^\mathcal M_\rho$, on trouve que
    $\Val_{\rho[x\mapsto t^\mathcal M_\rho]}(A) = 1$, mais l'expression de gauche
    correpsond exactement à $\Val_\rho(A[t/x])$, donc
    $\mathcal M,\rho\models A[t/x]$.
  \item Supposons que $\mathcal M,\rho\models X_\Gamma$, alors
    $\mathcal M,\rho\models A[t/x]$, mais comme
    $\Val_\rho(A[t/x])\leq\Val_\rho(\exists x,A)$ on en déduit que
    $\Val_\rho(\exists x,A) = 1$, donc $\mathcal M,\rho\models \exists x,A$.
  \item Supposons que $\mathcal M,\rho\models X_\Gamma$, alors
    $\mathcal M,\rho\models \exists x,A$, donc on peut trouver un élément
    $m\in|\mathcal M|$ tel que $\Val_{\rho[x\mapsto m]}(A) = 1$. Alors, en
    considérant $\rho[v\mapsto m]$, on a $\mathcal M,\rho \models X_\Gamma$, donc
    $Val_{\rho[x\mapsto m][v\mapsto m]}(A) = 1$, ce qui revient à dire que
    $\Val_{\rho[x\mapsto m]}(A[v/x]) = 1$, donc $\Val_{\rho[x\mapsto m]}(B) = 1$. Mais
    on sait que $x\notin\VL(B)$, donc $\Val_\rho(B) = 1$ : ainsi,
    $\mathcal M,\rho\models B$.
  \item Pour tout modèle $\mathcal M$ et environnement $\rho$, on a forcément
    $\Val_\rho(t=t) = 1$ puisque
    $(t^\mathcal M_\rho,t^\mathcal M_\rho)\in\{(m,m)\mid m\in|\mathcal M|\}$.
  \item Supposons que $\mathcal M,\rho\models X_\Gamma$, alors
    $\mathcal M,\rho\models A[t/x]$ et $\mathcal M,\rho\models t = u$. On en
    déduit que $\Val_\rho(t = u) = 1$, c'est-à-dire que
    $t^\mathcal M_\rho = u^\mathcal M_\rho$, donc
    $\Val_{\rho[x\mapsto t^\mathcal M_\rho]}(A) = \Val_{\rho[x\mapsto u^\mathcal M_\rho]}(A)$,
    et à partir du fait que $\mathcal M,\rho\models A[t/x]$ on en déduit donc
    que $\mathcal M,\rho\models A[u/x]$.
  \end{itemize}

  Ainsi, par induction, si $\mathcal M,\rho\models X_\Gamma$, alors
  $\mathcal M,\rho\models A$. En particulier, si $\mathcal M\models X_\Gamma$,
  alors $\mathcal M\models A$, et si pour
  $\mathcal F\subseteq \Formula(\Sigma)$, on a
  $\mathcal M\models \mathcal F$ et $\mathcal F\vdash A$, cela implique donc que
  $\mathcal M\models A$. Ainsi $\vdash\subseteq\vDash$.
\end{proof}

Il nous reste à prouver le sens réciproque. En réalité, le point critique pour
la démonstration est celui de l'existence d'un modèle. Plutôt que de montrer
réellement que $\vDash\subseteq\vdash$, nous allons montrer qu'une théorie
consistante, c'est-à-dire une théorie $\mathcal T$ telle que
$\mathcal T\not\vdash \bot$, possède un modèle. On va donc commencer par montrer
que notre résultat suffira à prouver $\vDash\subseteq\vdash$.

\begin{lemma}
  Supposons que pour toute théorie $\mathcal T$ telle que
  $\mathcal T\not\vdash\bot$, il existe une structure $\mathcal M$ telle que
  $\mathcal M\models \mathcal T$. Alors $\vDash\subseteq\vdash$.
\end{lemma}

\begin{proof}
  Supposons que $\mathcal F\vDash A$ pour $A\in\Formula(\Sigma)$ et
  $\mathcal F\subseteq\Formula(\Sigma)$. On voit donc que
  $\mathcal F\cup\{\lnot A\}$ n'a pas de modèle : par contraposée de notre
  hypothèse, cela signifie que $\mathcal F,\lnot A\vdash \bot$. En appliquant
  simplement $\bot_\mathrm c$, on en déduit que $\mathcal F\vdash A$. Ainsi
  $\vDash\subseteq\vdash$.
\end{proof}

\begin{remark}
  Nous avons défini une théorie comme un ensemble de formules closes, donc la
  démonstration précédente n'est pas tout à fait exacte. Une façon de corriger
  cela est d'enrichir le langage : pour chaque variable $x$ libre dans
  $\mathcal F$ ou $A$, on ajoute un symbole de constante $c_x$, et on considère
  ensuite la théorie où $x$ est remplacé par $c_x$. On a une correspondance
  entre l'existence d'un modèle avec $x$ ou avec $c_x$, puisque notre théorie
  sans $c_x$ pourra s'évaluer dans le modèle avec $x\mapsto c_x$.
\end{remark}

Il nous reste donc à démontrer qu'une théorie consistante admet bien un modèle.
Cette construction est technique, c'est pourquoi on va commencer par donner
l'idée de la preuve.

L'idée principale est de construire un modèle syntaxique, c'est-à-dire un modèle
dont les éléments sont exactement les termes du langage. Ceux-ci seront
quotientés par l'égalité, c'est-à-dire que si $\mathcal T \vdash t = u$, alors
les termes $t$ et $u$ seront identifiés. En construisant un tel modèle
$\mathcal M$, on va chercher à ce que $\mathcal T \vdash \varphi$ si et
seulement si $\mathcal M\vDash \varphi$ puisque ce que vérifie $\mathcal M$ est
exactement ce que $\mathcal T$ peut prouver. On a alors besoin de deux
caractéristiques essentielles :
\begin{itemize}
\item tout d'abord, il nous faut résoudre un problème quant à la quantification
  existentielle. Supposons que $\exists x,P$ soit vraie~: on veut pouvoir
  exhiber un élément $m$ tel que $P[m/x]$ est vrai, et nos éléments sont des
  termes. On en déduit donc qu'il faut pour chaque proposition $P$ avoir un
  terme $t_P$ correspondant tel que $\exists x, P \implies P[t_P/x]$. Cela n'est
  pas vrai \textit{a priori}, on va donc chercher à élargir notre langage pour
  ajouter à chaque fois des constantes témoignant pour une proposition
  $\exists x,P$ vraie qu'un terme correspond : c'est ce que l'on appelle la
  méthode des témoins de Henkin.
\item ensuite, $\mathcal T$ doit être complète. L'ensemble des énoncés
  vrais dans un modèle est une théorie complète, puisqu'un énoncé est vrai dans
  un modèle si et seulement si sa négation est fausse. Ainsi, si $\mathcal T$
  n'est pas complète, ça ne peut pas être la théorie d'un modèle. Moralement, on
  peut voir ça comme le fait qu'au moment de construire un modèle, il faut faire
  des choix parmi les propositions, car certains modèles de $\mathcal T$ peuvent
  vérifier telle proposition ou telle autre, si $\mathcal T$ n'est pas complète.
  Nous avons donc besoin d'étendre notre théorie en une théorie complète.
\item une fois cela fait, il ne nous restera plus qu'à appliquer la complétion
  de la théorie enrichie par témoins de Henkin pour obtenir une théorie
  $\overline{\mathcal T}$ contenant $\mathcal T$ et qui nous permettra de
  construire un modèle.
\end{itemize}

Pour pouvoir effectuer les deux premières étapes, il est important que celles-ci
soient compatibles entre elles.

\begin{definition}[Propriété de Henkin]
  On dit qu'une théorie $\mathcal T$ sur une signature $\Sigma$ a la propriété
  de Henkin si pour toute formule telle que $\mathcal T\vdash\exists x, P$, il
  existe un terme $c_P\in\Term(\Sigma)$ tel que $\mathcal T\vdash P[c_P/x]$.
\end{definition}

Le point important de cette propriété est qu'elle peut être vérifiée, au prix
d'une augmentation de la théorie et du langage. Il faut cependant vérifier,
alors, que la nouvelle théorie ne peut toujours pas prouver $\bot$.

\begin{definition}[Clôture par témoins]
  Soit $\Sigma$ une signature, et $\mathcal T$ une théorie sur $\Sigma$. On
  construit de façon itérative la suite $\Sigma_n$ et $\mathcal T_n$ de
  signatures et de théories, où $\mathcal T_n$ est une théorie sur $\Sigma_n$ :
  \begin{itemize}
  \item $\Sigma_0 = \Sigma$ et $\mathcal T_0 = \mathcal T$.
  \item Si $\Sigma_n$ et $\mathcal T_n$ sont construits, on définit la signature
    $\Sigma_{n+1}$ en ajoutant, pour chaque formule $F\in\Formula(\Sigma_n)$,
    une constante $c_F$. On définit $\mathcal T_{n+1}$ en ajoutant à la théorie
    $\mathcal T$ la famille de formules
    $\{(\exists x,F)\to F[c_F/x]\}_{F\in\Formula(\Sigma_n)}$.
  \end{itemize}
  On définit alors
  \[\hclose{\Sigma}\defeq \bigcup_{n\in\mathbb N}\Sigma_n
  \qquad\hclose{\mathcal T} \defeq \bigcup_{n \in \mathbb N}\mathcal T_n\]
  et $\hclose{\mathcal T}$ est une théorie sur $\hclose{\Sigma}$.
\end{definition}

\begin{property}
  La théorie $\hclose{\mathcal T}$ a la propriété de Henkin.
\end{property}

\begin{proof}
  Supposons que $\hclose{\mathcal T}\vdash\exists x,F$ où $F\in\hclose{\Sigma}$.
  Remarquons que $F$ ne peut contenir qu'un nombre fini de symboles de fonctions
  et de relations : on en déduit qu'il existe $n\in \mathbb N$ tel que
  $F\in\Formula(\Sigma_n)$. Cela signifie donc que
  $(\exists x,F)\to F[c_F/x] \in \mathcal T_{n+1}$. On en déduit
  \begin{prooftree}
    \AxiomC{}
    \RightLabel{Ax}
    \UnaryInfC{$\hclose{\mathcal T}\vdash (\exists x, F)\to F[c_F/x]$}
    \AxiomC{$\hclose{\mathcal T}\vdash \exists x,F$}
    \RightLabel{$\to_\mathrm e$}
    \BinaryInfC{$\hclose{\mathcal T}\vdash F[c_F/x]$}
  \end{prooftree}
\end{proof}

De plus, cette construction est stable par extension (ce qui nous servira
lorsque nous complèterons notre théorie).

\begin{property}\label{prop.henkin.ext}
  Si $\hclose{\mathcal T}\subseteq \mathcal S$ pour une certaine théorie
  $\mathcal S$, alors $\mathcal S$ a la propriété de Henkin.
\end{property}

\begin{proof}
  Comme $(\exists x,F)\to F[c_F/x]$ appartient à $\mathcal S$, l'arbre de preuve
  précédent fonctionne encore en remplaçant $\hclose{\mathcal T}$ par
  $\mathcal S$.
\end{proof}

Enfin, on veut montrer que la clôture par témoins conserve la cohérence. Pour
prouver cela, on a besoin d'abord de montrer un lemme important~: une constante
sur laquelle on ne fait aucune hypothèse revient à considérer une variable
libre.

\begin{lemma}[Simulation d'une variable par une constante]\label{lem.var.const}
  Soit $\Sigma$ une signature contenant une constante $c$,
  $A_1,\ldots,A_n\in\Formula(\Sigma\setminus\{c\})$ et $P\in\Formula(\Sigma)$.
  Les deux propositions sont alors équivalentes~:
  \begin{itemize}
  \item $A_1,\ldots,A_n\vdash P$
  \item $A_1,\ldots,A_n\vdash P[c/x]$
  \end{itemize}
\end{lemma}

\begin{proof}
  On raisonne par double implication. Par induction sur
  $A_1,\ldots,A_n\vdash P$ (on notera $\Gamma \defeq A_1,\ldots,A_n$)~:
  \begin{itemize}
  \item si $P\in \Gamma$, alors $x\notin\VL(P)$ donc $P[c/x]=P$ et
    le résultat est évident.
  \item si $P = \top$ alors $P[c/x]=P$ encore une fois.
  \item pour toutes les règles $\bot_\mathrm c,\lnot,\lor,\land,\to$, la
    validité est immédiate puisque l'hypothèse d'induction est stable dans
    toutes les règles (les variables ne changent jamais).
  \item si la dernière règle est $\forall_\mathrm i$, on suppose alors que
    $P = \forall y, A$ et que $\Gamma\vdash A[c/x]$~: on voit que $y$ est
    toujours libre dans $\Gamma$, donc le résultat tient toujours.
  \item si la denière règle est $\forall_\mathrm e$, on suppose alors que
    $P = A[t/y]$ pour un certain terme $t$, donc $P[c/x] = A[t/y][c/x]$ ou
    encore $P[c/x] = A[c/x][t[c/x]/y]$. Par hypothèse d'induction, on sait que
    $\Gamma\vdash \forall y, A[c/x]$, donc $\Gamma\vdash A[c/x][t[c/x]/y]$
    d'où le résultat.
  \item si la dernière règle est $\forall_\mathrm i$, on suppose alors que
    $P = \exists y, A$ et l'hypothèse d'induction nous donne
    $\Gamma\vdash A[t/y][c/x]$, mais on peut réécrire
    $A[t/y][c/x] = A[c/x][t[y/x]/y]$ et appliquer la règle $\exists_\mathrm i$
    pour avoir le résultat.
  \item si la dernière règle est $\exists_\mathrm e$, on suppose que
    $\Gamma\vdash \exists y, A$ et $\Gamma, A[v/x]\vdash P$ où $v$ n'apparait
    pas dans $\Gamma$ ou dans $P$. On utilise notre hypothèse d'induction sur
    la première hypothèse pour en déduire que $\Gamma\vdash \exists y, A[c/x]$,
    puis sur notre deuxième hypothèse réécrite en $\Gamma\vdash A[v/y]\to B$,
    pour obtenir $\Gamma\vdash A[v/y][c/x]\to B[c/x]$. On peut réécrire
    $A[v/y][c/x] = A[c/x][v/y]$, et on peut alors déduire que
    $\Gamma,A[c/x][v/y]\vdash B[c/x]$ donc en utilisant $\exists_\mathrm e$, que
    $\Gamma\vdash B[c/x]$.
  \item si la dernière règle est $=_\mathrm i$, le résultat est direct.
  \item si la dernière règle est $=_\mathrm e$, alors on suppose que $P = A[u/y]$
    et, en utilisant l'hypothèse d'induction, que $\Gamma\vdash A[t/y][c/x]$
    et $\Gamma\vdash t[c/x] = u[c/x]$. On peut alors montrer que
    $A[t/y][c/x] = A[c/x][t[c/x]/y]$, d'où $\Gamma\vdash A[c/x][u[c/x]/y]$, ce
    qui est exactement $\Gamma\vdash A[u/y][c/x]$.
  \end{itemize}
  Ainsi si $\Gamma\vdash P$ alors $\Gamma\vdash P[c/x]$. Le sens réciproque est
  une induction sur $\Gamma\vdash P[c/x]$ et est analogue, nous ne la détaillons
  donc pas.
\end{proof}

On peut maintenant prouver que la clôture par témoins de Henkin préserve la
cohérence.

\begin{property}
  Si $\mathcal T\nvdash \bot$, alors $\hclose{\mathcal T}\nvdash\bot$.
\end{property}

\begin{proof}
  On suppose que $\mathcal \nvdash\bot$, on montre alors par récurrence que
  $\mathcal T_n\not\vdash\bot$ :
  \begin{itemize}
  \item Comme $\mathcal T_0 =\mathcal T$, le résultat est direct.
  \item Supposons que $\mathcal T_n\not\vdash\bot$ et que
    $\mathcal T_{n+1}\vdash\bot$. Par définition, on trouve
    $\Gamma\in\List(\mathcal T_{n+1})$ tel que $\Gamma\vdash\bot$. On écrit
    alors $\Gamma = A_1,\ldots,A_m,B_1,\ldots,B_p$ où chaque $A_i$ est dans
    $\mathcal T_n$ et chaque $B_i$ dans
    $\mathcal T_n\setminus\mathcal T_{n+1}$. On sait que chaque $B_i$ est de
    la forme $(\exists x,F_i)\to F[c_{F_i}/x]$ où $F_i\in\Formula(\Sigma_n)$.
    En fait, on peut même préciser que
    $F_i \in\Formula(\Sigma_n)\setminus\Formula(\Sigma_{n-1})$ puisque
    sinon $B_i$ serait dans $\mathcal T_n$, ce qui signifie que
    $c_{F_i}\notin\Sigma_n$, nous permettant donc d'utiliser le
    \cref{lem.var.const}. On peut réécrire notre séquent en
    \[A_1,\ldots,A_m\vdash \left(\bigwedge_{i=1}^p B_i\right) \to \bot\]
    donc, en utilisant le \cref{lem.var.const}, on obtient avec de nouvelles
    variables fraîches $x_1,\ldots,x_p$~:
    \[A_1,\ldots,A_m\vdash
    \lnot \left(\bigwedge_{i = 1}^p ((\exists x, F_i)\to F_i[x_i/x])\right)\]
    On peut alors généraliser chaque $x_i$ pour obtenir
    \[A_1,\ldots,A_m\vdash \forall y_1,\ldots,y_p, \lnot \left(
    \bigwedge_{i = 1}^p ((\exists x, F_i) \to F_i[y_i/x])\right)\]
    En utilisant les lois de De Morgan, on peut inverser les quantifications sur
    les $y_i$ et la négation, nous donnant
    \begin{equation}\label{eq.henk}
      A_1,\ldots,A_m\vdash \lnot \left(\exists y_1,\ldots,y_p,
      \bigwedge_{i = 1}^p (\exists x, F_i) \to F_i[y_i/x]\right)
    \end{equation}

    On a alors besoin de deux lemmes qui sont laissés en exercices~: pour toutes
    formules $P,Q$ et variables $x,y$ où $x\notin\VL(Q), y \notin\VL(P)$, on
    a les équivalences suivantes ci-dessous.
    \[\exists x, \exists y, (P(x)\land Q(y)) \dashv\vdash (\exists x,P(x))\land
    \exists y,Q(y)
    \qquad \exists y, (P \to Q) \dashv\vdash P \to (\exists y, Q)\]

    On peut alors transformer la déduction \ref{eq.henk} en
    \[A_1,\ldots,A_m\vdash \lnot \left(\bigwedge_{i = 1}^p (\exists x, F_i)\to
    (\exists y, F_i[y/x])\right)\]
    or on peut prouver $(\exists x, F_i) \to (\exists y, F_i[y/x])$, en
    particulier dans $A_1,\ldots,A_m$, ce qui signifie par
    \textit{modus ponens} que $A_1,\ldots,A_m\vdash \bot$, donc que
    $\mathcal T_n\vdash \bot$, ce qui est contradictoire avec notre hypothèse.
    On en déduit donc que $\mathcal T_{n+1}\nvdash\bot$.
  \end{itemize}

  Ainsi, $\mathcal T_n\nvdash\bot$ pour tout $n\in \mathbb N$. Ceci suffit à
  notre preuve, car si $\hclose{\mathcal T}\vdash \bot$, alors il existe une
  liste (finie) $\Gamma\in\List(\hclose{\mathcal T})$ telle que
  $\Gamma\vdash\bot$, mais $\Gamma\in\List(\mathcal T_n)$ pour un certain $n$,
  puisque cette liste est finie. Comme $\mathcal T_n\nvdash\bot$, cela est
  absurde. Ainsi $\hclose{\mathcal T}\nvdash\bot$.
\end{proof}

On va maintenant énoncer qu'on peut compléter une théorie pour $\vdash$. Pour
prouver ce théorème, on va utiliser un contexte plus adapté qui est celui du
\cref{chp.ordres}.

\begin{theorem}[Extension complète d'une théorie]\label{thm.completion}
  Soit $\mathcal T$ une théorie telle que $\mathcal T \nvdash \bot$. Alors il
  existe une théorie $\mathcal T'$ complète pour $\vdash$, c'est-à-dire telle
  que pour tout formule $\varphi$, on a $\mathcal T'\vdash \varphi$ ou
  $\mathcal T'\vdash \lnot \varphi$, et $\mathcal T'\nvdash\bot$.
\end{theorem}

\begin{proof}
  Retardée à la \cref{sbsct.ultrafiltre}.
\end{proof}

En combinant les deux propriétés sur la complétion de Henkin avec le
\cref{thm.completion}, nous obtenons une extension de $\mathcal T$ complète et
vérifiant la propriété de Henkin.

\begin{lemma}
  Si $\mathcal T$ est une théorie sur $\Sigma$ telle que
  $\mathcal T\not\vdash\bot$, alors il existe $\Sigma\subseteq\hclose{\Sigma}$
  et une théorie $\mathcal T\subseteq\mathcal T'$ complète et possédant la
  propriété de Henkin.
\end{lemma}

\begin{proof}
  On applique le \cref{thm.completion} à $\hclose{\mathcal T}$ : comme on sait
  que $\hclose{\mathcal T}\not\vdash\bot$, on peut effectivement compléter cette
  théorie pour $\vdash$. Avec la \cref{prop.henkin.ext}, on sait que cette
  complétion vérifie aussi la propriété de Henkin.
\end{proof}

On fixe maintenant la théorie $\mathcal T'$ construite à partir de $\mathcal T$.
On fixe aussi la signature $\Sigma'$ comme étant $\hclose{\Sigma}$.

\begin{definition}[Modèle syntaxique]
  On définit le modèle syntaxique $\mathcal M_{\mathcal T'}$ par :
  \begin{itemize}
  \item $|\mathcal M_{\mathcal T'}|$ défini comme $\Term(\Sigma')/\equiv$ où
    $\equiv$ est défini par
    \[t\equiv u \defeq \mathcal T'\vdash t = u\]
  \item pour chaque symbole de fonction $f\in\Sigma'$ d'arité $n$, on associe la
    fonction
    \[\begin{array}{ccccc}
    f^{\mathcal T'} & : & |\mathcal M_{\mathcal T'}|^n & \longrightarrow &
    |\mathcal M_{\mathcal T'}|\\
    & & (\overline{t_1},\ldots,\overline{t_n}) & \longmapsto &
    \overline{f(t_1,\ldots,t_n)}
    \end{array}\]
  \item pour chaque symbole de relation $r\in\Sigma'$ d'arité $n$, on définit la
    relation $r^{\mathcal T'}$ par
    \[r^{\mathcal T'}(\overline{t_1},\ldots,\overline{t_n}) \defeq
    \mathcal T'\vdash r(t_1,\ldots,t_n)\]
  \end{itemize}
\end{definition}

\begin{proof}
  Pour que cette définition ait du sens, il convient de montrer que $\equiv$ est
  une relation d'équivalence, et qu'elle est compatible avec les symboles de
  fonction et d'équivalence (c'est-à-dire que nos définitions de
  $f^{\mathcal T'}$ et $r^{\mathcal T'}$ ne dépendent pas du représentant choisi).

  Montrons d'abord que $\equiv$ est une relation d'équivalence :
  \begin{itemize}
  \item Grâce à $=_\mathrm i$, on sait que $\mathcal T'\vdash t = t$ pour tout
    $t\in \Term(\Sigma')$, donc $\equiv$ est réflexive.
  \item Supposons que $t\equiv u$, c'est-à-dire que $\mathcal T'\vdash t = u$,
    on peut alors construire l'arbre de preuve suivant :
    \begin{prooftree}
      \AxiomC{}
      \RightLabel{$=_\mathrm i$}
      \UnaryInfC{$\mathcal T' \vdash t = t$}
      \AxiomC{$\mathcal T' \vdash t = u$}
      \RightLabel{$=_\mathrm e$}
      \BinaryInfC{$\mathcal T' \vdash u = t$}
    \end{prooftree}
    donc $u\equiv t$ ($t=t$ peut se lire comme $(x=t)[t/x]$).
  \item Supposons que $t\equiv u$ et $u\equiv v$, montrons alors que
    $t\equiv v$ :
    \begin{prooftree}
      \AxiomC{$\mathcal T'\vdash u = v$}
      \AxiomC{}
      \RightLabel{$=_\mathrm i$}
      \UnaryInfC{$\mathcal T' \vdash t = t$}
      \AxiomC{$\mathcal T' \vdash t = u$}
      \RightLabel{$=_\mathrm e$}
      \BinaryInfC{$\mathcal T' \vdash u = t$}
      \RightLabel{$=_\mathrm e$}
      \BinaryInfC{$\mathcal T'\vdash t = v$}
    \end{prooftree}
  \end{itemize}
  Ainsi $\equiv$ est bien une relation d'équivalence.

  Soit $f$ un symbole de fonction d'arité $n$. Pour simplifier la preuve, on
  suppose que $f$ est d'arité $1$ (il suffit ensuite de faire une récurrence
  sur $n$ pour généraliser la preuve que nous allons faire). Pour que $\equiv$
  soit compatible avec $f$, il faut que l'image de $f$ ne dépende pas du choix
  du représentant, c'est-à-dire que si $t\equiv u$ alors
  $f^{\mathcal T'}(\overline t) = f^{\mathcal T'}(\overline u)$, c'est-à-dire que
  $\mathcal T'\vdash f(t) = f(u)$, ce que l'on peut prouver par
  \begin{prooftree}
    \AxiomC{$\mathcal T'\vdash t = u$}
    \AxiomC{}
    \RightLabel{$=_\mathrm i$}
    \UnaryInfC{$\mathcal T' \vdash f(t) = f(t)$}
    \RightLabel{$=_\mathrm e$}
    \BinaryInfC{$\mathcal T'\vdash f(t) = f(u)$}
  \end{prooftree}
  donc $f^{\mathcal T'}$ est bien définie.

  De même, pour une relation $r$ prise pour simplifier d'arité $1$, il convient
  de montrer que si $t\equiv u$ alors
  $r^{\mathcal T'}(\overline t)\to r^{\mathcal T'}(\overline u)$ (il faudrait une
  équivalence, mais il suffit en fait de montrer l'implication puisque $\equiv$
  a été montrée symétrique). L'arbre de preuve suivant nous le montre :
  \begin{prooftree}
    \AxiomC{$\mathcal T'\vdash t = u$}
    \AxiomC{$\mathcal T'\vdash r(t)$}
    \RightLabel{$=_\mathrm e$}
    \BinaryInfC{$\mathcal T'\vdash r(u)$}
  \end{prooftree}
  donc $r^{\mathcal T'}$ est bien définie.
\end{proof}

On peut alors démontrer le lemme qui permettra de prouver le théorème de
complétude.

\begin{lemma}
  Pour toute proposition $\varphi \in \Formula(\Sigma')$, on a l'équivalence
  suivante :
  \[\mathcal T' \vdash \varphi \iff \mathcal M_{\mathcal T'}\models \varphi\]
\end{lemma}

\begin{proof}
  On démontre ce résultat par induction sur la structure de $\varphi$ :
  \begin{itemize}
  \item si $\varphi$ est $\top$ ou $\bot$, le résultat est évident (en
    particulier $\mathcal T'\not\vdash\bot$).
  \item si $\varphi = r(t_1,\ldots,t_n)$ alors
    $\mathcal T'\vdash r(t_1,\ldots,t_n)$ signifie exactement que
    $r^{\mathcal T'}(\overline{t_1},\ldots,\overline{t_n})$, donc
    $\mathcal M_{\mathcal T'}\models r(t_1,\ldots,t_n)$. Ce fait est une
    équivalence puisqu'il provient de la définition même de $r^{\mathcal T'}$.
  \item si $\varphi = \lnot \psi$, montrons que
    $\mathcal T'\vdash\varphi\implies \mathcal M_{\mathcal T'}\models \varphi$.
    Comme $\mathcal T'\vdash \lnot \psi$, on en déduit que
    $\mathcal T'\nvdash \psi$ (car $\mathcal T'\nvdash\bot$), donc par
    hypothèse d'induction $\mathcal M_{\mathcal T'}\not\models \psi$, donc
    $\mathcal M_{\mathcal T'}\models \varphi$. Montrons la réciproque~: on
    suppose que $\mathcal M_{\mathcal T'}\models \varphi$, c'est-à-dire que
    $\mathcal M_{\mathcal T'}\not\models \psi$. Par hypothèse d'induction, cela
    signifie que $\mathcal T'\nvdash \psi$, mais $\mathcal T'$ est complète
    donc $\mathcal T'\vdash\psi$. On a donc l'équivalence.
  \item si $\varphi = \psi \land \chi$, on suppose que
    $\mathcal T'\vdash \psi\land\chi$. Comme $\mathcal T'$ est complète, cela
    est équivalent à dire que $\mathcal T'\vdash \psi$ et
    $\mathcal T'\vdash \chi$ (puisque si $\lnot\psi$ ni $\lnot \chi$ ne sont
    prouvables), ce qui est équivalent par hypothèse d'induction à dire que
    $\mathcal M_{\mathcal T'}\models\psi$ et
    $\mathcal M_{\mathcal T'}\models\chi$, donc équivalent à
    $\mathcal M_{\mathcal T'}\models \psi\land \chi$.
  \item les cas de $\to$ et $\lor$ se ramènent directement à $\land$ et $\lnot$
    grâce aux lois de De Morgan (sémantiquement on peut voir facilement
    l'équivalence par l'égalité des valeurs de vérité).
  \item le cas de $\forall$ se ramène à celui de $\exists$ et $\lnot$ grâce
    aux lois de De Morgan, nous traitons donc simplement $\exists$.
  \item si $\varphi = \exists x, \psi$, alors supposons que
    $\mathcal T'\vdash \exists x, \psi$~: la propriété de Henkin sur
    $\mathcal T'$ nous donne un terme $t$ tel que $\mathcal T'\vdash \psi[t/x]$
    et donc un élément $\overline t \in |\mathcal M_{\mathcal T'}|$ tel que
    $\mathcal M_{\mathcal T'}, [x\leftarrow \overline t] \models \psi$, donc
    $\mathcal M_{\mathcal T'}\models \exists x, \psi$. Réciproquement, s'il
    existe $\overline t$ tel que
    $\mathcal M_{\mathcal T'}[x\leftarrow \overline t]\models \psi$, alors
    par hypothèse d'induction $\mathcal T'\vdash \psi[t/x]$ donc
    $\mathcal T'\vdash \exists x, \psi$.
  \end{itemize}

  D'où le résultat par induction.
\end{proof}

On peut maintenant prouver le théorème de complétude.

\begin{theorem}[Complétude de la déduction naturelle]
  Soit une signature $\Sigma$ et une théorie $\mathcal T$ sur $\Sigma$ telle que
  $\mathcal T\not\vdash\bot$. Alors il existe un modèle
  $\mathcal M\models \mathcal T$.
\end{theorem}

\begin{proof}
  En reprenant notre modèle $\mathcal M_{\mathcal T'}$, on sait à partir du lemme
  précédent que $\mathcal M_{\mathcal T'}\models \mathcal T'$ grâce à la règle
  d'axiome. Comme $\mathcal T\subseteq\mathcal T'$, on en déduit donc que
  $\mathcal M_{\mathcal T'}\models \mathcal T$.
\end{proof}

\begin{theorem}[Complétude, deuxième version]\label{thm.completude}
  Soit $\Sigma$ une signature. Alors les deux relations
  $\vdash,\vDash\subseteq \powerset(\Formula(\Sigma))\times\Formula(\Sigma)$
  coïncident.
\end{theorem}

On récupère en tant que conséquence un théorème essentiel de la théorie des
modèles : le théorème de compacité.

\begin{theorem}[Théorème de compacité]
  Soit $\Sigma$ une signature, $\mathcal F \subseteq\Formula(\Sigma)$ et
  $F\in\Formula(\Sigma)$, alors $\mathcal F\vDash F$ si et seulement s'il existe
  $\mathcal A\subseteq_{\mathrm{fin}} \mathcal F$ tel que
  $\mathcal A\vDash F$.
\end{theorem}

\begin{proof}
  En effet, si $\mathcal F\vDash F$ alors $\mathcal F\vdash F$, d'où par
  définition l'existence de $\Gamma\in\List(\mathcal F)$ tel que
  $\Gamma\vdash A$. Il n'y a qu'un nombre fini de propositions dans $\Gamma$,
  donc on peut trouver $\mathcal A\subseteq_{\mathrm{fin}}\mathcal F$ tel que
  $\Gamma\in\List(\mathcal A)$ : on en déduit donc que $\mathcal A\vdash F$,
  d'où par correction que $\mathcal A\vDash F$.
\end{proof}

En étudiant un peu en détail le modèle syntaxique que nous avons construit, il
est possible d'étudier son cardinal. En fait, comme le modèle est construit
comme un quotient de l'ensemble des termes, il nous suffit de dénombrer cet
ensemble. Cela mène au théorème de Löwenheim-Skolem descendant.

\begin{theorem}[Löwenheim-Skolem descendant]
  Soit $\mathcal T$ une théorie cohérente sur une signature $\Sigma$ qui
  n'admet aucun modèle fini. Alors il existe un modèle de $\mathcal T$ de
  cardinal inférieur $\max(|\Sigma_{\mathcal F}|,\aleph_0)$.
\end{theorem}

\begin{proof}
  En effet, soit le modèle $\mathcal M$ de $\mathcal T$ syntaxique, construit
  précédemment. Comme $\mathcal T$ n'admet aucun modèle fini, le modèle est de
  cardinal au moins $\aleph_0$. De plus, on a une surjection
  $\Term(\Sigma) \to \mathcal M$ par $t\mapsto \overline t$. L'ensemble
  $\Term(\Sigma)$ étant de cardinal $\max(|\Sigma_{\mathcal F}|,\aleph_0)$, on
  en déduit le résultat.
\end{proof}

Le théorème de compacité, lui, nous permet de déduire la version ascendante du
théorème.

\begin{theorem}[Löwenheim-Skolem ascendant]
  Soit $\mathcal T$ une théorie admettant un modèle infini, et
  soit $\lambda$ un cardinal infini. Alors il existe un modèle
  de $\mathcal T$ de cardinal supérieur à $\lambda$.
\end{theorem}

\begin{proof}
  On ajoute à $\Sigma$ l'ensemble $\{c_a\mid a \in \kappa\}$ de constantes, et
  on définit la théorie $\mathcal T'$ sur cette signature par
  \[\mathcal T'\defeq\mathcal T\cup \bigcup_{a,b\in \kappa}(a\neq b)\]
  ainsi un modèle de $\mathcal T'$ est naturellement un modèle de $\mathcal T$,
  et il a au moins $\kappa$ éléments. Par le théorème de compacité, pour
  que $\mathcal T'$ soit cohérent, il suffit que toute partie finie de
  $\mathcal T'$ soit cohérente. Mais le modèle de $\mathcal T$, infini, est
  un modèle de toute partie finie de $\mathcal T'$ en associant aux éléments
  $c_a$ des éléments (seul un nombre fini est considéré par la partie finie de
  $\mathcal T'$, les autres peuvent être associés au même élément). Ainsi on
  a un modèle de $\mathcal T'$.
\end{proof}

On peut ainsi énoncer le théorème de Löwenheim-Skolem sous une autre forme.

\begin{theorem}[Löwenheim-Skolem]
  Soit $\mathcal M$ une structure infinie sur une signature $\Sigma$ et
  $\kappa \geq \max(|\Sigma_{\mathcal F}|,\aleph_0)$. Alors il existe une
  structure $\mathcal N$ élémentairement équivalente à $\mathcal M$,
  c'est-à-dire telle que pour tout énoncé clos $\varphi\in\Clos(\Sigma)$, on a
  \[\mathcal M \models \varphi \iff \mathcal N \models \varphi\]
  et telle que $|\mathcal N| = \kappa$.
\end{theorem}

\begin{proof}
  On définit la théorie $\mathcal T_{\mathcal M}$ des énoncés vrais dans
  $\mathcal M$. Comme c'est une théorie cohérente sans modèle fini (puisque
  les axiomes exprimant qu'il y a au moins $n$ éléments sont vérifiés par
  $\mathcal M$), on peut trouver un modèle de cardinal $\kappa$ à
  $\mathcal T_{\mathcal M}$. Cette théorie étant complète, cela signifie que
  $\mathcal T_{\mathcal N} = \mathcal T_{\mathcal M}$, ce qui est le
  résultat attendu.
\end{proof}
