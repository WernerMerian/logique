\chapter{Théorie des ensembles ordonnés}
\label{chp.ordres}

\minitoc

\lettrine{D}{ans} ce chapitre, nous aborderons la théorie des ensembles ordonnés.
Celle-ci est fortement liée à la logique, et permet entre autre de formaliser des
structures logiques comme les algèbres de Boole ou les algèbres de Heyting. 

Les ensembles ordonnés peuvent être vus comme des cas dégénérés de catégories. Si
nous ne donnons pas dans cet ouvrage de rudiments de théorie des catégories,
l'étudiant catégoricien pourra voir dans plusieurs définitions des éléments
similaires à ce qu'il connait en théorie des catégories.

Nous commencerons par définir les idées générales sur les ensembles ordonnés.
Cette première partie introduira les notions élémentaires d'ordre, de borne
supérieure, inférieure etc.
Nous aborderons ensuite la théorie des treillis, des demi-treillis aux treillis
complets. Ces structures peuvent être vues comme des structure particulièrement
rudimentaires, nous verrons ensuite les structures liées à la logique que l'on
peut définir dessus : les algèbres de Boole et les algèbres de Heyting.
Cela motivera notre étude des filtres et ultrafiltres, dont le comportement dans
une algèbre de Boole est fortement simplifiée, et nous aurons alors l'occasion
de démontrer le théorème d'extension d'une théorie cohérente en une théorie
complète. Enfin, nous verrons des notions plus restreintes liées aux ordres :
la densité, la compatibilité, les chaînes et les antichaînes.

\section{Ensembles ordonnés}

Commençons par donner les définitions les plus élémentaires.

\subsection{Définitions}

\begin{definition}[Ensemble ordonné]
  Un ensemble ordonné est un couple $(X,\leq)$ où $\leq$ est une relation binaire
  sur $X$ vérifiant :
  \begin{itemize}
  \item $\leq$ est reflexive : pour tout $x\in X$, $x\leq x$.
  \item $\leq$ est antisymétrique : si deux éléments $x,y\in X$ sont tels que
    $x\leq y$ et $y\leq x$ alors $x = y$.
  \item $\leq$ est transitive : si trois éléments $x,y,z\in X$ sont tels que
    $x\leq y$ et $y\leq z$ alors $x\leq z$.
  \end{itemize}

  Un couple $(X,\leq)$ tel que $\leq$ est seulement réflexive et transitive est
  appelé un ensemble pré-ordonné, et $\leq$ est appelé un pré-ordre sur $X$.
\end{definition}

\begin{example}[Ensemble des parties]
  Soit $X$ un ensemble, l'ensemble $(\powerset(X),\subseteq)$ est un ensemble
  ordonné.
\end{example}

La plupart des résultats que nous énoncerons pourront aussi bien se traduire sur
des pré-ordres, en remplaçant en général $x = y$ par $x\leq y \land y \leq x$.
En fait, un pré-ordre peut se ramener à un ordre par un quotient.

\begin{property}
  Soit $(X,\leq)$ un ensemble pré-ordonné, alors la relation
  $\sim\subseteq X \times X$ définie par
  \[x\sim y \defeq x\leq y \land y \leq x\]
  est une relation d'équivalence, et $\leq$ induit un ordre sur $X/\sim$.
\end{property}

\begin{proof}
  Vérifions que $\sim$ est une relation d'équivalence :
  \begin{itemize}
  \item $\sim$ est transitive : $x\leq x$ et $x\leq x$, donc $x\sim x$.
  \item $\sim$ est symétrique : supposons que $x\sim y$, alors $x\leq y$ et
    $y\leq x$, donc $y\leq x$ et $x\leq y$, donc $y\sim x$.
  \item $\sim$ est réflexive : supposons que $x\sim y$ et $y\sim z$, alors
    $x\leq y$ et $y\leq z$, donc $x\leq z$, et de même $z\leq y$ et $y\leq x$
    donc $z\leq z$, ainsi $x\sim z$.
  \end{itemize}

  Pour vérifier que $\leq$ induit un ordre sur $X/\sim$, il suffit de vérifier
  que si $x\sim x'$ et $y\sim y'$, alors $x\leq y$ si et seulement si $y\leq x$.

  Si $x\leq y$ alors comme $x'\leq x$ et $y\leq y'$, on en déduit que $x'\leq y'$
  et récuproquement, comme $x\leq x'$ et $y'\leq y$, il vient que
  $x'\sim y'\implies x\sim y$.
\end{proof}

\begin{exercise}
  Soit une signature $\Sigma$, montrer que la relation
  $\vdash\subseteq\Formula(\Sigma)\times\Formula(\Sigma)$ définie comme
  restriction de la relation $\vdash$ que nous avons vue précédemment, mais où
  la liste de gauche ne contient qu'une proposition, est un pré-ordre.
\end{exercise}

\begin{remark}
  La relation $\dashv\vdash$ s'écrit aussi $\equiv$, ce qui est cohérent avec
  notre définition sémantique de $\equiv$, étant donné que l'on sait que
  $\vdash$ et $\vDash$ coïncident.

  L'ensemble ordonné construit par $\Formula(\Sigma)/\dashv\vdash$ est appelé
  l'algèbre de Lindenbaum-Tarski. Nous la verrons plus en détail plus tard.
\end{remark}

Une autre définition possible d'un ensemble ordonné est ce que l'on appelle
habituellement un ensemble strictement ordonné. Nous en donnons la définition et
montrons que l'on peut faire correspondre à chaque ordre un ordre strict (et
réciproquement).

\begin{definition}[Ordre strict]
  Un ensemble strictement ordonné est une paire $(X,<)$ où $<$ est une relation
  binaire sur $X$ vérifiant :
  \begin{itemize}
  \item $<$ est antiréflexive : pour tout $x\in X, x\not< x$.
  \item $<$ est transitive.
  \end{itemize}
\end{definition}

\begin{proposition}
  On a une bijection entre les ensembles ordonnés et les ensembles strictement
  ordonnés en associant à $(X,\leq)$ l'ensemble strictement ordonné $(X,<)$
  défini par
  \[x < y \defeq x \leq y \land x \neq y\]
\end{proposition}

\begin{proof}
  On définit, dans l'autre sens, la relation $\leq$ à partir de $<$ par
  \[x\leq y \defeq x < y \lor x = y\]
  Le fait que $((x \leq y)\land x \neq y) \lor x = y$ est équivalent à $x\leq y$
  et que $(x < y\lor x = y) \land x \neq y$ est équivalent à $x < y$ se vérifie
  directement.

  Montrons que $<$ défini à partir de $\leq$ est un ordre strict :
  \begin{itemize}
  \item par définition, $x < y\implies x \neq y$, donc $x\not< x$.
  \item si $x < y$ et $y < z$, alors $x \leq z$ par transitivité de $\leq$.
    Si $x = z$ alors $z < y$ par substitution de $x$ par $z$, donc par
    transitivité $y < y$, ce qui est absurde. Donc $x \neq z$.
  \end{itemize}

  Montrons que $\leq$ défini à partir de $<$ est un ordre :
  \begin{itemize}
  \item par définition, si $x = y$ alors $x\leq y$.
  \item si $x\leq y$ et $y\leq x$, alors soit $x = y$, soit $x < y$. Dans le
    premier cas, on a le résultat voulu. Dans le deuxième cas, on utilise le
    fait que $y\leq x$ pour en déduire que soit $y = x$, soit $y < x$. Encore
    une fois, le cas $y = x$ est directement ce qu'il faut démontrer.
    Si $y < x$, alors par transitivité avec $x < y$ on en déduit que $x < x$,
    ce qui est absurde.
  \item si $x \leq y$ et $y \leq z$, alors soit $x = y$, auquel cas on voit
    directement que $x\leq z$, soit $x < y$. En utilisant le fait que $y \leq z$,
    on en déduit que soit $y = z$, soit $y < z$. Dans le premier cas, il est
    évident que $x \leq z$. Dans le deuxième cas, par transitivité, on en déduit
    que $x < z$.
  \end{itemize}
\end{proof}

On définira donc des propriétés indifféremment sur un ordre ou sur un ordre
strict (si on dit qu'une propriété $P$ définie sur les ordres strictes est
vérifiée pour un ordre $\leq$, cela signifie que $<$ vérifie $P$).

Intéressons-nous d'abord aux éléments : comme un ordre est (sauf indication du
contraire) partiel, deux éléments peuvent ne pas être mis en relation. Dans ces
cas-là, deux notions distinctes apparaissent : la comparabilité et la
compatibilité. Pour deux éléments, être comparable est une condition forte,
mais parfois justement trop forte : la compatibilité signifie juste que deux
éléments peuvent être rejoints.

\begin{definition}[Comparabilité, compatibilité]
  Soit $(X,\leq)$ un ensemble ordonné. On dit que deux éléments $x$ et $y$ sont
  comparables si $x\leq y$ ou $y\leq x$. Deux éléments $x$ et $y$ sont dits
  compatibles s'il existe $z \in X$ tel que $z\leq x$ et $z\leq x$. Deux éléments
  sont dits incompatibles s'ils ne sont pas compatibles.

  Si tous les éléments de $X$ sont compatibles, on dit que $X$ est un ensemble
  totalement ordonné.
  Si tous les éléments de $X$ sont compatibles, on dit que $X$ est un ensemble
  ordonné filtrant vers le bas.
\end{definition}

\begin{example}
  Deux éléments de $\mathcal P(X)$ sont toujours compatibles en prenant comme
  élément $z$ leur intersection. Pourtant, $\{x\}$ et $\{y\}$ pour $x\neq y$ ne
  sont pas comparables.
\end{example}

\subsection{Dualité}

Relevons dès maintenant un phénomène important en théorie des ordres, appelé la
dualité : lorsque l'on a un ordre $\leq$, on peut définir la relation $\geq$
par $y\geq x \iff x\leq y$, et cette relation est aussi une relation d'ordre,
inversant quel élément est le plus grand et quel élément est le plus petit. Ainsi
lorsque l'on définit une notion sur un ensemble ordonné $(X,\leq)$, une notion
duale est directement définie en considérant l'objet dans $(X,\geq)$.

Ce phénomène permet de simplifier beaucoup de preuves : si l'on veut prouver deux
propositions $P$ et $Q$ où $Q$ est obtenue en remplaçant $\leq$ dans $P$ par
$\geq$ (et toutes les définitions par leurs définitions duales), si les
propriétés portent sur tous les ensembles ordonnés, alors prouver $P$ suffit à
déduire $Q$, qui est une conséquence de $P$ pour l'ordre dual.

Nous allons donc souvent écrire deux propositions pour une seule preuve, car la
proposition duale est souvent aussi importante que la proposition de base.

\subsection{Bornes et majorations}

On peut voir la théorie des ordres sous deux primes : le prisme algébrique et le
prisme analytique.

Suivant le prisme algébrique, nous étudions des structures
munies d'une relation (on peut voir cela comme de la théorie des modèles), et
l'accent est alors mis sur les propositions du premier ordre. Par exemple,
la comparabilité ou la compatibilité sont des notions algébriques. De même on
peut parler de la borne supérieure de deux éléments, ou d'être le maximum entre
$x$ et $y$, voire d'un ensemble fini d'éléments.

Suivant le prisme analytique, nous nous intéressons aux parties de l'ensemble
ordonné (potentiellement infinie). Nous parlons donc par exemple de la borne
supérieure d'un ensemble quelconque. Les structures pour lesquelles on peut
appliquer le point de vue analytique sont donc moins nombreuses, mais nous allons
étudier les définitions basiques de ces principes de bornes supérieures, de
majorant ou autre.

\begin{definition}[Majorant, minorant]
  Soit $(X,\leq)$ un ensemble ordonné et $F\subseteq X$ une partie de $X$. On
  dit que $x$ est un majorant de $F$ si
  \[\forall y\in F, y \leq x\]
  et que $x$ est un minorant de $F$ si
  \[\forall y\in F, x \leq y\]

  Par abus de notation, on écrira $F\leq x$ (respectivmeent $x\leq F$) pour dire
  que $x$ est un majorant (respectivement un minorant) de $F$.
\end{definition}

\begin{definition}[Borne supérieure, borne inférieure]
  Soit $(X,\leq)$ un ensemble ordonné et $F\subseteq X$ une partie de $X$. On
  dit que $x$ est la borne supérieure de $F$ si
  \[F\leq x \land (\forall y, F\leq y \implies x \leq y)\]
  et que $x$ est la borne inférieure de $F$ si
  \[x\leq F \land (\forall y, x\leq F \implies y \leq x)\]
\end{definition}

\begin{proof}
  Puisque nous avons dit \og la\fg{} il est nécessaire de montrer l'unicité de
  la borne supérieure (l'unicité de la borne inférieure se prouver par dualité).

  Supposons que $x,y$ sont des bornes supérieures de $F$. Comme $F\leq x$ et
  $y$ est une borne supérieure, $y\leq x$. De même, comme $F\leq y$ et $x$ est
  une borne supérieure, $x\leq y$. Donc $x = y$ par antisymétrie.
\end{proof}

\begin{notation}
  Si la borne supérieure de $F$ existe, on la notera $\bigvee F$. Si la borne
  inférieure de $F$ existe, on la notera $\bigwedge F$.
\end{notation}

\begin{exercise}
  Soit $(X,\leq)$ un ensemble ordonné et $A\subseteq \powerset X$ une partie
  telle que $\bigcup A$ admet une borne supérieure et tout élément de $A$ admet
  une borne supérieure, montrer alors que
  \[\bigvee (\bigcup A) =
  \bigvee\Big(\Big\{ \bigvee a \;\Big|\; a \in A\Big\}\Big)\]
\end{exercise}

\begin{definition}[Maximum,minimum]
  Soit $(X,\leq)$ un ensemble ordonné et $F\subseteq X$. On dit que $x$ est le
  maximum de $F$ si $F\leq x$ et $x\in F$. On dit que $x$ est le minimum de $F$
  si $x\leq F$ et $x\in F$.
\end{definition}

\begin{notation}
  On notera $\max F$ et $\min F$ le maximum (respectivement le minimum) de $F$.
\end{notation}

\begin{property}
  S'il existe, le maximum (respectivement le minimum) d'une partie est sa borne
  supérieure (respectivement inférieure).
\end{property}

\begin{proof}
  Si $x$ est un majorant de $F$, alors comme $\max F \in F$, on en déduit que
  $\max F \leq x$, d'où le résultat.
\end{proof}

\begin{definition}[\'Elément maximal, minimal]
  Soit un ensemble ordonné $(X,\leq)$ et une partie $F\subseteq X$. On dit que
  $x \in F$ est un élément maximal (respectivement minimal) dans $F$ s'il
  n'existe pas $y\in F$ tel que $x\leq y$ (respectivement tel que $y\leq x$).
\end{definition}

\begin{exercise}
  Montrer que le maximum d'une partie en est un élément maximal de cette partie.
\end{exercise}

Une autre partie intéressant est l'étude de parties sur lesquelles l'ordre induit
se comporte particulièrement bien. C'est ce que nous allons voir avec les
chaînes, les antichaînes et les parties filtrantes.

\begin{definition}[Chaîne, antichaîne]
  Soit $(X,\leq)$ un ensemble ordonné. On dit que $C\subseteq X$ est une chaîne
  de $X$ si l'ensemble $(C,\leq)$ (avec l'ordre induit) est un ensemble
  totalement ordonné. On dit que $A\subseteq X$ est une antichaîne de $X$ si
  l'ensemble $(A,\leq)$ (avec l'ordre induit) ne possède pas deux éléments
  comparables, c'est-à-dire si $(A,\leq)$ est l'ordre discret. On dit que
  $A\subseteq X$ est une antichaîne forte de $X$ si pour tous $x,y\in A$, $x$ et
  $y$ ne sont pas compatibles.
\end{definition}

Cela nous mène à la notion d'ensemble inductif, importante pour établir le lemme
de Zorn, qui est une version de l'axiome du choix plus maniable. Nous prouverons
ce résultat lorsque nous étudierons la théorie des ensembles, mais pouvons déjà
en donner le résultat.

\begin{definition}[Ensemble inductif]
  Un ensemble ordonné $(X,\leq)$ est dit inductif si pour chaîne admet un
  majorant.
\end{definition}

\begin{theorem}[Lemme de Zorn]
  Si $(X,\leq)$ est un ensemble inductif et $x\in X$, alors il existe un élément
  maximal $y \in X$ qui est supérieur à $x$.
\end{theorem}

\subsection{Ordre bien fondé et bon ordre}

Nous donnons maintenant les notions d'ordre bien fondé et de bon ordre. Les bons
ordres auront de l'importance pour la théorie des ensembles, et les ordres bien
fondés sont utiles pour étendre le théorème d'induction.

\begin{definition}[Ordre bien fondé]
  Soit $(X,\leq)$ un ensemble ordonné. On dit que $\leq$ est bien fondé si toute
  partie $F\subseteq X$ non vide possède un élément minimal, c'est-à-dire si
  \[\forall F \subseteq X, F\neq \varnothing \implies \exists x\in F,
  \forall y\in F, y\leq x \implies x = y\]
\end{definition}

\begin{remark}
  Une définition équivalente, sur les ordres stricts, est que $<$ est bien fondé
  si :
  \[\forall F \subseteq X, F\neq \varnothing \implies \exists x \in F,
  \forall y \in F, y\nless x\]
\end{remark}

\begin{definition}[Bon ordre]
  Un ensemble ordonné $(X,\leq)$ est un bon ordre lorsque toute partie
  $F\subseteq X$ possède un minimum.
\end{definition}

Une première conséquence évidente est qu'un bon ordre est un ordre bien fondé.
De plus, un bon ordre est total, puisque pour $x,y\in X$ il existe un minimum
à $\{x,y\}$. Ceci caractérise en fait les bons ordres parmi les ordres bien
fondés.

\begin{proposition}
  Si $(X,\leq)$ est un ordre bien fondé total, alors c'est un bon ordre.
\end{proposition}

\begin{proof}
  Supposons que $(X,\leq)$ est un ordre bien fondé total. Soit $F\subseteq X$,
  comme $\leq$ est un ordre bien fondé, on trouve $m$ un élément minimal de $F$.
  Soit $x\in F$, alors soit $x \leq m$ soit $m\leq x$. Dans le premier cas,
  par minimalité de $m$, on en déduit que $m = x$, donc que $m\leq x$. Ainsi,
  dans tous les cas, $m$ est un minorant de $F$ : $F$ admet donc un minimum.

  Donc $(X,\leq)$ est un bon ordre.
\end{proof}

Nous avons vu que la notion de structure inductive nous offre le principe
d'induction, qui dans le cas particulier de $\mathbb N$ correspond au principe de
récurrence. Le principe de récurrence forte, lui, est de nature légèrement
différente : il se base sur la notion d'ordre. Il dit que si pour tout
$k\in\mathbb N$, on peut prouver
\[(\forall i < k, P(i)) \implies P(k)\]
alors on peut en déduire que $\forall n, P(n)$.

Ce procédé peut se généraliser à un ensemble bien fondé : si l'on définit
l'ensemble
$x^{-1} = \{y \in X \mid y < x\}$, on veut montrer que $P(x)$ est vrai pour tout
$x$ à partir du fait que $(\forall y \in x^{-1}, P(y)) \implies P(x)$.

L'idée derrière cette démonstration est que pour un ordre bien fondé, on a un
moyen d'ordonner les éléments de sorte qu'on peut prouver $P$ élément par
élément en s'appuyant sur le fait que $P$ est vérifié aux étapes précédentes.

\begin{theorem}[Induction bien fondée]
  Soit $(X,\leq)$ un ensemble ordonné bien fondé. Soit $P\subseteq X$ une partie
  vérifiant
  \[\forall x, (\forall y < x, y\in P) \implies x\in P\]
  est égale à $P$.
\end{theorem}

\begin{proof}
  Supposons que $P\neq X$. On trouve alors $x\in X \setminus P$ minimal pour $<$.
  Comme $<$ est minimal dans
  $X\setminus P$, on en déduit que pour tout $y < x$, $y \notin X\setminus P$,
  c'est-à-dire $y\in X$. Par hypothèse, puisque $\forall y < x, y\in P$, on en
  déduit que $x\in P$ : c'est absurde. Ainsi, par l'absurde, on a prouvé que
  $X\setminus P = \varnothing$, donc que $P = X$.
\end{proof}

\begin{remark}
  En construisant un ensemble inductif à partir d'une signature $(C,\alpha)$ en
  prenant une suite d'ensembles $(X_i)_{i\in \mathbb N}$, on peut définir un ordre
  canonique bien fondé, l'ordre de sous-terme, par récurrence sur $i$ :
  \begin{itemize}
  \item pour $X_0$, on a facilement un ordre bien fondé sur $\varnothing$.
  \item supposons qu'on possède un ordre bien fondé sur $X_n$, alors on définit
    notre ordre sur $X_{n+1}$ par $x_i < c(x_1,\ldots,x_n)$ pour tout
    $i \in \{1,\ldots,n\}$, et en en prenant la clôture transitive.
  \end{itemize}

  Cet ordre, en particulier, permet de généraliser l'hypothèse d'induction de
  notre définition d'ensemble inductif en une induction forte.
\end{remark}

\begin{exercise}
  Montrer que l'ordre défini plus tôt est bien un ordre bien fondé.
\end{exercise}

\section{Treillis}

Nous allons maintenant étudier le cas des treillis, qui sont des ensembles
ordonnés munis de propriétés de clôture. Pour être exhaustif et situer exactement
quelles conditions mènent à quels résultats, nous allons définir nos structures
de la plus faible à la plus forte, en montrant tous les résultats que nous
souhaitons avoir sur une structure avant de passer à la suivante.

\subsection{Demi-treillis}

Commençons par étudier les demi-treillis : ceux-ci sont des ensembles ordonnés
avec une direction privilégiée, selon laquelle il existe toujours une borne
supérieure pour un ensemble fini.

\begin{definition}[Demi-treillis]
  Un ensemble $(X,\leq)$ est un sup demi-treillis si toute partie $F\subfin X$
  admet une borne supérieure $\bigvee F$. C'est un inf demi-treillis si toute
  partie $F\subfin X$ admet une borne inférieure $\bigwedge F$.
\end{definition}

\begin{property}
  De manière équivalent, un ensemble $(X,\leq)$ est un sup demi-treillis si et
  seulement s'il admet un élément $\bot$ et une opération
  $\lor : X \times X \to X$ tels que $\bot$ est un minorant de $X$ et
  $x\lor y$ est la borne supérieure de $x$ et $y$.

  C'est un inf demi-treillis si et seulement s'il admet un majorant $\top$ et
  une opération $\land : X \times X \to X$ prenant la borne inférieure de deux
  éléments.
\end{property}

\begin{proof}
  Si $(X,\leq)$ est un sup demi-treillis, on trouve $\bot$ en prenant
  $\bigvee\varnothing$ et $x\lor y$ en prenant $\bigvee\{x,y\}$.

  Réciproquement, montrons par récurrence sur le cardinal de $F\subfin X$ qu'il
  existe une borne supérieure à $F$ :
  \begin{itemize}
  \item si $F = \varnothing$, alors $\bigvee F = \bot$.
  \item si $F = F' \cup \{x\}$ et $\bigvee F'$ existe, alors
    $\bigvee F = \bigvee \{\bigvee F',\bigvee \{x\}\} = \bigvee F \lor x$
    donc $\bigvee F$ existe.
  \end{itemize}
\end{proof}

Cela nous donne une caractérisation plus algébrique d'un demi-treillis : c'est un
ensemble muni d'opérations. Cependant, ces opérations ont encore des définitions
trop proche de la théorie des ordres. Nous allons voir que l'on peut transformer
cette définition en une définition purement algébrique.

\begin{definition}[Monoïde commutatif idempotent]
  Un ensemble $X$ muni d'une opération $\cdot$ et d'un élément $e$ est un monoïde
  idempotent commutatif si les propriétés suivantes sont vérifiées :
  \begin{itemize}
  \item $\forall x\in X, x\cdot e = e \cdot x = x$
  \item $\forall x,y,z\in X, x \cdot (y \cdot z) = (x \cdot y) \cdot z$
  \item $\forall x,y\in X, x\cdot y = y \cdot x$
  \item $\forall x\in X, x\cdot x = x$
  \end{itemize}

  On définit sur un monoïde commutatif idempotent $(X,\cdot,e)$ les relations
  $\leq_\lor$ et $\leq_\land$ par
  \begin{align*}
    x \leq_\lor y &\defeq x \cdot y = y \\
    x \leq_\land y &\defeq x \cdot y = x
  \end{align*}
\end{definition}

\begin{proposition}
  L'ensemble $(X,\bot,\lor)$ est un sup demi-treillis pour $\leq_\lor$ si et
  seulement si $(X,\bot,\lor)$ est un monoïde commutatif idempotent.

  L'ensemble $(X,\top,\land)$ est inf demi-treillis pour $\leq_\land$ si et
  seulement si $(X,\top,\land)$ est un monoïde commutatif idempotent.
\end{proposition}

\begin{proof}
  Montrons d'abord que $\leq_\lor$ est bien un ordre pour un monoïde commutatif
  idempotent :
  \begin{itemize}
  \item par idempotence, $x\lor x = x$ donc $x\leq_\lor x$.
  \item si $x\leq_\lor y$ et $y\leq_\lor x$, alors $x\lor y = y$ et $y \lor x = x$
    donc
    \begin{align*}
      x &= y \lor x \\
      &= x \lor y \qquad\text{par commutativité}\\
      &= y
    \end{align*}
  \item si $x\leq_\lor y$ et $y\leq_\lor z$, alors $x\lor y = y$ et $y \lor z = z$
    donc
    \begin{align*}
      x\lor z &= x \lor (y \lor z) \\
      &= (x \lor y) \lor z \\
      &= y \lor z \\
      &= z
    \end{align*}
  \end{itemize}

  Supposons maintenant que $(X,\bot,\lor)$ est un sup demi-treillis, alors tout
  d'abord $x\leq y$ si et seulement si $y = \bigvee\{x,y\}$, si et seulement si
  $y = x \lor y$, d'où le fait que $\leq_\lor$ coïncide avec $\leq$. Montrons
  maintenant que $(X,\bot,\lor)$ est un monoïde commutatif idempotent :
  \begin{itemize}
  \item $\bot$ est un élément neutre pour $\lor$ : pour tout $x\in X$,
    $\bot\leq x$ donc $\bot \vee x = x$, et de même pour $x\vee\bot$.
  \item $\lor$ est associatif :
    $x\lor (y \lor z) = \bigvee\{x,y,z\} = (x\lor y)\lor z$.
  \item $\lor$ est commutatif : $x\lor y = \bigvee\{x,y\}=y\lor x$.
  \item $\lor$ est idempotent : $x\lor x = \bigvee\{x\}=x$.
  \end{itemize}

  Réciproquement, montrons que so $(X,\bot,\lor)$ est un monoïde commutatif
  idempotent, alors $\bot$ est un minorant de $X$ pour $\leq_\lor$ et $\lor$ est
  la borne supérieure de deux éléments pour $\leq_\lor$ :
  \begin{itemize}
  \item pour tout $x$, $\bot \lor x = x$ donc $\bot \leq_\lor x$.
  \item pour tous $x,y$, $x\lor (x\lor y) = (x\lor x)\lor y = x\lor y$ donc
    $x\lor y$ est un majorant de $x$ (de même on prouve que $x\lor y$ est un
    majorant de $y$). Supposons que $z$ est un majorant de $\{x,y\}$ pour
    $\leq_\lor$. Cela signifie que $x\lor z = z$ et $x\lor y = z$. Alors
    $(x\lor y)\lor z = x\lor (y\lor z) = x\lor z = z$ donc
    $x\lor y \leq_\lor z$, donc $x\lor y$ est inférieur à tout majorant de
    $\{x,y\}$ : c'est la borne supérieure de $\{x,y\}$.
  \end{itemize}
\end{proof}

Ainsi on parlera de sup demi-treillis (respectivement inf demi-treillis) en
utilisant $\leq$ et les opérations $\bot,\lor$ (respectivement $\top,\land$) sans
souci, puisque toutes les définitions nous donnent l'ensemble de ces notions.

\subsection{Treillis}

Un demi-treillis ne concerne qu'une direction entre le sup et l'inf. Un treillis,
comme son nom l'indique, est une structure qui possède les propriétés des deux
demi-treillis à la fois.

\begin{definition}[Treillis]
  Un ensemble ordonné $(X,\leq)$ est un treillis si toute partie finie
  $F\subfin X$ possède à la fois une borne supérieure $\bigvee F$ et une borne
  inférieure $\bigwedge F$.
\end{definition}
