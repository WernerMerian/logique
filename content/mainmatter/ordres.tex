\chapter{Théorie des ensembles ordonnés}
\label{chp.ordres}

\minitoc

\lettrine{D}{ans} ce chapitre, nous aborderons la théorie des ensembles ordonnés.
Celle-ci est fortement liée à la logique, et permet entre autre de formaliser des
structures logiques comme les algèbres de Boole ou les algèbres de Heyting. 

Les ensembles ordonnés peuvent être vus comme des cas dégénérés de catégories. Si
nous ne donnons pas dans cet ouvrage de rudiments de théorie des catégories,
l'étudiant catégoricien pourra voir dans plusieurs définitions des éléments
similaires à ce qu'il connait en théorie des catégories.

Nous commencerons par définir les idées générales sur les ensembles ordonnés.
Cette première partie introduira les notions élémentaires d'ordre, de borne
supérieure, inférieure, ainsi que les ordres bien fondés se reposant sur les
éléments minimaux.
Nous aborderons ensuite la théorie des treillis, en allant des demi-treillis aux
algèbres de Boole et de Heyting.
Cela motivera notre étude des filtres et ultrafiltres, dont le comportement dans
une algèbre de Boole est fortement simplifiée, et nous aurons alors l'occasion
de démontrer le théorème d'extension d'une théorie cohérente en une théorie
complète.

\section{Ensembles ordonnés}

Commençons par donner les définitions les plus élémentaires.

\subsection{Définitions}

\begin{definition}[Ensemble ordonné]
  Un ensemble ordonné est un couple $(X,\leq)$ où $\leq$ est une relation binaire
  sur $X$ vérifiant :
  \begin{itemize}
  \item $\leq$ est reflexive : pour tout $x\in X$, $x\leq x$.
  \item $\leq$ est antisymétrique : si deux éléments $x,y\in X$ sont tels que
    $x\leq y$ et $y\leq x$ alors $x = y$.
  \item $\leq$ est transitive : si trois éléments $x,y,z\in X$ sont tels que
    $x\leq y$ et $y\leq z$ alors $x\leq z$.
  \end{itemize}

  Un couple $(X,\leq)$ tel que $\leq$ est seulement réflexive et transitive est
  appelé un ensemble pré-ordonné, et $\leq$ est appelé un pré-ordre sur $X$.
\end{definition}

\begin{example}[Ensemble des parties]
  Soit $X$ un ensemble, l'ensemble $(\powerset(X),\subseteq)$ est un ensemble
  ordonné.
\end{example}

La plupart des résultats que nous énoncerons pourront aussi bien se traduire sur
des pré-ordres, en remplaçant en général $x = y$ par $x\leq y \land y \leq x$.
En fait, un pré-ordre peut se ramener à un ordre par un quotient.

\begin{property}
  Soit $(X,\leq)$ un ensemble pré-ordonné, alors la relation
  $\sim\subseteq X \times X$ définie par
  \[x\sim y \defeq x\leq y \land y \leq x\]
  est une relation d'équivalence, et $\leq$ induit un ordre sur $X/\sim$.
\end{property}

\begin{proof}
  Vérifions que $\sim$ est une relation d'équivalence :
  \begin{itemize}
  \item $\sim$ est transitive : $x\leq x$ et $x\leq x$, donc $x\sim x$.
  \item $\sim$ est symétrique : supposons que $x\sim y$, alors $x\leq y$ et
    $y\leq x$, donc $y\leq x$ et $x\leq y$, donc $y\sim x$.
  \item $\sim$ est réflexive : supposons que $x\sim y$ et $y\sim z$, alors
    $x\leq y$ et $y\leq z$, donc $x\leq z$, et de même $z\leq y$ et $y\leq x$
    donc $z\leq z$, ainsi $x\sim z$.
  \end{itemize}

  Pour vérifier que $\leq$ induit un ordre sur $X/\sim$, il suffit de vérifier
  que si $x\sim x'$ et $y\sim y'$, alors $x\leq y$ si et seulement si $y\leq x$.

  Si $x\leq y$ alors comme $x'\leq x$ et $y\leq y'$, on en déduit que $x'\leq y'$
  et récuproquement, comme $x\leq x'$ et $y'\leq y$, il vient que
  $x'\sim y'\implies x\sim y$.
\end{proof}

\begin{exercise}
  Soit une signature $\Sigma$, montrer que la relation
  $\vdash\subseteq\Formula(\Sigma)\times\Formula(\Sigma)$ définie comme
  restriction de la relation $\vdash$ que nous avons vue précédemment, mais où
  la liste de gauche ne contient qu'une proposition, est un pré-ordre.
\end{exercise}

\begin{remark}
  La relation $\dashv\vdash$ s'écrit aussi $\equiv$, ce qui est cohérent avec
  notre définition sémantique de $\equiv$, étant donné que l'on sait que
  $\vdash$ et $\vDash$ coïncident.

  L'ensemble ordonné construit par $\Formula(\Sigma)/\dashv\vdash$ est appelé
  l'algèbre de Lindenbaum-Tarski. Nous la verrons plus en détail plus tard.
\end{remark}

Une autre définition possible d'un ensemble ordonné est ce que l'on appelle
habituellement un ensemble strictement ordonné. Nous en donnons la définition et
montrons que l'on peut faire correspondre à chaque ordre un ordre strict (et
réciproquement).

\begin{definition}[Ordre strict]
  Un ensemble strictement ordonné est une paire $(X,<)$ où $<$ est une relation
  binaire sur $X$ vérifiant :
  \begin{itemize}
  \item $<$ est antiréflexive : pour tout $x\in X, x\not< x$.
  \item $<$ est transitive.
  \end{itemize}
\end{definition}

\begin{proposition}
  On a une bijection entre les ensembles ordonnés et les ensembles strictement
  ordonnés en associant à $(X,\leq)$ l'ensemble strictement ordonné $(X,<)$
  défini par
  \[x < y \defeq x \leq y \land x \neq y\]
\end{proposition}

\begin{proof}
  On définit, dans l'autre sens, la relation $\leq$ à partir de $<$ par
  \[x\leq y \defeq x < y \lor x = y\]
  Le fait que $((x \leq y)\land x \neq y) \lor x = y$ est équivalent à $x\leq y$
  et que $(x < y\lor x = y) \land x \neq y$ est équivalent à $x < y$ se vérifie
  directement.

  Montrons que $<$ défini à partir de $\leq$ est un ordre strict :
  \begin{itemize}
  \item par définition, $x < y\implies x \neq y$, donc $x\not< x$.
  \item si $x < y$ et $y < z$, alors $x \leq z$ par transitivité de $\leq$.
    Si $x = z$ alors $z < y$ par substitution de $x$ par $z$, donc par
    transitivité $y < y$, ce qui est absurde. Donc $x \neq z$.
  \end{itemize}

  Montrons que $\leq$ défini à partir de $<$ est un ordre :
  \begin{itemize}
  \item par définition, si $x = y$ alors $x\leq y$.
  \item si $x\leq y$ et $y\leq x$, alors soit $x = y$, soit $x < y$. Dans le
    premier cas, on a le résultat voulu. Dans le deuxième cas, on utilise le
    fait que $y\leq x$ pour en déduire que soit $y = x$, soit $y < x$. Encore
    une fois, le cas $y = x$ est directement ce qu'il faut démontrer.
    Si $y < x$, alors par transitivité avec $x < y$ on en déduit que $x < x$,
    ce qui est absurde.
  \item si $x \leq y$ et $y \leq z$, alors soit $x = y$, auquel cas on voit
    directement que $x\leq z$, soit $x < y$. En utilisant le fait que $y \leq z$,
    on en déduit que soit $y = z$, soit $y < z$. Dans le premier cas, il est
    évident que $x \leq z$. Dans le deuxième cas, par transitivité, on en déduit
    que $x < z$.
  \end{itemize}
\end{proof}

On définira donc des propriétés indifféremment sur un ordre ou sur un ordre
strict (si on dit qu'une propriété $P$ définie sur les ordres strictes est
vérifiée pour un ordre $\leq$, cela signifie que $<$ vérifie $P$).

Intéressons-nous d'abord aux éléments : comme un ordre est (sauf indication du
contraire) partiel, deux éléments peuvent ne pas être mis en relation. Dans ces
cas-là, deux notions distinctes apparaissent : la comparabilité et la
compatibilité. Pour deux éléments, être comparable est une condition forte,
mais parfois justement trop forte : la compatibilité signifie juste que deux
éléments peuvent être rejoints.

\begin{definition}[Comparabilité, compatibilité]
  Soit $(X,\leq)$ un ensemble ordonné. On dit que deux éléments $x$ et $y$ sont
  comparables si $x\leq y$ ou $y\leq x$. Deux éléments $x$ et $y$ sont dits
  compatibles s'il existe $z \in X$ tel que $z\leq x$ et $z\leq x$. Deux éléments
  sont dits incompatibles s'ils ne sont pas compatibles.

  Si tous les éléments de $X$ sont compatibles, on dit que $X$ est un ensemble
  totalement ordonné.
  Si tous les éléments de $X$ sont compatibles, on dit que $X$ est un ensemble
  ordonné filtrant vers le bas.
\end{definition}

\begin{example}
  Deux éléments de $\mathcal P(X)$ sont toujours compatibles en prenant comme
  élément $z$ leur intersection. Pourtant, $\{x\}$ et $\{y\}$ pour $x\neq y$ ne
  sont pas comparables.
\end{example}

\subsection{Dualité}

Relevons dès maintenant un phénomène important en théorie des ordres, appelé la
dualité : lorsque l'on a un ordre $\leq$, on peut définir la relation $\geq$
par $y\geq x \iff x\leq y$, et cette relation est aussi une relation d'ordre,
inversant quel élément est le plus grand et quel élément est le plus petit. Ainsi
lorsque l'on définit une notion sur un ensemble ordonné $(X,\leq)$, une notion
duale est directement définie en considérant l'objet dans $(X,\geq)$.

Ce phénomène permet de simplifier beaucoup de preuves : si l'on veut prouver deux
propositions $P$ et $Q$ où $Q$ est obtenue en remplaçant $\leq$ dans $P$ par
$\geq$ (et toutes les définitions par leurs définitions duales), si les
propriétés portent sur tous les ensembles ordonnés, alors prouver $P$ suffit à
déduire $Q$, qui est une conséquence de $P$ pour l'ordre dual.

Nous allons donc souvent écrire deux propositions pour une seule preuve, car la
proposition duale est souvent aussi importante que la proposition de base.

\subsection{Bornes et majorations}

On peut voir la théorie des ordres sous deux primes : le prisme algébrique et le
prisme analytique.

Suivant le prisme algébrique, nous étudions des structures
munies d'une relation (on peut voir cela comme de la théorie des modèles), et
l'accent est alors mis sur les propositions du premier ordre. Par exemple,
la comparabilité ou la compatibilité sont des notions algébriques. De même on
peut parler de la borne supérieure de deux éléments, ou d'être le maximum entre
$x$ et $y$, voire d'un ensemble fini d'éléments.

Suivant le prisme analytique, nous nous intéressons aux parties de l'ensemble
ordonné (potentiellement infinie). Nous parlons donc par exemple de la borne
supérieure d'un ensemble quelconque. Les structures pour lesquelles on peut
appliquer le point de vue analytique sont donc moins nombreuses, mais nous allons
étudier les définitions basiques de ces principes de bornes supérieures, de
majorant ou autre.

\begin{definition}[Majorant, minorant]
  Soit $(X,\leq)$ un ensemble ordonné et $F\subseteq X$ une partie de $X$. On
  dit que $x$ est un majorant de $F$ si
  \[\forall y\in F, y \leq x\]
  et que $x$ est un minorant de $F$ si
  \[\forall y\in F, x \leq y\]

  Par abus de notation, on écrira $F\leq x$ (respectivmeent $x\leq F$) pour dire
  que $x$ est un majorant (respectivement un minorant) de $F$.
\end{definition}

\begin{definition}[Borne supérieure, borne inférieure]
  Soit $(X,\leq)$ un ensemble ordonné et $F\subseteq X$ une partie de $X$. On
  dit que $x$ est la borne supérieure de $F$ si
  \[F\leq x \land (\forall y, F\leq y \implies x \leq y)\]
  et que $x$ est la borne inférieure de $F$ si
  \[x\leq F \land (\forall y, x\leq F \implies y \leq x)\]
\end{definition}

\begin{proof}
  Puisque nous avons dit \og la\fg{} il est nécessaire de montrer l'unicité de
  la borne supérieure (l'unicité de la borne inférieure se prouver par dualité).

  Supposons que $x,y$ sont des bornes supérieures de $F$. Comme $F\leq x$ et
  $y$ est une borne supérieure, $y\leq x$. De même, comme $F\leq y$ et $x$ est
  une borne supérieure, $x\leq y$. Donc $x = y$ par antisymétrie.
\end{proof}

\begin{notation}
  Si la borne supérieure de $F$ existe, on la notera $\bigvee F$. Si la borne
  inférieure de $F$ existe, on la notera $\bigwedge F$.
\end{notation}

\begin{exercise}
  Soit $(X,\leq)$ un ensemble ordonné et $A\subseteq \powerset X$ une partie
  telle que $\bigcup A$ admet une borne supérieure et tout élément de $A$ admet
  une borne supérieure, montrer alors que
  \[\bigvee (\bigcup A) =
  \bigvee\Big(\Big\{ \bigvee a \;\Big|\; a \in A\Big\}\Big)\]
\end{exercise}

\begin{definition}[Maximum,minimum]
  Soit $(X,\leq)$ un ensemble ordonné et $F\subseteq X$. On dit que $x$ est le
  maximum de $F$ si $F\leq x$ et $x\in F$. On dit que $x$ est le minimum de $F$
  si $x\leq F$ et $x\in F$.
\end{definition}

\begin{notation}
  On notera $\max F$ et $\min F$ le maximum (respectivement le minimum) de $F$.
\end{notation}

\begin{property}
  S'il existe, le maximum (respectivement le minimum) d'une partie est sa borne
  supérieure (respectivement inférieure).
\end{property}

\begin{proof}
  Si $x$ est un majorant de $F$, alors comme $\max F \in F$, on en déduit que
  $\max F \leq x$, d'où le résultat.
\end{proof}

\begin{definition}[\'Elément maximal, minimal]
  Soit un ensemble ordonné $(X,\leq)$ et une partie $F\subseteq X$. On dit que
  $x \in F$ est un élément maximal (respectivement minimal) dans $F$ s'il
  n'existe pas $y\in F$ tel que $x\leq y$ (respectivement tel que $y\leq x$).
\end{definition}

\begin{exercise}
  Montrer que le maximum d'une partie en est un élément maximal de cette partie.
\end{exercise}

Une autre partie intéressant est l'étude de parties sur lesquelles l'ordre induit
se comporte particulièrement bien. C'est ce que nous allons voir avec les
chaînes, les antichaînes et les parties filtrantes.

\begin{definition}[Chaîne, antichaîne]
  Soit $(X,\leq)$ un ensemble ordonné. On dit que $C\subseteq X$ est une chaîne
  de $X$ si l'ensemble $(C,\leq)$ (avec l'ordre induit) est un ensemble
  totalement ordonné. On dit que $A\subseteq X$ est une antichaîne de $X$ si
  l'ensemble $(A,\leq)$ (avec l'ordre induit) ne possède pas deux éléments
  comparables, c'est-à-dire si $(A,\leq)$ est l'ordre discret. On dit que
  $A\subseteq X$ est une antichaîne forte de $X$ si pour tous $x,y\in A$, $x$ et
  $y$ ne sont pas compatibles.
\end{definition}

Cela nous mène à la notion d'ensemble inductif, importante pour établir le lemme
de Zorn, qui est une version de l'axiome du choix plus maniable. Nous prouverons
ce résultat lorsque nous étudierons la théorie des ensembles, mais pouvons déjà
en donner le résultat.

\begin{definition}[Ensemble inductif]
  Un ensemble ordonné $(X,\leq)$ est dit inductif si toute chaîne admet un
  majorant.
\end{definition}

\begin{theorem}[Lemme de Zorn]\label{thm.Zorn}
  Si $(X,\leq)$ est un ensemble inductif et $x\in X$, alors il existe un élément
  maximal $y \in X$ qui est supérieur à $x$.
\end{theorem}

\subsection{Ordre bien fondé et bon ordre}

Nous donnons maintenant les notions d'ordre bien fondé et de bon ordre. Les bons
ordres auront de l'importance pour la théorie des ensembles, et les ordres bien
fondés sont utiles pour étendre le théorème d'induction.

\begin{definition}[Ordre bien fondé]
  Soit $(X,\leq)$ un ensemble ordonné. On dit que $\leq$ est bien fondé si toute
  partie $F\subseteq X$ non vide possède un élément minimal, c'est-à-dire si
  \[\forall F \subseteq X, F\neq \varnothing \implies \exists x\in F,
  \forall y\in F, y\leq x \implies x = y\]
\end{definition}

\begin{remark}
  Une définition équivalente, sur les ordres stricts, est que $<$ est bien fondé
  si :
  \[\forall F \subseteq X, F\neq \varnothing \implies \exists x \in F,
  \forall y \in F, y\nless x\]
\end{remark}

\begin{definition}[Bon ordre]
  Un ensemble ordonné $(X,\leq)$ est un bon ordre lorsque toute partie
  $F\subseteq X$ possède un minimum.
\end{definition}

Une première conséquence évidente est qu'un bon ordre est un ordre bien fondé.
De plus, un bon ordre est total, puisque pour $x,y\in X$ il existe un minimum
à $\{x,y\}$. Ceci caractérise en fait les bons ordres parmi les ordres bien
fondés.

\begin{proposition}
  Si $(X,\leq)$ est un ordre bien fondé total, alors c'est un bon ordre.
\end{proposition}

\begin{proof}
  Supposons que $(X,\leq)$ est un ordre bien fondé total. Soit $F\subseteq X$,
  comme $\leq$ est un ordre bien fondé, on trouve $m$ un élément minimal de $F$.
  Soit $x\in F$, alors soit $x \leq m$ soit $m\leq x$. Dans le premier cas,
  par minimalité de $m$, on en déduit que $m = x$, donc que $m\leq x$. Ainsi,
  dans tous les cas, $m$ est un minorant de $F$ : $F$ admet donc un minimum.

  Donc $(X,\leq)$ est un bon ordre.
\end{proof}

Nous avons vu que la notion de structure inductive nous offre le principe
d'induction, qui dans le cas particulier de $\mathbb N$ correspond au principe de
récurrence. Le principe de récurrence forte, lui, est de nature légèrement
différente : il se base sur la notion d'ordre. Il dit que si pour tout
$k\in\mathbb N$, on peut prouver
\[(\forall i < k, P(i)) \implies P(k)\]
alors on peut en déduire que $\forall n, P(n)$.

Ce procédé peut se généraliser à un ensemble bien fondé : si l'on définit
l'ensemble
$x^{-1} = \{y \in X \mid y < x\}$, on veut montrer que $P(x)$ est vrai pour tout
$x$ à partir du fait que $(\forall y \in x^{-1}, P(y)) \implies P(x)$.

L'idée derrière cette démonstration est que pour un ordre bien fondé, on a un
moyen d'ordonner les éléments de sorte qu'on peut prouver $P$ élément par
élément en s'appuyant sur le fait que $P$ est vérifié aux étapes précédentes.

\begin{theorem}[Induction bien fondée]
  Soit $(X,\leq)$ un ensemble ordonné bien fondé. Soit $P\subseteq X$ une partie
  vérifiant
  \[\forall x, (\forall y < x, y\in P) \implies x\in P\]
  est égale à $P$.
\end{theorem}

\begin{proof}
  Supposons que $P\neq X$. On trouve alors $x\in X \setminus P$ minimal pour $<$.
  Comme $<$ est minimal dans
  $X\setminus P$, on en déduit que pour tout $y < x$, $y \notin X\setminus P$,
  c'est-à-dire $y\in X$. Par hypothèse, puisque $\forall y < x, y\in P$, on en
  déduit que $x\in P$ : c'est absurde. Ainsi, par l'absurde, on a prouvé que
  $X\setminus P = \varnothing$, donc que $P = X$.
\end{proof}

\begin{remark}
  En construisant un ensemble inductif à partir d'une signature $(C,\alpha)$ en
  prenant une suite d'ensembles $(X_i)_{i\in \mathbb N}$, on peut définir un ordre
  canonique bien fondé, l'ordre de sous-terme, par récurrence sur $i$ :
  \begin{itemize}
  \item pour $X_0$, on a facilement un ordre bien fondé sur $\varnothing$.
  \item supposons qu'on possède un ordre bien fondé sur $X_n$, alors on définit
    notre ordre sur $X_{n+1}$ par $x_i < c(x_1,\ldots,x_n)$ pour tout
    $i \in \{1,\ldots,n\}$, et en en prenant la clôture transitive.
  \end{itemize}

  Cet ordre, en particulier, permet de généraliser l'hypothèse d'induction de
  notre définition d'ensemble inductif en une induction forte.
\end{remark}

\begin{exercise}
  Montrer que l'ordre défini plus tôt est bien un ordre bien fondé.
\end{exercise}

\section{Treillis}

Nous allons maintenant étudier le cas des treillis, qui sont des ensembles
ordonnés munis de propriétés de clôture. Pour être exhaustif et situer exactement
quelles conditions mènent à quels résultats, nous allons définir nos structures
de la plus faible à la plus forte, en montrant tous les résultats que nous
souhaitons avoir sur une structure avant de passer à la suivante.

\subsection{Demi-treillis}

Commençons par étudier les demi-treillis : ceux-ci sont des ensembles ordonnés
avec une direction privilégiée, selon laquelle il existe toujours une borne
supérieure pour un ensemble fini.

\begin{definition}[Demi-treillis]
  Un ensemble $(X,\leq)$ est un sup demi-treillis si toute partie $F\subfin X$
  admet une borne supérieure $\bigvee F$. C'est un inf demi-treillis si toute
  partie $F\subfin X$ admet une borne inférieure $\bigwedge F$.
\end{definition}

\begin{property}
  De manière équivalent, un ensemble $(X,\leq)$ est un sup demi-treillis si et
  seulement s'il admet un élément $\bot$ et une opération
  $\lor : X \times X \to X$ tels que $\bot$ est un minorant de $X$ et
  $x\lor y$ est la borne supérieure de $x$ et $y$.

  C'est un inf demi-treillis si et seulement s'il admet un majorant $\top$ et
  une opération $\land : X \times X \to X$ prenant la borne inférieure de deux
  éléments.
\end{property}

\begin{proof}
  Si $(X,\leq)$ est un sup demi-treillis, on trouve $\bot$ en prenant
  $\bigvee\varnothing$ et $x\lor y$ en prenant $\bigvee\{x,y\}$.

  Réciproquement, montrons par récurrence sur le cardinal de $F\subfin X$ qu'il
  existe une borne supérieure à $F$ :
  \begin{itemize}
  \item si $F = \varnothing$, alors $\bigvee F = \bot$.
  \item si $F = F' \cup \{x\}$ et $\bigvee F'$ existe, alors
    $\bigvee F = \bigvee \{\bigvee F',\bigvee \{x\}\} = \bigvee F \lor x$
    donc $\bigvee F$ existe.
  \end{itemize}
\end{proof}

Cela nous donne une caractérisation plus algébrique d'un demi-treillis : c'est un
ensemble muni d'opérations. Cependant, ces opérations ont encore des définitions
trop proche de la théorie des ordres. Nous allons voir que l'on peut transformer
cette définition en une définition purement algébrique.

\begin{definition}[Monoïde commutatif idempotent]
  Un ensemble $X$ muni d'une opération $\cdot$ et d'un élément $e$ est un monoïde
  idempotent commutatif si les propriétés suivantes sont vérifiées :
  \begin{itemize}
  \item $\forall x\in X, x\cdot e = e \cdot x = x$
  \item $\forall x,y,z\in X, x \cdot (y \cdot z) = (x \cdot y) \cdot z$
  \item $\forall x,y\in X, x\cdot y = y \cdot x$
  \item $\forall x\in X, x\cdot x = x$
  \end{itemize}

  On définit sur un monoïde commutatif idempotent $(X,\cdot,e)$ les relations
  $\leq_\lor$ et $\leq_\land$ par
  \begin{align*}
    x \leq_\lor y &\defeq x \cdot y = y \\
    x \leq_\land y &\defeq x \cdot y = x
  \end{align*}
\end{definition}

\begin{proposition}
  L'ensemble $(X,\bot,\lor)$ est un sup demi-treillis pour $\leq_\lor$ si et
  seulement si $(X,\bot,\lor)$ est un monoïde commutatif idempotent.

  L'ensemble $(X,\top,\land)$ est inf demi-treillis pour $\leq_\land$ si et
  seulement si $(X,\top,\land)$ est un monoïde commutatif idempotent.
\end{proposition}

\begin{proof}
  Montrons d'abord que $\leq_\lor$ est bien un ordre pour un monoïde commutatif
  idempotent :
  \begin{itemize}
  \item par idempotence, $x\lor x = x$ donc $x\leq_\lor x$.
  \item si $x\leq_\lor y$ et $y\leq_\lor x$, alors $x\lor y = y$ et $y \lor x = x$
    donc
    \begin{align*}
      x &= y \lor x \\
      &= x \lor y \qquad\text{par commutativité}\\
      &= y
    \end{align*}
  \item si $x\leq_\lor y$ et $y\leq_\lor z$, alors $x\lor y = y$ et $y \lor z = z$
    donc
    \begin{align*}
      x\lor z &= x \lor (y \lor z) \\
      &= (x \lor y) \lor z \\
      &= y \lor z \\
      &= z
    \end{align*}
  \end{itemize}

  Supposons maintenant que $(X,\bot,\lor)$ est un sup demi-treillis, alors tout
  d'abord $x\leq y$ si et seulement si $y = \bigvee\{x,y\}$, si et seulement si
  $y = x \lor y$, d'où le fait que $\leq_\lor$ coïncide avec $\leq$. Montrons
  maintenant que $(X,\bot,\lor)$ est un monoïde commutatif idempotent :
  \begin{itemize}
  \item $\bot$ est un élément neutre pour $\lor$ : pour tout $x\in X$,
    $\bot\leq x$ donc $\bot \vee x = x$, et de même pour $x\vee\bot$.
  \item $\lor$ est associatif :
    $x\lor (y \lor z) = \bigvee\{x,y,z\} = (x\lor y)\lor z$.
  \item $\lor$ est commutatif : $x\lor y = \bigvee\{x,y\}=y\lor x$.
  \item $\lor$ est idempotent : $x\lor x = \bigvee\{x\}=x$.
  \end{itemize}

  Réciproquement, montrons que so $(X,\bot,\lor)$ est un monoïde commutatif
  idempotent, alors $\bot$ est un minorant de $X$ pour $\leq_\lor$ et $\lor$ est
  la borne supérieure de deux éléments pour $\leq_\lor$ :
  \begin{itemize}
  \item pour tout $x$, $\bot \lor x = x$ donc $\bot \leq_\lor x$.
  \item pour tous $x,y$, $x\lor (x\lor y) = (x\lor x)\lor y = x\lor y$ donc
    $x\lor y$ est un majorant de $x$ (de même on prouve que $x\lor y$ est un
    majorant de $y$). Supposons que $z$ est un majorant de $\{x,y\}$ pour
    $\leq_\lor$. Cela signifie que $x\lor z = z$ et $x\lor y = z$. Alors
    $(x\lor y)\lor z = x\lor (y\lor z) = x\lor z = z$ donc
    $x\lor y \leq_\lor z$, donc $x\lor y$ est inférieur à tout majorant de
    $\{x,y\}$ : c'est la borne supérieure de $\{x,y\}$.
  \end{itemize}
\end{proof}

Ainsi on parlera de sup demi-treillis (respectivement inf demi-treillis) en
utilisant $\leq$ et les opérations $\bot,\lor$ (respectivement $\top,\land$) sans
souci, puisque toutes les définitions nous donnent l'ensemble de ces notions.

\begin{exercise}
  Soit $\Sigma$ une signature, on rappelle que l'algèbre de Lindenbaum-Tarski
  sur $\Sigma$, notée $\mathcal L(\Sigma)$, est l'ensemble quotient
  $\Formula(\Sigma)/\dashv\vdash$ muni de l'ordre $\vdash$. Montrer que
  cette structure est un inf demi-treillis et un sup demi-treillis où le majorant
  est $\top$, le minorant $\bot$, la borne supérieure $\lor$ et la borne
  inférieure $\land$.
\end{exercise}

\begin{exercise}
  Soit un ensemble $X$, montrer que $(\powerset(X),\subseteq)$ est à la fois un
  sup et un inf demi-treillis, où $X$, $\varnothing$, $\cup$ et $\cap$ sont
  respectivement le majorant, le minorant, la borne supérieure et la borne
  inférieure.
\end{exercise}

On peut aussi définir les morphismes de demi-treillis, qui sont des morphismes
entre structures au sens attendu.

\begin{definition}[Morphisme de demi-treillis]
  Soient $(X,\leq)$ et $(Y,\leq)$ deux sup demi-treillis (respectivement deux
  inf demi-treillis). Un morphisme de sup demi-treillis (respectivement d'inf
  demi-treillis) est une fonction $f : X \to Y$ croissante telle que
  $f(\bot_X) = \bot_Y$ (respectivement $f(\top_X) = \top_Y$) et telle que pour
  tous $x,y\in X$, $f(x\lor y) = f(x)\lor f(y)$ (respectivement telle que pour
  tous $x,y\in X$, $f(x\land y) = f(x)\land f(y)$).
\end{definition}

\subsection{Treillis}

Un demi-treillis ne concerne qu'une direction entre le sup et l'inf. Un treillis,
comme son nom l'indique, est une structure qui possède les propriétés des deux
demi-treillis à la fois.

\begin{definition}[Treillis]
  Un ensemble ordonné $(X,\leq)$ est un treillis si toute partie finie
  $F\subfin X$ possède à la fois une borne supérieure $\bigvee F$ et une borne
  inférieure $\bigwedge F$.
\end{definition}

En théorie des ordres, il est direct de voir qu'être un treillis est équivalent
à être à la fois un inf demi-treillis et un sup demi-treillis, mais cela n'est
pas aussi évident algébriquement, car un inf demi-treillis est défini
algébriquement par la relation $\leq_\land$ et un sup demi-treillis par la
relation $\leq_\lor$ : il faut vérifier que ces deux relations coïncident.

\begin{proposition}
  Un bimonoïde $(X,\bot,\lor,\top,\land)$ est un treillis pour $\leq_\lor$
  (respectivement pour $\lor_\land$) si et seulement si $(X,\bot,\lor)$ est un
  inf demi-treillis, $(X,\top,\land)$ est un inf demi-treillis, $\leq_\lor$ et
  $\leq_\land$ coïncident, cette dernière condition étant équivalente aux
  deux identités d'absorption suivantes~ :
  \begin{itemize}
  \item $\forall x,y\in X, (x\lor y)\land x = x$
  \item $\forall x,y\in X, (x\land y)\lor x = x$
  \end{itemize}
\end{proposition}

\begin{proof}
  Soit $(X,\bot,\lor,\top,\land)$ un bimonoïde commutatif idempotent. On
  suppose de plus que $\leq_\lor = \leq_\land$, montrons les deux identités :
  \begin{itemize}
  \item Soient $x,y\in X$, comme $x\leq_\lor x\lor y$, on en déduit que
    $x\leq_\land x\lor y$, c'est-à-dire que $x\land (x\lor y) = x$ d'où
    l'égalité par commutativité de $\land$.
  \item Soient $x,y\in X$, comme $x\land y\leq_\land x$, on en déduit que
    $x\land y \leq_\lor x$, c'est-à-dire que $(x\land y)\lor x = x$.
  \end{itemize}

  Réciproquement, supposons vraies les identités. Montrons par double inclusion
  que $\leq_\lor = \leq_\land$ :
  \begin{itemize}
  \item soient $x,y$ tels que $x\leq_\lor y$, alors $x\lor y = y$, montrons que
    $x\land y = x$ :
    \begin{align*}
      x \land y &= x \land (x \lor y)\\
      &= x
    \end{align*}
    par la première identité. Ainsi $x\leq_\land y$, donc
    $\leq_\lor\subseteq\leq_\land$.
  \item soient $x,y$ tels que $x\leq_\land y$, alors $x\land y = x$, montrons que
    $x\lor y = y$ :
    \begin{align*}
      x \lor y &= (x\land y) \lor y\\
      &= y
    \end{align*}
    par la deuxième identité. Ainsi $x\leq_\lor y$, donc
    $\leq_\land\subseteq\leq_\lor$.
  \end{itemize}
  Donc $\leq_\lor = \leq_\land$.
\end{proof}

Cela nous donne donc une définition purement algébrique d'un treillis (en
particulier, un treillis est un modèle d'une certaine théorie).

\begin{definition}[Morphisme de treillis]
  Soient $(X,\leq)$ et $(Y,\leq)$ deux treillis, un morphisme de treillis entre
  ces deux treillis est une fonction $f : X \to Y$ qui est à la fois un morphisme
  de sup demi-treillis et un morphisme d'inf demi-treillis.
\end{definition}

\begin{exercise}
  Soit une signature $\Sigma$, montrer que $\mathcal L(\Sigma)$ est un treillis.
\end{exercise}

\begin{exercise}
  Soit un ensemble $X$, montrer que $\powerset (X)$ est un treillis.
\end{exercise}

\begin{exercise}
  Soit une signature $\Sigma$. Soit $\mathcal M$ une structure sur $\Sigma$.
  Montrer que la fonction
  \[\begin{array}{ccccc}
  \Phi & : & \mathcal L(\Sigma_{\mathrm ZF}) & \longrightarrow &
  \powerset(\mathcal M)\\
  & & \varphi(x) & \longmapsto &
  \{x \in |\mathcal M| : \mathcal M\models \varphi(x)\}
  \end{array}\]
  est un morphisme de treillis.
\end{exercise}

Cependant, la partie sup et la partie inf d'un treillis n'interagissent pas
suffisamment bien en général, c'est pourquoi il est nécessaire d'ajouter la
distributivité de $\lor$ sur $\land$ et de $\land$ sur $\lor$, si l'on veut
pouvoir l'utiliser.

\begin{definition}[Treillis distributif]
  Soit $(X,\leq)$ un treillis. On dit que c'est un treillis distributif si de
  plus l'une des deux propositions équivalentes est vraie :
  \begin{enumerate}[label=(\roman*)]
  \item \label{enum.tr.distr1}
    $\forall x,y,z\in X, x\land (y\lor z) = (x\land y)\lor (x\land z)$
  \item \label{enum.tr.distr2}
    $\forall x,y,z\in X, x\lor (y\land z) = (x\lor y)\land (x\lor z)$
  \end{enumerate}
\end{definition}

\begin{proof}
  Vérifions que ces deux propositions sont équivalentes. Par dualité, il nous
  suffit de prouver que $\ref{enum.tr.distr1} \implies\ref{enum.tr.distr2}$.
  On suppose donc \ref{enum.tr.distr1}. Soient $x,y,z\in X$, par un calcul :

  A FAIRE

  D'où l'équivalence.
\end{proof}

\begin{exercise}
  Montrer que $\mathcal L(\Sigma)$ et $\powerset(X)$ sont des treillis
  distributifs.
\end{exercise}

\subsection{Algèbre de Boole}

Nous voyons maintenant notre premier cas de structure logique décrite par un
ensemble ordonné. Avec la structure de treillis, nous avons naturellement une
structure permettant d'encoder la conjonction et la disjonction. Pour travailler
sur la logique propositionnelle, il ne nous manque que la négation, l'implication
pouvant se coder grâce à l'équivalence logique $a\to b \equiv \lnot a \lor b$.

Nous souhaitons donc avoir, en plus de ce que nous avons déjà défini, une
fonction $\lnot : X \to X$ permettant de jouer le rôle de la négation. En logique
classique, qui est notre cadre de travail pour l'instant, deux propriétés
en particulier décrivent le comportement de $\lnot$ : le tiers exclu et le
principe de non contradiction, disant que $x\lor \lnot x$ est vraie et que
$x\land \lnot x$ est fausse. La notion de \og vrai\fg et de \og faux\fg sont
assez naturellement encodées par $\top$ et $\bot$, respectivement, d'où la
définition suivante.

\begin{definition}[Complément]
  Soit un treillis $(X,\leq)$. On dit que $y$ est un complément de $x$ si
  \[x\lor y = \top \qquad x \land y = \bot\]
\end{definition}

En fait, ceci suffit à caractériser la négation. En particulier, il n'y a qu'un
complément par élément, pour un treillis distributif.

\begin{proposition}
  Soit $(X,\leq)$ un treillis distributif et $x\in X$. Alors s'il existe un
  complément à $x$, celui-ci est unique.
\end{proposition}

\begin{proof}
  Supposons que $y,z$ soient deux compléments de $x$. Alors
  \begin{align*}
    y\lor z &= y\lor z\lor (z\land x)\\
    &= z\lor ((y\lor z)\land (y\lor x))\\
    &= z \lor ((y\lor z)\land \top)\\
    &= z \lor (y\lor z)\\
    &= z
  \end{align*}
  d'où $y\leq z$. En inversant le rôle de $y$ et $z$, on trouve aussi que
  $y\leq z$, d'où $y = z$.
\end{proof}

Une algèbre de Boole peut ainsi se décrire grâce à cette notion de complément.

\begin{definition}[Algèbre de Boole]
  Une algèbre de Boole $(B,\leq)$ est un treillis distributif complémenté. Pour
  un élément $x\in B$, on notera $\lnot x$ sont complément.
\end{definition}

\begin{exercise}
  Soit une signature $\Sigma$ et un ensemble $X$. Montrer que
  $\mathcal L(\Sigma)$ et $\powerset(X)$ sont des algèbres de Boole.
\end{exercise}

\begin{remark}
  En particulier, pour $X = \{\varnothing\}$, on obtient une algèbre de Boole
  de cardinal $2$, qui correspond en fait au corps $\mathbb Z/2\mathbb Z$.
  On notera cette algèbre de Boole $B_2$, celle-ci a la particularité d'être
  la plus petite algèbre de Boole non dégénérée (telle que $\bot \neq \top$) et
  d'être la seule algèbre de Boole qui a une structure de corps.
\end{remark}

\begin{exercise}
  Montrer que les identités de De Morgan sont vérifiées :
  \[\lnot\lnot x = x \qquad \lnot (x \lor y) = \lnot x \land \lnot y
  \qquad \lnot (x \land y) = \lnot x \lor \lnot y\]
\end{exercise}

\begin{exercise}
  Montrer que si $B$ et $B'$ sont deux algèbres de Boole, alors un morphisme $f$
  de treillis entre $B$ et $B'$ est en particulier un morphisme d'algèbre de
  Boole, au sens où la propriété suivante est vérifiée :
  \[\forall x \in B, f(\lnot x) = \lnot f(x)\]
\end{exercise}

L'exercice suivant donne une caractérisation purement algébrique d'une algèbre
de Boole : on peut le voir comme un anneau idempotent.

\begin{exercise}
  Soit $(B,\leq)$ une algèbre de Boole. On définit $\oplus$ sur $B$ par
  \[x\oplus y \defeq (x\lor y) \land (\lnot x \lor \lnot y)\]
  Montrer que $(B,\bot,\top,\oplus,\land)$ est un anneau (où $\bot$ est le neutre
  additif et $\top$ le neutre multiplicatif).

  Réciproquement, soit $(A,0,1,+,\times)$ un anneau idempotent, c'est-à-dire
  vérifiant
  \[\forall x\in A, a\times a = a\]
  montrer les propositions suivantes :
  \begin{enumerate}[label=(\roman*)]
  \item $A$ est un anneau de caractéristique $2$.
  \item la relation $\leq_\land$ définie par
    \[x\leq_\land y \defeq x\times y = x\]
    est une relation d'ordre.
  \item $(A,\leq_\land)$ est une algèbre de Boole, de majorant $1$, de minorant
    $0$, où $\times$ est la borne inférieure et où la borne supérieure est
    \[x\lor y \defeq x + y + x \times y\]
  \end{enumerate}
\end{exercise}

\subsection{Algèbre de Heyting}

Pour nous intéresser aux algèbres de Heyting, il nous faut d'abord parler de la
logique intuitionniste. Pour ne pas nous attarder sur les questions
philosophiques qui sont à l'origine de l'essor des mathématiques intuitionnistes
et constructives, nous nous contenterons de remarquer que c'est une logique plus
faible, mais donnant de meilleurs résultats (les résultats sont plus durs à
montrer mais apportent plus d'information).

La logique intuitionniste naît d'une volonté de travailler sans le tiers exclu.
D'un point de vue algorithmique, par exemple, ce principe peut être remis en
question : comment pouvons-nous effectivement décider toute proposition ? Il
est en effet difficile d'imaginer une machinerie finie qui pourrait décider,
par exemple, si tel objet est infini. Ainsi une part des mathématiques n'a ni
besoin, ni envie du principe du tiers exclu, qui est équivalent au raisonnement
par l'absurde. Nous étudierons en détail cette logique lors de notre étude de la
théorie de la démonstration, mais nous donnons ici la définition de la logique
intuitionniste.

\begin{definition}
  Soit une signature $\Sigma$. On définit la relation de déduction syntaxique
  $\vdNJ\subseteq\List(\Formula(\Sigma))\times\Formula(\Sigma)$ par les mêmes
  règles que pour $\vdash$, à l'exception de $\bot_\mathrm c$ que l'on remplace
  par la règle suivante :
  \begin{prooftree}
    \AxiomC{$\Gamma\vdNJ \bot$}
    \RightLabel{$\bot_\mathrm i$}
    \UnaryInfC{$\Gamma\vdNJ A$}
  \end{prooftree}
  appelée communément principe d'explosion, ou bien par sa dénomination latine
  \textit{ex falso quodlibet} (version raccourcie de
  \textit{ex falso sequitur quodlibet}).
\end{definition}

\begin{exercise}
  Soit la règle
  \begin{center}
    \AxiomC{}
    \RightLabel{EDN}
    \UnaryInfC{$\Gamma\vdash A\lor \lnot A$}
    \DisplayProof
  \end{center}
  Montrer que celle-ci est équivalente à $\bot_\mathrm c$ modulo $\vdNJ$,
  c'est-à-dire que si l'on considère $\vdNJ$ et qu'on rajoute l'une des deux
  règles, alors la seconde est dérivable.
\end{exercise}

\begin{exercise}
  Vérifier que dans le cas intuitionniste aussi, la relation $\vdNJ$ est une
  relation de pré-ordre. On notera $\mathcal L_\mathrm i(\Sigma)$ l'algèbre de
  Lindenbaum-Tarski intuitionniste associée à $\Sigma$.
\end{exercise}

\begin{exercise}
  Montrer que $\mathcal L_\mathrm i(\Sigma)$ est un treillis distributif.
\end{exercise}

On peut alors se demander quelle est la différence entre ce que nous pouvons
prouver en logique intuitionniste et ce que nous pouvons prouver en logique
classique. La réponse est difficile à donner directement, mais un point important
est que ce que l'on peut prouver en logique intuitionniste est constructif :
par exemple, prouver qu'un objet existe signifie qu'on a effectivement exhibé un
objet. En comparaison, on peut prouver grâce à l'absurde qu'un objet existe en
prouvant simplement qu'il ne peut pas ne pas existe.

Une conséquence de cet effet est que certaines tautologies ne sont plus valides,
en particulier la tautologie
\[\lnot(\forall x, \lnot A) \implies \exists x, A\]
exprimant qu'à partir de l'impossibilité que tous les $x$ rejettent $A$, on peut
extraire un certain $x$ acceptant $A$.

De même, la tautologie
\[\lnot (A \land B) \implies \lnot A \lor \lnot B\]
car pour la prouver, il faudrait de façon uniforme décider si $\lnot A$ est vraie
ou si $\lnot B$ est vraie. Comme cela dépend directement que la valeur de vérité
de $A$ et de celle de $B$, il est impossible de conclure en logique
intuitionniste. En échange, si l'on prouve $\vdNJ A \lor B$, cela signifie qu'on
peut soit prouver $\vdNJ A$, soit prouver $\vdNJ B$.

La question, maintenant, est de savoir s'il existe une structure similaire aux
algèbre de Boole, mais dans laquelle la logique appliquée est la logique
intuitionniste. C'est le cas des algèbres de Heyting, dont la définition repose
avant tout sur celle de l'opération $\to$, d'où l'on dérive le sens de $\lnot$
par $\lnot x = x \to \bot$, comme nous l'avons déjà fait.

\begin{definition}[Pseudo-complément relatif]
  Soit un treillis distributif $(X,\leq)$, et deux éléments $a,b\in X$. On dit
  que $x$ est le pseudo-complément de $a$ relatif à $b$ si $x$ est la borne
  supérieure de l'ensemble $\{x \in X \mid a \land x \leq b\}$. Si le
  pseudo-complément de $a$ relatif à $b$ existe, on le note $a\to b$. On appelle
  pseudo-complément de $a$ le pseudo complément de $a$ relatif à $\bot$, s'il
  existe, et on le note alors $\lnot x$.
\end{definition}

\begin{definition}[Algèbre de Heyting]
  Un treillis distributif $(X,\leq)$ est une algèbre de Heyting si tout élément
  admet un pseudo-complément relatif à tout autre élément, c'est-à-dire si
  pour tous $a,b\in X$, l'ensemble $\{x\in X \mid a \land x \leq b\}$ admet une
  borne supérieure.
\end{definition}

\begin{remark}
  Une inégalité par rapport à un pseudo-complément peut se réécrire par une
  conjonction : $a\leq b \to c$ revient directement à $a\land b \leq c$ étant
  donné que $b \to c$ est la borne supérieure des éléments $x$ tels que
  $x\land b \leq c$. On peut donc formaliser le modus ponens : si l'on a
  $a\leq b \to c$ et $a\leq b$, alors $a\land b \leq c$ et $a\land b = a$, donc
  $a\leq c$.
\end{remark}

\begin{exercise}
  Soit une signature $\Sigma$. Montrer que $\mathcal L_\mathrm i(\Sigma)$ est bien
  une algèbre de Heyting.
\end{exercise}

\begin{exercise}
  Montrer qu'une algèbre de Boole est en particulier une algèbre de Heyting, et
  que le complément d'un élément est en particulier son pseudo-complément.
\end{exercise}

\subsection{Treillis complet}

Notre première définition d'un treillis peut paraître légèrement plus
artificelle que sa caractérisation algébrique. En effet, dans la version
algébrique, on peut voir un treillis comme un certain bimonoïde, que l'on peut
ensuite étendre en l'existence d'une borne supérieure et d'une borne inférieure
pour toute partie finie (en raisonnant juste par récurrence). A l'inverse,
lorsque l'on part de la définition de treillis comme ensemble ordonné possédant
toutes les bornes sur des parties finies, une question naît immédiatement :
pourquoi ne pas autoriser des parties infinies ? C'est pour intégrer cette
possibilité que nous introduisons la notion de treillis complet.

\begin{definition}[Treillis complet]
  Un treillis complet est un ensemble ordonné $(X,\leq)$ tel que toute partie
  $Y\subseteq X$ possède à la fois une borne supérieure et une borne inférieure.
\end{definition}

Un treillis complet est, par définition, plus fort qu'un treillis. Dans la
pratique, on trouve effectivement plusieurs treillis qui ne sont pas complets
(le fait d'avoir deux notions n'est donc pas une distinction simplement formelle
mais revêt une vraie différence conceptuelle). Remarquons aussi qu'un treillis
complet peut se caractériser par l'existence d'une seule borne.

\begin{property}
  L'ensemble ordonné $(X,\leq)$ est un treillis complet si et seulement si toute
  partie possède une borne supérieure.
\end{property}

\begin{proof}
  Un sens est évident. Pour l'autre, supposons que toute partie de $X$ possède
  une borne supérieure, et montrons alors qu'elle possède une borne inférieure.

  Soit $Y\subseteq X$, on définit $Y\convcal$ par
  \[Y\convcal \defeq \{x\in X \mid \forall y \in Y, x \leq y\}\]
  et soit $y_0$ la borne supérieure de $Y\convcal$. Montrons que $y_0$ est la
  borne inférieure de $Y$. Tout d'abord, si $y\in Y$, alors par définition,
  $y$ est un majorant de $Y\convcal$, donc puisque $y_0$ est le plus petit
  majorant, $y_0$ est un minorant de $Y$. De plus, si $y$ est un minorant
  de $Y$, cela signifie que $y\in Y\convcal$, donc comme $y_0$ est un majorant de
  $Y\convcal$, on en déduit que $y\leq y_0$. Ainsi $y_0$ est le plus grand
  minorant de $Y$ : c'est la borne inférieure de $Y$.
\end{proof}

On peut maintenant prouver l'un des théorèmes importants que nous avons présenté
dans le \cref{chp.induction} : le \cref{thm.ind.KT}, mais dans une version bien
plus forte et générale.

\begin{theorem}[Knaster-Tarski]
  Soit $(X,\leq)$ un treillis complet et $f : X \to X$ une fonction croissante.
  Alors l'ensemble
  \[\fix(f) \defeq \{x\in X \mid f(x) = x\}\]
  est un treillis complet pour l'ordre induit par $\leq$.
\end{theorem}

\begin{proof}
  On définit tout d'abord l'ensemble des pré points fixes de $f$ :
  \[\prefix(f) \defeq \{x\in X\mid f(x)\leq x\}\]
  et on montre qu'il est un treillis complet pour $\leq$. Soit une partie
  $Y\subseteq\prefix(f)$, montrons que $Y$ possède une borne inférieure. Nous
  prenons pour candidat de la borne inférieure de $Y$ dans $\prefix(f)$, la
  borne inférieure $\bigwedge Y$ dans $X$ : montrons donc que $\bigwedge Y$ est
  bien un pré point fixe de $f$ (cela suffira à montrer qu'on a une borne
  inférieure, puisque la borne inférieure dans $Y$ est aussi la borne inférieure
  dans $\prefix(f)$, à partir du moment où la borne appartient effectivement à
  $\prefix(f)$).

  On montre que $f\big(\bigwedge Y\big) \leq \bigwedge Y$. Pour cela, il nous
  suffit de montrer que $f\big(\bigwedge Y\big)$ est un minorant de $Y$, puisque
  l'on sait que $\bigwedge Y$ est le plus grand minorant. Soit $y\in Y$, on
  remarque que $\bigwedge Y \leq y$, donc $f\big(\bigwedge Y\big) \leq f(y)$,
  mais comme par définition $f(y)\leq y$ et par transitivité, on en déduit que
  $f\big(\bigwedge Y\big) \leq y$. Ainsi
  $f\big(\bigwedge Y\big) \leq \bigwedge Y$.

  On sait donc que toute partie de $\prefix(f)$ admet une borne inférieure. Par
  la propriété précédente (son dual, plus précisément), $\prefix(f)$ est donc
  un treillis complet pour $\leq$. On a donc un treillis complet
  $(\prefix(f),\leq)$ et une fonction $f : \prefix(f) \to\prefix(f)$. En prenant
  le dual, on obtient donc que $\postfix(\prefix(f))$ est un treillis complet.

  Montrons maintenant que $\postfix(\prefix(f)) = \fix(f)$. Par définition,
  \[\postfix(\prefix(f)) = \{x\in \{x\in X \mid f(x)\leq x\} \mid x \leq f(x)\}\]
  ce qui correspond exactement à l'ensemble
  $\{x\in X\mid (f(x) \leq x) \land (x \leq f(x))\}$. Comme $\leq$ est
  transitif, cet ensemble est $\fix(f)$. Ainsi $\fix(f)$ est un treillis complet.
\end{proof}

\begin{remark}
  On en déduit donc, en particulier, qu'il existe un plus petit et un plus grand
  point fixe pour $f$. En rentrant un peu plus en détail dans la démonstration,
  on peut vérifier que le plus petit point fixe est le plus petit pré point fixe,
  et que le plus grand point fixe est le plus grand post point fixe.
\end{remark}

\begin{exercise}
  Montrer que $(\powerset(X),\subseteq)$ est un treillis complet, pour tout
  ensemble $X$. En déduire le \cref{thm.ind.KT}.
\end{exercise}

Enfin, donnons deux mots à propos des algèbre de Heyting complètes. On pourrait
s'attendre à ce qu'un treillis complet $X$ suffise à modéliser ce qu'on attend
d'une algèbre de Heyting complète, étant donné que pour tous $a,b$, l'ensemble
\[\{x \in X \mid a \land x \leq b\}\]
possède une borne inférieure. Ce qui bloque, cependant, est le caractère
distributif : on ne sait pas si le treillis complet est distributif. A la place,
une algèbre de Heyting complète va posséder une version complète de la
distributivité.

\begin{definition}[Algèbre de Heyting complète]
  Un ensemble ordonné $(H,\leq)$ est une algèbre de Heyting complète si
  $(H,\leq)$ est un treillis complet et s'il vérifie l'identité suivante, pour
  toute partie $X\subseteq H$ et tout élément $y \in H$ :
  \[\Bigg(\bigvee_{x\in X}x\Bigg) \land y = \bigvee_{x\in X}(x\land y)\]
\end{definition}

\section{Filtres}

Enfin, présentons la notion de filtre, et la notion duale d'idéal. Nous allons
détailler comment se comportent les filtres dans les structure de plus en plus
fortes : demi-treillis, treillis et algèbre de Boole, puis nous verrons les
ultrafiltres et le lemme de l'ultrafiltre.

\subsection{Définitions et caractérisations}

Commençons par définir ce qu'est un filtre.

\begin{definition}[Filtre]
  Soit $(X,\leq)$ un ensemble ordonné. Un filtre est une partie
  $\mathcal F\subseteq X$ vérifiant :
  \begin{itemize}
  \item $\mathcal F$ est non vide : $\mathcal F \neq\varnothing$.
  \item $\mathcal F$ est clos par le haut : pour tout $x\in \mathcal F$ et
    $y\in X$, si $x\leq y$ alors $y\in \mathcal F$.
  \item tous les éléments de $\mathcal F$ sont compatibles : pour tous
    $x,y\in\mathcal F$, il existe $z\in\mathcal F$ tel que $z\leq x$ et
    $z\leq y$.
  \end{itemize}

  Un filtre $\mathcal F$ est dit propre si $\mathcal F \neq X$.
\end{definition}

Un filtre peut donc être vu comme une description d'une notion de grandeur :
un élément d'un filtre est vu comme un \og grand\fg élément. En plus de cela,
cette notion de grandeur doit avoir une notion de compatibilité : deux grands
éléments doivent avoir en commun un grand élément. La notion duale, qui peut donc
être vue comme la spécification d'une notion de petitesse, est celle d'idéal.

\begin{definition}[Idéal]
  Soit $(X,\leq)$ un ensemble ordonné. Un idéal est une partie
  $\mathcal I \subseteq X$ vérifiant :
  \begin{itemize}
  \item $\mathcal I$ est non vide : $\mathcal I \neq\varnothing$.
  \item $\mathcal I$ est clos par le bas : pour tout $x\in \mathcal I$ et
    $y\in X$, si $y\leq x$ alors $y\in \mathcal I$.
  \item tous les éléments de $\mathcal I$ sont compatibles par le haut :
    pour tous $x,y\in \mathcal I$, il existe $z\in \mathcal I$ tel que
    $x\leq z$ et $y\leq z$.
  \end{itemize}

  Un idéal $\mathcal I$ est dit propre si $\mathcal I \neq X$.
\end{definition}

Nous ne travaillerons par sur les idéaux, mais donnons la définition car celle-ci
reste importante.

\begin{example}
  Soit un ensemble $X$ et un élément $x\in X$. On peut définir le filtre
  principal sur $\powerset(X)$ en $x$, $\mathcal F_x$, par
  \[\mathcal F_x \defeq \{A\subseteq X\mid x \in A\}\]

  On voit que $X\in \mathcal F_x$, que si $A\subseteq B$ et $x\in A$ alors
  $x\in B$, et que si $x\in A$ et $x\in B$ alors $x\in A\cap B$. Ainsi l'ensemble
  ordonné $\powerset(X)$ contient au moins autant de filtres que d'éléments dans
  $X$.
\end{example}

On peut donner une première caractérisation des filtres, dans le cas d'un
inf demi-treillis.

\begin{property}
  Soit $(X,\leq)$ un inf demi-treillis de majorant $\top$ et de borne inf
  $\land$. Alors $\mathcal F\subseteq X$ est un filtre sur $X$ si et seulement si
  les conditions suivantes sont vérifients :
  \begin{itemize}
  \item $\top \in X$.
  \item $\mathcal F$ est clos par le haut.
  \item si $x,y\in \mathcal F$, alors $x\land y\in \mathcal F$.
  \end{itemize}
\end{property}

\begin{proof}
  Supposons que $\mathcal F$ vérifie les propriétés listées. Le fait d'être
  clos par le haut est par définition. Comme $\top\in\mathcal F$,
  $\mathcal F$ est non vide. Si $x,y\in\mathcal F$, alors il existe effectivement
  $z\in \mathcal F$ tel que $z\leq x$ et $z\leq y$, en prenant $z = x\land y$.

  Réciproquement, supposons que $\mathcal F$ est un filtre. Alors, puisqu'il
  possède un élément, disons $x$, et que $x\leq \top$, on en déduit par clôture
  par le haut que $\top\in\mathcal F$. Le fait que $\mathcal F$ est clos par le
  haut est par hypothèse. Soient $x,y\in \mathcal F$, alors par hypothèse il
  existe $z\in \mathcal F$ tel que $z\leq x$ et $z\leq y$. Par définition de la
  borne supérieure, on a donc que $z\leq x\land y$, et par clôture par le haut
  on en déduit que $x\land y \in \mathcal F$.
\end{proof}

Les inf demi-treillis permettent donc de mieux caractériser les treillis. En
quelque sorte, puisqu'ils permettent de délimiter le fait de devenir plus petit,
il est plus aisé d'y définir ce qui est grand (c'est-à-dire ce qui est arrive à
rester grand quand on le rend plus petit). Les sup demi-treillis, eux, ne
permettent pas de mieux définir un treillis, mais permettent de décrire ses
propriétés.

\begin{property}
  Soit $(X,\leq)$ un sup demi-treillis et $\mathcal F \subseteq X$ un filtre.
  Alors $\mathcal F$ est propre si et seulement si $\bot\notin\mathcal F$.
\end{property}

\begin{proof}
  Puisque $\mathcal F$ est clos par le haut, si $\bot\in\mathcal F$ alors tout
  élément est dans $\mathcal F$, donc $\mathcal F$ n'est pas propre (et si
  $\mathcal F$ n'est pas propre, alors il contient $\bot$).
\end{proof}

Les sup demi-treillis permettent de définir la propriété d'être premier :

\begin{definition}[Filtre premier]
  Soit $(X,\leq)$ un sup demi-treillis et $\mathcal F$ un filtre sur $X$. On dit
  que $\mathcal F$ est premier si pour tous $x,y\in X$, si
  $x\lor y \in \mathcal F$ alors $x\in F$ ou $y\in F$.
\end{definition}

On peut voir une analogie ici avec un idéal dans un anneau : un idéal premier
est un idéal tel que, lorsqu'on a un produit $a\times b$ dedans, celui-ci doit
nécessairement provenir du produit par un élément de l'idéal. De la même façon,
ici, on sait naturellement que si $x\in\mathcal F$ alors $x\lor y \in\mathcal F$,
et la notion de primalité signifie que tout élément de la forme $x\lor y$ doit
provenir d'un cas de ce type, où $x$ (ou $y$) était déjà dans $\mathcal F$.

Pour synthétiser les différentes propriétés dans le cas d'un treillis : les
filtres propres (qui sont les filres intéressants) sont les parties closes par
le haut qui possède $\top$ mais pas $\bot$, et qui sont closes par intersection
finie.

Supposons maintenant qu'on possède une partie $X$ (non vide) qui n'est pas un
filtre. Comment trouver un filtre correspondant naturellement à $X$ ?
Premièrement, il nous faut d'abord clore $X$ par le haut : on prend l'ensemble
des éléments supérieurs à au moins un élément de $X$. Il nous faut aussi nous
assurer que pour tout $x,y$ dans notre filtre, un élément inférieur à $x$ et $y$
soit dans le filtre. Le plus simple ici est de considérer un inf demi-treillis :
on veut ajouter tous les $x\land y$, mais comme il est possible d'itérer ces
bornes inférieures, il nous faut ajouter tous les $\bigwedge F$ pour
$F\subfin X$. On obtient ainsi la construction suivante :

\begin{definition}[Filtre engendré par une partie]
  Soit $(X,\leq)$ un inf demi-treillis et $Y\subseteq X$ une partie non vide de
  $X$. On définit alors
  \[\filclose{Y} \defeq \{x\in X \mid \exists F\subfin Y, \bigwedge F \leq x\}\]
\end{definition}

\begin{proposition}
  Soit $(X,\leq)$ un inf demi-treillis et $Y\subseteq X$ une partie non cide.
  Alors $\filclose Y$ est un filtre sur $X$.
\end{proposition}

\begin{proof}
  Tout d'abord, on remarque que $Y\subseteq \filclose Y$, donc $\filclose Y$ est
  non vide. Soit $x\in \filclose Y$ et $y\in X$ tel que $x\leq y$. On trouve
  $F\subfin Y$ tel que $\bigwedge F\leq x$, alors $\bigwedge F \leq y$ par
  transitivité, donc $y\in \filclose F$. Soient $x$ et $y$ des éléments de
  $\filclose Y$, on trouve $F$ et $F'$ des parties finies de $Y$ telles que
  $\bigwedge F \leq x$ et $\bigwedge F' \leq y$. Alors
  \[\bigwedge (F\cup F') = \Bigg(\bigwedge F\Bigg)\land\Bigg(\bigwedge F\Bigg)
  \leq x\land y\]
  donc $x\land y \in \filclose Y$. Ainsi $\filclose Y$ est un filtre.
\end{proof}

Par défaut, le filtre engendré par une partie peut tout à fait contenir tous
les éléments. Prenons par exemple un ensemble $X$, une partie $Y$ et le filte
engendré par $\{Y,X\setminus Y\}$ : il est évident que ce filtre n'est pas
propre. Cependant, il existe une propriété assurant qu'un filtre engendré est
propre, est qui est tout à fait naturelle : on veut que les $\bigwedge F$ soient
tous différents de l'ensemble vide. Evidemment, cette propriété n'a d'intérêt que
dans le cas d'un treillis.

\begin{definition}[Propriété de l'intersection finie]
  Soit $(X,\leq)$ un treillis et $Y\subseteq X$. On dit que $Y$ a la propriété
  de l'intersection finie si
  \[\forall F\subfin Y, \bigwedge F\neq \bot\]
\end{definition}

\begin{proposition}
  Soit $(X,\leq)$ un treillis et $Y\subseteq X$ une partie non vide de $X$.
  Alors $\filclose Y$ est un filtre propre si et seulement si $Y$ a la propriété
  de l'intersection finie.
\end{proposition}

\begin{proof}
  Par définition, $\bot\in \filclose Y$ si et seulement s'il existe $F\subfin Y$
  tel que $\bigwedge F \leq \bot$, mais la condition que $\bigwedge F\leq \bot$
  est équivalente à ce que $\bigwedge F = \bot$, étant donné que l'autre
  inégalité est toujours vérifiée. Ainsi $\bot\in\filclose Y$ si et seulement
  s'il existe $F\subfin Y$ tel que $\bigwedge F = \bot$, donc $\filclose Y$ est
  propre si et seulement si $Y$ possède la propriété de l'intersection finie.
\end{proof}

\begin{exercise}
  Soit $\Sigma$ une signature. On considère $\mathcal T\subseteq\Clos(\Sigma)$
  une théorie. Montrer que $\mathcal T$ est cohérente (c'est-à-dire que
  $\mathcal T\nvdash \bot$) si et seulement si $\mathcal T'$, la projection
  de $\mathcal T$ dans $\mathcal L(\Sigma)$, a la propriété de l'intersection
  finie.
\end{exercise}

\begin{exercise}
  Soit $\Sigma$ une signature. Soit $\mathcal T\subseteq\Clos(\Sigma)$ une
  théorie. Montrer que la projection de $\vclose{\mathcal T}$ dans
  $\mathcal L(\Sigma)$ est le filtre engendré par la projection de $\mathcal T$
  dans $\mathcal L(\Sigma)$.
\end{exercise}

\subsection{Ultrafiltre}\label{sbsct.ultrafiltre}

On définit maintenant la notion d'ultrafiltre, qui est la notion duale de celle
de filtre maximale, souvent rencontrée en algèbre.

\begin{definition}[Ultrafiltre]
  Soit $(X,\leq)$ un ensemble ordonné. On dit qu'un filtre $\mathcal F$ est un
  ultrafiltre s'il est un filtre propre maximal pour l'inclusion, c'est-à-dire si
  pour tout autre filtre propre $\mathcal H$ de $X$, si
  $\mathcal F \subseteq\mathcal H$ alors $\mathcal F = \mathcal H$.
\end{definition}

Dans le cas d'un treillis distributif, on a une coïncidence entre les
ultrafiltres et les filtres premiers.

\begin{proposition}
  Soit $(X,\leq)$ un treillis distributif. Alors un filtre $\mathcal F$ est un
  ultrafiltre si et seulement s'il est un filtre premier.
\end{proposition}

\begin{proof}
  A FAIRE
\end{proof}

On a, de plus, une caractérisation encore meilleure dans le cas d'une algèbre de
Boole.

\begin{proposition}
  Soit $(B,\leq)$ une algèbre de Boole. Un filtre $\mathcal F$ est un ultrafiltre
  si et seulement si on a la propriété suivante, pour tout $x\in B$ :
  \[x\in \mathcal F \iff \lnot x \notin \mathcal F\]
\end{proposition}

\begin{proof}
  Si $\mathcal F$ est un filtre premier, alors comme $\top \in \mathcal F$ et
  $\top = x \lor \lnot x$, on sait que soit $x$ soit $\lnot x$ est dans
  $\mathcal F$. Comme par définition $\bot\notin\mathcal F$ et que
  $x\land\lnot x = \bot$, on en déduit qu'un seul des deux est dans $\mathcal F$,
  ce qui est le résultat qu'on veut démontrer.

  Supposons maintenant que pour tout $x\in B$,
  $x\in\mathcal F \iff \lnot x\notin \mathcal F$. Alors si
  $x\lor y\in \mathcal F$, supposons que $x\notin \mathcal F$ et
  $y\notin\mathcal F$. Alors $\lnot x \in \mathcal F$ et $\lnot y \in \mathcal F$
  grâce à notre hypothèse (et au fait que $\lnot\lnot x = x$). Mais alors
  $\lnot x \land \lnot y \in \mathcal F$, et $x\lor y \in \mathcal F$, donc
  $\lnot (x\lor y)$ et $x\lor y$ sont dans $\mathcal F$ : $\mathcal F$ n'est donc
  par propre. Par l'absurde, on en déduit que soit $x\in\mathcal F$, soit
  $y\in\mathcal F$.
\end{proof}

Pour finir, nous allons démontrer le lemme de l'ultrafiltre, qui est une
conséquence du lemme de Zorn. Il dit que tout filtre propre peut être étendu en
un ultrafiltre.

\begin{theorem}[Lemme de l'ultrafiltre]\label{thm.ultrafilter.lemma}
  Soit $(X,\leq)$ un treillis. Soit $\mathcal F$ un filtre sur $X$. Alors il
  existe un ultrafiltre $\mathcal U$ contenant $\mathcal F$.
\end{theorem}

\begin{proof}
  On définit l'ensemble ordonné
  \[\mathcal X \defeq \{\mathcal H \subseteq X\mid \mathcal H
  \text{ est un filtre propre}\}\]
  muni de l'inclusion ensembliste.

  On veut appliquer le lemme de Zorn sur cet ensemble. Pour cela, on vérifie
  que $\mathcal X$ est inductif.

  Soit $\{\mathcal F_i\}_{i\in I}$ une chaîne d'éléments de $\mathcal X$, montrons
  que $\mathcal F_0 = \displaystyle\bigcup_{i\in I} \mathcal F_i$ est un majorant
  de cette chaîne. Il nous suffit pour cela de vérifier que $\mathcal F_0$ est
  bien un filtre propre :
  \begin{itemize}
  \item soit $i\in I$, comme $\top \in \mathcal F_i$, on en déduit que
    $\top \in \mathcal F_0$.
  \item soit $x\in \mathcal F_0$ et $y\in X$ tel que $x\leq y$. Par définition
    de $\mathcal F_0$, on trouve $i\in I$ tel que $x\in \mathcal F_i$, et comme
    $\mathcal F_i$ est un filtre on en déduit que $y\in \mathcal F_i$, d'où
    $y\in\mathcal F_0$.
  \item soient $x,y\in \mathcal F_0$. On trouve $i,j\in I$ tels que
    $x\in \mathcal F_i$ et $y\in \mathcal F_j$. Comme $\{\mathcal F_i\}$ est une
    chaîne pour l'inclusion, alors soit $\mathcal F_i \subseteq \mathcal F_j$,
    soit $\mathcal F_j\subseteq \mathcal F_i$. Sans perte de généralité, on
    suppose donc que $\mathcal F_i\subseteq \mathcal F_j$, donc que
    $x\in \mathcal F_j$. Comme $\mathcal F_j$ est un filtre, on en déduit que
    $x\land y \in \mathcal F_j$, donc que $x\land y \in \mathcal F_0$.
  \item supposons que $\bot\in \mathcal F_0$. Alors on trouve $i\in I$ tel que
    $\bot\in \mathcal F_i$, mais comme tous les $\mathcal F_i$ sont supposés
    propres, $\bot\notin\mathcal F_i$. Par contradiction, on en déduit que
    $\mathcal F_0$ est propre.
  \end{itemize}

  On applique donc le \cref{thm.Zorn} : on trouve un élément maximal
  $\mathcal H$ de $\mathcal X$ qui est supérieur à $\mathcal F$, c'est-à-dire
  qui contient $\mathcal F$. Il ne nous reste qu'à vérifier que $\mathcal H$ est
  bien un ultrafiltre : si $\mathcal G$ est un filtre propre contenant
  $\mathcal H$, alors il contient $\mathcal F$, donc il appartient à
  $\mathcal X$. Par maximalité de $\mathcal H$ dans $\mathcal X$, on en déduit
  que $\mathcal H = \mathcal G$ : $\mathcal H$ est un ultrafiltre sur $X$
  contenant $\mathcal F$.
\end{proof}

On peut maintenant montrer le \cref{thm.completion} :

\begin{proof}
  Soit une signature $\Sigma$ et une théorie $\mathcal T$ sur $\Sigma$ supposée
  cohérente. On peut
  définir l'algèbre de Lindenbaum-Tarski pour la sémantique :
  \[\mathcal L(\Sigma) \defeq \Formula(\Sigma)/\equiv\]
  (comme nous prouvons un résultat permettant la démonstration du théorème de
  complétude, nous ne pouvons pas utiliser l'équivalence entre $\vdash$ et
  $\vDash$, donc nous construisons cette fois-ci l'algèbre en utilisant la
  relation $\vDash$ dès le début).

  On laisse au lecteur le soin de vérifier que cet ensemble ordonné est bien
  une algèbre de Boole avec les constructeurs induits par
  $\top, \bot, \land, \lor, \lnot$.

  La théorie $\mathcal T$ induit sur $\mathcal L(\Sigma)$ une partie vérifiant
  la propriété d'intersection finie, et la théorie $\vclose{\mathcal T}$ est
  ainsi un filtre propre : on peut donc grâce au \cref{thm.ultrafilter.lemma}
  trouver un ultrafiltre $\mathcal S$ contenant $\vclose{\mathcal T}$.

  Soit alors la théorie $\mathcal T' = \bigcup\mathcal S$, qui contient donc
  toutes les formules appartenant à une classe d'équivalence de $\mathcal S$.
  On remarque que $\mathcal T'$ est close par $\vDash$, donc puisque
  $\overline\bot \notin\mathcal S$, cela signifie que $\bot\notin\mathcal T'$.
  Il nous reste à vérifier que $\mathcal T'$ est complète : pour toute
  formule $\varphi$, on sait que $\varphi\lor\lnot\varphi \in \mathcal T'$, donc
  $\overline{\varphi}\lor\overline{\lnot\varphi}\in\mathcal S$, mais
  comme $\mathcal S$ est un ultrafiltre, il est en particulier premier. On
  en déduit que soit $\overline\varphi \in \mathcal S$, soit
  $\overline{\lnot\varphi}\in \mathcal S$ : dans le premier cas,
  $\varphi\in \mathcal T'$ et dans le deuxième cas,
  $\lnot\varphi\in \mathcal T'$. Ainsi pour toute formule $\varphi$, soit
  $\varphi\in \mathcal T'$ soit $\lnot\varphi\in \mathcal T'$ (c'est donc en
  particulier encore le cas pour $\vclose{\mathcal T'} = \mathcal T'$).

  Donc il existe une théorie complète $\mathcal T'$ qui contient la théorie
  $\mathcal T$.
\end{proof}
