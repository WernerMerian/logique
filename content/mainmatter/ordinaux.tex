\chapter{Ordinaux, cardinaux, cofinalité}
\label{chp.ordinaux}

\minitoc

Dans ce chapitre, on s'intéresse aux notions les plus incontournables de la
théorie des ensembles que sont les ordinaux, les cardinaux et la cofinalité.
Ces trois notions sont bâties l'une sur la suivante~: les cardinaux sont une
famille spécifique d'ordinaux et la cofinalité désigne une fonction retournant
des cardinaux.

Moralement, les ordinaux forment une hiérarchie d'ensembles canoniques. La
classe des ordinaux est une classe existant dans tout modèle de ZF, et se
comporte particulièrement bien. Cette classe nous permet de comparer des bons
ordres, mais nous verrons qu'elle revêt une importance plus particulière dans le
cadre de la théorie des ensembles~: elle est une classe ordonnée arbitrairement
grande, et permet donc de considérer des représentants pour des phénomènes
ensemblistes arbitrairement grands. De plus, c'est une classe bien ordonnée,
ce qui signifie qu'on a un principe de choix sur cette classe (si on a un
ensemble de parties d'un bon ordre, on peut prendre le minimum de chaque partie
dans l'ensemble), nous permettant en choisissant des minima d'obtenir un
représentant canonique.

Les cardinaux forment l'exemple le plus évident d'un tel procédé~: par défaut,
on imagine un cardinal comme une classe d'équipotence, c'est-à-dire comme
l'univers ensembliste $\mathcal U$ quotienté par la relation d'équivalence
\[X\sim Y \defeq \exists f : X \to Y, \mathrm{bij}(f)\]
Le souci, évidemment, est que ces classes ne sont pas des ensembles. Pour le
cardinal $1$, on peut par exemple prendre n'importe quel ensemble $X$ et
considérer $\{X\}$, montrant que tout l'univers $\mathcal U$ peut être injecté
dans la classe des ensembles de cardinal $1$. La solution adoptée en théorie des
ensembles est de choisir, pour chaque classe, un représentant canonique.
Plutôt que de définir ces classes d'équipotence, on remarque que chaque classe
rencontre la hiérarchie des ordinaux, et on prend alors l'ordinal minimum qui
appartient à la classe. Il reste un souci dans cette définition~: est-on sûrs
que la classe des cardinaux rencontre la hiérarchie des ordinaux~? Si l'on
admet $\AxC$, c'est le cas, comme nous le verrons, mais on traitera dans ce
chapitre plus en détail des implications de l'axiome du choix (on prouvera
en particulier lemme de Zorn).

Enfin, la cofinalité est une notion, différente de la cardinal, exprimant à quel
point un ordinal (ou un cardinal) a besoin d'un grand nombre d'étapes pour être
parcouru. L'idée de la cofinalité est assez technique et demande d'être
manipulée pour être pleinement assimilée. Nous verrons ce que sont les cardinaux
réguliers et singuliers, et étudierons quelques conséquences comme le théorème
de König.

\section{Ordinaux}

Commençons par donner une présentation informelle motivant la définition des
ordinaux.

\subsection{Description des ordinaux}

On rappelle qu'un bon ordre est un ordre tel que toute partie admet un minimum.
La classe des bons ordres est ainsi la classe des ensembles ordonnés $(X,<)$
où $<$ est un bon ordre sur $X$. On a par exemple $\{1\}$ et $\{2\}$ qui sont
deux bons ordres (avec le seul ordre possible sur un seul élément)~: ils sont
donc isomorphes, et représentent le même ordre, en ignorant l'étiquette des
éléments ordonnés eux-mêmes.

Dans le but de décrire efficacement la classe des bons ordres, il nous faut donc
trouver un moyen de représenter canoniquement un ensemble bien ordonné. Tout
d'abord, pour un ensemble $X$ quelconque, il peut exister un nombre considérable
de bons ordres (souvent infini), il nous faut donc fixer la relation pour un
ensemble donné. En repensant au lemme de Mostovski, on voit qu'on peut
représenter certains ensembles munis de relation par un ensemble muni de la
relation $\in$ et le besoin d'expliciter la relation est alors éliminé, puisque
la relation $\in$ existe dans la théorie même.

On souhaite donc représenter un ensemble bien ordonné $(X,<)$ par un ensemble
$X'$ pour lequel le bon ordre est $\in$. Comme nous l'avons vu pour le lemme
d'effondremment de Mostovski, un tel représentant existe bien et est unique si
l'on choisit le représentant comme étant transitif. Ainsi tout ensemble bien
ordonné est équivalent (on le montrera) à un unique ensemble transitif bien
ordonné pour la relation $\in$. Telle est donc notre définition d'ordinal.

\begin{definition}[Ordinal]
  On dit qu'un ensemble $\alpha$ est un ordinal si~:
  \begin{itemize}
  \item $\alpha$ est transitif, c'est-à-dire que
    $\forall x\in \alpha, \forall y \in x, y\in \alpha$
  \item $\alpha$ est bien ordonné pour $\in$, c'est-à-dire que $\in$ est un
    ordre sur $\alpha$~:
    \[(\forall x\in \alpha, x\notin \alpha)\land (\forall x,y,z\in \alpha,
    x\in y \land y \in z \implies x \in z)\]
    et que toutes ses parties ont un minimum~:
    \[\forall X\subseteq \alpha, \exists x \in \alpha, \forall y \in X,
    (x=y \lor x \in y)\]
  \end{itemize}

  On note $\Ord$ la classe des ordinaux, qui peut donc être définir par la
  conjonction des formules données ci-dessus.
\end{definition}

\begin{remark}
  On peut aussi décrire le fait d'être un bon ordre par le fait d'être total
  et d'être un ordre bien fondé. La propriété d'être bien fondée, en présence
  de l'axiome de fondation, est automatiquement vérifiée~: un ordinal est alors
  un ensemble transitif pour lequel la relation $\in$ est transitive et totale,
  c'est-à-dire que $x\in y \lor x = y \lor x\ni y$. Même si l'axiome de
  fondation simplifie les preuves, on décide de travailler pour l'instant dans
  $\ZF - \AxF$, pour montrer que toute la construction des ordinaux est possible
  avec peu d'axiomes.
\end{remark}

\begin{example}
  Donnons quelques cas pratiques d'ordinaux~:
  \begin{itemize}
  \item $0$ est un ordinal.
  \item Plus généralement, tout entier $n = \{0,\ldots,n-1\}$ est un ordinal.
  \item $\omega$, qui est le nom qu'on utilise en théorie des ensembles pour
    l'ensemble qu'on a décrété être $\mathbb N$, est aussi un ordinal. Les
    différents axiomes du fait d'être un ordinal peuvent se prouver par
    récurrence. C'est aussi le plus petit ordinal infini, puisque pour tout
    $n\in \omega$, $n$ est fini (nous donnerons plus tard une définition d'être
    un ordinal fini).
  \end{itemize}
\end{example}

\begin{notation}
  On désignera un ordinal par une lettre grecque du début de l'alphabet~:
  $\alpha,\beta,\gamma,\delta,\ldots$
\end{notation}

On donne maintenant trois propriétés basiques sur les ordinaux~: $\Ord$ est
stable par $S$, par intersection et l'inclusion entre ordinaux est équivalente
à l'appartenance ou l'égalité.

\begin{proposition}
  Soit $\alpha$ un ordinal, alors $S\;\alpha = \alpha\cup\{\alpha\}$ est
  aussi un ordinal.
\end{proposition}

\begin{proof}
  A FAIRE
\end{proof}

\begin{proposition}
  Soient $\alpha$ et $\beta$ deux ordinaux. Alors $\alpha\cap \beta$ est aussi
  un ordinal.
\end{proposition}

\begin{proof}
  A FAIRE
\end{proof}

\begin{proposition}
  Soient $\alpha$ et $\beta$ deux ordinaux. On suppose que
  $\alpha \subseteq \beta$, alors $\alpha \in \beta$ ou $\alpha = \beta$.
\end{proposition}

\begin{proof}
  On suppose que $\alpha \subsetneq \beta$, montrons qu'alors
  $\alpha \in \beta$. On sait donc que $\beta\setminus\alpha$ est une partie
  non vide, soit $z = \min \beta\setminus\alpha$. On montre maintenant que
  $z = \alpha$ par double inclusion.

  Si $x\in \alpha$, alors $x\notin\beta\setminus\alpha$ donc $x\neq z$. Montrons
  que $x\in z$. Supposons que $z\in x$. Alors, comme $\alpha$ est transitif,
  $z\in \alpha$, mais $z\in\beta\setminus\alpha$, ce qui est absurde. Donc
  $x\in z$, donc $\alpha\subseteq z$.

  Si $x\in z$, alors $x\notin\beta\setminus\alpha$ (car $z$ est le minimum de
  cet ensemble), donc $x\in\alpha$.

  Donc $\alpha = z$ et $z\in \beta$, donc $\alpha \in \beta$.
\end{proof}

La classe $\Ord$ est elle-même bien ordonnée pour $\in$, ce que l'on prouve dans
la proposition suivante.

\begin{proposition}
  Les propositions suivantes sont vraies~:
  \begin{enumerate}
  \item si $\alpha \in \beta \in \Ord$ alors $\alpha \in \Ord$.
  \item si $X\subseteq \Ord$, alors $\bigcap X \in \Ord$.
  \item $\Ord$ n'est pas un ensemble.
  \end{enumerate}
\end{proposition}

\begin{proof}
  On vérifie les différentes propositions~:
  \begin{enumerate}
  \item A FAIRE
  \item A FAIRE
  \item Supposons que $\Ord$ est un ensemble. Alors, par les propositions
    précédentes, c'est un ordinal~: on en déduit que $\Ord\in\Ord$, or par
    hypothèse, $\Ord\notin\Ord$.
  \end{enumerate}
\end{proof}

\begin{notation}
  Comme les ordinaux appartiennent à une classe elle-même ordonnée, on écrira
  souvent $\alpha < \beta$ ou $\alpha \leq \beta$ plutôt que $\alpha \in \beta$
  ou $\alpha \subseteq \beta$.
\end{notation}

\begin{property}
  Soit $X\subseteq\Ord$ un ensemble, alors $\bigcup X \in \Ord$.
\end{property}

\begin{proof}
  On vérifie les axiomes d'un ordinal~:
  \begin{itemize}
  \item si $x\in y\in \bigcup X$ alors on trouve $\alpha \in X$ tel que
    $x\in y \in \alpha$, et comme $\alpha$ est transitif, $x\in \alpha$, donc
    $x\in \bigcup X$.
  \item si $x\in X$ alors on trouve $\alpha \in X$ tel que $x\in \alpha$, donc
    comme $\alpha$ est un ordinal, $x\notin x$.
  \item si $x\in y\in z$ sont trois éléments de $\bigcup X$, alors on trouve
    $\alpha,\beta,\gamma\in X$ tels que $x\in \alpha,y\in \beta,z\in\gamma$.
    Comme $\Ord$ est bien ordonnée, on trouve un maximum entre $\alpha$,
    $\beta$ et $\gamma$, disons $\alpha$ quitte à renommer nos ordinaux.
    Il vient donc que $x,y,z\in\alpha$, donc que $x\in z$ comme $\in$
    est un ordre sur $\alpha$.
  \item soit $Y\subseteq \bigcup X$ une partie non vide de $X$. On trouve par
    hypothèse un élément $x\in Y$, et donc $\alpha \in X$ tel que $x\in\alpha$.
    Alors $Y\cap \alpha$ a un plus petit élément, notons-le $m$. Montrons que
    $m$ est un minorant de $Y$. Soit $y\in Y$, par hypothèse on trouve
    $\beta\in X$ tel que $y\in \beta$. Comme $y\in \beta$, $y$ est un ordinal,
    donc soit $y\subseteq \alpha$, soit $\alpha \in y$. Dans le premier cas,
    si $y\in \alpha$ alors $y\in Y \cap \alpha$ donc $m \leq y$ par définition,
    et si $y=\alpha$ alors $m\in Y \cap \alpha$ donc $m \leq \alpha$.
    Supposons maintenant que $\alpha \in y$. Alors on sait que
    $m\in \alpha\in y$, donc $m\in y$, d'où $m\leq y$.
  \end{itemize}

  $\bigcup X$ est donc un ordinal.
\end{proof}

Donnons maintenant un outil efficace pour prouver qu'un ensemble est un ordinal.

\begin{proposition}
  Soit $\alpha$ un ordinal et $Y\subseteq \alpha$ une partie close par le bas,
  c'est-à-dire telle que si $x\in Y$, $y\in \alpha$ et $y\in x$ alors
  $y\in \alpha$, alors $Y$ est un ordinal (on dit aussi que $Y$ est un segment
  initial de $\alpha$).
\end{proposition}

\begin{proof}
  A FAIRE
\end{proof}

Comme nous l'avons dit, le lemme de Mostovski appliqué aux ordinaux nous
permet de définir le représentant canonique d'un bon ordre.

\begin{proposition}
  Soit $(X,<)$ un bon ordre. Alors il existe un unique ordinal $\alpha$ et un
  unique isomorphisme $\pi : (X,<)\cong (\alpha,\in)$.
\end{proposition}

\begin{proof}
  On applique le lemme d'effondrement de Mostovski à $(X,<)$ qui est un ensemble
  muni d'une relation bien fondée. On trouve donc un unique ensemble $\alpha$
  transitif et un unique isomorphisme $\pi : (X,<) \cong (\alpha,\in)$. Comme
  $(\alpha,\in)$ est isomorphe à $(X,<)$, on en déduit directement que c'est
  un bon ordre. $\alpha$ est donc un ordinal, et tout autre ordinal isomorphe
  à $\alpha$ est alpha lui-même puisque ce sont deux ensembles transitifs
  bien fondés pour $\in$ (l'isomorphisme est donc l'identité).
\end{proof}

\begin{corollary}
  Soient deux ordinaux $\alpha,\beta$ tels qu'on a une fonction strictement
  croissante $f : \alpha \to \beta$ et une fonction strictement croissante
  $g : \beta \to \alpha$, alors $\alpha = \beta$ et $f = g = \id_\alpha$.
\end{corollary}

\begin{definition}
  Soit $(X,<)$ un bon ordre. On appelle type de $(X,<)$, souvent abrégé en
  type de $X$, que l'on note $\type(X)$ l'unique ordinal $\alpha$ tel que
  $(X,<)\cong (\alpha,\in)$.
\end{definition}

On a vu que la classe $\Ord$ contenait $\omega$, mais elle s'étend au-delà, déjà
en considérant $S\;\omega$. De même, on peut considérer $\omega + \omega$, que
l'on peut définir par
\[\omega + \omega \defeq \bigcup_{n \in \omega} S^n\;\omega\]
ou d'autres tels ordinaux toujours plus grands. Nous verrons plus tard
l'arithmétique ordinale, permettant de donner un sens plus précis à des
expressions telles que $\omega + \omega$, mais nous allons d'abord étudier les
différentes sortes d'ordinaux existants.

L'ordinal $0$ a un rôle particulier, puisqu'il est le plus petit ordinal
existant. A part lui, les ordinaux sont de deux sortes~: les ordinaux
successeurs et les ordinaux limites.

\begin{definition}[Ordinal successeur]
  Un ordinal $\alpha$ est dit successeur s'il existe $\beta \in \Ord$ tel que
  $\alpha = S\;\beta$.
\end{definition}

\begin{definition}[Ordinal limite]
  Un ordinal $\lambda$ est dit limite s'il est non nul et vérifie
  $\lambda = \sup_{\beta < \lambda} \beta$.
\end{definition}

\begin{property}
  Tout ordinal est soit $0$, soit un successeur, soit un ordinal limite, chaque
  cas étant exclusif.
\end{property}

\begin{proof}
  Le fait que $0$ n'est ni successeur ni limite découle de la définition.
  Montrons qu'un ordinal ne peut être à la fois limite et successeur. Soit
  $S\;\alpha$ un ordinal successeur, alors
  \[\bigcup_{\beta < S\;\alpha} \beta =
  \{\gamma \in \beta \mid \beta \in\alpha \cup \{\alpha\}\}\]
  par transitivité, si $\gamma\in\beta\in\alpha$ alors $\gamma\in \alpha$, et
  si $\beta \in\{\alpha\}$ alors $\gamma\in \alpha$ par définition. Dans l'autre
  sens, si $\beta \in \alpha$ alors $\beta\in\alpha\in\{\alpha\}$. Donc
  $\bigcup (S\;\alpha) = \alpha$. Ainsi un ordinal successeur ne peut pas être
  limite.

  Montrons que tout ordinal est soit $0$, soit un successeur, soit un ordinal
  limite. Supposons qu'il existe $\alpha$ qui n'est aucun des trois. On définit
  $\beta = \bigcup \alpha$. Comme $\alpha$ est transitif (c'est un ordinal),
  $\beta \subseteq \alpha$, donc $\beta \in \alpha$ ou $\beta = \alpha$.
  Si $\beta = \alpha$ alors $\alpha$ est limite, donc $\beta \in \alpha$.
  Montrons que $\alpha = S\;\beta$. Pour ça, montrons que le seul élément de
  $\alpha\setminus \beta$ est $\beta$. On sait que
  $\beta \in \alpha \setminus\beta$ car $\beta\notin\beta$. Soit
  $\gamma\in\alpha\setminus\beta$, montrons que $\gamma = \beta$.
  Comme $\gamma\notin\beta$, $\gamma \geq \beta$, montrons que
  $\gamma\leq\beta$. Comme $\gamma \in \alpha$, on en déduit que si
  $\xi\in\gamma$ alors $\xi \in \alpha$, donc $\gamma\subseteq \beta$. Donc
  $\gamma = \beta$, donc $\alpha = S\;\beta$.
\end{proof}

\subsection{Induction transfinie}

La structure de $\Ord$ ressemble fortement à celle des entiers, mais avec
l'ajout d'ordinaux limites. On s'attend donc à avoir un principe d'induction et
de récursion proche de celui de nos ensembles inductifs du \cref{chp.induction}.
C'est en effet le cas, et nous le donnerons sous deux formes~: avec l'induction
de bon ordre et avec la distinction des sortes d'ordinaux.

\begin{theorem}[Induction transfinie]
  Soit $\varphi(x_0,\ldots,x_n,y)$ une formule dans le langage des ensembles.
  On a alors la propriété
  \begin{multline*}
    \forall x_0,\ldots,x_n,(\forall \alpha\in\Ord,
    (\forall \beta < \alpha, \varphi(x_0,\ldots,x_n,\beta))\implies
    \varphi(x_0,\ldots,x_n,\alpha))\\
    \implies \forall \alpha \in \Ord, \varphi(x_0,\ldots,x_n,\alpha)
  \end{multline*}
\end{theorem}

\begin{proof}
  Soient $x_0,\ldots,x_n$ des ensembles. Supposons que pour tout
  $\alpha\in\Ord$, on a
  \[(\forall \beta < \alpha, \varphi(x_0,\ldots,x_n,\beta))\implies
  \varphi(x_0,\ldots,x_n,\alpha)\]
  Supposons par l'absurde qu'il existe $\alpha \in \Ord$ tel que
  $\lnot\varphi(x_0,\ldots,x_n,\alpha)$. On peut alors prendre un tel $\alpha$
  minimal~: on en déduit donc que pour tout $\beta < \alpha$,
  $\varphi(x_0,\ldots,x_n,\beta)$ est vraie, donc par hypothèse
  $\varphi(x_0,\ldots,x_n,\alpha)$ est vraie aussi, ce qui est une
  contradiction.
\end{proof}

Ce résultat est cependant souvent plus pratique à utiliser une fois combiné à la
disjonction de cas sur ce qu'est un ordinal.

\begin{proposition}[Induction transtfinie, variante]
  Soit $\varphi(x_0,\ldots,x_n,y)$ une formule dans le langage des ensembles.
  Soient $x_0,\ldots,x_n$ des ensembles. On suppose que~:
  \begin{itemize}
  \item $\varphi(x_0,\ldots,x_n,0)$ est vérifiée.
  \item pour tout ordinal $\alpha$, si $\varphi(x_0,\ldots,x_n,\alpha)$ alors
    $\varphi(x_0,\ldots,x_n,S\;\alpha)$.
  \item pour tout ordinal limite $\lambda$, si
    $\forall \beta < \lambda, \varphi(x_0,\ldots,x_n,\beta)$ alors
    $\varphi(x_0,\ldots,x_n,\lambda)$.
  \end{itemize}

  Alors $\varphi(x_0,\ldots,x_n,\alpha)$ est vraie pour tout ordinal $\alpha$.
\end{proposition}

\begin{proof}
  Par l'absurde, supposons que la classe
  \[\{\alpha \in\Ord\mid \mathcal \lnot\varphi(x_0,\ldots,x_n,\alpha)\}\]
  n'est pas vide. On trouve alors un minimum à cette classe, qu'on note
  $\alpha$. On sait que $\alpha \neq 0$ d'après nos hypothèses. De plus, si
  $\alpha$ est successeur, alors $\alpha = \beta$ et
  $\varphi(x_0,\ldots,x_n,\beta)$ est vérifiée, donc par hypothèse
  $\varphi(x_0,\ldots,x_n,\alpha)$, ce qui est absurde. Si $\alpha$ est limite,
  alors pour tout $\beta < \alpha$, $\varphi(x_0,\ldots,x_n,\beta)$ est
  vérifiée, d'où encore une absurdité.
  Ainsi la classe est vide, donc tout ordinal vérifie la formule.
\end{proof}

Notre principe d'induction est donc généralisé à des ensembles bien plus grands
que $\omega$. On va de même montrer un principe de récursion. Dans le cas des
entiers, on peut définir une suite $u_n$ à un rang $n$ donné à partir de $n$ et
de $u_{n-1}$. Pour un ordinal, on étend cette définition avec le cas d'un
ordinal limite, pour lequel $f(\lambda)$ est définie à partie de $f(\beta)$ pour
$\beta < \lambda$.

Pour préciser ces notions, nous allons introduire un peu de vocabulaire.

\begin{definition}[Fonction inductive pour une relation]
  Soit $H(x,y)$ une relation fonctionnelle à deux variables libres. Soit
  $\alpha$ un ordinal. On dit qu'une fonction $f : \alpha \to X$ est inductive
  pour $H$ si pour tout $\beta < \alpha$, on a~:
  \begin{itemize}
  \item $f\restr\beta$ est dans le domaine de $H$.
  \item $f(\beta) = H(f\restr{\beta})$.
  \end{itemize}
\end{definition}

\begin{remark}
  On ne s'intéresse pas à l'ensemble d'arrivée de $f$, puisque par le schéma de
  remplacement on peut toujours s'assurer que $f$ va dans $\im(f)$ qui est bien
  un ensemble.
\end{remark}

\begin{example}
  On va définir l'identité sur $\Ord$ par récursion. Pour cela, on définit
  $H_{\id}$~:
  \[H_{\id}(x,y) \defeq y = \{x(\alpha) \mid \alpha < x\}\]

  On peut alors vérifier que $\id_\alpha$ est inductive pour $H_{\id}$~:
  \begin{itemize}
  \item $\id_{\alpha}\restr{\beta}$ est dans le domaine de $H_{\id}$ pour tout
    $\beta < \alpha$.
  \item pour $\beta < \alpha$,
    $\id_\alpha(\beta) = \{\gamma\mid \gamma < \beta\}$ et
    $H(f\restr{\beta}) = \{\id_\beta(\gamma)\mid \gamma < \beta\}$ d'où
    l'égalité.
  \end{itemize}
\end{example}

On peut alors utiliser une relation $H$ pour définir une fonction dont le
domaine est $\Ord$~: l'image de $\alpha\in\Ord$ sera $H(f\restr{\alpha})$, pour
obtenir une unique fonction $H$-inductive. Pour assurer le bon fonctionnement
du procédé, on prouve d'abord qu'il existe au plus une fonction inductive pour
$H$ dont le domaine est $\alpha$.

\begin{lemma}
  Soit $H$ une relation fonctionnelle et $\alpha$ un ordinal. Alors il existe
  au plus une fonction $f : \alpha \to X$ qui est inductive pour $H$.
\end{lemma}

\begin{proof}
  Supposons qu'il existe $f,g$ deux fonctions inductives pour $H$. Par
  l'absurde,Supposons que $f\neq g$, soit alors $\beta$ l'ordinal minimal tel
  que $f(\beta)\neq g(\beta)$. Comme $\beta$ est minimal, on en déduit que
  $f\restr{\beta}=g\restr{\beta}$, donc par le fait que $f$ et $g$ sont
  inductives pour $H$, il vient que $f(\beta) = g(\beta)$, ce qui est une
  contradiction.
\end{proof}

On veut maintenant prouver qu'étant donnée une relation fonctionnelle $H$,
il existe effectivement une fonction $H$-inductive de domaine $\alpha$.

\begin{lemma}
  Soit $H$ une relation fonctionnelle dont le domaine est stable, c'est-à-dire
  que si $H(x,y)$ alors $y$ est lui-même dans le domaine. Pour tout
  $\alpha \in \Ord$, il existe une fonction $f : \alpha \to X$ qui est inductive
  pour $H$.
\end{lemma}

\begin{proof}
  Par l'absurde, prenons un $\alpha$ minimal tel que $f : \alpha \to X$ n'est
  pas définie (pour $f$ étant $H$-inductive). Comme $\alpha$ est minimal, on
  sait que pour tout $\beta < \alpha$, il existe une fonction
  $f_\beta :\beta\to X$ qui est inductive pour $H$. On définit alors
  \[\begin{array}{ccccc}
  f & : & \alpha & \longrightarrow & X\\
  & & \beta & \longmapsto & H(f_\beta)
  \end{array}\]
  On remarque, comme chaque $f_\beta$ est $H$-inductive, et par unicité d'une
  fonction $H$-inductive, que $f_\beta\restr{\gamma} = f_\gamma$ pour tous
  $\gamma < \beta < \alpha$.

  On vérifie que $f$ est bien inductive. Soit $\beta < \alpha$, alors~:
  \begin{itemize}
  \item $f\restr{\beta} = f_\beta$ est $f_\beta$ est inductive, donc elle
    est dans le domaine de $H$.
  \item $f(\beta) = H(f\restr{\beta}) = H(f_\beta)$ est vérifié par définition.
  \end{itemize}

  On en déduit donc qu'il existe une fonction $f : \alpha \to X$ inductive pour
  $H$.
\end{proof}

On en déduit le théorème de récursion.

\begin{theorem}[Principe de récursion transfinie]\label{thm.recurs.transf}
  Soit $H$ une relation fonctionnelle telle que pour tout $x\in \dom(H)$,
  $H(x)\in\dom(H)$. Il existe alors une unique fonction $f$ de $\Ord$ dans
  $\im(H)$ telle que $f\restr{\alpha}$ est $H$-inductive pour tout $\alpha$.
\end{theorem}

\begin{proof}
  On définit une fonction entre classes, ce qui est un abus de langage si l'on
  considère une fonction comme un ensemble. Le sens de cette assertion est qu'il
  existe un prédicat $\varphi$ fonctionnel dont le domaine est $\Ord$ et qui
  vérifie l'hypothèse que l'on veut. Pour écrire ce prédicat, il nous suffit
  d'utiliser la définition précédente~:
  \[\varphi(x,y) \defeq \exists (f : x \to X),
  \mathrm{ind}_H(f)\land H(f) = y\]

  On a vérifié précédemment que la fonction définie de cette façon est bien
  $H$-inductive, d'où l'existence.

  L'unicité de la fonction est donnée par le fait qu'une fonction $f$ qui est
  $H$-inductive de domaine $\alpha$ est unique, donc que $\varphi(\alpha)$ est
  unique.
\end{proof}

Remarquons que si l'on a la fonction $f\restr{\alpha}$, alors on peut récupérer
le $\alpha$ correspondant~:
\[\alpha = \bigcup \dom(f\restr{\alpha})\]
ce qui, pour construire une relation fonctionnelle $H$, signifie qu'on peut
définir le rang auquel on est dans la construction de l'image par la fonction
$H$-inductive associée en remplaçant $f\restr{\alpha}$ par l'argument de $H$.
On va appeler $\alpha_x$ l'ordinal $\bigcup\dom(x)$.

Comme on sait de plus qu'il existe une trichotomie sur les ordinaux suivant
s'ils sont successeurs ou limites, en utilisant les formules
\[\mathrm{estZero}(x) \defeq x = 0\qquad
\mathrm{estSucc}(x)\defeq \exists y. x = S\;y \qquad
\mathrm{estLimite}(x)\defeq x \neq 0 \land \bigcup x = x\]
et l'astuce précédente permettant de récupérer le rang de l'argument, on peut
donner une fonction $f : \Ord \to \mathcal U$ par trois données~:
\begin{itemize}
\item un ensemble $X_0$.
\item une relation fonctionnelle $H_s$ qui, à chaque ensemble $X_\alpha$ associe
  un nouvel ensemble $X_{S\;\alpha}$.
\item une relation fonctionnelle $H_\ell$ qui, si on a déjà défini $X_\beta$ pour
  tout $\beta < \lambda$ où $\lambda$ est un ordinal limite, construit un nouvel
  ensemble $X_\lambda$.
\end{itemize}

Si on souhaite expliciter la formule, celle-ci est alors
\begin{multline*}
  H(x,y) \defeq (\mathrm{estZero}(\alpha_x)\land y = X_0)\lor
  (\mathrm{estSucc}(\alpha_x)\land H_S(x(\alpha_x),y))\lor\\
  (\mathrm{estLimite}(\alpha_x)\land H_\ell(x,y))
\end{multline*}

On peut aussi généraliser notre \cref{thm.recurs.transf} en permettant la
définition de fonction $\Ord\times\Ord\to \mathcal U$. \'Etant donnée une
relation fonctionnelle $H$, on peut définir une fonction
$f : \Ord \to \mathcal U$ telle que $f\restr{\alpha}$ est une fonction. Soit
maintenant un énoncé fonctionnel $H$ à deux paramètres, disons $x$ et $\alpha$.
Pour tout $\alpha\in\Ord$, il est possible de définir la relation fonctionnelle
$H$ en fixant $\alpha$, donnant alors une fonction
$f_\alpha : \Ord \to \mathcal U$. On a donc ensuite, en prenant la fonction
$\alpha \mapsto f_\alpha$, une fonction $g : \Ord\to(\Ord\to\mathcal U)$, que
l'on peut réécrire en une fonction $g' : \Ord\times\Ord \to \mathcal U$. On
évitera bien sûr de détailler tout ce processus lorsque l'on définira des
fonctions par récursion transfinie, mais il est important de garder en tête que
l'on peut effectivement faire cela, et ce de façon purement syntaxique (sans
avoir recours à la notion de classe).

On donne maintenant deux exemples classiques de fonctions définies par récursion
transfinie~: l'addition et la multiplication.

\begin{definition}[Addition ordinale]
  On définit l'addition ordinal $\alpha + \beta$ par récursion sur $\beta$~:
  \begin{itemize}
  \item $\alpha + 0 = \alpha$
  \item $\alpha + S\;\beta = S\;(\alpha + \beta)$
  \item $\displaystyle\alpha+\lambda=\bigcup_{\beta < \lambda}(\alpha+\beta)$
    si $\lambda$ est limite.
  \end{itemize}
\end{definition}

\begin{notation}
  A partir de maintenant, on écrira $\alpha + 1$ plutôt que $S\;\alpha$.
\end{notation}
