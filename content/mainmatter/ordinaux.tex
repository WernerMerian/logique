\chapter{Ordinaux, cardinaux, cofinalité}
\label{chp.ordinaux}

\minitoc

\lettrine{D}{ans} ce chapitre, on s'intéresse aux notions les plus
incontournables de la théorie des ensembles que sont les ordinaux, les cardinaux
et la cofinalité. Ces trois notions sont bâties l'une sur la suivante~: les
cardinaux sont une famille spécifique d'ordinaux et la cofinalité désigne une
fonction retournant des cardinaux.

Moralement, les ordinaux forment une hiérarchie d'ensembles canoniques. La
classe des ordinaux est une classe existant dans tout modèle de ZF, et se
comporte particulièrement bien. Cette classe nous permet de comparer des bons
ordres, mais nous verrons qu'elle revêt une importance plus particulière dans le
cadre de la théorie des ensembles~: elle est une classe ordonnée arbitrairement
grande, et permet donc de considérer des représentants pour des phénomènes
ensemblistes arbitrairement grands. De plus, c'est une classe bien ordonnée,
ce qui signifie qu'on a un principe de choix sur cette classe (si on a un
ensemble de parties d'un bon ordre, on peut prendre le minimum de chaque partie
dans l'ensemble), nous permettant en choisissant des minima d'obtenir un
représentant canonique.

Les cardinaux forment l'exemple le plus évident d'un tel procédé~: par défaut,
on imagine un cardinal comme une classe d'équipotence, c'est-à-dire comme
l'univers ensembliste $\mathcal U$ quotienté par la relation d'équivalence
\[X\sim Y \defeq \exists f : X \to Y, \mathrm{bij}(f)\]
Le souci, évidemment, est que ces classes ne sont pas des ensembles. Pour le
cardinal $1$, on peut par exemple prendre n'importe quel ensemble $X$ et
considérer $\{X\}$, montrant que tout l'univers $\mathcal U$ peut être injecté
dans la classe des ensembles de cardinal $1$. La solution adoptée en théorie des
ensembles est de choisir, pour chaque classe, un représentant canonique.
Plutôt que de définir ces classes d'équipotence, on remarque que chaque classe
rencontre la hiérarchie des ordinaux, et on prend alors l'ordinal minimum qui
appartient à la classe. Il reste un souci dans cette définition~: est-on sûrs
que la classe des cardinaux rencontre la hiérarchie des ordinaux~? Si l'on
admet $\AxC$, c'est le cas, comme nous le verrons, mais on traitera dans ce
chapitre plus en détail des implications de l'axiome du choix (on prouvera
en particulier le lemme de Zorn).

Enfin, la cofinalité est une notion, différente de la cardinalité, exprimant à
quel point un ordinal (ou un cardinal) a besoin d'un grand nombre d'étapes pour
être parcouru. L'idée de la cofinalité est assez technique et demande d'être
manipulée pour être pleinement assimilée. Nous verrons ce que sont les cardinaux
réguliers et singuliers, et étudierons quelques conséquences comme le théorème
de König.

\section{Ordinaux}

Commençons par donner une présentation informelle motivant la définition des
ordinaux.

\subsection{Description des ordinaux}

On rappelle qu'un bon ordre est un ordre tel que toute partie admet un minimum.
La classe des bons ordres est ainsi la classe des ensembles ordonnés $(X,<)$
où $<$ est un bon ordre sur $X$. On a par exemple $\{1\}$ et $\{2\}$ qui sont
deux bons ordres (avec le seul ordre possible sur un seul élément)~: ils sont
donc isomorphes, et représentent le même ordre, en ignorant l'étiquette des
éléments ordonnés eux-mêmes.

Dans le but de décrire efficacement la classe des bons ordres, il nous faut donc
trouver un moyen de représenter canoniquement un ensemble bien ordonné. Tout
d'abord, pour un ensemble $X$ quelconque, il peut exister un nombre considérable
de bons ordres (souvent infini), il nous faut donc fixer la relation pour un
ensemble donné. En repensant au lemme de Mostovski, on voit qu'on peut
représenter certains ensembles munis de relation par un ensemble muni de la
relation $\in$ et le besoin d'expliciter la relation est alors éliminé, puisque
la relation $\in$ existe dans la théorie même.

On souhaite donc représenter un ensemble bien ordonné $(X,<)$ par un ensemble
$X'$ pour lequel le bon ordre est $\in$. Comme nous l'avons vu pour le lemme
d'effondremment de Mostovski, un tel représentant existe bien et est unique si
l'on choisit le représentant comme étant transitif. Ainsi tout ensemble bien
ordonné est équivalent (on le montrera) à un unique ensemble transitif bien
ordonné pour la relation $\in$. Telle est donc notre définition d'ordinal.

\begin{definition}[Ordinal]
  On dit qu'un ensemble $\alpha$ est un ordinal si~:
  \begin{itemize}
  \item $\alpha$ est transitif, c'est-à-dire que
    $\forall x\in \alpha, \forall y \in x, y\in \alpha$
  \item $\alpha$ est bien ordonné pour $\in$, c'est-à-dire que $\in$ est un
    ordre sur $\alpha$~:
    \[(\forall x\in \alpha, x\notin \alpha)\land (\forall x,y,z\in \alpha,
    x\in y \land y \in z \implies x \in z)\]
    et que toutes ses parties ont un minimum~:
    \[\forall X\subseteq \alpha, \exists x \in \alpha, \forall y \in X,
    (x=y \lor x \in y)\]
  \end{itemize}

  On note $\Ord$ la classe des ordinaux, qui peut donc être définir par la
  conjonction des formules données ci-dessus.
\end{definition}

\begin{remark}
  On peut aussi décrire le fait d'être un bon ordre par le fait d'être total
  et d'être un ordre bien fondé. La propriété d'être bien fondée, en présence
  de l'axiome de fondation, est automatiquement vérifiée~: un ordinal est alors
  un ensemble transitif pour lequel la relation $\in$ est transitive et totale,
  c'est-à-dire que $x\in y \lor x = y \lor x\ni y$. Même si l'axiome de
  fondation simplifie les preuves, on décide de travailler pour l'instant dans
  $\ZF - \AxF$, pour montrer que toute la construction des ordinaux est possible
  avec peu d'axiomes.
\end{remark}

\begin{example}
  Donnons quelques cas pratiques d'ordinaux~:
  \begin{itemize}
  \item $0$ est un ordinal.
  \item Plus généralement, tout entier $n = \{0,\ldots,n-1\}$ est un ordinal.
  \item $\omega$, qui est le nom qu'on utilise en théorie des ensembles pour
    l'ensemble qu'on a décrété être $\mathbb N$, est aussi un ordinal. Les
    différents axiomes du fait d'être un ordinal peuvent se prouver par
    récurrence. C'est aussi le plus petit ordinal infini, puisque pour tout
    $n\in \omega$, $n$ est fini (nous donnerons plus tard une définition d'être
    un ordinal fini).
  \end{itemize}
\end{example}

\begin{notation}
  On désignera un ordinal par une lettre grecque du début de l'alphabet~:
  $\alpha,\beta,\gamma,\delta,\ldots$
\end{notation}

On donne maintenant trois propriétés basiques sur les ordinaux~: $\Ord$ est
stable par $S$, par intersection et l'inclusion entre ordinaux est équivalente
à l'appartenance ou l'égalité.

\begin{proposition}
  Soit $\alpha$ un ordinal, alors $S\;\alpha = \alpha\cup\{\alpha\}$ est
  aussi un ordinal.
\end{proposition}

\begin{proof}
  A FAIRE
\end{proof}

\begin{proposition}
  Soient $\alpha$ et $\beta$ deux ordinaux. Alors $\alpha\cap \beta$ est aussi
  un ordinal.
\end{proposition}

\begin{proof}
  A FAIRE
\end{proof}

\begin{proposition}
  Soient $\alpha$ et $\beta$ deux ordinaux. On suppose que
  $\alpha \subseteq \beta$, alors $\alpha \in \beta$ ou $\alpha = \beta$.
\end{proposition}

\begin{proof}
  On suppose que $\alpha \subsetneq \beta$, montrons qu'alors
  $\alpha \in \beta$. On sait donc que $\beta\setminus\alpha$ est une partie
  non vide, soit $z = \min \beta\setminus\alpha$. On montre maintenant que
  $z = \alpha$ par double inclusion.

  Si $x\in \alpha$, alors $x\notin\beta\setminus\alpha$ donc $x\neq z$. Montrons
  que $x\in z$. Supposons que $z\in x$. Alors, comme $\alpha$ est transitif,
  $z\in \alpha$, mais $z\in\beta\setminus\alpha$, ce qui est absurde. Donc
  $x\in z$, donc $\alpha\subseteq z$.

  Si $x\in z$, alors $x\notin\beta\setminus\alpha$ (car $z$ est le minimum de
  cet ensemble), donc $x\in\alpha$.

  Donc $\alpha = z$ et $z\in \beta$, donc $\alpha \in \beta$.
\end{proof}

La classe $\Ord$ est elle-même bien ordonnée pour $\in$, ce que l'on prouve dans
la proposition suivante.

\begin{proposition}
  Les propositions suivantes sont vraies~:
  \begin{enumerate}
  \item si $\alpha \in \beta \in \Ord$ alors $\alpha \in \Ord$.
  \item si $X\subseteq \Ord$, alors $\bigcap X \in \Ord$.
  \item $\Ord$ n'est pas un ensemble.
  \end{enumerate}
\end{proposition}

\begin{proof}
  On vérifie les différentes propositions~:
  \begin{enumerate}
  \item A FAIRE
  \item A FAIRE
  \item Supposons que $\Ord$ est un ensemble. Alors, par les propositions
    précédentes, c'est un ordinal~: on en déduit que $\Ord\in\Ord$, or par
    hypothèse, $\Ord\notin\Ord$.
  \end{enumerate}
\end{proof}

\begin{notation}
  Comme les ordinaux appartiennent à une classe elle-même ordonnée, on écrira
  souvent $\alpha < \beta$ ou $\alpha \leq \beta$ plutôt que $\alpha \in \beta$
  ou $\alpha \subseteq \beta$.
\end{notation}

\begin{property}
  Soit $X\subseteq\Ord$ un ensemble, alors $\bigcup X \in \Ord$.
\end{property}

\begin{proof}
  On vérifie les axiomes d'un ordinal~:
  \begin{itemize}
  \item si $x\in y\in \bigcup X$ alors on trouve $\alpha \in X$ tel que
    $x\in y \in \alpha$, et comme $\alpha$ est transitif, $x\in \alpha$, donc
    $x\in \bigcup X$.
  \item si $x\in X$ alors on trouve $\alpha \in X$ tel que $x\in \alpha$, donc
    comme $\alpha$ est un ordinal, $x\notin x$.
  \item si $x\in y\in z$ sont trois éléments de $\bigcup X$, alors on trouve
    $\alpha,\beta,\gamma\in X$ tels que $x\in \alpha,y\in \beta,z\in\gamma$.
    Comme $\Ord$ est bien ordonnée, on trouve un maximum entre $\alpha$,
    $\beta$ et $\gamma$, disons $\alpha$ quitte à renommer nos ordinaux.
    Il vient donc que $x,y,z\in\alpha$, donc que $x\in z$ comme $\in$
    est un ordre sur $\alpha$.
  \item soit $Y\subseteq \bigcup X$ une partie non vide de $X$. On trouve par
    hypothèse un élément $x\in Y$, et donc $\alpha \in X$ tel que $x\in\alpha$.
    Alors $Y\cap \alpha$ a un plus petit élément, notons-le $m$. Montrons que
    $m$ est un minorant de $Y$. Soit $y\in Y$, par hypothèse on trouve
    $\beta\in X$ tel que $y\in \beta$. Comme $y\in \beta$, $y$ est un ordinal,
    donc soit $y\subseteq \alpha$, soit $\alpha \in y$. Dans le premier cas,
    si $y\in \alpha$ alors $y\in Y \cap \alpha$ donc $m \leq y$ par définition,
    et si $y=\alpha$ alors $m\in Y \cap \alpha$ donc $m \leq \alpha$.
    Supposons maintenant que $\alpha \in y$. Alors on sait que
    $m\in \alpha\in y$, donc $m\in y$, d'où $m\leq y$.
  \end{itemize}

  $\bigcup X$ est donc un ordinal.
\end{proof}

Donnons maintenant un outil efficace pour prouver qu'un ensemble est un ordinal.

\begin{proposition}
  Soit $\alpha$ un ordinal et $Y\subseteq \alpha$ une partie close par le bas,
  c'est-à-dire telle que si $x\in Y$, $y\in \alpha$ et $y\in x$ alors
  $y\in \alpha$, alors $Y$ est un ordinal (on dit aussi que $Y$ est un segment
  initial de $\alpha$).
\end{proposition}

\begin{proof}
  A FAIRE
\end{proof}

On voudrait maintenant utiliser le lemme d'effondrement de Mostovski pour
prouver qu'un ensemble bien ordonné est isomorphe à un unique ordinal. Comme
nous avons fait le choix de travailler dans $\ZF - \AxF$, le lemme
d'effondrement n'est pas utilisable. On va donc en remontrer une version adaptée
aux ordinaux, qui vérifient l'axiome de fondation.

On montre d'abord un lemme similaire à celui utilisé pour la démonstration du
lemme d'effondremnet de Mostovski.

\begin{lemma}
  Soient $\alpha$ et $\beta$ deux ordinaux, avec
  $\varphi : (\alpha,\in) \cong (\beta,\in)$, alors $\varphi = \id$ et
  $\alpha = \beta$.
\end{lemma}

\begin{proof}
  Supposons par l'absurde que $\varphi\neq\id$. On définit donc l'ensemble non
  vide $\{\gamma \in \alpha \mid \varphi(\gamma)\neq\gamma\}$, soit $\gamma$
  le plus petit élément de cet ensemble. On applique le même raisonnement que
  dans le \cref{lem.most} pour en déduire que $\gamma = \varphi(\gamma)$, donc
  que $\varphi\neq\id$ est absurde.
\end{proof}

\begin{proposition}
  Soit $(X,<)$ un bon ordre. Alors il existe un unique ordinal $\alpha$ et un
  unique isomorphisme $\pi : (X,<)\cong (\alpha,\in)$.
\end{proposition}

\begin{proof}
  On définit la fonction $\pi : X \to \Ord$ par récursion bien fondée sur
  $X$~:
  \[\pi(x)\defeq \{ \pi(y) \mid y < x\}\]
  qui, d'après l'axiome de remplacement et par une induction bien fondée,
  peut bien être défini comme une fonction $\pi : X \to Y$ avec
  $Y\subseteq \Ord$. On vérifie alors que $\pi$ est un isomorphisme d'ensemble
  ordonné vers un ordinal~:
  \begin{itemize}
  \item $\pi$ est surjective par définition, puisqu'on considère son codomaine
    comme étant son image.
  \item $\pi$ est injective~: si $x\neq y$ alors $x < y$ sans perte de
    généralité, et $\pi(x) \subsetneq \pi(y)$, donc $\pi(x)\neq\pi(y)$.
  \item si $x < y$ alors $\pi(x) \in \pi(y)$.
  \item si $\pi(x) \in \pi(y)$, alors $y < x$ est impossible puisque dans ce
    cas, $\pi(y) \in \pi(x)$, donc (comme $<$ est total) $x < y$.
  \item $Y$ est un segment initial de $\Ord$~: si $\alpha \in Y$ alors on trouve
    $x \in X$ tel que $\alpha = \pi(x)$, et si $\beta \in \alpha$ alors
    on trouve $y\in X$ tel que $\beta = \pi(y)$ et $y < x$, par définition de
    $\pi$, donc $\beta \in Y$.
  \end{itemize}
  On a donc trouvé un isomorphisme entre $(X,<)$ et un ordinal $\alpha$ donné.
  Si on avait deux ordinaux isomorphes, alors par le lemme précédent,
  l'isomorphisme induit entre eux est l'idéentité, donc on a en fait un seul
  tel ordinal.
\end{proof}

\begin{corollary}
  Soient deux ordinaux $\alpha,\beta$ tels qu'on a une fonction strictement
  croissante $f : \alpha \to \beta$ et une fonction strictement croissante
  $g : \beta \to \alpha$, alors $\alpha = \beta$ et $f = g = \id_\alpha$.
\end{corollary}

\begin{definition}
  Soit $(X,<)$ un bon ordre. On appelle type de $(X,<)$, souvent abrégé en
  type de $X$, que l'on note $\type(X)$ l'unique ordinal $\alpha$ tel que
  $(X,<)\cong (\alpha,\in)$.
\end{definition}

On a vu que la classe $\Ord$ contenait $\omega$, mais elle s'étend au-delà, déjà
en considérant $S\;\omega$. De même, on peut considérer $\omega + \omega$, que
l'on peut définir par
\[\omega + \omega \defeq \bigcup_{n \in \omega} S^n\;\omega\]
ou d'autres tels ordinaux toujours plus grands. Nous verrons plus tard
l'arithmétique ordinale, permettant de donner un sens plus précis à des
expressions telles que $\omega + \omega$, mais nous allons d'abord étudier les
différentes sortes d'ordinaux existants.

L'ordinal $0$ a un rôle particulier, puisqu'il est le plus petit ordinal
existant. A part lui, les ordinaux sont de deux sortes~: les ordinaux
successeurs et les ordinaux limites.

\begin{definition}[Ordinal successeur]
  Un ordinal $\alpha$ est dit successeur s'il existe $\beta \in \Ord$ tel que
  $\alpha = S\;\beta$.
\end{definition}

\begin{definition}[Ordinal limite]
  Un ordinal $\lambda$ est dit limite s'il est non nul et vérifie
  $\lambda = \sup_{\beta < \lambda} \beta$.
\end{definition}

\begin{property}
  Tout ordinal est soit $0$, soit un successeur, soit un ordinal limite, chaque
  cas étant exclusif.
\end{property}

\begin{proof}
  Le fait que $0$ n'est ni successeur ni limite découle de la définition.
  Montrons qu'un ordinal ne peut être à la fois limite et successeur. Soit
  $S\;\alpha$ un ordinal successeur, alors
  \[\bigcup_{\beta < S\;\alpha} \beta =
  \{\gamma \in \beta \mid \beta \in\alpha \cup \{\alpha\}\}\]
  par transitivité, si $\gamma\in\beta\in\alpha$ alors $\gamma\in \alpha$, et
  si $\beta \in\{\alpha\}$ alors $\gamma\in \alpha$ par définition. Dans l'autre
  sens, si $\beta \in \alpha$ alors $\beta\in\alpha\in\{\alpha\}$. Donc
  $\bigcup (S\;\alpha) = \alpha$. Ainsi un ordinal successeur ne peut pas être
  limite.

  Montrons que tout ordinal est soit $0$, soit un successeur, soit un ordinal
  limite. Supposons qu'il existe $\alpha$ qui n'est aucun des trois. On définit
  $\beta = \bigcup \alpha$. Comme $\alpha$ est transitif (c'est un ordinal),
  $\beta \subseteq \alpha$, donc $\beta \in \alpha$ ou $\beta = \alpha$.
  Si $\beta = \alpha$ alors $\alpha$ est limite, donc $\beta \in \alpha$.
  Montrons que $\alpha = S\;\beta$. Pour ça, montrons que le seul élément de
  $\alpha\setminus \beta$ est $\beta$. On sait que
  $\beta \in \alpha \setminus\beta$ car $\beta\notin\beta$. Soit
  $\gamma\in\alpha\setminus\beta$, montrons que $\gamma = \beta$.
  Comme $\gamma\notin\beta$, $\gamma \geq \beta$, montrons que
  $\gamma\leq\beta$. Comme $\gamma \in \alpha$, on en déduit que si
  $\xi\in\gamma$ alors $\xi \in \alpha$, donc $\gamma\subseteq \beta$. Donc
  $\gamma = \beta$, donc $\alpha = S\;\beta$.
\end{proof}

\subsection{Induction transfinie}

La structure de $\Ord$ ressemble fortement à celle des entiers, mais avec
l'ajout d'ordinaux limites. On s'attend donc à avoir un principe d'induction et
de récursion proche de celui de nos ensembles inductifs du \cref{chp.induction}.
C'est en effet le cas, et nous le donnerons sous deux formes~: avec l'induction
de bon ordre et avec la distinction des sortes d'ordinaux.

\begin{theorem}[Induction transfinie]
  Soit $\varphi(x_0,\ldots,x_n,y)$ une formule dans le langage des ensembles.
  On a alors la propriété
  \begin{multline*}
    \forall x_0,\ldots,x_n,(\forall \alpha\in\Ord,
    (\forall \beta < \alpha, \varphi(x_0,\ldots,x_n,\beta))\implies
    \varphi(x_0,\ldots,x_n,\alpha))\\
    \implies \forall \alpha \in \Ord, \varphi(x_0,\ldots,x_n,\alpha)
  \end{multline*}
\end{theorem}

\begin{proof}
  Soient $x_0,\ldots,x_n$ des ensembles. Supposons que pour tout
  $\alpha\in\Ord$, on a
  \[(\forall \beta < \alpha, \varphi(x_0,\ldots,x_n,\beta))\implies
  \varphi(x_0,\ldots,x_n,\alpha)\]
  Supposons par l'absurde qu'il existe $\alpha \in \Ord$ tel que
  $\lnot\varphi(x_0,\ldots,x_n,\alpha)$. On peut alors prendre un tel $\alpha$
  minimal~: on en déduit donc que pour tout $\beta < \alpha$,
  $\varphi(x_0,\ldots,x_n,\beta)$ est vraie, donc par hypothèse
  $\varphi(x_0,\ldots,x_n,\alpha)$ est vraie aussi, ce qui est une
  contradiction.
\end{proof}

Ce résultat est cependant souvent plus pratique à utiliser une fois combiné à la
disjonction de cas sur ce qu'est un ordinal.

\begin{proposition}[Induction transfinie, variante]
  Soit $\varphi(x_0,\ldots,x_n,y)$ une formule dans le langage des ensembles.
  Soient $x_0,\ldots,x_n$ des ensembles. On suppose que~:
  \begin{itemize}
  \item $\varphi(x_0,\ldots,x_n,0)$ est vérifiée.
  \item pour tout ordinal $\alpha$, si $\varphi(x_0,\ldots,x_n,\alpha)$ alors
    $\varphi(x_0,\ldots,x_n,S\;\alpha)$.
  \item pour tout ordinal limite $\lambda$, si
    $\forall \beta < \lambda, \varphi(x_0,\ldots,x_n,\beta)$ alors
    $\varphi(x_0,\ldots,x_n,\lambda)$.
  \end{itemize}

  Alors $\varphi(x_0,\ldots,x_n,\alpha)$ est vraie pour tout ordinal $\alpha$.
\end{proposition}

\begin{proof}
  Par l'absurde, supposons que la classe
  \[\{\alpha \in\Ord\mid \mathcal \lnot\varphi(x_0,\ldots,x_n,\alpha)\}\]
  n'est pas vide. On trouve alors un minimum à cette classe, qu'on note
  $\alpha$. On sait que $\alpha \neq 0$ d'après nos hypothèses. De plus, si
  $\alpha$ est successeur, alors $\alpha = \beta$ et
  $\varphi(x_0,\ldots,x_n,\beta)$ est vérifiée, donc par hypothèse
  $\varphi(x_0,\ldots,x_n,\alpha)$, ce qui est absurde. Si $\alpha$ est limite,
  alors pour tout $\beta < \alpha$, $\varphi(x_0,\ldots,x_n,\beta)$ est
  vérifiée, d'où encore une absurdité.
  Ainsi la classe est vide, donc tout ordinal vérifie la formule.
\end{proof}

Notre principe d'induction est donc généralisé à des ensembles bien plus grands
que $\omega$. On va de même montrer un principe de récursion. Dans le cas des
entiers, on peut définir une suite $u_n$ à un rang $n$ donné à partir de $n$ et
de $u_{n-1}$. Pour un ordinal, on étend cette définition avec le cas d'un
ordinal limite, pour lequel $f(\lambda)$ est définie à partie des tous les
$f(\beta)$ et $\beta$, pour $\beta < \lambda$.

Pour préciser ces notions, nous allons introduire un peu de vocabulaire.

\begin{definition}[Fonction inductive pour une relation]
  Soit $H(x,y)$ une relation fonctionnelle à deux variables libres. Soit
  $\alpha$ un ordinal. On dit qu'une fonction $f : \alpha \to \mathcal U$ est
  inductive pour $H$ si pour tout $\beta < \alpha$, on a~:
  \begin{itemize}
  \item $f\restr\beta$ est dans le domaine de $H$.
  \item $f(\beta) = H(f\restr{\beta})$.
  \end{itemize}
\end{definition}

\begin{remark}
  On ne s'intéresse pas à l'ensemble d'arrivée de $f$, puisque par le schéma de
  remplacement on peut toujours s'assurer que $f$ va dans $\im(f)$ qui est bien
  un ensemble.
\end{remark}

\begin{example}
  On va définir l'identité sur $\Ord$ par récursion. Pour cela, on définit
  $H_{\id}$~:
  \[H_{\id}(x,y) \defeq y = \{x(\alpha) \mid \alpha < x\}\]

  On peut alors vérifier que $\id_\alpha$ est inductive pour $H_{\id}$~:
  \begin{itemize}
  \item $\id_{\alpha}\restr{\beta}$ est dans le domaine de $H_{\id}$ pour tout
    $\beta < \alpha$.
  \item pour $\beta < \alpha$,
    $\id_\alpha(\beta) = \{\gamma\mid \gamma < \beta\}$ et
    $H(f\restr{\beta}) = \{\id_\beta(\gamma)\mid \gamma < \beta\}$ d'où
    l'égalité.
  \end{itemize}
\end{example}

On peut alors utiliser une relation $H$ pour définir une fonction dont le
domaine est $\Ord$~: l'image de $\alpha\in\Ord$ sera $H(f\restr{\alpha})$, pour
obtenir une unique fonction $H$-inductive. Pour assurer le bon fonctionnement
du procédé, on prouve d'abord qu'il existe au plus une fonction inductive pour
$H$ dont le domaine est $\alpha$.

\begin{lemma}
  Soit $H$ une relation fonctionnelle et $\alpha$ un ordinal. Alors il existe
  au plus une fonction $f : \alpha \to X$ qui est inductive pour $H$.
\end{lemma}

\begin{proof}
  Supposons qu'il existe $f,g$ deux fonctions inductives pour $H$. Par
  l'absurde, supposons que $f\neq g$, soit alors $\beta$ l'ordinal minimal tel
  que $f(\beta)\neq g(\beta)$. Comme $\beta$ est minimal, on en déduit que
  $f\restr{\beta}=g\restr{\beta}$, donc par le fait que $f$ et $g$ sont
  inductives pour $H$, il vient que $f(\beta) = g(\beta)$, ce qui est une
  contradiction.
\end{proof}

On veut maintenant prouver qu'étant donnée une relation fonctionnelle $H$,
il existe effectivement une fonction $H$-inductive de domaine $\alpha$.

\begin{lemma}
  Soit $H$ une relation fonctionnelle dont le domaine est stable, c'est-à-dire
  que si $H(x,y)$ alors $y$ est lui-même dans le domaine. Pour tout
  $\alpha \in \Ord$, il existe une fonction $f : \alpha \to X$ qui est inductive
  pour $H$.
\end{lemma}

\begin{proof}
  Par l'absurde, prenons un $\alpha$ minimal tel que $f : \alpha \to X$ n'est
  pas définie (pour $f$ étant $H$-inductive). Comme $\alpha$ est minimal, on
  sait que pour tout $\beta < \alpha$, il existe une fonction
  $f_\beta :\beta\to X$ qui est inductive pour $H$. On définit alors
  \[\begin{array}{ccccc}
  f & : & \alpha & \longrightarrow & X\\
  & & \beta & \longmapsto & H(f_\beta)
  \end{array}\]
  On remarque, comme chaque $f_\beta$ est $H$-inductive, et par unicité d'une
  fonction $H$-inductive, que $f_\beta\restr{\gamma} = f_\gamma$ pour tous
  $\gamma < \beta < \alpha$.

  On vérifie que $f$ est bien inductive. Soit $\beta < \alpha$, alors~:
  \begin{itemize}
  \item $f\restr{\beta} = f_\beta$ est $f_\beta$ est inductive, donc elle
    est dans le domaine de $H$.
  \item $f(\beta) = H(f\restr{\beta}) = H(f_\beta)$ est vérifié par définition.
  \end{itemize}

  On en déduit donc qu'il existe une fonction $f : \alpha \to X$ inductive pour
  $H$.
\end{proof}

On en déduit le théorème de récursion.

\begin{theorem}[Principe de récursion transfinie]\label{thm.recurs.transf}
  Soit $H$ une relation fonctionnelle telle que pour tout $x\in \dom(H)$,
  $H(x)\in\dom(H)$. Il existe alors une unique fonction $f$ de $\Ord$ dans
  $\im(H)$ telle que $f\restr{\alpha}$ est $H$-inductive pour tout $\alpha$.
\end{theorem}

\begin{proof}
  On définit une fonction entre classes, ce qui est un abus de langage si l'on
  considère une fonction comme un ensemble. Le sens de cette assertion est qu'il
  existe un prédicat $\varphi$ fonctionnel dont le domaine est $\Ord$ et qui
  vérifie l'hypothèse que l'on veut. Pour écrire ce prédicat, il nous suffit
  d'utiliser la définition précédente~:
  \[\varphi(x,y) \defeq \exists (f : x \to X),
  \mathrm{ind}_H(f)\land H(f) = y\]

  On a vérifié précédemment que la fonction définie de cette façon est bien
  $H$-inductive, d'où l'existence.

  L'unicité de la fonction est donnée par le fait qu'une fonction $f$ qui est
  $H$-inductive de domaine $\alpha$ est unique, donc que $\varphi(\alpha)$ est
  unique.
\end{proof}

Remarquons que si l'on a la fonction $f\restr{\alpha}$, alors on peut récupérer
le $\alpha$ correspondant~:
\[\alpha = \dom(f\restr{\alpha})\]
ce qui, pour construire une relation fonctionnelle $H$, signifie qu'on peut
définir le rang auquel on est dans la construction de l'image par la fonction
$H$-inductive associée en remplaçant $f\restr{\alpha}$ par l'argument de $H$.
On va appeler $\alpha_x$ l'ordinal $\bigcup\dom(x)$.

Comme on sait de plus qu'il existe une trichotomie sur les ordinaux suivant
s'ils sont successeurs ou limites, en utilisant les formules
\[\mathrm{estZero}(x) \defeq x = 0\qquad
\mathrm{estSucc}(x)\defeq \exists y. x = S\;y \qquad
\mathrm{estLimite}(x)\defeq x \neq 0 \land \bigcup x = x\]
et l'astuce précédente permettant de récupérer le rang de l'argument, on peut
donner une fonction $f : \Ord \to \mathcal U$ par trois données~:
\begin{itemize}
\item un ensemble $X_0$.
\item une relation fonctionnelle $H_s$ qui, à chaque ensemble $X_\alpha$ associe
  un nouvel ensemble $X_{S\;\alpha}$.
\item une relation fonctionnelle $H_\ell$ qui, si on a déjà défini $X_\beta$ pour
  tout $\beta < \lambda$ où $\lambda$ est un ordinal limite, construit un nouvel
  ensemble $X_\lambda$.
\end{itemize}

Si on souhaite expliciter la formule, celle-ci est alors
\begin{multline*}
  H(x,y) \defeq (\mathrm{estZero}(\alpha_x)\land y = X_0)\lor
  (\mathrm{estSucc}(\alpha_x)\land H_S(x(\bigcup \alpha_x),y))\lor\\
  (\mathrm{estLimite}(\alpha_x)\land H_\ell(x,y))
\end{multline*}

On peut aussi généraliser notre \cref{thm.recurs.transf} en permettant la
définition de fonction $\Ord\times\Ord\to \mathcal U$. \'Etant donnée une
relation fonctionnelle $H$, on peut définir une fonction
$f : \Ord \to \mathcal U$ telle que $f\restr{\alpha}$ est une fonction. Soit
maintenant un énoncé fonctionnel $H$ à deux paramètres, disons $x$ et $\alpha$.
Pour tout $\alpha\in\Ord$, il est possible de définir la relation fonctionnelle
$H$ en fixant $\alpha$, donnant alors une fonction
$f_\alpha : \Ord \to \mathcal U$. On a donc ensuite, en prenant la fonction
$\alpha \mapsto f_\alpha$, une fonction $g : \Ord\to(\Ord\to\mathcal U)$, que
l'on peut réécrire en une fonction $g' : \Ord\times\Ord \to \mathcal U$. On
évitera bien sûr de détailler tout ce processus lorsque l'on définira des
fonctions par récursion transfinie, mais il est important de garder en tête que
l'on peut effectivement faire cela, et ce de façon purement syntaxique (sans
avoir recours à la notion de classe).

On donne maintenant deux exemples classiques de fonctions définies par récursion
transfinie~: l'addition et la multiplication.

\begin{definition}[Addition ordinale]
  On définit l'addition ordinal $\alpha + \beta$ par récursion sur $\beta$~:
  \begin{itemize}
  \item $\alpha + 0 = \alpha$
  \item $\alpha + S\;\beta = S\;(\alpha + \beta)$
  \item $\displaystyle\alpha+\lambda=\bigcup_{\beta < \lambda}(\alpha+\beta)$
    si $\lambda$ est limite.
  \end{itemize}
\end{definition}

\begin{notation}
  A partir de maintenant, on écrira $\alpha + 1$ plutôt que $S\;\alpha$.
\end{notation}

\begin{exercise}[Sur la somme ensembliste]
  Soit $(X_i)_{i\in I}$ une famille d'ensembles. On définit l'ensemble
  \[\sum_{i\in I} X_i \defeq \{(i,x)\mid i\in I, x \in X_i\}\]
  Supposons que $(\alpha_i)_{i\in \beta}$ est une famille d'ordinaux indicée par
  un ordinal. On muni $\sum_{i\in\beta} \alpha_i$ de l'ordre $\lexl$ défini par
  \[(i,x)\lexl (j,y) \defeq i < j \lor (i = j \land x < y)\]

  Montrer que $\displaystyle\Big(\sum_{i\in \beta}\alpha_i,\lexl\Big)$ est un
  ensemble bien ordonné. On notera $\displaystyle\sum_{i\in\beta}\alpha_i$
  l'ordinal correspondant au type d'ordre de cet ensemble. Montrer que si
  $\beta = 2$ alors $\displaystyle\sum_{i\in\beta}\alpha_i = \alpha_0 + \alpha_1$
  (avec la définition de $+$ donnée ci-dessus).
\end{exercise}

\begin{exercise}[Autre définition du produit cartésien]
  L'ensemble $\sum_{i\in I} X_i$ induit naturellement une fonction
  \[\begin{array}{ccccc}
  \pi & : & \sum_{i\in I} X_i & \longrightarrow & I\\
  & & (i,x) & \longmapsto & i
  \end{array}\]
  Montrer qu'on a une bijection
  \[\prod_{i \in I} X_i \cong \Big\{f : I \to \sum_{i \in I}\;\Big|\;
  \pi\circ f = \id_I\Big\}\]
\end{exercise}

\begin{definition}[Multiplication ordinale]
  On définit la multiplication ordinale $\alpha\times \beta$ par récursion sur
  $\beta$~:
  \begin{itemize}
  \item $\alpha \times 0 = 0$
  \item $\alpha \times (\beta + 1) = \alpha \times \beta + \alpha$
  \item $\displaystyle \alpha \times \lambda = \bigcup_{\beta < \lambda} \alpha
    \times \beta$ si $\lambda$ est un ordinal limite.
  \end{itemize}
\end{definition}

\begin{exercise}
  Montrer que si $(\alpha_i)_{i\in \beta}$ est une famille d'ordinaux indicée par
  un ordinal, alors $\displaystyle\Big(\prod_{i\in \beta}\alpha_i,\lexl\Big)$
  est un ensemble bien ordonné. On notera $\prod_{i\in\beta}\alpha_i$ sont type
  d'ordre. Montrer que
  $\displaystyle\prod_{i < 2}\alpha_i = \alpha_0\times\alpha_1$.
\end{exercise}

On peut aussi définir l'exponentiation $\alpha^\beta$ par récursion ou par un
type d'ordre. Par récursion, on a les équations suivantes~:
\begin{itemize}
\item $\alpha^0 = 1$
\item $\alpha^{\beta+1} = \alpha^\beta \times \alpha$
\item $\alpha^\lambda = \displaystyle\bigcup_{\beta < \lambda}\alpha^\beta$
  si $\lambda$ est limite.
\end{itemize}
En tant que type d'ordre, on considère l'ensemble $\alpha\to_\mathrm{fin}\beta$
des fonctions de $\alpha$ dans $\beta$ à support fini, c'est-à-dire qui sont non
nulles pour un nombre fini de valeurs, et on le muni de l'ordre lexicographique
sur les valeurs prises, c'est-à-dire que $f < g$ si pour le plus petit élément
$\alpha$ tel que $f(\alpha)\neq g(\alpha)$, on a $f(\alpha) < g(\alpha)$.

On peut maintenant prouver plusieurs résultats d'arithmétique ordinale.

\begin{property}
  Soient $\alpha,\beta,\gamma\in\Ord$, on a~:
  \begin{itemize}
  \item $\alpha + (\beta + \gamma) = (\alpha + \beta) + \gamma$
  \item $\alpha\times(\beta\times\gamma) = (\alpha\times\beta)\times \gamma$
  \item $\alpha < \beta \implies \alpha + \gamma < \beta + \gamma \text{ et }
    \gamma + \alpha < \gamma + \beta$
  \item $\alpha + \beta = \alpha + \gamma \implies \beta = \gamma$
  \item $\alpha \times 1 = 1 \times \alpha = \alpha$
  \item $\alpha\times\beta = 0 \implies \alpha = 0 \text{ ou }\beta=0$
  \item $\alpha\times(\beta+\gamma) = \alpha\times\beta+\alpha\times\gamma$
  \item si $\beta > 0$ alors il existe un unique couple $(\delta,\varepsilon)$
    tel que $\alpha = \beta \times\delta + \varepsilon$ et
    $\varepsilon < \beta$, c'est la division euclidienne ordinale.
  \end{itemize}
\end{property}

\begin{proof}
  A FAIRE
\end{proof}

On peut aussi dégager des propriétés arithmétiques sur l'exponentiation, mais
celles-ci sont laissées en exercice car elles ne seront pas essentielles.

\begin{exercise}
  Montrer les propriétés suivantes sur l'exponentiation ordinale, pour tous
  $\alpha,\beta,\gamma\in\Ord$~:
  \begin{itemize}
  \item $\alpha^{\beta+\gamma} = \alpha^\beta\times\alpha^\gamma$
  \item $\alpha \leq \beta \implies \alpha^\gamma\leq\beta^\gamma$
  \item $(\alpha^\beta)^\gamma = \alpha^{\beta\times\gamma}$
  \item si $\alpha > 1$ et $\beta >0$, alors il existe un unique triplet
    $(\delta,\varepsilon,\eta)$ tel que
    $\beta = \alpha^\delta\times\eta + \varepsilon$, $\varepsilon < \alpha^\delta$
    et $0 < \eta < \alpha$.
  \end{itemize}
\end{exercise}

\subsection{Ordinal de Hartogs et axiome du choix}\label{sbsct.Zorn}

L'outil de la récursion transfinie permet d'enfin montrer le lemme de Zorn
(\cref{thm.Zorn}). La preuve de ce lemme repose sur une récursion transfinie
dont on prouve qu'elle ne peut pas se prolonger jusqu'à tout $\Ord$. Pour cela,
on montre d'abord qu'un ensemble quelconque peut voir majorer sa taille par un
ordinal, qu'on appelle l'ordinal de Hartogs.

\begin{definition}[Ordinal de Hartogs]
  Soit $X$ un ensemble. Alors il existe un plus petit ordinal $\alpha_X$ tel
  que $X$ ne s'injecte par dans $\alpha_X$, on appelle $\alpha_X$ l'ordinal
  de Hartogs de $X$.
\end{definition}

\begin{proof}
  On veut donc prouver qu'un tel ordinal existe. Pour ce faire, on commence par
  définir
  \[\Ord(X) \defeq \{(Y,R)\mid
  Y\subseteq X, R\subseteq Y\times Y, R\text{ est un bon ordre sur }Y\}\]
  On sait donc que pour chaque $(Y,R)\in\Ord(X)$, il existe un unique
  $\alpha_{(Y,R)}\in\Ord$ qui lui correspond. Ainsi, par le schéma de
  remplacement, la collection
  \[\alpha_X \defeq \{\alpha \in \Ord\mid \exists (Y,R) \in \Ord(X),
  \alpha = \alpha_{(Y,R)}\}\]
  est bien un ensemble. Pour vérifier que $\alpha_X$ est bien un ordinal, il
  nous suffit maintenant de vérifier que c'est un segment initial de $\Ord$.
  Si $\gamma\in\beta\in\alpha_X$ alors on trouve $(Y,R)$ tel que
  $\beta = \alpha_{(Y,R)}$. Comme $\gamma\in\beta$, $\gamma$ induit
  par restriction une partie bien ordonnée $(Z,R')\subseteq (Y,R)$, donc
  $\gamma = \alpha_{(Z,R')}$, donc $\gamma\in \alpha_X$. Ainsi $\alpha_X$ est
  bien un ordinal.

  On vérifie maintenant que $X$ ne s'injecte pas dans $\alpha_X$. Si c'était le
  cas, alors l'image de $\alpha_X$ définirait un élément de $\Ord(X)$ (avec
  l'ordre induit par $\in$ sur l'image de $\alpha_X$), donc par construction
  $\alpha_X\in\alpha_X$, ce qui est impossible pour un ordinal. On en déduit
  que $X$ ne s'injecte pas dans $\alpha_X$.

  Pour vérifie que $\alpha_X$ est le plus petit tel ordinal, prenons un autre
  ordinal $\beta$ tel que $X$ ne s'injecte pas dans $\beta$. On sait que
  soit $\beta \in \alpha_X$, soit $\alpha_X\subseteq \beta$. Dans le deuxième
  cas, on a le résultat souhaité. Dans le premier cas, $\beta$ est par
  définition $\alpha_{(Y,R)}$ pour un certain $(Y,R)\in\Ord(X)$, ce qui impose
  que $X$ s'injecte dans $\beta$, ce qui n'est pas.
\end{proof}

L'ordinal de Hartogs permet de relier un ensemble quelconque à la hiérarchie
ordinale. Dans le cas de ZF, sans l'axiome du choix, il est le meilleur lien
possible (on ne connait pas, \textit{a priori}, un ordinal en bijection avec
l'ensemble donné). On verra plus tard que l'ordinal de Hartogs est en fait un
cardinal, et qu'il est même le cardinal successeur quand on l'applique à un
cardinal.

Montrons maintenant le lemme de Zorn, qui utilise l'axiome du choix.

\begin{theorem}[Lemme de Zorn]
  Soit $(X,<)$ un ensemble inductif, c'est-à-dire tel que toute chaîne
  $C\subseteq X$ possède un majorant, et soit $x_0\in X$. Alors il existe un
  élément maximal $x$ dans $X$ supérieur à $x_0$, c'est-à-dire tel que
  \[\forall y \in X, \lnot (x < y)\]
\end{theorem}

\begin{proof}
  On va construire une fonction $f : \Ord \to X$ strictement croissante tant que
  l'on n'a pas atteint un élément maximal. Pour faciliter la rédaction, on
  procède par l'absurde. Supposons donc qu'il n'existe pas d'élément maximal
  dans $X$ qui est supérieur à $x_0$, on définit pour tout $x\in X$ l'ensemble
  \[x\uparrow\defeq \{y \in X \mid x < y\}\]
  qui est toujours un ensemble non vide pour $x \geq x_0$ ($x\uparrow$ est vide
  si et seulement si $x$ est un élément maximal). On se donne, grâce à l'axiome
  du choix, une fonction de choix
  $\varphi : \powerset(X)\setminus\{\varnothing\}\to X$ telle que
  $\varphi(Y) \in Y$ pour tout $Y\in\dom(\varphi)$.

  On définit maintenant par récursion transfinie la fonction $f : \Ord\to X$
  strictement croissante, par~:
  \begin{itemize}
  \item $f(0) = x_0$.
  \item pour tout $\alpha\in\Ord$, $f(\alpha+1) = \varphi(f(\alpha)\uparrow)$.
  \item pour tout $\lambda\in\Ord$ limite,
    $f(\lambda) = \varphi(\{f(\beta)\mid \beta < \lambda\}\uparrow)$ qui existe
    bien car l'ensemble est une chaîne, et possède donc un majorant (strict car
    s'il était atteint, cela contredirait la stricte croissante de $f$
    sur les éléments inférieurs à $\lambda$).
  \end{itemize}
  Le fait qu'on ajoute chaque fois un élément strictement supérieur aux éléments
  déjà définis assure la stricte croissance à chaque étape de notre récursion.
  De plus, comme la fonction est strictement croissante, on est assurés que
  les éléments rencontrés ne sont pas maximaux, puisqu'ils sont tous supérieurs
  à $x_0$. On a donc une fonction injective $f : \Ord \to X$, ce qui signifie en
  particulier qu'on a trouvé $f : \alpha_X\to X$ injective, ce qui n'est pas
  possible par hypothèse.

  Par l'absurde, on en déduit donc qu'il existe bien $x > x_0$ maximal.
\end{proof}

Montrons aussi la réciproque~: le lemme de Zorn implique l'axiome du choix.

\begin{proposition}
  Dans $\ZF - \AxC - \AxF$, si on admet le lemme de Zorn alors $\AxC$ est vrai.
\end{proposition}

\begin{proof}
  A FAIRE
\end{proof}

\begin{exercise}[Une caractérisation des ordinaux finis]
  Montrer qu'un ordinal $\alpha$ est fini si et seulement si $(\alpha,\ni)$ est
  un ensemble bien ordonné.
\end{exercise}

\begin{exercise}[Une caractérisation des ordinaux infinis]
  Un ensemble $X$ est dit infini au sens de Dedekind s'il existe une partie
  $Y\subsetneq X$ telle que $X$ s'injecte dans $Y$. Montrer qu'un ensemble $X$
  est infini au sens de Dedekind si et seulement s'il est infini au sens usuel,
  c'est-à-dire que $\omega$ s'injecte dans $X$.
\end{exercise}

\section{Cardinaux}

Nous étudions maintenant la notion de cardinal. Comme nous l'avons mentionné
au début du chapitre, les cardinaux sont des choix de représentants canoniques
de classes d'équipotence. Dans cette section, nous allons rapidement avoir
besoin de l'axiome du choix, mais l'axiome de fondation n'est pas nécessaire.

Sans l'axiome du choix, comme on l'a dit pour définir l'ordinal de Hartogs, il
nous manque un moyen de prouver que tout ensemble est équipotent à un ordinal.
Nous allons donc, en utilisant l'axiome du choix, définir une mesure plus
précise de la taille d'un ensemble, ce qui revient au final à munir un ensemble
quelconque d'un bon ordre~: c'est le théorème de Zermelo.

Quand nous aurons traité de la définition des ordinaux, nous verrons les
fonctions $\aleph$ et $\beth$, ainsi que l'hypothèse du continu.

\subsection{Axiome du choix et dichotomie cardinale}

Pour introduire la notion de cardinal, on va définir la relation de subpotence,
qui correspond intuitivement à dire qu'un ensemble est plus petit qu'un autre.

\begin{definition}[Subpotence]
  Soient $X,Y$ deux ensembles. On dit que $X$ est subpotent à $Y$, ce que l'on
  notera $X\preceq Y$, lorsqu'il existe une injection $f : X \to Y$.
\end{definition}

\begin{property}
  La relation $\preceq$ est une relation de pré-ordre.
\end{property}

\begin{proof}
  On vérifie les axiomes d'un pré-ordre.
  \begin{itemize}
  \item pour tout ensemble $X$, $\id_X$ est une injection de $X$ dans $X$.
  \item si $f : X \to Y$ et $g : Y \to Z$ sont des injections, alors
    $g\circ f : X \to Z$ est une injection, donc $\preceq$ est transitive.
  \end{itemize}
  Ainsi $\preceq$ est une relation d'ordre.
\end{proof}

La relation d'équipotence est alors la relation d'équivalence associée à ce
pré-ordre. On la définit aussi par le fait que $X\simeq Y$ si et seulement s'il
existe $f : X \to Y$ bijective. La coïncidence des deux définitions n'est pas
immédiate~: c'est le théorème de Cantor-Bernstein.

\begin{theorem}[Cantor-Bernstein]
  Soient $X,Y$ des ensembles. S'il existe une injection $f : X \to Y$ et une
  injection $g : Y \to X$, alors il existe une bijection $h : X \simeq Y$.
\end{theorem}

\begin{proof}
  A FAIRE
\end{proof}

La relation $\preceq$ est donc une relation d'ordre sur les classes
d'équipotence, mis à part l'impossibilité de décrire ces classes par des
ensembles. On peut prouver que cette relation est totale si et seulement si
l'axiome du choix est vrai. Nous montrons ici que l'axiome du choix implique
que $\preceq$ est totale, et nous verrons ensuite avec le théorème de Zermelo
que c'est en fait une équivalence.

\begin{property}
  L'axiome du choix implique la proposition suivante~:
  \[\forall X,Y,(\exists f : X \to Y, \mathrm{inj}(f))\lor(\exists f : Y \to X,
  \mathrm{inj}(f))\]
\end{property}

\begin{proof}
  On utilise le lemme de Zorn sur l'ensemble
  \[\{f : A \to B \mid A \subseteq X, B \subseteq Y, \mathrm{inj}(f)\}\]
  ordonné par $\subseteq$. Si on a une chaîne $\{f_i\}_{i\in I}$ d'éléments,
  alors on veut montrer que
  \[g\defeq \bigcup_{i\in I} f_i\]
  est une injection entre une partie de $X$ et une partie de $Y$~:
  \begin{itemize}
  \item $g$ est bien définie. Si $(x,y)\in g$ et $(x,z)\in g$, alors comme
    $\{f_i\}$ est une chaîne, il existe $i\in I$ tel que $(x,y)\in f_i$ et
    $(x,z)\in f_i$, et comme $f_i$ est une fonction, $y=z$.
  \item $g$ est injective. Si $(x,y)\in g$ et $(z,y)\in g$, alors comme
    $\{f_i\}$ est une chaîne, on trouve $i\in I$ tel que $(x,y)\in f_i$ et
    $(z,y)\in f_i$, et comme $f_i$ est injective, $x=z$.
  \end{itemize}

  On applique donc le lemme de Zorn~: on trouve $f : A \to B$ maximale pour
  l'inclusion, et injective. Supposons que $A\subsetneq X$ et $B\subsetneq Y$,
  alors on trouve $x\in X\setminus A$ et $y\in Y\setminus B$, et on peut
  construire $g\cup\{(x,y)\}$ qui est une injection strictement plus grande
  que $f$. On en déduit que $A = X$ ou $B = Y$. Si $A = X$, alors on a trouvé
  une injection de $X$ dans $Y$. Si $B = Y$, alors la transposée de $f$,
  c'est-à-dire l'ensemble $\{(y,x)\mid (x,y)\in f\}$, est une injection de
  $Y$ dans $X$.
\end{proof}

Le théorème de Zermelo indique qu'il est toujours possible de munir un ensemble
d'un bon ordre.

\begin{theorem}[Zermelo]
  Soit $X$ un ensemble. Alors il existe $<$ qui est un bon ordre sur $X$.
\end{theorem}

\begin{proof}
  Ce théorème est une conséquence de la proposition précédente~: on sait que
  $\alpha_X$ ne s'injecte pas dans $X$, donc $X$ s'injecte dans $\alpha_X$,
  donc l'ordre induit par $\alpha_X$ est un bon ordre sur $X$.
\end{proof}

\begin{property}
  Le théorème de Zermelo implique, dans $\ZF - \AxF$, l'axiome du choix.
\end{property}

\begin{proof}
  Soit $X$ un ensemble non vide dont tous les éléments sont non vides.
  On trouve, par Zermelo, un bon ordre $<$ sur $X$. On définit alors la
  fonction de choix
  \[\begin{array}{ccccc}
  \varphi & : & \powerset(X)\setminus\{\varnothing\} & \longrightarrow
  & X\\
  & & X &\longmapsto & \min_{<} X
  \end{array}\]
  et comme $\min X\in X$, on en déduit que cette fonction vérifie bien l'énoncé
  de l'axiome du choix.
\end{proof}

Comme on a prouvé que la dichotomie cardinale ($\preceq$ est un ordre total)
implique le théorème de Zermelo, on en déduit que les deux énoncés impliquent
l'axiome du choix.

On a donc plusieurs énoncés utiles équivalents à l'axiome du choix~:
\begin{itemize}
\item le lemme de Zorn.
\item le théorème de Zermelo.
\item le fait que $\preceq$ est un pré-ordre total.
\end{itemize}

\subsection{Nombre cardinal}

Avec le théorème de Zermelo, il devient direct que tout ensemble est en
bijection avec un ordinal, puisqu'un isomorphisme d'ensembles ordonné est en
particulier une bijection, et que tout ensemble bien ordonné est isomorphe à
un unique ordinal. Cependant, on n'a pas unicité du bon ordre, et donc de
l'ordinal auquel un ensemble quelconque est isomorphe. Même au sein des ordinaux
il y a un défaut d'unicité, puisque $\omega \simeq \omega + 1$ (il suffit
d'envoyer $\omega$ sur $0$ et $n$ sur $n + 1$). On veut donc représenter une
classe d'équipotence par la hiérarchie ordinale en prenant un unique
représentant, ce qui se fait en utilisant le plus petit ordinal de la classe.
Cela motive donc notre définition de la hiérarchie cardinale.

\begin{definition}[Cardinal]
  Soit $\alpha \in \Ord$. On dit que $\alpha$ est un cardinal si pour tout
  $\beta < \alpha$, $\alpha\not\preceq \beta$. On dénote la classe des cardinaux
  par $\Card$.

  Pour un ensemble quelconque $X$, on appelle son cardinal, noté $\Card(X)$,
  comme le plus grand cardinal inférieur à l'ordinal associé à $(X,<)$ où $<$
  est un bon ordre sur $X$.
\end{definition}

\begin{proof}
  Il nous faut montrer que le choix du bon ordre sur $X$ n'intervient pas dans
  la définition de $\Card(X)$. Si $<,<'$ sont deux bons ordres sur $X$, alors
  l'ordinal associé à $(X,<)$ et celui associé à $(X,<')$ sont en bijection,
  donc un cardinal est inférieur à l'ordinal associé à $(X,<)$ si et seulement
  s'il est inférieur à l'ordinal associé à $(X,<')$, donc le plus grand
  cardinal inférieur à l'un d'eux est aussi le plus grand cardinal inférieur à
  l'autre.
\end{proof}

Un cardinal est donc un ordinal qui ne s'injecte dans aucun de ses éléments.
En observant la définition de l'ordinal de Hartogs, le fait que cet ordinal est
même un cardinal est peu étonnant, puisqu'on l'énonce par le fait qu'il ne
s'injecte pas dans un plus petit ensemble.

\begin{property}
  Pour tout ensemble $X$, l'ordinal de Hartogs $\alpha_X$ est un cardinal.
  Il est le plus petit cardinal ne s'injectant pas dans $X$.
\end{property}

\begin{proof}
  Si $\beta\in\alpha_X$ alors $\beta$ s'injecte dans $X$, donc $\alpha_X$ ne
  peut pas s'injecter dans $\beta$ (parce qu'alors il s'injecte dans $X$). Si
  un cardinal $\kappa$ ne s'injecte pas dans $X$, alors $X$ s'injecte dans ce
  cardinal. Par trichotomie sur les ordinaux, $\kappa = \alpha_X$ ou
  $\kappa \in \alpha_X$ ou $\alpha_X \in \kappa$. Seul le cas
  $\kappa\in\alpha_X$ est à traiter (les autres nous donnent bien que
  $\alpha_X$ est le plus petit cardinal possible). Si $\kappa\in\alpha_X$,
  alors $\kappa$ est associé à un certain bon ordre partiel $(Y,R)\in\Ord(X)$,
  donc $\kappa$ s'injecte dans $X$, ce qui est faux par hypothèse.
\end{proof}

\begin{definition}[Cardinal successeur]
  On définit le cardinal successeur d'un cardinal $\kappa\in\Card$ comme
  $\kappa^+ = \alpha_\kappa$, l'ordinal de Hartogs de $\kappa$. C'est le plus
  petit cardinal strictement supérieur à $\kappa$.
\end{definition}

\begin{example}
  Donnons quelques exemples de cardinaux~:
  \begin{itemize}
  \item tous les ordinaux finis, c'est-à-dire les éléments de $\omega$, sont
    des cardinaux.
  \item $\omega$ lui-même est un cardinal.
  \item $\omega_1 = \omega^+$, l'ensemble des ordinaux dénombrables, est un
    cardinal.
  \end{itemize}
\end{example}

\begin{exercise}
  Montrer qu'un cardinal infini est un ordinal limite.
\end{exercise}

On définit maintenant la fonction $\aleph$, qui décrit les cardinaux infinis à
partir des ordinaux. Pour que notre définition ait du sens, on va montrer un
premier lemme.

\begin{lemma}
  Soit $(\kappa_i)_{i\in I}$ une famille de cardinaux. Alors
  $\displaystyle\bigcup_{i\in I} \kappa_i$ est un cardinal.
\end{lemma}

\begin{proof}
  Soit $\beta < \bigcup_{i\in I}\kappa_i$, par définition on trouve un
  $i \in I$ tel que $\beta < \kappa_i$, donc il n'existe pas d'injection
  $\kappa_i \to \beta$, et en particulier pas d'injection
  $\bigcup_{i\in I}\kappa_i\to\beta$.
\end{proof}

\begin{definition}[Fonction aleph]
  On définit par induction transfinie la fonction entre classes
  $\aleph : \Ord \to \Card$~:
  \begin{itemize}
  \item $\aleph_0 = \omega$
  \item $\aleph_{\alpha + 1} = \aleph_\alpha^+$
  \item pour $\lambda$ limite,
    $\displaystyle\aleph_\lambda = \bigcup_{\beta < \lambda}\aleph_\beta$
  \end{itemize}
\end{definition}

Montrons plusieurs propriétés de la fonction $\aleph$~: elle est strictement
croissante, supérieure à l'identité et bijective entre $\Ord$ et la classe des
cardinaux infinis.

\begin{property}
  Les propositions suivantes sont vérifiées~:
  \begin{enumerate}[label=(\roman*)]
  \item pour tous $\alpha < \beta$, $\aleph_\alpha < \aleph_\beta$.
  \item pour tout $\alpha \in \Ord$, $\alpha \leq \aleph_\alpha$.
  \item pour tout $\kappa \in \Card$ infini, il existe $\alpha \in \Ord$ tel
    que $\kappa = \aleph_\alpha$.
  \end{enumerate}
\end{property}

\begin{proof}
  On prouve chaque assertion.
  \begin{enumerate}[label=(\roman*)]
  \item A FAIRE
  \item A FAIRE
  \item Par l'absurde, supposons qu'il existe des valeurs non atteintes. Soit la
    plus petite valeur non atteinte, $\kappa$. On pose alors
    \[\alpha = \{\gamma \in \Ord \mid \aleph_\gamma < \kappa\}\]
    qui est un ordinal.
    
    Si $\alpha = \beta + 1$ pour un certain $\beta$, alors on sait que
    $\aleph_{\alpha}$ est le plus petit cardinal strictement supérieur à
    $\aleph_\beta$, c'est-à-dire $\kappa$.
    
    Si $\alpha$ est limite, alors
    $\displaystyle\kappa = \bigcup_{\beta \in \alpha} \aleph_\alpha$
    donc $\kappa = \aleph_\alpha$.

    Par l'absurde, $\aleph$ est surjective.
  \end{enumerate}
\end{proof}

\begin{definition}[Cardinal limite]
  On dit qu'un cardinal $\kappa$ est limite s'il est un cardinal de la forme
  $\aleph_{\lambda}$ pour $\lambda$ limite. De façon équivalente, cela signifie
  qu'il n'est pas un cardinal successeur.
\end{definition}

\begin{remark}
  Le fait de considérer que $\omega$ est ou non un cardinal limite dépend des
  conventions, mais les cardinaux limite d'intérêt sont surtout les cardinaux
  non dénombrables.
\end{remark}

\subsection{Arithmétique cardinale}

L'arithmétique cardinale, contrairement à l'arithmétique ordinale, est très
pauvre (et donc très simple). En effet, pour des cardinaux infinis, il n'existe
que le maximum et l'exponentiation. On va d'abord prouver le résultat central
qui rend l'arithmétique cardinale si simple.

\begin{theorem}[Hessenberg]
  Si $X$ est un ensemble infini, alors $X\times X \simeq X$.
\end{theorem}

\begin{proof}
  A FAIRE
\end{proof}

A partir de ce théorème, on peut montrer ce que l'on souhaite sur l'addition
et le produit.

\begin{proposition}
  Soient deux ensembles $X,Y$ infinis, on a les deux identités
  \[\Card(X+Y) = \max(\Card(X),\Card(Y))\qquad
  \Card(X\times Y) = \max(\Card(X),\Card(Y))\]
\end{proposition}

\begin{proof}
  On a une injection $X+Y\to X\times Y$, une injection $X\to X\times Y$ et une
  injection $Y\to X\times Y$. Il nous suffit donc de montrer que
  $X\times Y$ s'injecte dans le plus grand ensemble entre $X$ et $Y$, disons
  sans perte de généralité que le plus grand est $Y$. Comme $X$ s'injecte dans
  $Y$ et que $Y\times Y$ s'injecte dans $Y$ par le théorème précédent,
  $X\times Y$ s'injecte dans $Y$, ce qui est le résultat souhaité. On en déduit
  les deux égalités.
\end{proof}

On peut montrer qu'en fait ce résultat se généralise pour des unions bien plus
grandes.

\begin{proposition}
  Soit $(X_i)_{i\in I}$ une famille d'ensembles de cardinal inférieur à
  $\kappa$ indicée par un ensemble $I$ de cardinal inférieur à $\kappa$, où
  $\kappa$ est un certain cardinal. Alors
  \[\Card\Big(\bigcup_{i\in I} X_i\Big)\leq \kappa\]
\end{proposition}

\begin{proof}
  A FAIRE
\end{proof}

Il nous reste à traiter du cas de l'exponentiation. Si l'on décidait simplement
d'ordonner les fonctions à support fini, le cardinal resterait le même. A la
place, on considère l'ensemble de toutes les fonctions, qui ne pouvait pas
bien s'ordonner par une simple construction ordinale. On peut ici lui donner
un bon ordre grâce au théorème de Zermelo, mais on verra que celui-ci ne se
comporte pas aussi bien qu'on le voudrait.

\begin{definition}[Exponentiation cardinale]
  Soient $\kappa$ et $\kappa'$ deux ordinaux. On définit l'exponentiation
  par
  \[\kappa^{\kappa'} \defeq \Card(\Funct(\kappa,\kappa'))\]
\end{definition}

\begin{notation}
  On a ici une ambigüité~: $\omega^\omega$ peut vouloir dire deux choses
  suivants si on considère l'exponentiation ordinale ou cardinale. Pour lever
  cette ambigüité, on notera chaque fois $\aleph_0$ (respectivement
  $\aleph_\alpha$) pour montrer qu'on utilise les opérations ordinales, ou on
  utilisera la lettre $\kappa$ réservée implicitement aux cardinaux.
\end{notation}

\begin{remark}
  Le cardinal $2^\kappa$ est exactement le cardinal de $\powerset(\kappa)$.
\end{remark}

\begin{proposition}
  On a l'égalité suivante, pour tout ensemble $X$ infini~:
  \[\Card(X)^{\Card(X)} = 2^{\Card(X)}\]
\end{proposition}

\begin{proof}
  Comme $X$ est infini, $X\times X \simeq X$, et $X \simeq 2\times X$, donc
  \begin{align*}
    \Card(X)^{\Card(X)} &= (2\times\Card(X))^{\Card(X)}\\
    &= (2^{\Card(X)})^{\Card(X)} \\
    &= 2^{\Card(X)\times\Card(X)} \\
    &= 2^{\Card(X)}
  \end{align*}
\end{proof}

On voit donc que l'exponentiation est surtout utile pour considérer le cardinal
de l'ensemble des parties. On va donc définir la hiérarchie $\beth$ de façon
analogue à $\aleph$, mais en itérant l'exponentiation.

\begin{definition}[Fonction beth]
  On définit la fonction $\beth : \Ord \to \Card$ par récursion transfinie~:
  \begin{itemize}
  \item $\beth_0 = \aleph_0$
  \item $\beth_{\alpha + 1} = 2^{\beth_\alpha}$
  \item pour tout ordinal $\lambda$ limite,
    $\displaystyle\beth_\lambda = \bigcup_{\beta < \lambda} \beth_\beta$
  \end{itemize}
\end{definition}

La fonction $\beth$ est strictement croissante grâce au théorème de Cantor.

\begin{theorem}[Cantor]
  Soit $X$ un ensemble, alors $X \prec \powerset(X)$ (il y a une injection mais
  pas de surjection de $X$ dans $\powerset(X)$).
\end{theorem}

\begin{proof}
  L'injection est donnée par $x \mapsto \{x\}$.

  Supposons qu'il existe une surjection $f : X \to \powerset(X)$. Soit alors
  l'ensemble
  \[Y\defeq \{x\in X\mid x\notin f(x)\}\]
  Comme $f$ est surjective, on trouve $y$ tel que $f(y) = Y$. Alors si
  $y\in Y$, on en déduit que $y\in f(y)$ et donc que $y\notin Y$. Si
  $y\notin Y$ alors $y\notin f(y)$ donc $y\in Y$. La situation est donc absurde,
  donc il n'existe pas de telle surjection.
\end{proof}

La question vient donc naturellement~: à quel point les deux hiérarchies
coïncident-elles ? Cette question mène à deux énoncés.

\begin{definition}[Hypothèse du continu]
  On appelle hypotèse du continu l'assertion
  \[\HC \defeq \aleph_1 = \beth_1\]
  et l'hypothèse du continu généralisée l'assertion
  \[\HCG \defeq \forall \alpha \in \Ord, \aleph_\alpha = \beth_\alpha\]
\end{definition}

Le fait est assez connu que $\HC$ et $\HCG$ sont en fait des énoncés
indépendants de $\ZFC$, c'est-à-dire que si $\ZFC$ est cohérent, alors
$\ZFC + \HC$ comme $\ZFC + \lnot\HC$ sont cohérents (et de même pour $\HCG$).

\section{Cofinalité}\label{sct.cof}

Pour introduire la notion de cofinalité, commençons par observer deux exemples.
Tout d'abord, étant donnée une famille finie $(n_i)_{i\in F}$ d'entiers, la borne
supérieure de cette famille est un entier. Ainsi si $(\alpha_i)_{i\in \beta}$ est
une famille d'ordinaux tous inférieurs strictement à $\omega$, et où
$\beta < \omega$, alors $\sup \alpha_i < \omega$.

Ce phénomène n'a rien de surprenant et, écrit comme ça, ne nous apporte pas
vraiment d'informations. Cependant, en le généralisant à $\omega_1$, la chose
devient alors moins intuitive. En effet, si $(\alpha_i)_{i\in \beta}$ est une
famille d'ordinaux dénombrables, et $\beta$ est dénombrable, alors
$\sup \alpha_i$ est dénombrable, donc pour une famille
$(\alpha_i)_{i\in \beta}$ où $\alpha_i < \omega_1$ pour tout $i$ et
$\beta < \omega_1$, $\sup \alpha_i < \omega_1$. Le cardinal $\omega_1$ ne peut
donc pas être atteint par une suite.

Au contraire, le cardinal $\aleph_\omega$ peut parfaitement être atteint par
une suite, dénombrable~: $n \mapsto \aleph_n$ est une telle suite. On voit donc
apparaître une nouvelle propriété sur les ordinaux, qui n'est pas simplement
liée à la taille.

Nous allons dans un premier temps nous concentrer sur la définition de
cofinalité, de suite cofinale et sur les propriétés usuelles de cette notion.
Nous verrons ensuite le théorème de König et ses conséquences sur le cardinal
d'un ensemble de parties.

\subsection{Premières définitions}

Nous avons parlé ci-dessus d'atteindre un ordinal. Une façon plus formelle de
définir cette idée est d'avoir une partie non strictement majorée, c'est-à-dire
une partie $X\subseteq \alpha$ telle que pour tout $\beta \in \alpha$ il existe
$\gamma \in X$ tel que $\beta \leq \gamma$.

\begin{definition}[Cofinalité]
  Soit $\alpha\in\Ord$, on dit qu'une partie $X\subseteq \alpha$ est cofinale si
  \[\forall \beta \in \alpha, \exists \gamma \in X, \beta \leq \gamma\]

  On définit la cofinalité d'un ordinal $\alpha$ par~:
  \[\cof(\alpha) \defeq \min \{\type(X)
  \mid X\subseteq\alpha, X\text{ est cofinale}\}\]
\end{definition}

Remarquons que, par définition, $\alpha$ est une partie cofinale dans lui-même,
on a donc l'inégalité $\cof(\alpha) \leq \alpha$.

\begin{definition}[Ordinal régulier, cardinal singulier]
  Un ordinal $\alpha$ tel que $\cof(\alpha) = \alpha$ est un ordinal régulier.
  Un cardinal qui n'est pas régulier est dit singulier.
\end{definition}

\begin{example}
  $\omega$ est un ordinal régulier, et $\omega_1$ aussi, comme nous l'avons dit
  au début. $\aleph_\omega$ est un cardinal singulier, puisqu'il a une partie
  cofinale de type $\omega$.
\end{example}

\'Eliminons maintenant le cas de la cofinalité finie. Le cas de la cofinalité
nulle n'est possible que pour l'ordinal nul, qui n'est pas très intéressant.

\begin{proposition}
  Un ordinal $\alpha$ a une cofinalité finie non nulle si et seulement si
  $\cof(\alpha) = 1$, si et seulement si $\alpha$ est un ordinal successeur.
\end{proposition}

\begin{proof}
  Si $\alpha$ a une cofinalité finie non nulle, alors soit $X$ une partie
  cofinale de type $n$. On peut en prendre le maximum, qui est une partie
  cofinale, donc $\cof(\alpha) = 1$. Si $\cof(\alpha) = 1$ alors on trouve
  $\beta$ tel que $\forall \gamma \in \alpha, \gamma \leq \beta$, donc
  $\alpha = \beta + 1$. Si $\alpha$ est successeur, alors en prenant $\{\beta\}$
  tel que $\alpha = \beta + 1$ on trouve une partie cofinale finie.
\end{proof}

On va maintenant s'intéresser à l'effet des opérations ordinales sur la
cofinalité, mais nous avons d'abord besoin d'un lemme assez naturel.

\begin{lemma}\label{lem.cof.type}
  Soit $\alpha$ un ordinal limite et $X$ une partie cofinale dans $\alpha$.
  Alors on a l'égalité $\cof(\alpha) = \cof(\type(X))$.
\end{lemma}

\begin{proof}
  Soit $Y\subseteq X$ telle que $\type(Y) = \cof(\type(X))$. Comme $X$ est
  cofinale dans $\alpha$ et $Y$ est cofinale dans $X$, $Y$ est cofinale dans
  $\alpha$, donc $\cof(\alpha) \leq \type(Y)$, donc
  $\cof(\alpha) \leq \cof(\type(X))$.

  Réciproquement, soit $Y\subseteq \alpha$ cofinale telle que
  $\type(Y) = \cof(\alpha)$. On définit alors la fonction
  \[\begin{array}{ccccc}
  f & : & Y & \longmapsto & X\\
  & & y &\longmapsto &\min (X\setminus y)
  \end{array}\]
  L'image de $f$ dans $X$ est cofinale. En effet, si on prend $\beta \in X$,
  alors on peut trouver $\gamma \in \alpha$ tel que $\beta < \gamma$ (car
  $\alpha$ est limite). Comme $Y$ est cofinale dans $\alpha$, on trouve
  $\delta \in Y$ tel que $\beta < \delta$, mais alors $f(\delta) > \delta$
  par définition, donc $\beta < f(\delta)$, donc on trouve un élément dans
  $\im(f)$ qui est supérieur à $\beta$. Comme $\im(f)$ est cofinale, on en
  déduit que $\cof(\type(X)) \leq \type(Y)$, donc que
  $\cof(\alpha) = \cof(\type(X))$ par double inégalité.
\end{proof}

On peut maintenant comprendre quel effet a une somme sur la cofinalité.
Moralement, l'ordinal $\alpha + \beta$ peut se voir comme la droite représentant
$\alpha$ à laquelle on a accolé la droite représentant $\beta$ à sa fin. Ainsi,
pour avoir une partie non bornée de $\alpha + \beta$, la partie en $\alpha$
n'importe pas, puisqu'il suffit de dépasser les éléments de $\beta$. De même
$\alpha \times \beta$ peut se voir comme $\beta$ successions de l'ordinal
$\alpha$. Si $\beta$ est successeur, alors il suffit d'avoir une cofin à
$\alpha$ et de l'appliquer au dernier segment de $\alpha\times\beta$. Si
$\beta$ est limite, alors il faut au contraire avoir une cofin à $\beta$ pour
l'appliquer à chaque copie de $\alpha$.

\begin{proposition}\label{prop.cof.oper}
  Soit $\alpha$ un ordinal limite, et $\beta$ un ordinal. Alors
  \begin{itemize}
  \item $\cof(\alpha + \beta) = \cof(\beta)$
  \item si $\beta$ est successeur, $\cof(\alpha \times \beta) = \cof(\alpha)$
  \item si $\beta$ est limite, $\cof(\alpha\times\beta) = \cof(\beta)$
  \end{itemize}
\end{proposition}

\begin{proof}
  On prouve chaque point.
  \begin{itemize}
  \item On a vu qu'une autre façon d'écrire $\alpha + \beta$ était comme
    $\type(\{0\}\times\alpha \cup \{1\}\times\beta, \lexl)$. Ainsi $\beta$ est
    cofinal dans $\alpha + \beta$, à l'encodage près, donc
    $\cof(\beta) = \cof(\alpha + \beta)$.
  \item Par définition, $\alpha \times \beta = \alpha \times \gamma + \alpha$
    où $\beta = \gamma + 1$, donc avec l'argument précédent,
    $\cof(\alpha\times\beta) = \cof(\alpha)$.
  \item Dans le cas où $\beta$ est limite, en utilisant la représentation de
    $\alpha \times \beta$ comme $\type(\alpha\times\beta,\lexl)$ avec le produit
    cartésien, on remarque que
    \[F \defeq \{(0,\gamma) \mid \gamma \in \beta\}\]
    est cofinal dans $\alpha \times \beta$. Donc
    $\cof(\alpha\times \beta) = \cof(\type(F))$ et $\type(F) = \beta$, d'où le
    résultat.
  \end{itemize}
  Chaque point est donc prouvé.
\end{proof}

Pour donner les propriétés essentielles de la cofinalité, nous allons avoir
besoin d'un lemme technique pour comparer des cofinalités à partir de fonctions.

\begin{proposition}\label{lem.cof.compar.surj}
  Soient $\alpha,\beta$ deux ordinaux, $f : \beta \to \alpha$ une fonction telle
  que l'image de $f$ est cofinale dans $\alpha$. Alors
  $\cof(\alpha) \leq \beta$.
\end{proposition}

\begin{proof}
  A FAIRE
\end{proof}

On en déduit une version plus précise dans le cas d'une fonction croissante.

\begin{proposition}
  Soient $\alpha,\beta$ deux ordinaux, $f : \beta \to \alpha$ une fonction
  croissante (au sens large) telle que l'image de $f$ est cofinale dans
  $\alpha$. Alors $\cof(\alpha)\leq\cof(\beta)$.
\end{proposition}

\begin{proof}
  A FAIRE
\end{proof}

On peut maintenant prouver les propriétés essentielles sur la cofinalité, et en
particulier qu'un ordinal régulier est en fait un cardinal.

\begin{proposition}
  Soit $\alpha$ un ordinal limite. Alors on a la suite d'inégalités~:
  \[\omega\leq \cof(\alpha) \leq |\alpha|\leq \alpha\]
\end{proposition}

\begin{proof}
  Le fait que $|\alpha|\leq \alpha$ est par définition, de même
  $\omega\leq\cof(\alpha)$ se déduit du résultat sur la cofinalité finie. On
  veut donc montrer que $\cof(\alpha) \leq|\alpha|$. On sait par définition
  qu'on dispose d'une fonction $f : |\alpha|\to \alpha$ surjective, donc
  cofinale. En utilisant le \cref{lem.cof.compar.surj}, on en déduit donc que
  $\cof(\alpha)\leq|\alpha|$.
\end{proof}

\begin{property}
  Soit $\alpha$ un ordinal régulier, alors $\alpha$ est un cardinal.
\end{property}

\begin{proof}
  On sait que $\cof(\alpha) \leq |\alpha| \leq \alpha$, donc dans le cas où
  $\alpha = \cof(\alpha)$ on en déduit que $|\alpha| = \alpha$, donc $\alpha$
  est un cardinal.
\end{proof}

\begin{property}
  Soit $\alpha\in\Ord$, $\cof(\alpha)$ est un cardinal régulier.
\end{property}

\begin{proof}
  On trouve une partie $X\subseteq\alpha$ de type $\cof(\alpha)$ cofinale dans
  $\alpha$, par définition de $\cof(\alpha)$. Alors en utilisant le
  \cref{lem.cof.type}, on en déduit que $\cof(\alpha) = \cof(\cof(\alpha))$.
\end{proof}

On a vu dans le cas de $\omega$ et $\omega_1$ que des suites finies
(respectivement dénombrables) d'éléments finis (respectivement dénombrables)
ont une limite finie (respectivement dénombrable). Cette propriété se généralise
pour caractériser les inégalités sur les cofinalités.

\begin{proposition}
  Soient $\kappa$ et $\lambda$ deux cardinaux infinis. L'inégalité
  $\kappa < \cof(\lambda)$ est vraie si et seulement si toute fonction
  $f : \kappa \to \lambda$ est bornée.
\end{proposition}

\begin{proof}
  A FAIRE
\end{proof}

On va maintenant affiner le critère de régularité. Tout d'abord, un cardinal
successeur est régulier.

\begin{theorem}
  Soit $\kappa$ un cardinal infini. Alors $\kappa^+$ est régulier.
\end{theorem}

\begin{proof}
  A FAIRE
\end{proof}

Pour les cardinaux limites, on a un résultat beaucoup plus faible.

\begin{proposition}
  Soit $\kappa = \aleph_\gamma$ où $\gamma$ est un ordinal limite. Alors
  $\cof(\kappa) = \cof(\gamma)$.
\end{proposition}

\begin{proof}
  Soit $X = \{\aleph_\beta \mid \beta < \gamma\}$, alors $\type(X) = \gamma$,
  donc par le \cref{lem.cof.type}, $\cof(\kappa) = \cof(\type(X))$.
\end{proof}

On voit en particulier que dans le cas d'un point fixe de $\aleph$, cette
propriété n'aide pas.

\begin{definition}[Cardinal faiblement inaccessible]
  On dit qu'un cardinal $\kappa$ est faiblement inaccessible si $\kappa$ est à
  la fois un cardinal limite et un cardinal régulier.
\end{definition}

\subsection{Répartition des cofinalités}

On a vu dans la \cref{prop.cof.oper} que la cofinalité interagit de façon assez
inattendue avec les opérations. En particulier, celle-ci n'est pas croissante,
et la répartition des cofinalités en-dessous d'un cardinal donné est assez
anarchique.

Nous allons voir dans cette sous-section qu'en fait, dans un cardinal $\kappa$
fixé, toutes les cofinalités en-dessous de $\kappa$ sont représentées
arbitrairement haut.

\begin{proposition}
  Soient $\kappa$ et $\mu$ deux cardinaux infinis. On peut trouver $\lambda$ tel
  que $\mu \leq \lambda$, $\cof(\lambda) = \cof(\kappa)$, et même que
  $\aleph_\lambda = \lambda$.
\end{proposition}

\begin{proof}
  A FAIRE
\end{proof}

\begin{proposition}
  Soient $\kappa, \lambda$ deux ordinaux réguliers infinis, avec
  $\kappa < \lambda$. Alors
  \[E_\kappa^\lambda \defeq \{\alpha < \lambda \mid \cof(\alpha) = \kappa\}\]
  est un ensemble cofinal dans $\lambda$.
\end{proposition}

\begin{proof}
  A FAIRE
\end{proof}

\subsection{Théorème de König}

Cette sous-section s'intéresse au théorème de König, qui est une des principales
utilisations de la cofinalité. Nous allons en voir les principales conséquences,
mais commençons par le théorème lui-même.

\begin{theorem}[König]\label{thm.Konig}
  Soit $\kappa$ un cardinal. Alors~:
  \begin{itemize}
  \item si $\kappa$ est régulier, alors pour toute famille $\mathcal F$
    d'ensembles telle que $|\mathcal F| < \kappa$ et pour tout $X\in\mathcal F$,
    $|X| < \kappa$, alors $\displaystyle\Big|\bigcup \mathcal F\Big| < \kappa$.
  \item si $\cof(\kappa) < \kappa$, alors il existe une famille $\mathcal F$
    de parties de $\kappa$ telle que $|\mathcal F| = \cof(\kappa)$ et pour tout
    $X\in\mathcal F$, $|X|<\kappa$, et $\displaystyle\bigcup\mathcal F =\kappa$.
  \end{itemize}
\end{theorem}

\begin{proof}
  On prouve les deux points~:
  \begin{itemize}
  \item A FAIRE
  \item A FAIRE
  \end{itemize}
  D'où le résultat.
\end{proof}

On peut maintenant en tirer deux conséquences sur l'exponentiation cardinale.

\begin{proposition}
  Soit $\kappa$ un cardinal infini. Alors $\kappa^{\cof(\kappa)} > \kappa$.
\end{proposition}

\begin{proof}
  A FAIRE
\end{proof}

\begin{theorem}
  Soit $\kappa \geq 2$ et $\lambda$ un cardinal infini. Alors
  $\cof(\kappa^\lambda) > \lambda$.
\end{theorem}

\begin{proof}
  A FAIRE
\end{proof}

On sait donc que la fonction $\kappa \mapsto 2^\kappa$ vérifie les propriétés
suivantes~:
\begin{itemize}
\item $\kappa < 2^\kappa$ (théorème de Cantor)
\item si $\kappa < \lambda$ alors $2^\kappa \leq 2^\lambda$
\item $\kappa < \cof(2^\kappa)$
\end{itemize}

En fait, on peut prouver qu'il est impossible d'être plus précis que ces trois
conditions dans le cas des cardinaux réguliers~: c'est le théorème d'Easton.
Plus précisément, pour toute fonction $f : C \to \Card$ croissante, où $C$ est
un ensemble de cardinaux réguliers, qui vérifie $\kappa < \cof(f(\kappa))$, il
existe un modèle de $\ZFC$ vérifiant que $\kappa \mapsto 2^\kappa$ coïncide avec
$f$.
