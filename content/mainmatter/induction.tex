\chapter[Induction]{Induction}
\chaptermark{Induction}
\label{chp.induction}

\minitoc

L'un des outils fondamentaux en logique est l'induction. Intuitivement, elle
peut se voir comme une généralisation du principe de récurrence. Nous allons
cependant adopter un formalisme différent de celui utilisé pour faire une simple
récurrence. Les objets principaux sur lesquels nous utiliserons l'induction sont
les ensembles inductifs et les relations inductives, que nous présenterons. Ces
deux objets sont associés à des formalismes différents : le premier aux
grammaire en forme de Backus-Naur, et le deuxième aux points fixes et à la
théorie des treillis. Comme la théorie des treillis sera étudiée plus tard dans
cette partie préliminaire, nous ne traiterons qu'un cas restreint suffisant pour
le travail sur l'induction.

\section{Ensemble inductif}

\subsection{Construction d'un ensemble inductif}

Au niveau intuitif, les ensembles finis semblent avoir une réalité plus
robuste que les ensembles infinis. Il est en effet très facile de se convaincre
à partir de règles simples qu'il existe un ensemble à $3$ éléments, ou à $n$
éléments pour $n$ aussi grand que l'on veut (bien que se convaincre qu'il existe
un ensemble à $300!$ éléments semble légèrement plus long). Cette robustesse
découle du fait qu'on peut explicitement les construire, et cette possibilité
n'existe que pour un ensemble fini. Pourtant, l'ensemble $\mathbb N$ tend aussi
à être plus facilement accepté qu'un ensmeble tel que $\mathbb R / \mathbb Q$.
Un point essentiel qui rend le premier ensemble logiquement plus simple que le
deuxième est qu'il est facile à engendrer : l'ensemble $\mathbb N$ est constitué
de l'élément $0$ et de l'opération $S$ définie par $n \mapsto n + 1$, et tout
autre élément de $\mathbb N$ peut être construit à partir de ces deux éléments.
Sa structure est donc fondamentalement simple, et peut être décrite en des
termes finis.

C'est exactement cette idée de structure générée par des termes finis qui est
formalisée par les ensembles inductifs. Un ensemble inductif va être un ensemble
obtenu par une liste de générateurs, chaque générateur ayant une arité (un
nombre d'objets qu'il prend en entrée). Dans cette définition d'ensemble
inductif, l'exemple canonique est bien sûr $\mathbb N$ lui-même, qu'on peut
définir par :
\begin{itemize}
\item un constructeur sans argument, $0$
\item un constructeur à un argument, $S$
\end{itemize}

Avant de donner la définition d'ensemble inductif, nous allons donner un
formalisme pour parler des constructeurs.

\begin{definition}[Signature]
  Une signature est un couple $C,\alpha$ tel que $\alpha : C \to \mathbb N$.
  On appelle $C$ l'ensemble des constructeurs et, pour $c \in C$, $\alpha(c)$
  est appelé l'arité de $c$.
\end{definition}

Une signature sera généralement donnée sous forme dite de Backus-Naur. Cette
présentation se décompose de la façon suivante :
$$a,b,\ldots ::= \mathrm{cas}\;1 \mid \mathrm{cas}\;2 \mid \ldots$$
où $a,b,\ldots$ représentent les éléments que les constructeurs définissent,
et où chaque cas est la définition d'un nouveau constructeur (ou d'une famille
de constructeurs). Par exemple pour le cas de $\mathbb N$, nous avons :
$$n ::= 0 \mid S\;n$$
Il est fréquent d'employer des variables qui seront quantifiées hors de la
définition à proprement parler, comme
$$l ::= \mathrm{nil} \mid \mathrm{cons}(a,l)$$
où $a \in A$ et $A$ est certain ensemble fixé au préalable. Cette définition
doit se lire comme l'ensemble $(\{\mathrm{nil}\} \cup A,\alpha)$ où $\alpha$
est défini par :
$$
\begin{array}{rcccl}
  \alpha &:& C & \longrightarrow & \mathbb N\\
  & & \mathrm{nil} & \longmapsto & 0 \\
  & & a (\in A) & \longmapsto & 1
\end{array}
$$

Voyons maintenant comment associer à une signature un ensemble généré par les
constructeurs donnés dans la signature. L'ensemble généré doit être un ensemble
$X$ contenant, pour chaque $x_1,\ldots,x_n \in X$ et $c \in C$ d'arité $n$,
l'objet $c(x_1,\ldots,x_n)$, et ne doit contenir que les objets de cette forme.
Nous procédons alors par le bas : un premier ensemble est construit par
l'ensemble $C_0 = \{ c \in C \mid \alpha(c) = 0\}$, puis l'ensemble $C_1$ est
construit par $C_1 = C_0 \cup \{ c(x_1,\ldots,x_n) \mid c \in C, x_1,\ldots,
x_n \in C_0, \alpha(c) = n\}$ et ainsi de suite. Comme $c$ est simplement un
élément dans notre cas, écrire $c(x_1,\ldots,x_n)$ n'a pas de sens, c'est
pourquoi l'on va utiliser à la place $(c,x_1,\ldots,x_n)$.

\begin{definition}[Ensemble inductif sur une signature]
  Soit $(C,\alpha)$ une signature, on définit la suite d'ensembles
  $(X_i)_{i\in\mathbb N}$ par :
  \begin{itemize}
  \item $X_0 = \varnothing$
  \item $X_{n+1} = \{ (c,x_1,\ldots,x_p) \mid c\in C, (x_1,\ldots,x_p)\in (X_n)^p,
    \alpha(c) = p \}$
  \end{itemize}

  L'ensemble inductif engendré par $(C,\alpha)$ est alors l'ensemble
  $$X = \bigcup_{n\in \mathbb N} X_n$$
\end{definition}

La définition donnée n'est pas exactement celle décrite plus haut, mais la
proposition suivante assure que l'union finale génère bien le même ensemble
avec les deux méthodes.

\begin{proposition}
  Soit $(C,\alpha)$ une signature, $X$ l'ensemble inductif engendré par cette
  signature et $(X_n)$ la suite précédemment construite. Alors
  $$\forall n,m \in \mathbb N, n \leq m \implies X_n \subseteq X_m$$
\end{proposition}

\begin{proof}
  On procède par récurrence sur $n$ :
  \begin{itemize}
  \item comme $X_0 = \varnothing$, il est évident que
    $\varnothing \subseteq X_m$ pour tout $m \in\mathbb N$.
  \item supposons que $X_n \subseteq X_m$ pour tout $m \geq n$. Alors
    $$X_{n+1} = \{(c,x_1,\ldots,x_p)\mid c\in C, (x_1,\ldots,x_p)\in(X_n)^p,
    \alpha(c) = p\}$$
    mais par inclusion, comme $(x_1,\ldots,x_p)\in (X_n)^p$, on en déduit que
    $(x_1,\ldots,x_p)$ est aussi dans $(X_m)^p$, pour tout $m \geq n$. Ainsi
    $(c,x_1,\ldots,x_p) \in X_{m+1}$ pour tout $m \geq n$, donc
    $X_{n+1}\subseteq X_m$ pour tout $m \geq n+1$.
  \end{itemize}
\end{proof}

\subsection{Récursion et induction}

Maintenant qu'une construction a été donnée d'un ensemble inductif, il faut
vérifier que le comportement que l'on a décrit est en accord avec le
comportement réel de l'ensemble que l'on a construit. Nous avons dit que
l'ensemble engendré par $(C,\alpha)$ doit contenir exactement les éléments de
la forme $c(x_1,\ldots,x_n)$ où $x_i \in X$ pour tout $i\in\{1,\ldots,n\}$, mais
une autre façon de penser le fait que l'ensemble ne contient que des
applications de constructeurs est le fait qu'une fonction partant d'un ensemble
inductif est exactement spécifiée par son comportement sur les constructeurs.
De telle fonctions sont appelées récursives, car elles peuvent faire appel à
elles-mêmes pour s'appliquer sur les arguments d'un constructeurs, comme nous le
verrons en pratique. Nous verrons ensuite que ce principe de définition
récursive peut se modifier pour donner le principe d'induction, un analogue à
la preuve par récurrence pour un ensemble inductif quelconque.

\begin{theorem}[Propriété universelle des ensembles inductifs]
  Soit $(C,\alpha)$ une signature, et $X$ l'ensemble associé à cette signature.
  Soit un ensemble $Y$ quelconque.
  Soit une famille de fonctions $\{f_c\}_{c\in C}$ telles que pour tout $c\in C$,
  $f_c : Y^{\alpha(c)} \to Y$ (avec la convention que $Y^0 = \{*\}$ est un
  singleton quelconque). Alors il existe une unique fonction $f : X \to Y$
  telle que
  $$\forall c\in C, \forall (x_1,\ldots,x_p)\in X^{\alpha(c)},
  f((c,x_1,\ldots,x_p)) = f_c (f(x_1),\ldots,f(x_p))$$
\end{theorem}

On peut représenter l'équation précédente par le diagramme suivant, où l'égalité
signifie que le diagramme commute, c'est-à-dire que les deux chemins possibles
pour aller d'un coin à l'autre du carré sont égaux.

\begin{center}
  \begin{tikzcd}
    X^{\alpha(c)} \ar[r,"f^{\alpha(c)}"]\ar[d,"c"] & Y^{\alpha(c)} \ar[d,"f_c"] \\
    X \ar[r,"f"] & Y
  \end{tikzcd}
\end{center}

\begin{proof}
  Soit $(X_n)$ la suite d'ensemble construite précédemment pour définir $X$. On
  va prouver par récurrence sur $n$ la proposition suivante :
  $$\forall n\in \mathbb N, \exists ! f : X_n \to Y, \forall c \in C,
  \forall (x_1,\ldots,x_p)\in (X_n)^{\alpha(c)}, f((c,x_1,\ldots,x_p))
  = f_c(f_n(x_1),\ldots,f_n(x_p))$$
  \begin{itemize}
  \item Si $n = 0$, il existe une unique fonction $f_0 : \varnothing \to Y$ et
    elle vérifie la propriété par vacuité.
  \item Soit $n \in \mathbb N$. Supposons qu'il existe une unique fonction
    $f_n : X_n \to Y$ telle que
    $$\forall c\in C, \forall (x_1,\ldots,x_p)\in (X_n)^{\alpha(c)},
    f((c,x_1,\ldots,x_p)) = f_c(f(x_1),\ldots,f(x_p))$$
    On définit alors
    $$\begin{array}{rcccl}
      f_{n+1} &:& X_{n+1} &\longrightarrow & Y \\
      & & (c,x_1,\ldots,x_p) & \longmapsto & f_c(f_n(x_1),\ldots,f_n(x_p))
    \end{array}$$
    On remarque que cette fonction vérifie bien la propriété. De plus, si une
    autre fonction $g$ vérifie la propriété, alors pour $x\in X_{n+1}$, on trouve
    $c\in C$ et $x_1,\ldots,x_p \in X_n$ tels que $x = (c,x_1,\ldots,x_p)$, on a
    alors
    \begin{align*}
      f_{n+1}(x) &= f((c,x_1,\ldots,x_p)) \\
      &= f_c(f_n(x_1),\ldots,f_n(x_n)) \\
      &= g((c,x_1,\ldots,x_p))\\
      &= g (x)
    \end{align*}
    Donc pour tout $x\in X_{n+1}$, $f_{n+1} = g(x)$, ce qui signifie que
    $f_{n+1} = g$, d'où l'unicité de $f_{n+1}$.
  \end{itemize}

  Soit $x \in X$, par définition on trouve $n\in \mathbb N$ tel que $x \in X_n$,
  et on peut donc définir $f(x) = f_n(x)$. Pour montrer que la fonction est
  unique, il suffit de remarquer que toute fonction $X \to Y$ vérifiant les
  prémisses du théorème induit une fonction sur chaque $X_n$, et doit donc
  coïncider avec chaque $f_n$ sur $X_n$.
\end{proof}

Cet outil nous permet maintenant de définir des fonctions dont le domaine est
un ensemble inductif en utilisant sa structure.

Donnons un premier exemple de fonction récursive : la fonction $d :n\mapsto 2n$,
définie de $\mathbb N$ dans $\mathbb N$. En effet, on peut la décrire par
\begin{itemize}
\item $d(0) = 0$
\item $d(S\;n) = S\;S\;d(n)$
\end{itemize}
nous donnant alors une définition de $d$ grâce au théorème précédent.

Un exemple à la fois d'ensemble inductif et de fonction récursive est le
suivant. Soit $A$ un ensemble quelconque, on définit la signature des listes sur
$A$ par la grammaire suivante :
$$l ::= \mathrm{nil}\mid \mathrm{cons}(a,l)$$
où $a\in A$. L'ensemble $\mathrm{List}\;A$ est alors l'ensemble inductif
associé. On définit alors la fonction $|-|$, ou $l$, donnant la longueur d'une
liste :
\begin{itemize}
\item $|\mathrm{nil}| = 0$
\item $|\mathrm{cons}(a,l)| = 1 + |l|$
\end{itemize}
