\chapter{\'Eléments de calculabilité}
\label{chp.calculabilité}

\minitoc

\lettrine{L}{es} chapitres précédents ont permis d'introduire à la fois des
modèles de calcul, et d'appréhender des outils de base utilisés en
calculabilité, la forme normale de Kleene et l'existence d'une machine de
Turing universelle en tête.

Dans ce chapitre, nous allons étudier les théorèmes principaux de la
calculabilité, en nous éloignantdu choix particulier d'un modèle de calcul.
Grâce à l'équivalence des modèles et à la \cref{thesis.CT}, il nous est possible
d'étudier la structure de monde calculable de façon abstraite avec simplement
une énumération $\Enum$ des fonctions calculables et en ne considérant que les
fonctions $\bN \to \bN$ (ou $\bN^k \to \bN^n$).

L'étude de ce paradigme plus moderne de la calculabilité commence donc par la
notion d'énumération des fonctions calculables, avec ses propriétés essentielles
dont on montre ensuite qu'elles suffisent à caractériser, en un sens à préciser,
les énumérations de fonctions calculables. Nous donnons ensuite une introduction
à la notion de problème de décision, en présentant en particulier la classe des
problèmes calculables et calculatoirement énumérables, qui sont fondamentaux en
calculabilité. Enfin, nous présentons les réductions many-one et Turing, pour
étudier plus en détails les problèmes de décision~: nous concluons en présentant
le théorème de Post, donnant ainsi une hiérarchie de problèmes canoniques pour
caractériser la hiérarchie arithmétique (introduite dans les classes many-one).

\section{Paradigmes de la calculabilité}

L'objectif de cette section est de se familiariser avec les choix notationnels
et théoriques de la calculabilité. Il y a en effet un certain écart entre
le point de vue adopté dans un cours présentant des modèles de calcul et dans un
cours dédié à de la pure calculabilité, et nous présentons ici les notions
permettant de réduire cet écart. L'objectif premier est de décrire une
énumération des fonctions calculables à travers ses propriétés~: théorème
$s_m^n$, théorème du point fixe de Kleene\ldots

\subsection{Abstraction du modèle de calcul}

Commençons donc par définir la notion d'énumération.

\begin{definition}[\'Enumération des fonctions calculables]
  On appelle énumération des fonctions calculables une famille de fonctions
  \[(\Enum^k)_{k \in \bN} : \bN \to (\bN^k \partialto \bN)\]
  dont l'image est la classe des fonctions calculables $\Calc(\bN^k,\bN)$.

  Pour une fonction $f \in \Calc(\bN^k,\bN)$, on dira que $e \in \bN$ est un
  indice (ou un code) de $f$ lorsque $f = \Enum^k_e$ (on utilise cette notation
  au lieu de $\Enum^k(e)$ pour mettre en avant les arguments qui viennent
  après).
\end{definition}

\begin{remark}
  En utilisant le \cref{thm.norm.Kleene}, on vérifie qu'il existe bien de telles
  énumération~: on choisit donc l'énumération $(\Enum^k)_{k \in \bN}$ donnée
  par ce théorème.
\end{remark}

On a, de plus, une propriété permettant d'ajouter en paramètre le code de la
fonction qu'on considère.

\begin{property}[Paramétricité]
  A FAIRE
\end{property}
