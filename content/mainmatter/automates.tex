\chapter{Automates finis et langages réguliers}
\label{chp.auto}

\minitoc

\lettrine{P}{our} aborder la calculabilité, plusieurs approches existent. L'une
d'entre elles est de considérer directement les formalismes de fonctions
calculables. Cette approche est raisonnable puisqu'elle permet d'aller à
l'essentiel, mais cela donne une vision assez restrictive des modèles de calculs
plus faibles existants, et qui sont souvent très utiles.

C'est pourquoi nous faisons le choix dans cet ouvrage (où nous ne sommes limités
ni par le temps d'un cours ni par l'espace d'un livre imprimé) de traiter, avant
les formalismes de fonctions calculables, les formalismes liés à la hiérarchie
de Chomsky. Nous pensons en particulier que les automates permettent d'illuminer
profondément le fonctionnement intuitif d'une machine de Turing, plus complexe
mais basée sur des notions très proches.

Ce chapitre couvre donc les bases de l'étude des langages réguliers et des
automates finis. Cette classe de langage peut être décrite par de nombreux
formalismes, mais nous nous contenterons des automates finis, des expressions
régulières et des monoïdes.

Ce chapitre commence par la base de la théorie des langages~: les définitions
de ce qu'est un alphabet, un mot, un langage\ldots Nous aborderons ensuite la
classe des langages réguliers en abordant les trois formalismes décrits plus
haut, et la preuve de leur équivalence. Enfin, nous verrons les théorèmes de
structure sur ces langages~: le lemme de pompage et le théorème de
Myhill-Nérode, ainsi que ses conséquences.

\section{Alphabet, mot, langage}

