\chapter{Automates finis et langages réguliers}
\label{chp.auto}

\minitoc

\lettrine{P}{our} aborder la calculabilité, plusieurs approches existent. L'une
d'entre elles est de considérer directement les formalismes de fonctions
calculables. Cette approche est raisonnable puisqu'elle permet d'aller à
l'essentiel, mais cela donne une vision assez restrictive des modèles de calculs
plus faibles existants, et qui sont souvent très utiles.

C'est pourquoi nous faisons le choix dans cet ouvrage (où nous ne sommes limités
ni par le temps d'un cours ni par l'espace d'un livre imprimé) de traiter, avant
les formalismes de fonctions calculables, les formalismes liés à la hiérarchie
de Chomsky. Nous pensons en particulier que les automates permettent d'illuminer
profondément le fonctionnement intuitif d'une machine de Turing, plus complexe
mais basée sur des notions très proches.

Ce chapitre couvre donc les bases de l'étude des langages réguliers et des
automates finis. Cette classe de langage peut être décrite par de nombreux
formalismes, mais nous nous contenterons des automates finis, des expressions
régulières et des monoïdes.

Ce chapitre commence par la base de la théorie des langages~: les définitions
de ce qu'est un alphabet, un mot, un langage\ldots Nous aborderons ensuite la
classe des langages réguliers en abordant les trois formalismes décrits que sont
les automates, les expressions régulières et les langages reconnus par des
monoïdes, puis la preuve de leur équivalence. Enfin, nous verrons les théorèmes
de structure sur ces langages~: le lemme de pompage et le théorème de
Myhill-Nérode, ainsi que ses conséquences.

\section{Alphabet, mot, langage}

Pour étudier de façon plus systématique les langages, nous étudierons aussi de
l'algèbre avec la structure de monoïde, qui est intimement liée à celle
d'ensemble de mots. Nous verrons donc tout d'abord l'aspect concret de ce qu'est
un mot et un langage, puis nous aborderons les côtés théoriques de l'étude des
monoïdes.

\subsection{Premières définitions}

Tout d'abord, il nous faut définit ce qu'est un alphabet. Intuitivement, cela
désigne l'ensemble des symboles que nous utilisons pour écrire. Pour avoir un
trairement uniforme, on considère que des symboles comme \og ( \fg font aussi
partie de l'alphabet. C'est alors un objet particulièrement quelconque~: on peut
imaginer écrire avec à peu près n'importe quel ensemble de symboles. D'où la
définition semblant inutile d'un alphabet~: parler d'alphabet permet surtout de
situer le contexte d'étude.

\begin{definition}[Alphabet]
  Un alphabet $\Sigma$ est un ensemble fini non vide.
\end{definition}

L'intérêt d'un alphabet est avant tout de pouvoir écrire des mots avec. La
notion de mot est la notion finie par excellence~: un mot est donc simplement
une suite finie de lettres.

\begin{definition}[Mot]
  Un mot $u$ sur un alphabet $\Sigma$ est une suite finie à valeurs dans
  $\Sigma$, c'est-à-dire
  $u = (u_0,\ldots,u_{n-1})\in \Sigma^n$ pour un certain entier $n\in\mathbb N$.

  On note $\Sigma^*$ l'ensemble des mots sur $\Sigma$~:
  \[\Sigma^* \defeq \bigcup_{n\in \mathbb N} \Sigma^n\]
\end{definition}

\begin{notation}
  Pour tout mot $u\in\Sigma^n$, on note $u_i$ pour la
  \ordinalnumeralfeminin{$i$} lettre, c'est-à-dire l'image de $u$ par la
  projection $\pi_i : \Sigma^n \to \Sigma$.

  Pour tout mot $u\in\Sigma^*$, on note $|u|\in\mathbb N$ l'entier $n$ tel que
  $u\in \Sigma^n$.
\end{notation}

Remarquons qu'une suite vide est une suite finie~: on peut donc définir le mot
vide, qu'on notera $\varepsilon$.

On peut désormais définir un langage~: c'est un ensemble de mots.

\begin{definition}[Langage]
  Soit $\Sigma$ un alphabet. Un langage $\mathcal L$ sur $\Sigma$ est une partie
  $\mathcal L\subseteq\Sigma^*$.
\end{definition}

Parlons dès maintenant de stabilité~:

\begin{definition}[Stabilité]
  Soit $\Sigma$ un alphabet et $\star$ une opération binaire sur $\Sigma^*$. On
  dit qu'un langage $\mathcal L$ sur $\Sigma$ est stable par $\star$ si
  \[\forall u,v\in \mathcal L, u\star v \in \mathcal L\]
\end{definition}

Cette notion de stabilité se généralise très bien à des classes de langages,
dont il convient donc d'introduire la définition.

\begin{definition}[Classe de langages]
  Soit un alphabet $\Sigma$. On appelle classe de langages une partie
  $\mathcal C\subseteq \powerset(\Sigma^*)$ (ses éléments sont donc des
  langages).
\end{definition}

\begin{definition}[Stabilité d'une classe de langages]
  Soit $\Sigma$ un alphabet et $\star$ une opération binaire sur
  $\powerset(\Sigma^*)$. On dit qu'une classe de langages $\mathcal C$ sur
  $\Sigma$ est stable par $\star$ si
  \[\forall \mathcal L,\mathcal M\in\mathcal C,
  \mathcal L\star \mathcal M\in\mathcal C\]
\end{definition}

Pour le reste de cette sous-section, nous allons juste voir des exemples de
langages, d'opérations et de classes de langages.

\begin{example}
  Nous allons dès maintenant définir l'alphabet qui nous sera le plus utile~:
  \[\btwo \defeq \{0,1\}\]

  Voici quelques langages sur $\btwo$ ou un alphabet $\Sigma$ quelconque~:
  \begin{itemize}
  \item le langage des mots commençant par $0$~:
    \[\mathcal L_0 \defeq \{u\in\Sigma^*\mid u_0 = 0\}\]
  \item le langage des mots ayant une taille paire~:
    \[\mathcal L_{\mathrm{pair}} \defeq \{u\in\Sigma^*\mid |u|\in2\mathbb N\}\]
  \item le langage des mots dont le début et la fin sont la même lettre~:
    \[\mathcal L_{\mathrm{debut}=\mathrm{fin}}\defeq
    \{u\in\Sigma^*\mid u_0 = u_{|u| - 1}\}\]
  \item pour un langage $\mathcal L$, le langage $\overline{\mathcal L}$ des
    mots pris dans l'autre sens~:
    \[\overline{\mathcal L}\defeq
    \{u\in\Sigma^*\mid u_{|u|-1}\ldots u_0 \in \mathcal L\}\]
  \item l'intersection de deux langages.
  \item l'union de deux langages.
  \end{itemize}
\end{example}

Nous allons définir une nouvelle opération, fortement liée à la structure de
$\Sigma^*$, qui nous permettra de donner plus d'exemples de langages.

\begin{definition}[Concaténation]
  Soit un alphabet $\Sigma$. On définit l'opération de concaténation
  $\star : \Sigma^*\times\Sigma^*\to\Sigma^*$ comme le fait, étant donnés deux
  mots $u,v\in\Sigma^*$, de lire les deux mots à la suite. Formellement, on
  définit
  \[u\star v \defeq (u_0,\ldots,u_{|u|-1},v_0,\ldots,v_{|v|-1})\]

  Pour tout $n\in \mathbb N$, on notera $u^n$ pour l'itération de l'opération
  $\star$~:
  \[u^0 \defeq \varepsilon \qquad
  \forall n \in \mathbb N, u^{n+1} = u^n \star u\]
\end{definition}

\begin{remark}
  Le mot vide, $\varepsilon$, est neutre pour $\star$~: $u\star\varepsilon = u$
  et $\varepsilon \star u = u$.
\end{remark}

\begin{example}
  Voici plusieurs autres langages possibles sur un alphabet $\Sigma$~:
  \begin{itemize}
  \item pour deux langages $\mathcal L, \mathcal L'$, le langage concaténé des
    deux langages~:
    \[\mathcal L\star\mathcal L'\defeq
    \{u\star u'\mid u \in \mathcal L, u' \in \mathcal L' \}\]
  \item pour un langage $\mathcal L$, l'itération de concaténation sur ce
    langage~:
    \[\mathcal L^{\star 0} \defeq \{\varepsilon\}\qquad
    \forall n \in \mathbb N, \mathcal L^{\star(n+1)} \defeq
    \mathcal L^{\star n}\star\mathcal L\]
  \item pour un langage $\mathcal L$, le langage des mots dont toutes les
    concaténations sont encore dans $\mathcal L$~:
    \[\mathcal L_{\mathrm{stable}\star}\defeq
    \{u\in \Sigma^*\mid \forall n \in \mathbb N, u^n \in \mathcal L\}\]
  \item pour un langage $\mathcal L$, le langage des itérations de mots de
    $\mathcal L$~:
    \[\mathcal L^\star \defeq
    \{u_0\star\cdots\star u_{n-1}\mid
    n \in \mathbb N, \forall i\in \{0,\ldots,n -1\}, u_i \in \mathcal L\}\]
  \end{itemize}
\end{example}

\begin{exercise}
  Montrer que pour tout langage $\mathcal L$ sur un alphabet $\Sigma$, on a
  l'égalité suivante~:
  \[\mathcal L^\star = \bigcup_{n \in \mathbb N} \mathcal L^{\star n}\]
\end{exercise}

\begin{exercise}
  Pour tout langage $\mathcal L$ sur un alphabet $\Sigma$, on a
  $\varepsilon \in \mathcal L^\star$.
\end{exercise}

\begin{example}
  Donnons quelques exemples de classes de langages sur un alphabet $\Sigma$~:
  \begin{itemize}
  \item la classe des langages contenant $\varepsilon$~:
    \[\mathcal C_\varepsilon \defeq \{\mathcal L \subseteq \Sigma^*\mid
    \varepsilon \in \mathcal L\}\]
  \item la classe des langages stables par $\star$~:
    \[\mathcal C_\star \defeq \{\mathcal L \subseteq \Sigma^*\mid
    \forall u,v\in\mathcal L, u\star v \in \mathcal L\}\]
  \end{itemize}
\end{example}

\subsection{Aspect algébrique des langages}

Nous nous intéressons maintenant à l'aspect algébrique des langages. L'objectif
de cette sous-section est d'aboutir à la caractérisation de l'ensemble des mots
comme construction universelle (celle du monoïde libre). Pour cela, commençons
par voir ce qu'est un monoïde.

\begin{definition}[Monoïde]
  Un monoïde est un ensemble $M$ muni d'une opération binaire
  $\cdot : M \times M \to M$ et d'un élément neutre $e$,
  c'est-à-dire un élément $e\in M$ tel que
  \[\forall x\in M, x\cdot e = e \cdot x = x\]
  et tel que $\cdot$ est associative, c'est-à-dire
  \[\forall x,y,z\in M, x\cdot (y\cdot z) = (x\cdot y) \cdot z\]
\end{definition}

Définissons la notion de morphisme associée à la structure de monoïde.

\begin{definition}[Morphisme de monoïde]
  Soient $(M,\cdot,e)$ et $(N,\diamond,e')$ deux monoïdes. Un morphisme entre
  ces deux monoïdes est une fonction $f : M \to N$ vérifiant les deux conditions
  \begin{equation}
    \forall x,y\in M, f(x\cdot y) = f(x)\diamond f(y)
  \end{equation}
  \begin{equation}
    f(e) = e'
  \end{equation}
\end{definition}

Remarquons que contrairement au cas des groupes, les morphismes de monoïdes ont
besoin de stabiliser l'élément neutre.

L'exemple le plus évident de monoïde est $(\mathbb N,+,0)$, mais nous allons
voir que la structure des mots généralise de façon naturelle ce cas.

\begin{exercise}
  Soit $\Sigma$ un alphabet, montrer que $(\Sigma^*, \star,\varepsilon)$ est un
  monoïde.
\end{exercise}

Ce fait généralise le cas de $\mathbb N$ car on peut représenter un entier en
unaire comme un mot sur l'alphabet $\{*\}$~: la seule information d'un tel mot
est sa longueur.

L'étude des monoïdes permet donc d'inclure l'étude de la structure de
$\Sigma^*$. Nous allons maintenant voir qu'en fait, $\Sigma^*$ peut être
construit comme la structure de monoïde la plus simple engendrée par $\Sigma$.
Cela se traduit par la propriété universelle du monoïde libre.

\begin{theorem}[Propriété universelle du monoïde libre]
  Soit $\Sigma$ un alphabet, $(M,\diamond,e)$ un monoïde et $f : \Sigma \to M$
  une fonction (quelconque). Alors il existe un unique morphisme de monoïde
  $\tilde f$ entre $\Sigma^*$ et $M$ tel que $\tilde f (x) = f(x)$ pour tout
  $x \in \Sigma$.
\end{theorem}

\begin{proof}
  On procède par analyse-synthèse~:
  \begin{itemize}
  \item supposons que $\tilde f$ existe et vérifie les conditions décrites.
    Alors pour tout mot $u = u_0\ldots u_{n-1}$, on sait par récurrence sur
    $n$ que $\tilde f(u) = f(u_0)\diamond \cdots \diamond f(u_{n-1})$. La
    fonction ainsi définie est unique, étant donnée la construction.
  \item en considérant la fonction proposée précédemment, on voit d'abord que
    $\tilde f(x) = f(x)$ pour $x\in \Sigma$, et que si $u = v\star w$ alors
    $\tilde f(u) = \tilde f(v)\diamond \tilde f(w)$, où l'associativité est
    nécessaire pour pouvoir couper le produit au milieu sans changer le
    résultat.
  \end{itemize}
  On voit donc qu'un seul candidat est possible pour la fonction donnée, et que
  ce candidat vérifie la proposition attendue.
\end{proof}

Remarquons qu'en choisissant une fonction $f : \Sigma \to M$ sans propriété
particulière, on peut potentiellement atteindre avec $\tilde f$ des valeurs qui
ne sont pas atteintes par $f$ directement. Par exemple si on prend la fonction
\[\begin{array}{ccccc}
f & : & \Sigma & \longrightarrow & \mathbb N\\
& & \alpha & \longmapsto & 1
\end{array}\]
alors l'image d'un mot de taille $n$ est $n$. De cette façon, on vient de voir
que l'ensemble des mots de taille $n$ peut être décrit à partir du monoïde
$(\mathbb N,+,1)$ en considérant une partie du monoïde et une fonction
$f : \Sigma \to \mathbb N$.

\begin{definition}
  Soit $\Sigma$ un alphabet, $(M,\cdot,e)$ un monoïde, $h : \Sigma \to M$ une
  fonction et $S\subseteq M$. On dit que $\mathcal L(\Sigma,M,\cdot,e,h)$ est
  le langage induit par $(\Sigma,M,\cdot,e,h,S)$, qui est défini par~:
  \[\mathcal L(\Sigma,M,\cdot,e,h,S)\defeq
  \{u\in\Sigma^*\mid \tilde h(u)\in S\}\]
\end{definition}

En reprenant l'exemple de $\alpha \mapsto 1$, on voit qu'en prenant $S = \{n\}$,
on induit le langage $\Sigma^n$.

\section{Automate fini et langage reconnaissable}

On peut désormais s'intéresser plus en profondeur aux automates finis et aux
langages reconnaissables.

\subsection{Définition d'un automate}

Pour présenter un automate, le meilleur moyen est de commencer par une approche
graphique~: un automate se visualise comme une machine effectuant des opérations
élémentaires sur des mots. Dans notre premier cas, l'opération est
particulièrement simple puisque la machine se contente de lire le mot. Un
automate fini est une machine lisant linéairement un mot (elle lit chaque lettre
l'une après l'autre, sans pouvoir faire demi-tour). A chaque lettre lue,
l'automate change (ou non) d'état, parmi un ensemble fini d'états~: le mot lu
est accepté si la lecture aboutit à un certain état considéré comme acceptant.

Le formalisme derrière cette définion est lourd, mais nécessaire, nous donnons
donc la définition formelle d'un automate.

\begin{definition}[Automate fini déterministe]
  Soit un alphabet $\Sigma$. Un automate fini déterministe $\mathcal A$ sur
  $\Sigma$ est un quadruplet $(Q,q_0,\delta,F)$ où~:
  \begin{itemize}
  \item $Q$ est un ensemble fini appelé ensemble des états de $\mathcal A$,
  \item $q_0 \in Q$ est appelé l'état initial,
  \item $\delta : Q\times \Sigma \partialto Q$ est une fonction partielle
    appelée la fonction (ou table) de transitions,
  \item $F \subseteq Q$ est appelé l'ensemble des états terminaux.
  \end{itemize}

  On notera $\mathcal A_\Sigma$ l'ensemble des automates sur $\Sigma$.
\end{definition}

Le processus de lecture décrit plus haut, en reprenant les notations
introduites, consiste à prendre un mot $u = u_0\ldots u_n$ et calculer
$\delta(\delta(\cdots\delta(q_0,u_0)\cdots),u_n)$ puis vérifier que cette
valeur appartient à $F$.

\begin{definition}[Fonction de transition étendue]
  Soit un automate fini déterministe $\mathcal A$ sur un alphabet $\Sigma$ et
  $\delta$ sa fonction de transition. On appelle fonction de transition
  étendue la fonction $\delta^* : Q \times \Sigma^* \partialto Q$ définie par
  induction sur $\Sigma^*$ par~:
  \[\delta^* (q,\varepsilon) = q \qquad
  \delta^* (q,u\star a) = \delta(\delta^*(q,u),a)\]
\end{definition}

\begin{definition}[Langage accepté par un automate]
  Soit $\mathcal A$ un automate sur un alphabet $\Sigma$. On dit que
  $\mathcal A$ accepte (ou reconnait) le mot $u \in \Sigma^*$ si
  $\delta^*(q_0,u) \in F$, ce qu'on note $\mathcal A \models u$. On définit le
  langage reconnu par $\mathcal A$ par~:
  \[\mathcal L_\mathcal A \defeq \{u \in \Sigma^* \mid \mathcal A \models u\}\]
\end{definition}

Maintenant que les définitions ont été introduites, donnons une intuition
graphique de ce qu'il se passe (intuition largement indispensable pour
comprendre les automates). Pour représenter un automate $\mathcal A$, la
convention est de dessiner un graphe donc les sommets sont les états (l'ensemble
$Q$), et où $\delta$ est représentée par des arcs dirigés et étiquetés par
$\Sigma$. Par exemple, entre deux états $q$ et $q'$, s'il existe $a$ tel que
$\delta(a,q) = q'$, alors on aura une flèche $q \xrightarrow{a} q'$. L'état
initial est représenté avec une flèche entrante, et les états terminaux sont
représentés par des doubles cercles.

Prenons par exemple un premier automate sur $\btwo$~:

\begin{figure}[h]
  \centering
  \begin{tikzpicture}[node distance = 2cm, on grid, auto]
    \node[state, initial] (q_0) {$q_0$};
    \node[state,accepting] (q_1) [right = of q_0] {$q_1$};
    \node[state] (q_2) [above right = of q_1] {$q_2$};
    \node[state, accepting] (q_3) [below right = of q_1] {$q_3$};
    \path[->,>=latex]
    (q_0) edge node {0,1} (q_1)
    (q_1) edge node {1} (q_2)
    edge node {0} (q_3)
    (q_2) edge node {0} (q_3)
    edge [loop above] node {1} ()
    (q_3) edge node {1} (q_0);
  \end{tikzpicture}
\end{figure}

En lisant simplement ce dessin, on peut voir que l'automate reconnaît, par
exemple, $00$. Cette représentation a l'avantage, en plus de la lisibilité,
de donner l'ensemble des informations nécessaires pour retrouver l'objet
mathématique représenté associé.

On a ainsi construit une classe de langages~: la classe des langages
reconnaissables.

\begin{definition}[Langage reconnaissable]
  Soit $\Sigma$ un alphabet. On appelle classe des langages reconnaissbles sur
  $\Sigma$ l'ensemble des langages reconnus par les automates sur $\Sigma$~:
  \[\Reco(\Sigma) \defeq \{\mathcal L \mid
  \exists \mathcal A \in \mathcal A_\Sigma, \mathcal L = \mathcal L_\mathcal A\}\]
\end{definition}

\begin{exercise}
  Trouver le langage reconnu par l'automate suivant sur $\btwo$ (et prouver que
  le langage reconnu est bien celui attendu)~:
  \begin{figure}[h]
    \centering
    \begin{tikzpicture}[node distance = 2cm, on grid, auto]
      \node[state, initial,accepting] (q_0) {$q_0$};
      \node[state] (q_1) [right = of q_0] {$q_1$};
      \path[->,>=latex]
      (q_0) edge [bend left] node {0} (q_1)
      edge [loop above] node {1} (q_0)
      (q_1) edge [bend left] node {0} (q_0)
      edge [loop above] node {1} (q_1);
    \end{tikzpicture}
  \end{figure}
\end{exercise}

Cette premier classe de langage est en fait le centre de l'étude de ce chapitre.
On le verra, mais plusieurs définitions de cette classe existent, par des
formalismes très différents, ce qui rend cette classe de langages assez
remarquable.

L'un de ces formalismes a déjà été présenté~: on verra que les langages associés
à des monoïdes sont en fait des langages reconnus par des automates.

\subsection{Langage rationnel}

Avant de donner notre dernier formalisme, décrivons une autre classe de
langages, qui s'avérera en fait être la même classe.

\begin{definition}[Langage rationnel]
  La classe des langages rationnels sur un alphabet $\Sigma$ est la plus petite
  classe $\mathcal R$ de langages sur $\Sigma$ telle que~:
  \begin{itemize}
  \item $\varnothing \in \mathcal R$
  \item pour tout $a \in \Sigma$, $\{a\}\in \mathcal R$
  \item si $\mathcal L,\mathcal L' \in \mathcal R$ alors
    $\mathcal L\cup \mathcal L' \in \mathcal R$
  \item si $\mathcal L,\mathcal L' \in \mathcal R$ alors
    $\mathcal L\star \mathcal L' \in \mathcal R$
  \item si $\mathcal L \in \mathcal R$ alors $\mathcal L^\star \in \mathcal R$.
  \end{itemize}

  On note $\Ratio(\Sigma)$ cette classe $\mathcal R$.
\end{definition}

\begin{remark}
  Le langage $\{\varepsilon\}$ est rationnel, car il peut s'écrire
  $\varnothing^\star$. On peut donc ajouter si on le souhaite le mot vide tout
  en gardant la rationnalité d'un langage.
\end{remark}

Cette définition est une définition par induction, comme on en a rencontré
de nombreuses fois depuis le début de ce livre. Cependant, il n'est pas
particulièrement facile de manipuler un tel ensemble de langages, en particulier
pour représenter un langage de façon efficace~: devoir décrire le procédé par
lequel on obtient le langage est, pour l'instant, fastidieux.

Heureusement, il existe un formalisme permettant exactement de décrire comment
on obtient un langage rationnel à partir de ces opérations inductives, c'est ce
qu'on appelle les expressions rationnelles.

\begin{definition}[Expression rationnelle]
  Soit un alphabet $\Sigma$. On définit l'ensemble des expressions rationnelles
  par la grammaire suivante~:
  \[e,e' \Coloneq \varnothing\mid a \mid e + e' \mid e e' \mid e^\star\]
  où $a \in \Sigma$. On note $\Regex(\Sigma)$ l'ensemble des expressions
  rationnelles sur $\Sigma$.
\end{definition}

\begin{remark}
  On appelle aussi ces expressions et langages des expressions régulières et des
  langages réguliers, d'où le terme \textit{regular expression} donnant lieu à
  \textit{regex}.
\end{remark}

La correspondance est assez naturelle~: une expression de la forme $a$
représente le langage $\{a\}$, $\varnothing$ le langage vide et les trois
opérations représentent les opérations par lesquelles est stable la classe des
langages rationnels.

On définit donc une fonction d'interprétation des expressions rationnelles. De
plus, on ajoute $\varepsilon$ dans nos expressoins rationnelles, par souci de
lisbilité.

\begin{definition}[Interprétation d'une expression rationnelle]
  On définit la fonction $\Val : \Regex(\Sigma)\to\Ratio(\Sigma)$ par
  induction~:
  \begin{itemize}
  \item $\Val(\varnothing) = \varnothing$
  \item $\Val(a) = \{a\}$
  \item $\Val(e+e') = \Val(e) \cup \Val(e')$
  \item $\Val(ee') = \Val(e) \star \Val(e')$
  \item $\Val(e^\star) = (\Val(e))^\star$
  \item $\Val(\varepsilon) = \{\varepsilon\}$
  \end{itemize}
\end{definition}

\begin{exercise}
  Montrer que $\Val$ est une surjection.
\end{exercise}

\begin{exercise}
  Trouver le langage interprété par l'expression rationnelle suivante sur
  $\btwo$~:
  \[(01^\star 0)^\star\]
\end{exercise}

\begin{remark}
  Cette surjection n'est pas pour autant injective, et ce même en retirant
  $\varepsilon$ des expressions rationnelles. Par exemple, $a + a$ a la même
  interprétation que $a$.
\end{remark}

On a ainsi une façon plus simple de manipuler la classe des langages rationnels.
Notre première équivalence va donc consister en deux étapes~:
\begin{itemize}
\item montrer que les langages reconnaissables sont stables par les
  constructions de la classe des langages rationnels.
\item étant donnée un automate $\mathcal A$, construire une expression
  rationnelle $e$ dont l'interprétation est le langage reconnu par $\mathcal A$.
\end{itemize}

Pour la première étape, nous aurons besoin de faire des constructions sur les
automates.

\subsection{Constructions d'automates}

Les langages rationnels contiennent les singletons $\{a\}$ pour $a \in \Sigma$,
l'ensemble vide et sont stables par trois opérations~:
\begin{itemize}
\item l'union,
\item la concaténation de langage,
\item l'étoile d'un langage.
\end{itemize}

Il est clair qu'on peut construire des automates pour les singletons~:
\begin{figure}[h]
  \centering
  \begin{tikzpicture}[node distance = 2cm, on grid, auto]
      \node[state, initial] (q_0) {$q_0$};
      \node[state,accepting] (q_1) [right = of q_0] {$q_1$};
      \path[->,>=latex]
      (q_0) edge [bend left] node {$a$} (q_1);
  \end{tikzpicture}
\end{figure}
et pour l'ensemble vide~:
\begin{figure}[h]
  \centering
  \begin{tikzpicture}[node distance = 2cm, on grid, auto]
      \node[state, initial] (q_0) {$q_0$};
  \end{tikzpicture}
\end{figure}

Le point essentiel est donc la stabilité par les trois opérations. On va donc
devoir introduire deux constructions d'automates~: l'automate union, et les
$\varepsilon$-transitions.

\begin{definition}[Automate union]
  Soient $\mathcal A$ et $\mathcal A'$ deux automates sur un alphabet $\Sigma$.
  On définit l'automate union $\mathcal A \sqcup \mathcal A'$ par~:
  \begin{itemize}
  \item $Q_{\mathcal A\sqcup \mathcal A'} \defeq Q\times Q'$
  \item $q_{0,\mathcal A\sqcup \mathcal A'} \defeq (q_0,q'_0)$
  \item $\delta_{\mathcal A \sqcup \mathcal A'}(q,a) \defeq (\delta_{\mathcal A}(q,a),
    \delta_{\mathcal A'}(q,a))$
  \item $F_{\mathcal A \sqcup \mathcal A'} \defeq F\times Q' \cup Q\times F'$
  \end{itemize}
\end{definition}

\begin{proposition}\label{prop.union.ratio}
  Pour tous automates $\mathcal A, \mathcal A'$ sur un alphabet $\Sigma$ donné,
  on a l'équation
  \[\mathcal L_{\mathcal A} \cup \mathcal L_{\mathcal A'} =
  \mathcal L_{\mathcal A \sqcup \mathcal A'}\]
\end{proposition}

\begin{proof}
  On montre d'abord par récurrence sur $u$ que pour tout $u \in \Sigma^\star$,
  on a
  \[\delta_{\mathcal A\sqcup\mathcal A'}^\star((q_0,q'_0),u) =
  (\delta_\mathcal A^\star(q_0,u),\delta_{\mathcal A'}^\star(q'_0,u))\]
  \begin{itemize}
  \item pour $u = \varepsilon$, le résultat est par définition.
  \item si l'égalité est vraie pour $u$ et pour tout $a \in \Sigma$, on a
    \begin{align*}
      \delta_{\mathcal A\sqcup\mathcal A'}^\star ((q_0,q'_0),u\star a) &=
      \delta_{\mathcal A\sqcup \mathcal A'}(
      \delta_{\mathcal A\sqcup \mathcal A'}^\star(q_0,q'_0),a)\\
      &= \delta_{\mathcal A\sqcup\mathcal A'}(
      (\delta_\mathcal A^\star(q_0,u),\delta_{\mathcal A'}^\star(q'_0,u)),a)\\
      &= (\delta_\mathcal A(\delta_\mathcal A^\star(q_0,u)),\delta_{\mathcal A'}(
      \delta_{\mathcal A'}^\star(q'_0,u)))\\
      &= (\delta_\mathcal A^\star(q_0,u\star a),
      \delta_{\mathcal A'}^\star(q'_0,u\star a))
    \end{align*}
  \end{itemize}

  De plus, $(q,q') \in F\times Q' \cup Q\times F'$ signifie que $q \in F$ ou
  $q' \in F'$. On en déduit donc, en utilisant le résultat précédent, que
  \[\mathcal A \sqcup \mathcal A'\models u \iff
  \mathcal A \models u \lor \mathcal A'\models u\]
\end{proof}

\begin{exercise}[Automate produit]
  Soient $\mathcal A$ et $\mathcal A'$ deux automates sur un alphabet $\Sigma$.
  On définit l'automate produit $\mathcal A \times \mathcal A'$ par~:
  \begin{itemize}
  \item $Q_{\mathcal A\sqcup \mathcal A'} \defeq Q\times Q'$
  \item $q_{0,\mathcal A\sqcup \mathcal A'} \defeq (q_0,q'_0)$
  \item $\delta_{\mathcal A \sqcup \mathcal A'}(q,a) \defeq (\delta_{\mathcal A}(q,a),
    \delta_{\mathcal A'}(q_a))$
  \item $F_{\mathcal A \sqcup \mathcal A'} \defeq F\times F'$
  \end{itemize}

  Montrer que cet automate reconnait
  $\mathcal L_\mathcal A \cap \mathcal L_{\mathcal A'}$, en déduire que les
  langages reconnaissables sont stables par intersection (binaire).
\end{exercise}

On veut maintenant montrer la stabilité par concaténation de deux langages et
par étoile d'un langage. L'idée derrière ces deux constructions est relativement
simple~: pour deux automates $\mathcal A,\mathcal A'$, l'automate reconnaissant
la concaténation est une copie de chacun des deux automates, en ajoutant des
transitions entre les états acceptants de $\mathcal A$ et l'état initial de
$\mathcal A'$. De même pour l'étoile d'un langage, on ajoute un nouvel état
initial, acceptant, on le relie à l'ancien état initial et on crée une
transition entre chaque état acceptant et le nouvel état initial.

Cependant, ces transitions demandent à pouvoir être effectuées automatiquement,
en ayant la possibilité pour un état de faire (ou non) cette transition avant
la lecture d'une nouvelle lettre. Cela nous pousse à ajouter deux choses~: le
non déterminisme, pour pouvoir choisir ou non si l'on fait une transition, et
les $\varepsilon$-transitions, qui sont des transitions qui, comme décrites
précédemment, peuvent s'exécuter ou non et sans lecture de lettre.

Commençons par présenter un automate non déterministe~: celui-ci est un automate
dans lequel, au lieu d'une fonction de transition indiquant à partir d'un état
et d'une lettre quel est l'état lu ensuite, on dispose d'une relation de
transition indiquant pour deux états et une lettre s'il existe une transition
entre ces deux états étiquetée par cette lettre. On lui préfère en général une
présentation équivalente~: à un état et une lettre on associe l'ensemble des
états possibles à atteindre.

\begin{definition}[Automate fini non déterministe]
  Soit $\Sigma$ un alphabet. On appelle automate fini non déterministe
  $\mathcal A$ un quadruplet $(Q,Q_0,\delta,F)$ où~:
  \begin{itemize}
  \item $Q$ est un ensemble fini appelé ensemble des états de $\mathcal A$,
  \item $Q_0\subseteq Q$ est l'ensemble des états initiaux,
  \item $\delta : Q\times \Sigma \partialto \powerset(Q)$ est la fonction de
    transition non déterministe,
  \item $F\subseteq Q$ est l'ensemble des états terminaux.
  \end{itemize}
\end{definition}

On peut définir la fonction de transition étendue, dont le principe est de
considérer l'union à chaque fois qu'on prend un ensemble d'états possibles.

\begin{definition}[Fonction de transition étendue]
  Soit un automate fini non déterministe $\mathcal A$ sur un alphabet $\Sigma$,
  on note $\delta^\star : Q \times \Sigma^\star \to \powerset(Q)$ sa fonction de
  transition étendue, définie par induction par~:
  \begin{itemize}
  \item $\delta^\star(q,\varepsilon) = \{q\}$
  \item $\displaystyle\delta^\star(q,u\star a) =
    \bigcup_{q' \in \delta^\star(q,u)} \delta(q',a)$
  \end{itemize}
\end{definition}

\begin{definition}[Langage accepté]
  Soit $\Sigma$ un alphabet et $\mathcal A$ un automate fini non déterministe
  sur $\Sigma$. On dit que $\mathcal A$ accepte (ou reconnaît) le mot
  $u \in \Sigma^\star$ s'il existe $q_0 \in Q_0$ tel que
  $\delta^*(q_0,u) \cap F\neq\varnothing$, ce qu'on note $\mathcal A \models u$.
  On définit le langage reconnu par $\mathcal A$ par~:
  \[\mathcal L_\mathcal A\defeq \{u \in \Sigma^\star \mid \mathcal A \models u\}\]
\end{definition}

Un automate fini non déterministe se représente comme un automate fini
déterministe, à ceci près qu'on accepte d'avoir plusieurs flèches depuis un
état qui sont étiquetées par la même lettre, et qu'on peut avoir plusieurs
états initiaux. Accepter un mot dans un tel automate est alors simplement
trouver au moins un chemin partant d'un état initial et arrivant à un état
terminal en suivant des transitions étiquetées par les lettres du mot lu.

Cela mène à une définition alternative de l'acceptation d'un mot, qu'on traite
en exercice. La notion de trace est assez générale, et représente l'historique
de l'exécution d'une entrée sur une machine~: ici, la trace d'un automate est
alors la suite des états dans lesquels est la machine en lisant un mot.

\begin{exercise}\label{exo.traces.auto}
  Soit $\Sigma$ un alphabet et $\mathcal A$ un automate fini non déterministe
  sur $\Sigma$. On définit, pour tout $u \in \Sigma^\star$, l'ensemble $\tr(u)$
  des traces de $u$ sur $\mathcal A$, comme l'ensemble des suites finies
  $(q_0,\ldots,q_n)$ d'états de $\mathcal A$ dont le premier élément est un
  élément de $Q_0$ et tel que $\delta(q_i,u_i) = q_{i+1}$ pour tout
  $i \in \{0,\ldots,n-1\}$. Montrer que $\mathcal A \models u$ si et seulement
  s'il existe un élément $p \in \tr(u)$ dont le dernier état est un état
  terminal.
\end{exercise}

On pourrait s'attendre à ce que relâcher tant de contraintes augmente la
puissance des automates, mais ce n'est en réalité pas le cas. Il est clair qu'un
automate déterministe peut être transformé en un automate non déterministe, en
prenant $Q_0 \defeq \{q_0\}$ et pour fonction de transition non déterministe le
singleton de l'image par la fonction de transition déterministe. La réciproque,
elle, demande une construction plus subtile, appelée l'automate des parties.

\begin{definition}[Automate des parties]
  Soit un alphabet $\Sigma$ et un automate fini non déterministe $\mathcal A$
  sur $\Sigma$. On définit l'automate des parties de $\mathcal A$, noté
  $\powerset(\mathcal A)$, par~:
  \begin{itemize}
  \item $Q_{\powerset(\mathcal A)} \defeq \powerset(Q_\mathcal A)$
  \item $q_0 \defeq Q_0$
  \item $\delta_{\powerset(\mathcal A)} (R,a) \defeq
    \bigcup_{q \in R} \delta_{\mathcal A}(q,a)$
  \item $F_{\powerset(\mathcal A)} \defeq \{R \subseteq Q\mid
    R\cap F \neq \varnothing\}$
  \end{itemize}
\end{definition}

\begin{proposition}
  Soit un automate non déterministe $\mathcal A$ sur un alphabet $\Sigma$, on a
  l'égalité
  \[\mathcal L_{\mathcal A} = \mathcal L_{\powerset(\mathcal A)}\]
\end{proposition}

\begin{proof}
  On prouve par induction sur $u$ que pour tout $u \in \Sigma^\star$, on a
  l'égalité suivante pour tout $R \subseteq Q$~:
  \[\delta^\star_{\powerset(\mathcal A)}(R,u) = \delta^\star_{\mathcal A}(R,u)\]
  \begin{itemize}
  \item dans le cas où $u = \varepsilon$, le résultat est direct.
  \item supposons que l'hypothèse d'induction est valide pour $u$ et montrons
    le cas $u \star a$~:
    \begin{align*}
      \delta^\star_{\powerset(\mathcal A)}(R,u\star a) &=
      \bigcup_{q \in \delta^\star_{\powerset(\mathcal A)}(R,u)}
      \delta_{\mathcal A}(q,a)\\
      &= \bigcup_{q \in \delta^\star_\mathcal A(R,u)} \delta_\mathcal A(q,a)\\
      &= \delta_\mathcal A^\star(R,u\star a)
    \end{align*}
  \end{itemize}
  D'où le résultat par induction. On voit alors que la condition pour que
  $u$ soit accepté, dans les deux automates, revient à
  \[\delta_{\powerset(\mathcal A)}^\star (Q_0,u) \cap F \neq \varnothing\]
  donc $\mathcal L_{\powerset(\mathcal A)} = \mathcal L_\mathcal A$.
\end{proof}

\begin{remark}
  Cette construction n'est pas gratuite en pratique~: elle fait passer d'un
  ensemble d'états à son ensemble des parties, ce qui donne une complexité
  exponentielle à la construction.
\end{remark}

On peut donc utiliser des automates non déterministes pour étudier les langages
reconnaissables. On introduit maintenant les $\varepsilon$-transitions, qui
ont plus de sens dans le contexte des automates non déterministes.

\begin{definition}[Automate avec $\varepsilon$-transitions]
  On définit un automate fini non déterministe avec $\varepsilon$-transitions
  comme un automate fini non déterministe, mais dans lequel la fonction de
  transition est de la forme
  $\delta : Q\times (\Sigma \cup \{\varepsilon\}) \partialto Q$.
\end{definition}

Pour définir l'acceptation d'un mot par un automate avec
$\varepsilon$-transitions, on utilisera le formalisme des traces, introduit dans
l'\cref{exo.traces.auto}.

\begin{definition}[Langage reconnu par un automate avec
    $\varepsilon$-transitions]
  Soit $\Sigma$ un alphabet et $\mathcal A$ un automate non déterministe avec
  $\varepsilon$-transitions sur $\Sigma$. On définit l'ensemble $\tr(u)$ des
  traces sur un mot $u$ comme l'ensemble des suites finies $(q_0,\ldots,q_n)$
  d'états de $Q$ où $q_0 \in Q_0$ et telles qu'il existe pour tout
  $i \in \{0,\ldots,n-1\}$ une lettre $a_i$ telle que les deux conditions
  suivantes sont vérifiées~:
  \begin{itemize}
  \item $\delta(q_i,a_i) = q_{i+1}$
  \item $u = a_0\star\cdots\star a_{n-1}$
  \end{itemize}
  On dit que $\mathcal A$ accepte le mot $u$
  s'il existe une suite $(q_0,\ldots,q_n) \in \tr(u)$ telle que $q_n \in F$.

  On définit le langage reconnu par $\mathcal A$ comme l'ensemble des mots
  reconnus par $\mathcal A$.
\end{definition}

Les traces dans un automate avec $\varepsilon$-transitions représentent juste
les exécutions possibles lors de la lecture d'un mot, mais où il est possible
d'effectuer gratuitement des transitions étiquetées par $\varepsilon$.

Là encore, il est clair qu'un automate sans $\varepsilon$-transition peut être
simulé par un automate sans $\varepsilon$-transition, en n'ajoutant aucune
$\varepsilon$-transition.

On introduit donc maintenant l'élimination des $\varepsilon$-transitions,
permettant de construire un automate sans $\varepsilon$-transition mais
reconnaissant le même langage.

L'idée de l'élimination des $\varepsilon$-transitions est assez simple~: au lieu
de faire des sauts spontanés dans une $\varepsilon$-transition, on décide de
faire une grande transition contenant toutes les $\varepsilon$-transitions ainsi
que la transition qui suit. Il nous faut aussi modifier l'ensemble des états
terminaux pour ajouter les états qui peuvent atteindre un état terminal
uniquement en effectuant des $\varepsilon$-transitions.

\begin{definition}[\'Elimination des $\varepsilon$-transitions]
  Soit $\Sigma$ un alphabet et $\mathcal A$ un automate avec
  $\varepsilon$-transitions sur $\Sigma$. On définit la relation $\to^\star$ sur
  les états de $\mathcal A$ comme la plus petite relation  réflexive et
  transitive contenant la relation
  \[q\to q' \defeq q' \in \delta(q,\varepsilon)\]

  On définit l'automate $\mathcal A_{\not\varepsilon}$ non déterministe sans
  $\varepsilon$-transition par la construction suivante~:
  \begin{itemize}
  \item $Q_{\mathcal A_{\not\varepsilon}} \defeq Q_\mathcal A$
  \item $Q_{0,\mathcal A_{\not\varepsilon}} \defeq Q_{0,\mathcal A}$
  \item $\displaystyle\delta_{\mathcal A_{\not\varepsilon}}(q,a) \defeq
    \bigcup_{{q' \in Q}\atop{q\to^\star q'}} \delta_{\mathcal A}(q',a)$
  \item $F_{\mathcal A_{\not\varepsilon}} =\{q \in Q
    \mid \exists q' \in Q, q \to^\star q' \land q' \in F_\mathcal A\}$
  \end{itemize}
\end{definition}

\begin{proposition}
  Soit $\Sigma$ un alphabet et $\mathcal A$ un automate avec
  $\varepsilon$-transitions sur $\Sigma$. Alors
  \[\mathcal L_{\mathcal A} = \mathcal L_{\mathcal A_{\not\varepsilon}}\]
\end{proposition}

\begin{proof}
  On prouve d'abord que $q'\in \delta_{\mathcal A_{\not\varepsilon}}(q,a)$ si et
  seulement s'il existe une suite finie $(q_0,\ldots,q_n)$ telle que~:
  \begin{itemize}
  \item $q_0 = q$
  \item $\forall i \in \{0,\ldots,n-2\}, q_{i+1} \in
    \delta_{\mathcal A}(q_i,\varepsilon)$
  \item $q_n \in \delta_{\mathcal A}(q_{n-1},a)$
  \item $q_n = q'$
  \end{itemize}
  Pour un sens, on remarque que dans une telle suite, $q_1 \to^\star q_{n-1}$
  donc $\delta_{\mathcal A_{\not\varepsilon}}(q,a)\supseteq \delta_\mathcal A(q_{n-1},a)$.
  Pour l'autre sens, on sait par définition qu'il existe $q''$ tel que
  $q\to^\star q''$ et $q'\in\delta_\mathcal A (q'',a)$, donc il nous reste à
  vérifier que si $q\to^\star q''$ alors il existe une suite d'états uniquement
  reliés par des $\varepsilon$-transition, ce qui se fait par induction sur
  $\to^\star$ (on laisse la vérification au lecteur).

  Ainsi, on a une trace $(q_0,\ldots,q_n)$ dans $\mathcal A$ pour un mot $u$
  donné si et seulement si $(q'_0,\ldots,q'_p)$, obtenue en supprimant les
  états reliés par des $\varepsilon$-transitions, est une trace pour $u$ dans
  $\mathcal A_{\not\varepsilon}$ (réciproquement si on peut ajouter des états
  reliés par des $\varepsilon$-transitions).

  De plus, un état $q_n$ est terminal dans $\mathcal A_{\not\varepsilon}$ si et
  seulement s'il existe une suite d'$\varepsilon$-transitions menant à un état
  terminal~: ainsi une trace terminant sur un tel état dans $\mathcal A$ donne
  une trace acceptante en ajoutant des $\varepsilon$-transitions et
  réciproquement on récupère une trace acceptante dans
  $\mathcal A_{\not\varepsilon}$ en supprimant ces $\varepsilon$-transitions.

  Ainsi
  \[\mathcal L_{\mathcal A} = \mathcal L_{\mathcal A_{\not\varepsilon}}\]
\end{proof}

On sait donc maintenant que les automates déterministes, non déterministes et
non déterministes avec $\varepsilon$-transitions, sont tous équivalents du
point de vue de la classe de langage qu'ils reconnaissent.

On peut maintenant prouver la stabilité des langages reconnaissables par
concaténation de langages et par étoile.

\begin{proposition}\label{prop.concat.ratio}
  Soit $\Sigma$ un alphabet et $\mathcal A, \mathcal A'$ deux automates non
  déterministes avec $\varepsilon$-transitions sur $\Sigma$. Alors il existe
  un automate reconnaissant
  \[\mathcal L_\mathcal A \star \mathcal L_{\mathcal A'}\]
\end{proposition}

\begin{proof}
  On construit l'automate $\mathcal A''$ dont les états est la somme directe des
  états de $\mathcal A$ et $\mathcal A'$ (pour simplifier les notations on va
  considérer les deux ensembles d'états disjoints et considérer leur union), les
  transitions sont uniquement celles définies au sein de $\mathcal A$ et
  $\mathcal A'$ sauf entre chaque état $q_f$ de $\mathcal A$ et chaque état
  initial $q_i$ de $\mathcal A'$ où l'on rajoute une transition
  $q_f\xrightarrow{\varepsilon} q_i$. Les états initiaux sont ceux de
  $\mathcal A$ et les états terminaux ceux de $\mathcal A'$.

  Accepter un mot dans cet automate signifie donc avoir une trace
  $(q_0,\ldots,q_n)$ commençant en un état initial de $\mathcal A$ et terminant
  dans un état terminal de $\mathcal A'$. Comme la seule façon de passer d'un
  état de $\mathcal A$ a un état de $\mathcal A'$ est d'utiliser une des
  $\varepsilon$-transitions ajoutées, on sait que la trace peut se décomposer
  en $(q_0,\ldots,q_p)$ et $(q_{p+1},\ldots,q_n)$ et ces deux traces sont
  acceptantes respectivement pour un mot dans $\mathcal L_\mathcal A$ et pour un
  mot dans $\mathcal L_{\mathcal A'}$, donc la trace est acceptant pour un
  mot dans $\mathcal L_\mathcal A \star \mathcal L_{\mathcal A'}$.

  Réciproquement, si on a deux traces $(q_0,\ldots,q_p)$ et
  $(q_{p+1},\ldots,q_n)$ pour respectivement $u \in \mathcal L_{\mathcal A}$ et
  $v\in\mathcal L_{\mathcal A'}$, alors $(q_0,\ldots,q_n)$ est une trace
  acceptante pour $u\star v$.
\end{proof}

On prouve de même qu'ajouter simplement des transitions entre les états
terminaux et initiaux d'un automate donne le langage de l'étoile. Comme la
preuve est très proche de la preuve précédente, nous préférons la donner en
exercice.

\begin{definition}[Automate étoile]
  Soit $\Sigma$ un alphabet et $\mathcal A$ un automate sur $\Sigma$. On définit
  l'automate étoile $\mathcal A^\star$ par~:
  \begin{itemize}
  \item $Q_{\mathcal A^\star} \defeq Q_\mathcal A \sqcup \{p\}$
  \item $Q_{0,\mathcal A^\star} \defeq \{p\}$
  \item $\delta_{\mathcal A^\star}$ est définie comme $\delta_\mathcal A$ mais en
    ajoutant des $\varepsilon$-transitions entre $p$ et chaque
    $q_0\in Q_{0,\mathcal A}$, ainsi qu'entre chaque $q_f \in F_\mathcal A$ et
    $p$.
  \item $F_{\mathcal A^\star} \defeq \{p\}$.
  \end{itemize}
\end{definition}

\begin{exercise}\label{prop.etoile.ratio}
  Soit $\Sigma$ un alphabet et $\mathcal A$ un automate sur $\Sigma$, montrer
  que
  \[\mathcal L_{\mathcal A^\star} = \mathcal L_{\mathcal A}^\star\]
\end{exercise}

On déduit donc des constructions effectuées la stabilité par les opérations
rationnelles.

\begin{corollary}
  Soit $\Sigma$ un alphabet. La classe $\Reco(\Sigma)$ contient la classe
  $\Ratio(\Sigma)$.
\end{corollary}

\begin{proof}
  On a déjà dit que $\varnothing \in \Reco(\Sigma)$ et que
  $\{a\}\in \Reco(\Sigma)$ pour tout $a \in \Sigma$. De plus, en utilisant la
  \cref{prop.union.ratio}, la \cref{prop.concat.ratio} et
  l'\cref{prop.etoile.ratio}, on en déduit que $\Reco$ est stable par union,
  concaténation et étoile. On en déduit donc, puisque $\Ratio(\Sigma)$ est la
  plus petite telle classe, que $\Ratio(\Sigma)\subseteq\Reco(\Sigma)$.
\end{proof}

Pour montrer l'inclusion réciproque, on commence par s'intéresser aux propriétés
des automates. Certaines ne nous seront pas utiles directement, mais nous
pensons que faire un inventaire des propriétés usuelles des automates et des
constructions permettant d'obtenir certaines propriétés est important pour se
forger une intuition sur les automates.

\subsection{Propriétés d'automates}

Plusieurs propriétés d'intérêt existent, à propos des automates. Pour faciliter
la génération d'une expression rationnelle à pratir d'un automate, nous allons
présenter des propriétés basiques à leur propos. La première est celle
d'avoir une table de transition totale plutôt que partielle.

\begin{definition}[Automate complet]
  Un automate $\mathcal A$ est dit complet si la fonction de transition $\delta$
  est une fonction totale.
\end{definition}

Il est facile d'assurer cette propriété en ajoutant un état dit \og puits\fg
vers lequel pointent toutes les transitions à ajouter, et les transitions
partant de cet état.

\begin{proposition}
  Soit $\mathcal A$ un automate sur un alphabet $\Sigma$, il existe un automate
  $\mathcal A'$ qui est complet et reconnaît le même langage que $\mathcal A$.
\end{proposition}

\begin{proof}
  On construit l'automate $\mathcal A'$ comme décrit ci-dessus~:
  \begin{itemize}
  \item $Q' \defeq Q \sqcup \{p\}$
  \item $q'_0 \defeq q_0$
  \item $\delta' : Q' \times \Sigma \to Q'$ est défini par disjonction de cas~:
    \begin{itemize}
    \item si $(q,a) \in \dom(\delta)$, alors $\delta'(q,a) = \delta(q,a)$
    \item sinon, $\delta'(q,a) = p$
    \end{itemize}
  \item $F' = F$
  \end{itemize}

  On veut maintenant prouver que les deux automates reconnaissent le même
  langage. Soit $u \in \Sigma^\star$, montrons que $\delta^\star(q_0,u)$ est
  indéfini si et seulement si $\delta'^\star(q_0,u) = p$ et que sinon
  $\delta'^\star(q_0,u) = \delta^\star(q_0,u)$, par induction sur $u$~:
  \begin{itemize}
  \item dans le cas de $\varepsilon$, on sait que
    $\delta'^\star(q_0,\varepsilon) = \delta^\star(q_0,\varepsilon) = q_0$.
  \item supposons l'hypothèse vraie pour $u$, et soit $a \in \Sigma$. Supposons
    que $\delta^\star(q_0,u\star a)$ est indéfinie, on a alors deux cas
    possibles~:
    \begin{itemize}
    \item Si $\delta^\star(q_0,u)$ est indéfinie, alors par hypothèse
      d'induction $\delta'^\star(q_0,u) = p$ et comme $\delta'(p,a) = p$, on en
      déduit que $\delta'^\star(q_0,u\star a) = p$.
    \item Si $\delta^\star(q_0,u) = q$ et $\delta(q,a)$ est indéfini, alors
      par hypothèse d'induction $\delta'^\star(q_0,u) = q$ et, par définition,
      $\delta'(q,a) = p$.
    \end{itemize}
    
    Réciproquement, si $\delta'^\star(q_0,u\star a) = p$ alors c'est que la
    transition a atteint un état puits à un moment, et cela correspondait à
    une transition non définie dans $\delta$ (la preuve est parfaitement
    analogue à celle du sens direct).

    Dans l'autre cas, si $\delta'^\star(q_0,u\star a) = q$ où $q \in Q$, alors
    on sait par induction que $\delta'^\star(q_0,u) = \delta(q_0,u) = q'$ par
    hypothèse d'induction, et $\delta'(q,a) = \delta(q,a)$ par définition,
    donc $\delta'^\star(q_0,u\star a) = \delta^\star(q_0,u\star a)$.
  \end{itemize}

  On en déduit donc~:
  \[\delta^\star(q_0,u) \in F \iff \delta'^\star(q_0,u)\in F\]
  ce qui est le résultat voulu.
\end{proof}

Une autre propriété importante sur un automate est celle d'automate émondé.
Pour donner une idée de leur intérêt, commençons
