\chapter{Automates finis et langages réguliers}
\label{chp.auto}

\minitoc

\lettrine{P}{our} aborder la calculabilité, plusieurs approches existent. L'une
d'entre elles est de considérer directement les formalismes de fonctions
calculables. Cette approche est raisonnable puisqu'elle permet d'aller à
l'essentiel, mais cela donne une vision assez restrictive des modèles de calculs
plus faibles existants, et qui sont souvent très utiles.

C'est pourquoi nous faisons le choix dans cet ouvrage (où nous ne sommes limités
ni par le temps d'un cours ni par l'espace d'un livre imprimé) de traiter, avant
les formalismes de fonctions calculables, les formalismes liés à la hiérarchie
de Chomsky. Nous pensons en particulier que les automates permettent d'illuminer
profondément le fonctionnement intuitif d'une machine de Turing, plus complexe
mais basée sur des notions très proches.

Ce chapitre couvre donc les bases de l'étude des langages réguliers et des
automates finis. Cette classe de langage peut être décrite par de nombreux
formalismes, mais nous nous contenterons des automates finis, des expressions
régulières et des monoïdes.

Ce chapitre commence par la base de la théorie des langages~: les définitions
de ce qu'est un alphabet, un mot, un langage\ldots Nous aborderons ensuite la
classe des langages réguliers en abordant les trois formalismes décrits plus
haut, et la preuve de leur équivalence. Enfin, nous verrons les théorèmes de
structure sur ces langages~: le lemme de pompage et le théorème de
Myhill-Nérode, ainsi que ses conséquences.

\section{Alphabet, mot, langage}

Pour étudier de façon plus systématique les langages, nous étudierons aussi de
l'algèbre avec la structure de monoïde, qui est intimement liée à celle
d'ensemble de mots. Nous verrons donc tout d'abord l'aspect concret de ce qu'est
un mot et un langage, puis nous aborderons les côtés théoriques de l'étude des
monoïdes.

\subsection{Premières définitions}

Tout d'abord, il nous faut définit ce qu'est un alphabet. Intuitivement, cela
désigne l'ensemble des symboles que nous utilisons pour écrire. Pour avoir un
trairement uniforme, on considère que des symboles comme \og ( \fg font aussi
partie de l'alphabet. C'est alors un objet particulièrement quelconque~: on peut
imaginer écrire avec à peu près n'importe quel ensemble de symboles. D'où la
définition semblant inutile d'un alphabet~: parler d'alphabet permet surtout de
situer le contexte d'étude.

\begin{definition}[Alphabet]
  Un alphabet $\Sigma$ est un ensemble fini non vide.
\end{definition}

L'intérêt d'un alphabet est avant tout de pouvoir écrire des mots avec. La
notion de mot est la notion finie par excellence~: un mot est donc simplement
une suite finie de lettres.

\begin{definition}[Mot]
  Un mot $u$ sur un alphabet $\Sigma$ est une suite finie à valeurs dans
  $\Sigma$, c'est-à-dire
  $u = (u_0,\ldots,u_{n-1})\in \Sigma^n$ pour un certain entier $n\in\mathbb N$.

  On note $\Sigma^*$ l'ensemble des mots sur $\Sigma$~:
  \[\Sigma^* \defeq \bigcup_{n\in \mathbb N} \Sigma^n\]
\end{definition}

\begin{notation}
  Pour tout mot $u\in\Sigma^n$, on note $u_i$ pour la
  \ordinalnumeralfeminin{$i$} lettre, c'est-à-dire l'image de $u$ par la
  projection $\pi_i : \Sigma^n \to \Sigma$.

  Pour tout mot $u\in\Sigma^*$, on note $|u|\in\mathbb N$ l'entier $n$ tel que
  $u\in \Sigma^*$.
\end{notation}

Remarquons qu'une suite vide est une suite finie~: on peut donc définir le mot
vide, qu'on notera $\varepsilon$.

On peut désormais définir un langage~: c'est un ensemble de mots.

\begin{definition}[Langage]
  Soit $\Sigma$ un alphabet. Un langage $\mathcal L$ sur $\Sigma$ est une partie
  $\mathcal L\subseteq\Sigma^*$.
\end{definition}

Parlons dès maintenant de stabilité~:

\begin{definition}[Stabilité]
  Soit $\Sigma$ un alphabet et $\star$ une opération binaire sur $\Sigma^*$. On
  dit qu'un langage $\mathcal L$ sur $\Sigma$ est stable par $\star$ si
  \[\forall u,v\in \mathcal L, u\star v \in \mathcal L\]
\end{definition}

Cette notion de stabilité se généralise très bien à des classes de langages,
dont il convient donc d'introduire la définition.

\begin{definition}[Classe de langages]
  Soit un alphabet $\Sigma$. On appelle classe de langages une partie
  $\mathcal C\subseteq \powerset(\Sigma^*)$ (ses éléments sont donc des
  langages).
\end{definition}

\begin{definition}[Stabilité d'une classe de langages]
  Soit $\Sigma$ un alphabet et $\star$ une opération binaire sur
  $\powerset(\Sigma)^*$. On dit qu'une classe de langages $\mathcal C$ sur
  $\Sigma$ est stable par $\star$ si
  \[\forall \mathcal L,\mathcal M\in\mathcal C,
  \mathcal L\star \mathcal M\in\mathcal C\]
\end{definition}

Pour le reste de cette sous-section, nous allons juste voir des exemples de
langages, d'opérations et de classes de langages.

\begin{example}
  Nous allons dès maintenant définir l'alphabet qui nous sera le plus utile~:
  \[\btwo \defeq \{0,1\}\]

  Voici quelques langages sur $\btwo$ ou un alphabet $\Sigma$ quelconque~:
  \begin{itemize}
  \item le langage des mots commençant par $0$~:
    \[\mathcal L_0 \defeq \{u\in\Sigma^*\mid u_0 = 0\}\]
  \item le langage des mots ayant une taille paire~:
    \[\mathcal L_{\mathrm{pair}} \defeq \{u\in\Sigma^*\mid |u|\in2\mathbb N\}\]
  \item le langage des mots dont le début et la fin sont la même lettre~:
    \[\mathcal L_{\mathrm{début}=\mathrm{fin}}\defeq
    \{u\in\Sigma^*\mid u_0 = u_{|u| - 1}\}\]
  \item pour un langage $\mathcal L$, le langage $\overline{\mathcal L}$ des
    mots pris dans l'autre sens~:
    \[\overline{\mathcal L}\defeq
    \{u\in\Sigma^*\mid u_{|u|-1}\ldots u_0 \in \mathcal L\}\]
  \item l'intersection de deux langages.
  \item l'union de deux langages.
  \end{itemize}
\end{example}

Nous allons définir une nouvelle opération, fortement liée à la structure de
$\Sigma^*$, qui nous permettra de donner plus d'exemples de langages.

\begin{definition}[Concaténation]
  Soit un alphabet $\Sigma$. On définit l'opération de concaténation
  $\star : \Sigma^*\times\Sigma^*\to\Sigma^*$ comme le fait, étant donnés deux
  mots $u,v\in\Sigma^*$, de lire les deux mots à la suite. Formellement, on
  définit
  \[u\star v \defeq (u_0,\ldots,u_{|u|-1},v_0,\ldots,v_{|v|-1})\]

  Pour tout $n\in \mathbb N$, on notera $u^n$ pour l'itération de l'opération
  $\star$~:
  \[u^0 \defeq \varepsilon \qquad
  \forall n \in \mathbb N, u^{n+1} = u^n \star u\]
\end{definition}

\begin{remark}
  Le mot vide, $\varepsilon$, est neutre pour $\star$~: $u\star\varepsilon = u$.
\end{remark}

\begin{example}
  Voici plusieurs autres langages possibles sur un alphabet $\Sigma$~:
  \begin{itemize}
  \item pour deux langages $\mathcal L, \mathcal L'$, le langage concaténé des
    deux langages~:
    \[\mathcal L\star\mathcal L'\defeq
    \{u\star u'\mid u \in \mathcal L, u' \in \mathcal L' \}\]
  \item pour un langage $\mathcal L$, l'itération de concaténation sur ce
    langage~:
    \[\mathcal L^{\star 0} \defeq \{\varepsilon\}\qquad
    \forall n \in \mathbb N, \mathcal L^{\star(n+1)} \defeq
    \mathcal L^{\star n}\star\mathcal L\]
  \item pour un langage $\mathcal L$, le langage des mots dont toutes les
    concaténations sont encore dans $\mathcal L$~:
    \[\mathcal L_{\mathrm{stable}\star}\defeq
    \{u\in \Sigma^*\mid \forall n \in \mathbb N, u^n \in \mathcal L\}\]
  \item pour un langage $\mathcal L$, le langage des itérations de mots de
    $\mathcal L$~:
    \[\mathcal L^\star \defeq
    \{u_0\star\cdots\star u_{n-1}\mid
    n \in \mathbb N, \forall i\in \{0,\ldots,n -1\}, u_i \in \mathcal L\}\]
  \end{itemize}
\end{example}

\begin{exercise}
  Montrer que pour tout langage $\mathcal L$ sur un alphabet $\Sigma$, on a
  l'égalité suivante~:
  \[\mathcal L^\star = \bigcup_{n \in \mathbb N} \mathcal L^{\star n}\]
\end{exercise}

\begin{example}
  Donnons quelques exemples de classes de langages sur un alphabet $\Sigma$~:
  \begin{itemize}
  \item la classe des langages contenant $\varepsilon$~:
    \[\mathcal C_\varepsilon \defeq \{\mathcal L \subseteq \Sigma^*\mid
    \varepsilon \in \mathcal L\}\]
  \item la classe des langages stables par $\star$~:
    \[\mathcal C_\star \defeq \{\mathcal L \subseteq \Sigma^*\mid
    \forall u,v\in\mathcal L, u\star v \in \mathcal L\}\]
  \end{itemize}
\end{example}

\subsection{Aspect algébrique des langages}

Nous nous intéressons maintenant à l'aspect algébrique des langages. L'objectif
de cette sous-section est d'aboutir à la caractérisation de l'ensemble des mots
comme construction universelle (celle du monoïde libre). Pour cela, commençons
par voir ce qu'est un monoïde.

\begin{definition}[Monoïde]
  Un monoïde est un ensemble $M$ muni d'une opération binaire
  $\cdot : M \times M \to M$ et d'un élément neutre $e$,
  c'est-à-dire un élément $e\in M$ tel que
  \[\forall x\in M, x\cdot e = e \cdot x = x\]
  et tel que $\cdot$ est associative, c'est-à-dire
  \[\forall x,y,z\in M, x\cdot (y\cdot z) = (x\cdot y) \cdot z\]
\end{definition}

Définissons la notion de morphisme associée à la structure de monoïde.

\begin{definition}[Morphisme de monoïde]
  Soient $(M,\cdot,e)$ et $(N,\diamond,e')$ deux monoïdes. Un morphisme entre
  ces deux monoïdes est une fonction $f : M \to N$ vérifiant les deux conditions
  \begin{equation}
    \forall x,y\in M, f(x\cdot y) = f(x)\diamond f(y)
  \end{equation}
  \begin{equation}
    f(e) = e'
  \end{equation}
\end{definition}

Remarquons que contrairement au cas des groupes, les morphismes de monoïdes ont
besoin de stabiliser l'élément neutre.

L'exemple le plus évident de monoïde est $(\mathbb N,+,0)$, mais nous allons
voir que la structure des mots généralise de façon naturelle ce cas.

\begin{exercise}
  Soit $\Sigma$ un alphabet, montrer que $(\Sigma^*, \star,\varepsilon)$ est un
  monoïde.
\end{exercise}

Ce fait généralise le cas de $\mathbb N$ car on peut représenter un entier en
unaire comme un mot sur l'alphabet $\{*\}$~: la seule information d'un tel mot
est sa longueur.

L'étude des monoïdes permet donc d'inclure l'étude de la structure de
$\Sigma^*$. Nous allons maintenant voir qu'en fait, $\Sigma^*$ peut être
construit comme la structure de monoïde la plus simple engendrée par $\Sigma$.
Cela se traduit par la propriété universelle du monoïde libre.

\begin{theorem}[Propriété universelle du monoïde libre]
  Soit $\Sigma$ un alphabet, $(M,\diamond,e)$ un monoïde et $f : \Sigma \to M$
  une fonction (quelconque). Alors il existe un unique morphisme de monoïde
  $\tilde f$ entre $\Sigma^*$ et $M$ tel que $\tilde f (x) = f(x)$ pour tout
  $x \in \Sigma$.
\end{theorem}

\begin{proof}
  On procède par analyse-synthèse~:
  \begin{itemize}
  \item supposons que $\tilde f$ existe et vérifie les conditions décrites.
    Alors pour tout mot $u = u_0\ldots u_{n-1}$, on sait par récurrence sur
    $n$ que $\tilde f(u) = f(u_0)\diamond \cdots \diamond f(u_{n-1})$. La
    fonction ainsi définie est unique, étant donnée la construction.
  \item en considérant la fonction proposée précédemment, on voit d'abord que
    $\tilde f(x) = f(x)$ pour $x\in \Sigma$, et que si $u = v\star w$ alors
    $\tilde f(u) = \tilde f(v)\diamond \tilde f(w)$, où l'associativité est
    nécessaire pour pouvoir couper le produit au milieu sans changer le
    résultat.
  \end{itemize}
  On voit donc qu'un seul candidat est possible pour la fonction donnée, et que
  ce candidat vérifie la proposition attendue.
\end{proof}

Remarquons qu'en choisissant une fonction $f : \Sigma \to M$ sans propriété
particulière, on peut potentiellement atteindre avec $\tilde f$ des valeurs qui
ne sont pas atteintes par $f$ directement. Par exemple si on prend la fonction
\[\begin{array}{ccccc}
f & : & \Sigma & \longrightarrow & \mathbb N\\
& & \alpha & \longmapsto & 1
\end{array}\]
alors l'image d'un mot de taille $n$ est $n$. De cette façon, on vient de voir
que l'ensemble des mots de taille $n$ peut être décrit à partir du monoïde
$(\mathbb N,+,1)$ en considérant une partie du monoïde et une fonction
$f : \Sigma \to \mathbb N$.

\begin{definition}
  Soit $\Sigma$ un alphabet, $(M,\cdot,e)$ un monoïde, $h : \Sigma \to M$ une
  fonction et $S\subseteq M$. On dit que $\mathcal L(\Sigma,M,\cdot,e,h)$ est
  le langage induit par $(\Sigma,M,\cdot,e,h,S)$, qui est défini par~:
  \[\mathcal L(\Sigma,M,\cdot,e,h,S)\defeq
  \{u\in\Sigma^*\mid \tilde h(u)\in S\}\]
\end{definition}

En reprenant l'exemple de $\alpha \mapsto 1$, on voit qu'en prenant $S = \{n\}$,
on induit le langage $\Sigma^n$.

\section{Automate fini et langage reconnaissable}

On peut désormais s'intéresser plus en profondeur aux automates finis et aux
langages reconnaissables.
