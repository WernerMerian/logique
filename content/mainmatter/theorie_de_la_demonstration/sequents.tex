\chapter[Calcul des séquents]{Théorème de complétude méta et
  calcul des séquents}\label{chp.sequents}

\minitoc

\lettrine{D}{ans} le \cref{chp.logpred}, nous avons prouvé le théorème de
complétude pour la déduction naturelle dans le cas de la logique du premier
ordre. La déduction naturelle, comme nous l'avons vu, est une syntaxe de
démonstration permettant de formaliser des étapes élémentaires de raisonnements
sous forme d'un arbre de preuve.

La structure de la preuve du théorème de complétude, du moins de sa partie
technique où l'on construit un modèle d'une théorie cohérente, se décompose
ainsi~:
\begin{itemize}
\item on ajoute des témoins de Henkin à la théorie~;
\item on complète la théorie~;
\item on construit le modèle syntaxique, dont les éléments sont les termes clos
  quotientés par l'égalité prouvable dans la théorie
\item on montre par induction que la théorie complétée prouve une formule si et
  seulement si le modèle syntaxique la satisfait.
\end{itemize}

Ce théorème est donc relativement long et technique à prouver, et notre preuve
ne fonctionne que dans le cas très précis de la déduction naturelle. Est-il
possible de généraliser ce résultat à d'autres syntaxes de preuves ? C'est en
répondant à cette question que nous introduisons une autre syntaxe, celle du
calcul des séquents, que nous étudierons ce chapitre.

Ce chapitre se compose donc d'abord d'une preuve d'un théorème de complétude
\og méta\fg, qui permet de prouver avec quelques conditions qu'une large classe
de syntaxes de preuves sont complètes vis à vis des modèles déjà présentés dans
le \cref{chp.logpred}. Grâce à cette preuve, nous abordons ensuite le calcul des
séquents classique, et son principal théorème~: l'élimination des coupures.

Enfin, nous présentons le calcul des séquents intuitionnistes. Pour cela, nous
préférons prendre un chemin légèrement plus long mais plus riche~: nous étudions
sommairement la logique modale et la sémantique de Kripke qui leur est associé.
Cela nous permet de conclure par un théorème de complétude similaire à celui du
calcul des séquents, mais entre sa version intuitionniste et la classe des
modèles de Kripke.

\section{Théorème de complétude généralisé}

Dans notre première démonstration du théorème de complétude, nous construisons
le modèle syntaxique d'une théorie cohérente en ayant un système syntaxique
déjà donné, ce qui n'est pas notre cas ici. Nous avons donc besoin de considérer
avant tout des conditions qu'un système syntaxique doit vérifier pour nous
sembler acceptable.

Nous nous autorisons à en ajouter autant que nécessaire, mais nous souhaitons
garder un critère essentiel~: chaque condition doit être raisonnable, au sens
où un système nous paraissant refléter ce qu'est une preuve doit forcément
vérifier une telle condition.

Tout d'abord, des conditions dites structurelles s'imposent. On considère que
notre relation $\vdash$ de prouvabilité syntaxique relie une théorie et une
formule, on a donc, pour une signature $\Sigma$ donnée,
$\vdash \subseteq \powerset(\Prop(\Sigma))\times \Formula(\Sigma)$. Dire qu'une
théorie $\mathcal T$ prouve une formule $\varphi$ demande qu'on écrive
effectivement une preuve, ce qui signifie qu'on doit avoir un ensemble, qu'on
notera $\Pi$, de preuves. Cet ensemble est muni d'une fonction donnant les
hypothèses d'une preuves et sa conclusion, c'est-à-dire une fonction
$\hypo : \Pi \to \powerfin(\Formula(\Sigma))$ et d'une fonction
$\concl : \Pi \to \Formula(\Sigma)$,
de telle sorte qu'on puisse écrire
\[\forall \mathcal T, \varphi, \mathcal T \vdash \varphi \iff
\exists \pi \in \Pi, (\hypo(\pi) \subseteq \mathcal T) \land
(\concl(\pi) = \varphi)\]
Remarquons ici qu'on choisit de poser $\hypo$ comme ayant ses valeurs dans un
ensemble fini de formules. Le fait de choisir un ensemble fini est poussé par
le fait qu'une preuve est toujours considérée comme un objet finitaire~: écrire
une démonstration repose sur un nombre fini de symboles. Le fait de considérer
des formules et non seulement des propositions permet de considérer des preuves
qui utilisent des formules non closes comme hypothèses.

Nous voyons donc plusieurs conditions apparaître grâce à cette structure.
Tout d'abord, cela force la propriété d'affaiblissement~:
\[
(\textit{Aff})\quad : \quad \forall \mathcal T,\mathcal T', \varphi,
(\mathcal T \vdash \varphi) \land (\mathcal T \subseteq \mathcal T') \implies
\mathcal T' \vdash \varphi
\]
ainsi que la propriété de compacité~:
\[
(\textit{Comp})\quad : \quad \forall \mathcal T, \varphi,
\mathcal T \vdash \varphi \implies \exists F \subfin \mathcal T,
F\vdash \varphi
\]

Au niveau structurel, il est aussi attendu d'une démonstration qu'elle permette
de faire deux choses considérées évidentes~: une démonstration peut utiliser une
hypothèse pour en faire sa conclusion, et une démonstration peut utiliser un
lemme. Ces deux principes de raisonnements élémentaires se traduisent par deux
conditions, l'axiome~:
\[
(\textit{Ax})\quad : \quad \forall \mathcal T, \varphi, \varphi \in \mathcal T
\implies \mathcal T \vdash \varphi
\]
et la coupure~:
\[
(\textit{Cut})\quad : \quad \forall \mathcal T, \varphi, \psi,
(\mathcal T \vdash \varphi) \land (\mathcal T \cup \{\varphi\}\vdash \psi)
\implies \mathcal T \vdash \psi
\]

On ajoute aussi une condition liée au fait qu'on travaille en logique
classique~: on admet le tiers exclu, ou plutôt ici une version équivalente
qu'est le raisonnement par l'absurde. Pour faciliter l'écriture de cette
condition, on définit deux nouvelles notations~: on notera $\mathcal T \vdash$
pour dire que $\mathcal T$ peut prouver une formule $\varphi$ et son contraire
$\lnot\varphi$, et on notera $\mathcal T\nvdash$ pour dire que ce n'est pas le
cas (ce qui correspond au fait que $\mathcal T$ est cohérente). On peut alors
formuler le raisonnement par l'absurde ainsi~:
\[
(\textit{Raa})\quad : \quad \forall \mathcal T, \varphi,
(\mathcal T\cup\{\lnot \varphi\}\vdash) \implies \mathcal T \vdash \varphi
\]

Avec ces conditions, toutes largement raisonnables pour considérer une notion de
prouvabilité entre une théorie et une formule, on peut déjà remarquer deux
éléments~:
\begin{itemize}
\item avec $(\textit{Raa})$, prouver que $\vDash\subseteq \vdash$ ne demande que
  de prouver que pour toute théorie $\mathcal T$, si $\mathcal T\nvdash$ alors
  il existe un modèle de $\mathcal T$. Cela revient au
  \cref{lem.vDashvdashabs} que nous avons vu avant.
\item sans reconstruire l'algèbre de Lindenbaum-Tarski pour la
  relation syntaxique, le fait d'avoir $(\textit{Aff})$ et $(\textit{Comp})$
  nous permet d'utiliser le lemme de Zorn pour étendre toute théorie en une
  théorie cohérente maximale pour $\vdash$, vu comme un pré-ordre grâce aux
  conditions $(\textit{Ax})$ et $(\textit{Cut})$ qui assurent respectivement
  la réflexivité et la transitivité de la relation.
\end{itemize}

Grâce à ces quatre conditions structurelles, on cherche maintenant des
conditions pour assurer que la construction effectuée dans le \cref{chp.logpred}
fonctionne encore. On souhaite donc construire le modèle syntaxique. Pour cela,
on a besoin de quotienter l'ensemble des termes clos par la relation
\[t \sim u \defeq \mathcal T \vdash t = u\]
Mais il faut donc vérifier que cette relation est à la fois une relation
d'équivalence, et une congruence pour les symboles de fonction et de relation.

On a déjà vu que deux règles suffisaient pour ça, qui sont la réflexivité~:
\[(\textit{r}=)\quad : \quad \forall \mathcal T, t, \mathcal T \vdash t = t\]
et la loi de Leibniz~:
\[(\textit{l}=)\quad : \quad \forall \mathcal T, t, u, \varphi,
(\mathcal T \vdash \varphi[u/x]) \implies
\mathcal T \cup\{ t = u \} \vdash \varphi[t/x]
\]

On peut donc construire notre modèle syntaxique. L'étape suivante est de prouver
par induction sur les formules que, en notant notre modèle syntaxique
$\mathcal M$ et notre théorie maximale $\mathcal T$, pour tout environnement
d'interprétation des variables du premier ordre $\rho$ et toute formule
$\varphi$ dont les variables libres sont dans le domaine de $\rho$, on a
\[\mathcal M, \rho \models \varphi \iff \mathcal T \vdash \rho(\varphi)\]
Sans s'attarder sur les substitutions de variables et en passant les cas
atomiques (le cas de $\top$ est automatique, le cas de $\bot$ est dû au fait
que la théorie $\mathcal T$ est cohérente, et les autres formules atomiques sont
données par définition de l'interprétation des relations), l'induction nous
donne~:
\[\begin{array}{ccccc}
\mathcal M \models \lnot \varphi 
& \overset{\textit{ind. hyp.}}{\iff} &
\mathcal T \nvdash \varphi & \overset{\text{?}}{\iff} &
\mathcal T \vdash \lnot\varphi \\
\mathcal M \models \varphi\land \psi
& \overset{\textit{ind. hyp.}}{\iff} &
(\mathcal T \vdash \varphi) \land (\mathcal T \vdash \psi)&
\overset{\text{?}}{\iff} &
\mathcal T \vdash \varphi \land \psi \\
\mathcal M \models \varphi\lor \psi
& \overset{\textit{ind. hyp.}}{\iff} &
(\mathcal T \vdash \varphi) \lor (\mathcal T \vdash \psi)&
\overset{\text{?}}{\iff} &
\mathcal T \vdash \varphi \lor \psi \\
\mathcal M \models \varphi\to \psi
& \overset{\textit{ind. hyp.}}{\iff} &
(\mathcal T \nvdash \varphi) \lor (\mathcal T \vdash \psi)&
\overset{\text{?}}{\iff} &
\mathcal T \vdash \varphi \lor \psi \\
\mathcal M \models \forall x, \varphi
& \overset{\textit{ind. hyp.}}{\iff} &
\forall m \in |\mathcal M |, \mathcal T \vdash \varphi[t/x]
& \overset{\text{?}}{\iff} &
\mathcal T \vdash \forall x, \varphi \\
\mathcal M \models \exists x, \varphi
& \overset{\textit{ind. hyp.}}{\iff} &
\exists m \in |\mathcal M |, \mathcal T \vdash \varphi[t/x]
& \overset{\text{?}}{\iff} &
\mathcal T \vdash \exists x, \varphi \\
\end{array}\]

On a ainsi une liste de nouvelles conditions à satisfaire pour assurer que
l'induction fonctionne. En séparant les équivalences en deux implications, et
en prenant la contraposée pour l'une des deux implications, on
obtient (en notant $\mathcal T, \varphi$ pour $\mathcal T \cup \{\varphi\}$)~:
\[\begin{array}{ccccc}
(l\lnot) & : & \mathcal T \vdash \varphi & \implies &
\mathcal T, \lnot \varphi \vdash \\
(r\lnot) & : & \mathcal T, \varphi \vdash & \implies &
\mathcal T \vdash \lnot \varphi \\
(l\land) & : & (\mathcal T, \varphi \vdash) \lor
(\mathcal T, \psi \vdash)
& \implies & \mathcal T, \varphi \land \psi\vdash \\
(r\land) & : & (\mathcal T \vdash \varphi) \land (\mathcal T \vdash \psi)
& \implies & \mathcal T \vdash \varphi \land \psi \\
(l\lor) & : & (\mathcal T, \varphi \vdash) \land (\mathcal T, \psi \vdash)
& \implies & \mathcal T, \varphi \lor \psi \vdash \\
(r\lor) & : & (\mathcal T \vdash \varphi) \lor (\mathcal T \vdash \psi)
& \implies & \mathcal T \vdash \varphi \lor \psi \\
(l\to) & : & (\mathcal T \vdash \varphi) \land (\mathcal T, \psi \vdash)
& \implies & \mathcal T, \varphi \to \psi \vdash \\
(r\to) & : & (\mathcal T,\varphi \vdash) \lor (\mathcal T \vdash \psi)
& \implies & \mathcal T \vdash \varphi \to \psi \\
(l\forall) & : & (\exists t \in |\mathcal M |, \mathcal T, \varphi[t/x] \vdash)
& \implies & \mathcal T, \forall x, \varphi \vdash \\
(r\forall) & : & (\forall t \in |\mathcal M |, \mathcal T \vdash \varphi [t/x])
& \overset{\text{?}}{\implies} & \mathcal T \vdash \forall x, \varphi \\
(l\exists) & : & (\forall t \in |\mathcal M |, \mathcal T, \varphi[t/x] \vdash)
& \overset{\text{?}}{\implies} & \mathcal T, \exists x, \varphi \vdash \\
(r\exists) & : & (\exists t \in |\mathcal M |, \mathcal T \vdash \varphi[t/x])
& \implies & \mathcal T \vdash \exists x, \varphi
\end{array}\]

Toutes ces conditions, à l'exceptions de deux marquées par un
$\overset{\text{?}}{\implies}$, sont des conditions parfaitement raisonnables à
attendre d'un système syntaxique de démonstration. Les deux autres conditions,
en revanche, n'ont pas de raison d'être vérifiées~: une implication telle que
$\forall t \in |\mathcal T|, \mathcal T \vdash \varphi[t/x] \implies \mathcal T
\vdash \forall x, \varphi$
signifie en particulier que tout objet du premier ordre est représenté par un
terme clos. On peut facilement se convaincre que ça n'est pas le cas en prenant
le langage et la théorie des groupes~: pour un certain groupe fixé, disons le
groupe $\mathbb Z$, seul $0$ peut s'écrire comme un terme clos.

On modifie donc les deux conditions $(r\forall)$ et $(l\exists)$ en demandant,
non pas que la prémisse soit vérifiée en quantifiant sur les termes clos,
mais qu'elle soit vérifiée en prenant une constante $c$ n'apparaissant pas dans
$\varphi$ ni dans $\mathcal T$~:
\[\begin{array}{ccccc}
(r\forall)^\dagger & : & \forall c \notin \mathcal T, \varphi,
\mathcal T \vdash \varphi[c/x]
& \implies & \mathcal T \vdash \forall x, \varphi \\
(l\exists)^\dagger & : & \forall c \notin \mathcal T, \varphi,
\mathcal T, \varphi[c/x]\vdash
& \implies & \mathcal T, \exists x, \varphi \vdash
\end{array}\]
Ces deux conditions sont, elles, raisonnables, puisqu'elles correspondent à une
généralisation sur une constante indépendante du contexte. Cependant, elles ne
suffisent pas à assurer l'implication initiale. Nous utilisons donc la
construction des témoins de Henkin. Dans le cas présent, comme le système
syntaxique n'est pas défini et qu'on ne connait que quelques unes de ses
propriétés, on préfère ajouter des témoins de façon plus parcimonieuse.

En effet, on considère maintenant les formules $\varphi$ à une variable libre
telle que $\exists x, \varphi$ appartient à $\mathcal T$ (ou est prouvable dans
$\mathcal T$, comme $\mathcal T$ est choisie maximale), on ajoute seulement
une constante $c_{\exists x, \varphi}$ dans ce cas, et la théorie $\mathcal T'$
enrichie est alors augmentée seulement de l'axiome
$\varphi[c_{\exists x, \varphi}/x]$.

On peut alors itérer cette construction pour, à partir de la théorie
$\mathcal T$, construire une théorie $\hclose{\mathcal T}$ qui possède la
propriété des témoins de Henkin, c'est-à-dire telle que pour toute formule
$\varphi$ à une variable libre telle que $\exists x, \varphi$ est prouvable
dans $\hclose{\mathcal T}$, il existe une constante $c$ telle que
$\varphi[c/x]$ est prouvable dans $\hclose{\mathcal T}$.

Il se pose alors un nouveau problème. Lorsque nous construisons la version à
la fois complète et possédant les témoins de Henkin de notre théorie initiale,
dans le \cref{chp.logpred}, nous utilisons le \cref{prop.henkin.ext} pour,
depuis le complété de la clôture par témoins de Henkin, avoir une théorie
possédant les deux propriétés. Dans le cas présent, cette propriété n'a pas de
raison d'être vérifiée~: augmenter par maximalité notre théorie
$\hclose{\mathcal T}$ dans le nouveau langage nous fait perdre la propriété des
témoins de Henkin, et ajouter des témoins de Henkin nous fait perdre la
maximalité de la théorie.

On décide donc de créer une suite de théorie et de langages (on ne précisera pas
les langages) $\mathcal T_n$ où $\mathcal T_{2n+1}$ est l'extension par témoins
de Henkin de $\mathcal T_{2n}$, et $\mathcal T_{2n+2}$ est l'extension par
maximalité de $\mathcal T_{2n+1}$. En prenant alors l'union de toutes ces
théories, qu'on notera $\mathcal T'$ ici, on sait que~:
\begin{itemize}
\item si $\exists x, \varphi$ appartient à $\mathcal T'$, alors elle appartient
  à une certaine théorie $\mathcal T_n$, et on trouve donc une constante
  $c_{\varphi}$ telle que $\varphi[c_{\varphi}/x] \in \mathcal T_{n+2}$, donc
  $\mathcal T'$ possède la propriété des témoins de Henkin~;
\item soit $\varphi$ une formule close, exprimée dans le langage de
  $\mathcal T'$. Comme $\varphi$ est un objet fini, ses symboles appartiennent
  tous à un certain langage $\Sigma_n$ (en notant $\Sigma_n$ la suite de
  langages lors de la construction des $\mathcal T_n$). On sait alors que,
  $\mathcal T_{2n+2}$ contenant l'extension maximale d'une théorie sur
  $\Sigma_n$, soit $\varphi$ soit $\lnot\varphi$ y appartient. L'extension est
  donc maximale.
\end{itemize}

Avec tous ces éléments, il est maintenant possible de généraliser l'énoncé du
\cref{thm.completude}.

\begin{theorem}[Théorème de complétude généralisé]\label{thm.completude.gen}
  Soit une relation $\vdash$ vérifiant les règles suivantes~:
  \[\begin{array}{ccc}
  \AxiomC{$\varphi \in \mathcal T$}
  \RightLabel{(\textit{Ax})}
  \UnaryInfC{$\mathcal T \vdash \varphi$}
  \DisplayProof
  & \qquad &
  \AxiomC{$\mathcal T \vdash \varphi$}
  \AxiomC{$\mathcal T, \varphi \vdash \psi$}
  \RightLabel{(\textit{Cut})}
  \BinaryInfC{$\mathcal T \vdash \psi$}
  \DisplayProof
  \\
  \\
  \AxiomC{$\mathcal T \vdash \varphi$}
  \AxiomC{$\mathcal T \subseteq \mathcal T'$}
  \RightLabel{(\textit{Aff})}
  \BinaryInfC{$\mathcal T' \vdash \varphi$}
  \DisplayProof
  & &
  \AxiomC{$\mathcal T \vdash \varphi$}
  \RightLabel{(\textit{Comp})}
  \UnaryInfC{$\exists F \subfin \mathcal T, F\vdash \varphi$}
  \DisplayProof
  \\
  \\
  \AxiomC{$\mathcal T\vdash \varphi$}
  \RightLabel{$(l\lnot)$}
  \UnaryInfC{$\mathcal T, \lnot\varphi\vdash$}
  \DisplayProof
  & &
  \AxiomC{$\mathcal T, \varphi \vdash$}
  \RightLabel{$(r\lnot)$}
  \UnaryInfC{$\mathcal T\vdash \lnot \varphi$}
  \DisplayProof
  \\
  \\
  \AxiomC{$\mathcal T, \varphi \vdash$}
  \AxiomC{$\mathcal T, \psi \vdash$}
  \RightLabel{$(l\lor)$}
  \BinaryInfC{$\mathcal T, \varphi\lor\psi\vdash$}
  \DisplayProof
  & &
  \AxiomC{$\mathcal T \vdash \varphi$}
  \RightLabel{$(r\lor_1)$}
  \UnaryInfC{$\mathcal T\vdash \varphi\lor\psi$}
  \DisplayProof
  \quad
  \AxiomC{$\mathcal T \vdash \psi$}
  \RightLabel{$(r\lor_2)$}
  \UnaryInfC{$\mathcal T\vdash \varphi \lor \psi$}
  \DisplayProof
  \\
  \\
  \AxiomC{$\mathcal T, \varphi \vdash$}
  \RightLabel{$(l\land_1)$}
  \UnaryInfC{$\mathcal T,\varphi\land\psi\vdash$}
  \DisplayProof
  \quad
  \AxiomC{$\mathcal T, \psi \vdash$}
  \RightLabel{$(l\land_2)$}
  \UnaryInfC{$\mathcal T,\varphi \land \psi\vdash$}
  \DisplayProof
  & &
  \AxiomC{$\mathcal T\vdash \varphi$}
  \AxiomC{$\mathcal T \vdash \psi$}
  \RightLabel{$(r\land)$}
  \BinaryInfC{$\mathcal T\vdash \varphi\land\psi$}
  \DisplayProof
  \\
  \\
  \AxiomC{$\mathcal T \vdash \varphi$}
  \AxiomC{$\mathcal T, \psi \vdash$}
  \RightLabel{$(l\to)$}
  \BinaryInfC{$\mathcal T, \varphi\to\psi\vdash$}
  \DisplayProof
  & &
  \AxiomC{$\mathcal T, \varphi \vdash$}
  \RightLabel{$(r\to_1)$}
  \UnaryInfC{$\mathcal T\vdash \varphi\to\psi$}
  \DisplayProof
  \quad
  \AxiomC{$\mathcal T \vdash \psi$}
  \RightLabel{$(r\to_2)$}
  \UnaryInfC{$\mathcal T\vdash \varphi \to \psi$}
  \DisplayProof
  \\
  \\
  \AxiomC{$\mathcal T, \varphi[t/x] \vdash$}
  \RightLabel{$(l\forall)$}
  \UnaryInfC{$\mathcal T, \forall x, \varphi \vdash$}
  \DisplayProof
  & &
  \AxiomC{$c \notin \mathcal T, \varphi$}
  \AxiomC{$\mathcal T \vdash \varphi[c/x]$}
  \RightLabel{$(r\forall)^\dagger$}
  \BinaryInfC{$\mathcal T \vdash \forall x, \varphi$}
  \DisplayProof
  \\
  \\
  \AxiomC{$c \notin \mathcal T, \varphi$}
  \AxiomC{$\mathcal T, \varphi[c/x] \vdash$}
  \RightLabel{$(l\exists)^\dagger$}
  \BinaryInfC{$\mathcal T, \exists x, \varphi \vdash$}
  \DisplayProof
  & &
  \AxiomC{$\mathcal T \vdash \varphi[t/x]$}
  \RightLabel{$(r\exists)$}
  \UnaryInfC{$\mathcal T \vdash \exists x, \varphi$}
  \DisplayProof
  \\
  \\
  \AxiomC{$\mathcal T \vdash \varphi[u / x]$}
  \RightLabel{$(l=)$}
  \UnaryInfC{$\mathcal T, t = u \vdash \varphi[t / x]$}
  \DisplayProof
  & &
  \AxiomC{}
  \RightLabel{$(r=)$}
  \UnaryInfC{$\mathcal T \vdash t = t$}
  \DisplayProof
  \end{array}\]
  \[
  \AxiomC{$\mathcal T, \lnot\varphi \vdash$}
  \RightLabel{(\textit{Raa})}
  \UnaryInfC{$\mathcal T \vdash \varphi$}
  \DisplayProof\]
  Alors on a l'inclusion $\vDash \subseteq \vdash$.
\end{theorem}

Il nous est donc possible, désormais, de prouver des théorèmes de complétude
bien plus facilement, en montrant simplement la validité d'un ensemble de
conditions. On remarque en fait que toutes ces conditions peuvent déjà donner
lieu à un premier système syntaxique. A la place de considérer une théorie
$\mathcal T$, on considère des séquents de la forme $\Gamma \vdash [\varphi]$
où $[\varphi]$ signifie que la conclusion est soit vide (dans le cas d'une
contradiction), soit la formule $\varphi$, et la relation $\vdash$ donnée par
les conditions modulo cette réécriture est déjà un système syntaxique de
démonstration complet vis à vis de $\vDash$. Il est possible de prouver qu'il
est correct, par un argument similaire au \cref{thm.correction}.

Cependant, plusieurs éléments méritent d'être relevés~:
\begin{itemize}
\item en supprimant simplement la règle $(Raa)$ du système, on obtient un
  système syntaxique intuitionniste, puisque l'affaiblissement permet à partir
  de $\mathcal T \vdash$, de déduire $\mathcal T \vdash \varphi$ pour n'importe
  quelle formule $\varphi$~;
\item il est donc possible d'avoir une théorie intuitionniste, mais le théorème
  de complétude ne s'y applique alors plus, car ce qui nous permet de passer de
  la construction d'un modèle à l'inclusion $\vDash\subseteq\vdash$ est le fait
  que, si $\mathcal T \cup\{\lnot\varphi\}$ n'a pas de modèle, alors
  $\mathcal T, \lnot\varphi\vdash$, donc que
  $\mathcal T \vdash \lnot\lnot\varphi$ sans la règle de raisonnement par
  l'absurde~;
\item une symétrie peut s'observer entre les règles droites et les règles
  gauches~: les règles $\lnot$ permettent de changer de côté du $\vdash$ en
  ajoutant un $\lnot$, les règles $\lor$ et $\land$ sont exactement identiques
  modulo une inversion des positions, et la même chose peut être dite pour les
  règles $\exists$ et $\forall$. Ces couples $\lor/\land$ et $\exists/\forall$
  sont ce qu'on appelle des duaux de De Morgan, c'est-à-dire qu'on a par exemple
  $\varphi \lor \psi \equiv \lnot (\lnot \varphi \land \lnot \psi)$~;
\item contrairement au \cref{chp.logpred}, les règles $(r\forall)^\dagger$ et
  $(l\exists)^\dagger$ utilisent des symboles de constantes, ce qui permet en
  fait avant tout de faciliter la preuve que nous avons donnée du théorème de
  complétude~: cela nous évite d'ajouter le théorème de simulation d'une
  constante par une variable, mais ce théorème est généralement vérifier pour
  un système syntaxique raisonnable, et on prendra donc généralement plutôt
  cette version~;
\item à l'exception des règles sur l'égalité, qui ont un statut particulier du
  fait que l'égalité n'est pas un symbole logique aussi simple à étudier que les
  autres, et de la règle $(Raa)$, toutes les autres règles ont la propriété dite
  de la sous-formule~: les séquents en haut d'une règle sont constitués de
  sous-formules du séquent en bas de la règle. C'est en fait un critère central
  pour considérer qu'un formalisme est un calcul des séquents, comme nous le
  verrons.
\end{itemize}

Nous avons donc deux options, que nous allons explorer au fil de ce chapitres~:
créer des séquents symétriques, ou travailler dans un système intuitionniste.

\section{Calcul des séquents LK}

La recherche d'un théorème de complétude plus général nous a en fait poussé vers
un nouveau formalisme~: en considérant les conditions du
\cref{thm.completude.gen}, on peut en fait remplacer les séquents de la forme
$\mathcal T \vdash \varphi$ par des séquents de la forme $\Gamma \vdash \varphi$
où $\Gamma$ est un ensemble fini de formules, pour obtenir un système
syntaxique (finitaire). Compte tenu des nombreuses symétries entre le côté
gauche et le côté droit du $\vdash$ dans les règles, il est alors naturel de
vouloir construire des séquents de la forme $\Gamma\vdash \Delta$, où
$\Delta$ est aussi un ensemble fini de formules. S'il est assez clair que
$\Gamma$, dans le séquent $\Gamma\vdash \Delta$, se lit comme une conjonction
(on lit le séquent \og en supposant vraies toutes les formules de $\Gamma$\fg,
c'est-à-dire qu'on suppose la conjonction de toutes les formules), le dual
naturel est alors une disjonction sur $\Delta$~: ainsi le séquent
$\Gamma\vdash \Delta$ s'interprète comme l'énoncé \og en supposant vraies toutes
les formules de $\Gamma$, une au moins des formules de $\Delta$ est vraie\fg.
Naturellement, le cas où $\Gamma$ est vide donne alors une conclusion toujours
valide, et le cas où $\Delta$ est vide donne alors une contradiction des
hypothèses, les deux pouvant donc se réécrire respectivement comme
$\top \vdash \Delta$ et $\Gamma\vdash \bot$.

On peut maintenant définir notre premier calcul des séquents, $\LK$.

\begin{definition}[Calcul des séquents $\LK$]
  Soit $\Sigma$ une signature du premier ordre. On définit inductivement
  la relation
  $\vdLK\subseteq \powerfin(\Formula(\Sigma))\times\powerfin(\Formula(\Sigma))$
  par les règles~:
  \[\begin{array}{cc}
  \AxiomC{}
  \RightLabel{\textit{ax}}
  \UnaryInfC{$\Gamma, \varphi\vdLK \varphi$}
  \DisplayProof
  &
  \AxiomC{$\Gamma \vdLK \varphi,\Delta$}
  \AxiomC{$\Theta, \varphi \vdLK \Xi$}
  \RightLabel{\textit{cut}}
  \BinaryInfC{$\Gamma, \Theta\vdLK \Delta, \Xi$}
  \DisplayProof
  \\
  \\
  \AxiomC{$\Gamma \vdLK \Delta$}
  \RightLabel{\textit{lw}}
  \UnaryInfC{$\Gamma, \varphi \vdLK \Delta$}
  \DisplayProof
  &
  \AxiomC{$\Gamma \vdLK \Delta$}
  \RightLabel{\textit{rw}}
  \UnaryInfC{$\Gamma\vdLK \Delta, \varphi$}
  \DisplayProof
  \\
  \\
  \AxiomC{$\Gamma\vdLK \Delta, \varphi$}
  \RightLabel{$l\lnot$}
  \UnaryInfC{$\Gamma, \lnot\varphi \vdLK \Delta$}
  \DisplayProof
  &
  \AxiomC{$\Gamma, \varphi\vdLK \Delta$}
  \RightLabel{$r\lnot$}
  \UnaryInfC{$\Gamma\vdLK \Delta, \lnot\varphi$}
  \DisplayProof
  \\
  \\
  \AxiomC{$\Gamma, \varphi \vdLK \Delta$}
  \AxiomC{$\Theta, \psi \vdLK \Xi$}
  \RightLabel{$l\lor$}
  \BinaryInfC{$\Gamma, \Theta, \varphi\lor\psi \vdLK \Delta, \Xi$}
  \DisplayProof
  &
  \AxiomC{$\Gamma \vdLK \Delta,\varphi$}
  \RightLabel{$r\lor_1$}
  \UnaryInfC{$\Gamma\vdLK \Delta,\varphi\lor\psi$}
  \DisplayProof
  \quad
  \AxiomC{$\Gamma\vdLK \Delta,\psi$}
  \RightLabel{$r\lor_2$}
  \UnaryInfC{$\Gamma\vdLK \Delta, \varphi \lor \psi$}
  \DisplayProof
  \\
  \\
  \AxiomC{$\Gamma, \varphi \vdLK \Delta$}
  \RightLabel{$l\land_1$}
  \UnaryInfC{$\Gamma,\varphi\land\psi\vdLK \Delta$}
  \DisplayProof
  \quad
  \AxiomC{$\Gamma, \psi \vdLK \Delta$}
  \RightLabel{$l\land_2$}
  \UnaryInfC{$\Gamma,\varphi \land \psi\vdLK \Delta$}
  \DisplayProof
  &
  \AxiomC{$\Gamma\vdLK \Delta,\varphi$}
  \AxiomC{$\Theta\vdLK \Xi,\psi$}
  \RightLabel{$r\land$}
  \BinaryInfC{$\Gamma, \Theta\vdLK \Delta, \Xi, \varphi\land\psi$}
  \DisplayProof
  \\
  \\
  \AxiomC{$\Gamma \vdLK \Delta,\varphi$}
  \AxiomC{$\Theta, \psi \vdLK \Xi$}
  \RightLabel{$l\to$}
  \BinaryInfC{$\Gamma, \Theta,\varphi\to\psi\vdLK \Delta, \Xi$}
  \DisplayProof
  &
  \AxiomC{$\Gamma, \varphi \vdLK\Delta$}
  \RightLabel{$r\to_1$}
  \UnaryInfC{$\Gamma\vdLK \Delta,\varphi\to\psi$}
  \DisplayProof
  \quad
  \AxiomC{$\Gamma \vdLK \Delta,\psi$}
  \RightLabel{$r\to_2$}
  \UnaryInfC{$\Gamma\vdLK\Delta, \varphi \to \psi$}
  \DisplayProof
  \\
  \\
  \AxiomC{$\Gamma, \varphi[t/x] \vdLK\Delta$}
  \RightLabel{$l\forall$}
  \UnaryInfC{$\Gamma, \forall x, \varphi \vdLK\Delta$}
  \DisplayProof
  &
  \AxiomC{$\Gamma\vdLK \Delta,\varphi$}
  \RightLabel{$r\forall^\dagger$}
  \UnaryInfC{$\Gamma \vdLK \Delta,\forall x, \varphi$}
  \DisplayProof
  \\
  \\
  \AxiomC{$\Gamma, \varphi \vdLK\Delta$}
  \RightLabel{$l\exists^\dagger$}
  \UnaryInfC{$\Gamma, \exists x, \varphi \vdLK\Delta$}
  \DisplayProof
  &
  \AxiomC{$\Gamma \vdLK \Delta,\varphi[t/x]$}
  \RightLabel{$r\exists$}
  \UnaryInfC{$\Gamma \vdLK \Delta,\exists x, \varphi$}
  \DisplayProof
  \\
  \\
  \AxiomC{$\Gamma[u / x] \vdLK \Delta[u / x]$}
  \RightLabel{$l=$}
  \UnaryInfC{$\Gamma[t / x], t = u \vdLK \Delta[t / x]$}
  \DisplayProof
  &
  \AxiomC{}
  \RightLabel{$r=$}
  \UnaryInfC{$\vdLK t = t$}
  \DisplayProof
  \end{array}\]

  où les règles $l\exists^\dagger$ et $r\forall^\dagger$ demandent que
  $x$ soit libre dans $\Gamma$ et $\Delta$.
\end{definition}
