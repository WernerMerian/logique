\chapter{Déduction naturelle}
\label{chp.deduc.nat}

\minitoc

\lettrine{D}{ans} son article présentant le calcul des séquents
\cite{Gentzen1935}, Gentzen présente aussi un deuxième formalisme de
démonstration~: la déduction naturelle. C'est ce système qui a été présenté dans
le \cref{chp.logpred}, mais nous allons ici étudier ses aspects structurels.

Comme le relavait déjà Gentzen dans son article original, la déduction naturelle
comporte deux variantes que sont $\NK$ et $\NJ$, respectivement la déduction
naturelle classique et intuitionniste, mais la première est en fait un
formalisme, en un sens, maladroit.

Le chapitre présente donc les deux variantes, mais étudie principalement la
variante intuitionniste. La variante classique peut en fait être construire de
façon plus naturelle, mais ces considérations dépassent le cadre d'une
introduction à la théorie de la démonstration.

Nous donnons ensuite une version adaptée à la déduction naturelle de
l'élimination des coupures, dans laquelle on introduit les notions essentielles
de système de réécriture abstrait. La preuve de la forte normalisation des
arbres de preuve de $\NJ$ est, elle, reportée au CHAPITRE LAMBDA CALCUL TYPE.

\section{Variantes de la déduction naturelle}

\subsection{Déduction naturelle intuitionniste et classique}

On a déjà vu une première présentation de la déduction naturelle dans le
\cref{chp.logpred}. Dans cette présentation, les points importants sont les
suivants~:
\begin{itemize}
\item les séquents des arbres de preuve ressemblent à des séquents
  intuitionnistes, mais contiennent exactement une formule à droite (aucune
  formule à droite dans un séquent intuitionniste pouvant être simulé par la
  formule $\bot$)~;
\item les règles ne sont plus des variantes gauche/droite mais des règles dites
  d'introduction et d'élimination, la première permettant d'introduire un
  constructeur à droite et la seconde permettant d'utiliser une formule du
  constructeur.
\end{itemize}

La première présentation de Gentzen du calcul en est proche, mais n'utilise pas
de séquent. Dans le calcul des séquents, seules existent des formules dans des
arbres, et les hypothèses d'une preuve peuvent être récupérées en considérant
les feuilles de l'arbre. Le second point, lui, reste identique dans cette
présentation.

\begin{table}[t!]
  \centering
  \begin{tabular}{cc}
    \bottomAlignProof
    \AxiomC{$\varphi$}
    \DisplayProof
    &
    \bottomAlignProof
    \AxiomC{$\bot$}
    \RightLabel{$\bot_\mathrm e$}
    \UnaryInfC{$\varphi$}
    \DisplayProof
    \\
    \\
    \bottomAlignProof
    \AxiomC{$[\varphi]^\alpha$}
    \noLine
    \UnaryInfC{$\vdots$}
    \noLine
    \UnaryInfC{$\bot$}
    \RightLabel{$\lnot_\mathrm i,\alpha$}
    \UnaryInfC{$\lnot\varphi$}
    \DisplayProof
    &
    \bottomAlignProof
    \AxiomC{$\lnot \varphi$}
    \AxiomC{$\varphi$}
    \RightLabel{$\lnot_\mathrm e$}
    \BinaryInfC{$\bot$}
    \DisplayProof
    \\
    \\
    \bottomAlignProof
    \AxiomC{$\varphi$}
    \RightLabel{$\lor_\mathrm i^1$}
    \UnaryInfC{$\varphi\lor\psi$}
    \DisplayProof
    \quad
    \bottomAlignProof
    \AxiomC{$\psi$}
    \RightLabel{$\lor_\mathrm i^2$}
    \UnaryInfC{$\varphi\lor\psi$}
    \DisplayProof
    &
    \bottomAlignProof
    \AxiomC{$\varphi\lor\psi$}
    \AxiomC{$[\varphi]^\alpha$}
    \noLine
    \UnaryInfC{$\vdots$}
    \noLine
    \UnaryInfC{$\chi$}
    \AxiomC{$[\psi]^\alpha$}
    \noLine
    \UnaryInfC{$\vdots$}
    \noLine
    \UnaryInfC{$\chi$}
    \RightLabel{$\lor_\mathrm e,\alpha$}
    \TrinaryInfC{$\chi$}
    \DisplayProof
    \\
    \\
    \bottomAlignProof
    \AxiomC{$\varphi$}
    \AxiomC{$\psi$}
    \RightLabel{$\land_\mathrm i$}
    \BinaryInfC{$\varphi\land\psi$}
    \DisplayProof
    &
    \bottomAlignProof
    \AxiomC{$\varphi\land\psi$}
    \RightLabel{$\land_\mathrm e^1$}
    \UnaryInfC{$\varphi$}
    \DisplayProof
    \quad
    \bottomAlignProof
    \AxiomC{$\varphi\land\psi$}
    \RightLabel{$\land_\mathrm e^2$}
    \UnaryInfC{$\psi$}
    \DisplayProof
    \\
    \\
    \bottomAlignProof
    \AxiomC{$[\varphi]^\alpha$}
    \noLine
    \UnaryInfC{$\vdots$}
    \noLine
    \UnaryInfC{$\psi$}
    \RightLabel{$\to_\mathrm i,\alpha$}
    \UnaryInfC{$\varphi\to\psi$}
    \DisplayProof
    &
    \bottomAlignProof
    \AxiomC{$\varphi\to\psi$}
    \AxiomC{$\psi$}
    \RightLabel{$\to_\mathrm e$}
    \BinaryInfC{$\psi$}
    \DisplayProof
    \\
    \\
    \bottomAlignProof
    \AxiomC{$\varphi$}
    \RightLabel{$\forall_\mathrm i^\dagger$}
    \UnaryInfC{$\forall x, \varphi$}
    \DisplayProof
    &
    \bottomAlignProof
    \AxiomC{$\forall x, \varphi$}
    \RightLabel{$\forall_\mathrm e$}
    \UnaryInfC{$\varphi[t/x]$}
    \DisplayProof
    \\
    \\
    \bottomAlignProof
    \AxiomC{$\varphi[t/x]$}
    \RightLabel{$\exists_\mathrm i$}
    \UnaryInfC{$\exists x, \varphi$}
    \DisplayProof
    &
    \bottomAlignProof
    \AxiomC{$\exists x, \varphi$}
    \AxiomC{$[\varphi]^\alpha$}
    \noLine
    \UnaryInfC{$\vdots$}
    \noLine
    \UnaryInfC{$\psi$}
    \RightLabel{$\exists_\mathrm e^\dagger,\alpha$}
    \BinaryInfC{$\psi$}
    \DisplayProof
    \\
    \\
    \bottomAlignProof
    \AxiomC{}
    \RightLabel{$=_\mathrm i$}
    \UnaryInfC{$t=t$}
    \DisplayProof
    &
    \bottomAlignProof
    \AxiomC{$t = u$}
    \AxiomC{$\varphi[t/x]$}
    \RightLabel{$=_\mathrm e$}
    \BinaryInfC{$\varphi[u/x]$}
    \DisplayProof
  \end{tabular}

  \vspace{0.2cm}
  \hrule
  
  \vspace{0.4cm}
  \caption{Règles de $\NJ$}
  \label{tbl.NJ}
  \hrule
\end{table}

On présente d'abord la variante intuitionniste, car $\NK$ en est un système
dérivé dans lequel on ajoute une règle.

\begin{definition}[Déduction naturelle $\NJ$]
  On définit l'ensemble $\ProofNJ$ des arbres de preuves de $\NJ$ par les
  règles de formation données dans la \cref{tbl.NJ}. Dans un arbre de preuves,
  on va appeler feuille les formules tout en haut, qui n'ont pas de règle
  au-dessus d'elles (en particulier, $=_\mathrm i$ ne donne pas de feuille).
  Les feuilles de l'arbres sont donc de deux sortes~:
  \begin{itemize}
  \item les formules marquées, de la forme $[\varphi]^\alpha$ où $\alpha$ est
    appelé une marque~;
  \item les formules non marquées.
  \end{itemize}
  Un arbre $\pi\in \ProofNJ$ a pour conclusion le séquent $\Gamma\vdNJ \varphi$
  si la formule à sa racine est $\varphi$ et si ses feuilles non marquées sont
  des formules de $\Gamma$.

  Les règles avec la condition $\dagger$ demandent que la variable libre $x$
  n'apparaisse pas dans les feuilles non marquées de l'arbre, ou les feuilles
  avec une marque liée à une règle plus bas dans l'arbre que l'instance liant
  $x$.

  On considère uniquement les arbres dont toutes les marques sont liées par
  des règles, c'est-à-dire que pour une formule marquée $[\varphi]^\alpha$
  dans un arbre $\pi$, il existe une règle $\mathrm r_\alpha$ dans $\pi$.

  La relation de prouvabilité dans une théorie $\mathcal T$ est donnée par
  \[\mathcal T \vdNJ \varphi \defeq \exists \Gamma \in \List(\mathcal T),
  \exists \pi \in \ProofNJ, \pi\concl \Gamma\vdNJ\varphi\]
\end{definition}

\begin{remark}
  Dans les règles introduisant les marques telles que $\to_\mathrm i,\alpha$,
  il n'existe pas forcément de formule marquée par $\alpha$, et il peut y en
  avoir plusieurs marquées par le même $\alpha$. L'intérêt de ces marques est
  de supprimer du contexte global des hypothèses en ne les comptant plus comme
  des feuilles d'hypothèses (c'est pour cela qu'on ne considère que les feuilles
  non marquées).
\end{remark}

La déduction naturelle $\NK$ est simplement l'ajout d'une règle de logique
classique. Il existe plusieurs axiomes ou règles qui permettent de passer de la
logique intuitionniste à la logique classique~: l'\cref{exo.EDN.RAA} consiste
par exemple à montrer qu'il est équivalent de considérer l'ajout des axiomes
$\varphi\lor\lnot\varphi$ ou l'ajout des axiomes $\lnot\lnot\varphi\to\varphi$,
les deux pouvant se réécrire comme des règles. On choisit d'ajouter la règle du
raisonnement par l'absurde, plus naturelle à considérer pour formaliser des
raisonnements mathématiques.

\begin{definition}[Déduction naturelle $\NK$]
  L'ensemble $\ProofNK$ des arbres de preuve de déduction naturelle classique
  est donné par les règles de la \cref{tbl.NJ} ainsi que par la règle
  \begin{prooftree}
    \AxiomC{$[\lnot\varphi]$}\noLine\UnaryInfC{$\vdots$}\noLine
    \UnaryInfC{$\bot$}
    \RightLabel{Raa}
    \UnaryInfC{$\varphi$}
  \end{prooftree}
  La relation $\vdNK$ est définie de façon analogue à $\vdNJ$.
\end{definition}

\begin{remark}
  \'Etant donné un arbre $\pi\concl \psi_1,\cdots,\psi_n\vdNJ \varphi$ et
  un arbre
  $\pi'\concl \Delta\vdNJ\psi_i$, on peut placer l'arbre $\pi'$ au-dessus des
  feuilles de $\pi$ étiquetées par $\psi_i$ pour obtenir un arbre
  \[\pi \circ_i \pi' \concl \psi_1,\ldots,\psi_{i-1},\Delta,
  \psi_{i+1},\ldots,\psi_n\vdNJ\varphi\]
  Le lecteur ayant vu la notion d'opérade notera des similarités avec ces
  objets.
\end{remark}

\begin{example}
  Donnons un exemple d'arbre de preuve dans $\NJ$. On reprend l'exemple donné
  dans le \cref{chp.logpred}, pour mettre en valeur les différences de style
  avec la première version de déduction naturelle donnée.
  On fixe la signature
  $\Sigma \defeq \{f^1,R^1\}$ où $R$ est un symbole de relation et $f$ un
  symbole de fonction. On veut prouver que
  $\forall x, R(x)\to R(f(x))\vdNJ \forall x, R(x)\to R(f(f(x)))$~:
  \begin{prooftree}
    \AxiomC{$\forall x, R(x)\to R(f(x))$}
    \RightLabel{$\forall_\mathrm e$}
    \UnaryInfC{$R(f(x))\to R(f(f(x)))$}
    \AxiomC{$\forall x, R(x)\to R(f(x))$}
    \RightLabel{$\forall_\mathrm e$}
    \UnaryInfC{$R(x) \to R(f(x))$}
    \AxiomC{$[R(x)]^\alpha$}
    \RightLabel{$\to_\mathrm e$}
    \BinaryInfC{$R(f(x))$}
    \RightLabel{$\to_\mathrm e$}
    \BinaryInfC{$R(f(f(x)))$}
    \RightLabel{$\to_\mathrm i,\alpha$}
    \UnaryInfC{$R(x) \to R(f(f(x)))$}
    \RightLabel{$\forall_\mathrm i^\dagger$}
    \UnaryInfC{$\forall x, R(x)\to R(f(f(x)))$}
  \end{prooftree}
\end{example}

Puisque nous avons donné une nouvelle syntaxe de preuve, à la fois pour la
logique classique et la logique intuitionniste, il convient de montrer la
complétude de ces calculs.

\subsection{Complétude de la déduction naturelle}

Dans le cas de la logique classique, on peut simplement appliquer le théorème de
complétude.

\begin{exercise}
  Montrer que $\NK$ vérifie les hypothèse du \cref{thm.completude.gen}. En
  déduire que $\vdNK = \vDash$.
\end{exercise}

Pour la logique intuitionniste, on peut tenter de construire un théorème de
complétude directement inspiré de celui utilisé pour prouver la complétude de
$\LJ$ pour la sémantique de Kripke. Si cette méthode fonctionne, on propose de
donner une preuve plus algorithmique de l'équivalence entre $\LJ$ et $\NJ$.

\begin{definition}[Construction d'arbre depuis le calcul des séquents]
  On définit une fonction $\tradNJ - : \ProofLJ \to \ProofNJ$ par induction sur
  les arbres, de telle sorte que si $\pi\concl \Gamma\vdLJ \varphi$ alors
  $\tradNJ\pi\concl\Gamma\vdNJ\varphi$ et si $\pi\concl\Gamma\vdLJ$ alors
  $\tradNJ\pi\concl\Gamma\vdNJ\bot$~:
  \begin{itemize}
  \item si $\pi = \bottomAlignProof\AxiomC{}\RightLabel{$ax$}
    \UnaryInfC{$\varphi\vdLJ\varphi$}\DisplayProof$
    alors $\tradNJ\pi \defeq \varphi$.
  \item si $\pi = \bottomAlignProof
    \AxiomC{$\pi_0$}\noLine\UnaryInfC{$\Gamma\vdLJ\varphi$}
    \AxiomC{$\pi_1$}\noLine\UnaryInfC{$\Delta,\varphi\vdLJ[\psi]$}
    \RightLabel{$cut$}\BinaryInfC{$\Gamma,\Delta\vdLJ[\psi]$}
    \DisplayProof$
    alors $\tradNJ\pi \defeq \tradNJ{\pi_1}\circ_i\tradNJ{\pi_0}$ où
    $i$ est la position de $\varphi$ dans $\Delta,\varphi$.
  \item pour les règles structurelles sauf $rw$, la traduction ne change pas
    l'arbre.
  \item si $\pi = \bottomAlignProof
    \AxiomC{$\pi_0$}\noLine\UnaryInfC{$\Gamma\vdLJ$}
    \RightLabel{$rw$}\UnaryInfC{$\Gamma\vdLJ\varphi$}\DisplayProof$
    alors \[\tradNJ\pi \defeq
    \bottomAlignProof
    \def\fCenter{\mbox{\Large$\vdots$}}
    \Axiom$\fCenter\tradNJ{\pi_0}$
    \noLine\UnaryInfC{$\bot$}
    \RightLabel{$\bot_\mathrm e$}
    \UnaryInfC{$\varphi$}
    \DisplayProof\]
  \item si $\pi = \bottomAlignProof
    \AxiomC{$\pi_0$}\noLine
    \UnaryInfC{$\Gamma\vdLJ\varphi$}
    \RightLabel{$l\lnot$}\UnaryInfC{$\Gamma,\lnot\varphi\vdLJ$}\DisplayProof$
    alors
    \[\tradNJ{\pi} \defeq \bottomAlignProof
    \AxiomC{$\lnot\varphi$}
    \Axiom$\fCenter\tradNJ{\pi_0}$
    \noLine\UnaryInfC{$\varphi$}
    \RightLabel{$\lnot_\mathrm e$}\BinaryInfC{$\bot$}
    \DisplayProof\]
  \item si $\pi = \bottomAlignProof
    \AxiomC{$\pi_0$}\noLine\UnaryInfC{$\Gamma,\varphi\vdLJ$}
    \RightLabel{$r\lnot$}\UnaryInfC{$\Gamma\vdLJ\lnot\varphi$}
    \DisplayProof$, alors
    \[\tradNJ\pi \defeq \bottomAlignProof
    \def\fCenter{\mbox{\Large$\vdots$}}
    \AxiomC{$[\varphi]^\alpha$}\noLine
    \UnaryInf$\fCenter\tradNJ{\pi_0}$
    \noLine
    \UnaryInfC{$\bot$}
    \RightLabel{$\lnot_\mathrm i,\alpha$}\UnaryInfC{$\lnot\varphi$}
    \DisplayProof\]
  \item si $\pi = \bottomAlignProof
    \AxiomC{$\pi_0$}\noLine\UnaryInfC{$\Gamma,\varphi\vdLJ[\chi]$}
    \AxiomC{$\pi_1$}\noLine\UnaryInfC{$\Gamma,\psi\vdLJ[\chi]$}
    \RightLabel{$l\lor$}\BinaryInfC{$\Gamma,\varphi\lor\psi\vdLJ[\chi]$}
    \DisplayProof$, alors
    \[\tradNJ\pi\defeq \bottomAlignProof
    \def\fCenter{\mbox{\Large$\vdots$}}
    \AxiomC{$\varphi\lor\psi$}
    \AxiomC{$[\varphi]^\alpha$}\noLine
    \UnaryInf$\fCenter\tradNJ{\pi_0}$
    \noLine
    \UnaryInfC{$[\chi]$}
    \AxiomC{$[\psi]^\alpha$}\noLine
    \UnaryInf$\fCenter\tradNJ{\pi_1}$
    \noLine
    \UnaryInfC{$[\chi]$}
    \RightLabel{$\lor_\mathrm e,\alpha$}\TrinaryInfC{$[\chi]$}
    \DisplayProof\]
  \item si $\pi = \bottomAlignProof
    \AxiomC{$\pi'$}\noLine\UnaryInfC{$\Gamma\vdLJ\varphi_i$}
    \RightLabel{$r\lor_i$}\UnaryInfC{$\Gamma\vdLJ\varphi_1\lor\varphi_2$}
    \DisplayProof$, alors
    \[\tradNJ\pi\defeq\bottomAlignProof
    \def\fCenter{\mbox{\Large$\vdots$}}
    \Axiom$\fCenter\tradNJ{\pi'}$
    \noLine
    \UnaryInfC{$\varphi_i$}
    \RightLabel{$\lor_\mathrm i^i$}\UnaryInfC{$\varphi_1\lor\varphi_2$}
    \DisplayProof\]
  \item si $\pi = \bottomAlignProof
    \AxiomC{$\pi'$}\noLine\UnaryInfC{$\Gamma,\varphi_i\vdLJ [\psi]$}
    \RightLabel{$l\land_i$}
    \UnaryInfC{$\Gamma,\varphi_1\land\varphi_2\vdLJ[\psi]$}\DisplayProof$,
    alors
    \[\tradNJ\pi\defeq\bottomAlignProof
    \def\fCenter{\mbox{\Large$\vdots$}}
    \AxiomC{$\varphi_1\land\varphi_2$}\RightLabel{$\land_\mathrm e^i$}
    \UnaryInfC{$\varphi_i$}\noLine
    \UnaryInf$\fCenter\tradNJ{\pi'}$
    \noLine
    \UnaryInfC{$[\psi]$}\DisplayProof\]
  \item si $\pi = \bottomAlignProof
    \AxiomC{$\pi_0$}\noLine\UnaryInfC{$\Gamma\vdLJ\varphi$}
    \AxiomC{$\pi_1$}\noLine\UnaryInfC{$\Gamma\vdLJ\psi$}
    \RightLabel{$r\land$}\BinaryInfC{$\Gamma\vdLJ\varphi\land\psi$}
    \DisplayProof$, alors
    \[\tradNJ\pi\defeq \bottomAlignProof
    \def\fCenter{\mbox{\Large$\vdots$}}
    \Axiom$\fCenter\tradNJ{\pi_0}$
    \noLine\UnaryInfC{$\varphi$}
    \Axiom$\fCenter\tradNJ{\pi_1}$
    \noLine\UnaryInfC{$\psi$}
    \RightLabel{$\land_\mathrm i$}\BinaryInfC{$\varphi\land\psi$}\DisplayProof\]
  \item si $\pi = \bottomAlignProof
    \AxiomC{$\pi_0$}\noLine\UnaryInfC{$\Gamma\vdLJ\varphi$}
    \AxiomC{$\pi_1$}\noLine\UnaryInfC{$\Gamma,\psi\vdLJ[\chi]$}
    \RightLabel{$l\to$}\BinaryInfC{$\Gamma,\varphi\to\psi\vdLJ[\chi]$}
    \DisplayProof$, alors
    \[\tradNJ\pi\defeq\bottomAlignProof
    \def\fCenter{\mbox{\Large$\vdots$}}
    \AxiomC{$\varphi\to\psi$}
    \Axiom$\fCenter\tradNJ{\pi_0}$
    \noLine
    \UnaryInfC{$\varphi$}
    \RightLabel{$\to_\mathrm e$}
    \BinaryInfC{$\psi$}
    \noLine
    \UnaryInf$\fCenter\tradNJ{\pi_1}$
    \noLine\UnaryInfC{$[\chi]$}
    \DisplayProof\]
  \item si $\pi = \bottomAlignProof
    \AxiomC{$\pi'$}\noLine\UnaryInfC{$\Gamma,\varphi\vdLJ\psi$}
    \RightLabel{$r\to$}\UnaryInfC{$\Gamma\vdLJ\varphi\to\psi$}
    \DisplayProof$, alors
    \[\tradNJ\pi\defeq\bottomAlignProof
    \def\fCenter{\mbox{\Large$\vdots$}}
    \AxiomC{$[\varphi]^\alpha$}\noLine
    \UnaryInf$\fCenter\tradNJ{\pi'}$
    \noLine
    \UnaryInfC{$\psi$}
    \RightLabel{$\to_\mathrm i,\alpha$}\UnaryInfC{$\varphi\to\psi$}
    \DisplayProof\]
  \item si $\pi = \bottomAlignProof
    \AxiomC{$\pi'$}\noLine\UnaryInfC{$\Gamma,\varphi[t/x]\vdLJ[\psi]$}
    \RightLabel{$l\forall$}\UnaryInfC{$\Gamma,\forall x,\varphi\vdLJ[\psi]$}
    \DisplayProof$, alors
    \[\tradNJ\pi\defeq\bottomAlignProof
    \def\fCenter{\mbox{\Large$\vdots$}}
    \AxiomC{$\forall x, \varphi$}\RightLabel{$\forall_\mathrm e$}
    \UnaryInfC{$\varphi[t/x]$}\noLine
    \UnaryInf$\fCenter\tradNJ{\pi'}$
    \noLine
    \UnaryInfC{$[\psi]$}
    \DisplayProof\]
  \item si $\pi = \bottomAlignProof
    \AxiomC{$\pi'$}\noLine\UnaryInfC{$\Gamma\vdLJ\varphi$}
    \RightLabel{$r\forall^\dagger$}\UnaryInfC{$\Gamma\vdLJ\forall x, \varphi$}
    \DisplayProof$, alors
    \[\tradNJ\pi\defeq\bottomAlignProof
    \def\fCenter{\mbox{\Large$\vdots$}}
    \Axiom$\fCenter\tradNJ{\pi'}$
    \noLine
    \UnaryInfC{$\varphi$}
    \RightLabel{$\forall_\mathrm i^\dagger$}\UnaryInfC{$\forall x, \varphi$}
    \DisplayProof\]
  \item si $\pi = \bottomAlignProof
    \AxiomC{$\pi'$}\noLine\UnaryInfC{$\Gamma,\varphi\vdLJ[\psi]$}
    \RightLabel{$l\exists^\dagger$}
    \UnaryInfC{$\Gamma,\exists x, \varphi\vdLJ[\psi]$}
    \DisplayProof$, alors
    \[\tradNJ\pi\defeq\bottomAlignProof
    \def\fCenter{\mbox{\Large$\vdots$}}
    \AxiomC{$\exists x, \varphi$}
    \AxiomC{$[\varphi]^\alpha$}\noLine
    \UnaryInf$\fCenter\tradNJ{\pi'}$
    \noLine
    \UnaryInfC{$[\psi]$}\RightLabel{$\exists_\mathrm e^\dagger$}
    \BinaryInfC{$[\psi]$}\DisplayProof\]
  \item si $\pi = \bottomAlignProof
    \AxiomC{$\pi'$}\noLine\UnaryInfC{$\Gamma\vdLJ\varphi[t/x]$}
    \RightLabel{$r\exists$}\UnaryInfC{$\Gamma\vdLJ\exists x, \varphi$}
    \DisplayProof$, alors
    \[\tradNJ\pi\defeq \bottomAlignProof
    \def\fCenter{\mbox{\Large$\vdots$}}
    \Axiom$\fCenter\tradNJ{\pi'}$
    \noLine
    \UnaryInfC{$\varphi[t/x]$}
    \RightLabel{$\exists_\mathrm i$}\UnaryInfC{$\exists x, \varphi$}
    \DisplayProof\]
  \item si $\pi = \bottomAlignProof
    \AxiomC{$\pi'$}\noLine
    \UnaryInfC{$\psi_1[u/x],\ldots,\psi_n[u/x]\vdLJ[\varphi[u/x]]$}
    \RightLabel{$l=$}
    \UnaryInfC{$\psi_1[t/x],\ldots,\psi_n[t/x], t = u \vdLJ[\varphi[t/x]]$}
    \DisplayProof$, alors
    \[\tradNJ\pi\defeq\bottomAlignProof
    \def\fCenter{\mbox{\Large$\vdots$}}
    \AxiomC{$t = u$}
    \AxiomC{}
    \RightLabel{$=_\mathrm i$}
    \UnaryInfC{$t = t$}
    \RightLabel{$=_\mathrm e$}
    \BinaryInfC{$u = t$}
    \AxiomC{$t = u$}
    \AxiomC{$\psi_1[t/x]$}
    \RightLabel{$=_\mathrm e$}
    \BinaryInfC{$\psi_1[u/x]$}
    \AxiomC{$\cdots$}
    \AxiomC{$t = u$}
    \AxiomC{$\psi_n[t/x]$}
    \RightLabel{$=_\mathrm e$}
    \BinaryInfC{$\psi_n[u/x]$}
    \dashedLine
    \TrinaryInf$\fCenter\tradNJ{\pi'}$
    \noLine
    \UnaryInfC{$[\varphi[u/x]]$}
    \RightLabel{$=_\mathrm e$}
    \BinaryInfC{$[\varphi[t/x]]$}
    \DisplayProof
    \]
  \item si $\pi = \bottomAlignProof\AxiomC{}\RightLabel{$r=$}
    \UnaryInfC{$t =t$}\DisplayProof$, alors
    $\tradNJ{\pi}\defeq \AxiomC{}\RightLabel{$=_\mathrm i$}
    \UnaryInfC{$t = t$}\DisplayProof$.
  \end{itemize}
\end{definition}

\begin{corollary}
  Un séquent intuitionniste est prouvable dans $\NJ$ si et seulement s'il est
  vrai dans tout modèle de Kripke intuitionniste.
\end{corollary}

\begin{proof}
  On vérifie par une induction sans difficulté que le calcul $\NJ$ est correct.
  Si un séquent est vrai dans tout modèle de Kripke intuitionniste, alors il
  est prouvable dans $\LJ$ par complétude, mais pour $\pi$ la preuve ainsi
  obtenue, $\tradNJ\pi$ est un arbre de preuve du même séquent dans $\NJ$.
\end{proof}

Cette construction se sert crucialement du fait que les séquent sont
intuitionnistes dans $\LJ$, car les arbres de preuves de déduction naturelle
sont construits avec une unique conclusion. On peut donc se demander comment
cette construction peut s'adapter au cas de $\LK$.

On peut donner une première façon de gérer ce cas classique. Les possibilités
supplémentaires lorsqu'on considère $\LK$ en comparaison de $\LJ$ sont liées à
la capacité à avoir plusieurs formules à droite, et il est possible d'avoir
plusieurs formules à gauche même dans un séquent intuitionniste. On peut donc
chercher à déplacer toutes les formules à gauche, de sorte que les raisonnements
classiques peuvent se faire de façon identique dans les deux formalismes. La
différence principale est qu'alors, si on a par exemple
\[\Gamma,\lnot\Delta\vdLK\]
représentant le séquent $\Gamma\vdLK\Delta$ (la négation permettant de faire
passer toutes les formules de $\Delta$ à gauche), on peut récupérer
\[\Gamma\vdLK\Delta\]
en procédant à des coupures contre $\vdLK\lnot\psi,\psi$ pour chaque
$\psi \in \Delta$.

Dans le cas intuitionniste, la seule possibilité est, d'abord, de transformer
$\lnot\Delta$ en une seule formule $\bigwedge\lnot\Delta$, puis de la passer à
droite avec $r\lnot$. On obtient ainsi une double négation, dans le cas d'une
seule formule. On cherche donc à prouver un énoncé de la forme~:
\[\Gamma\vdLK\varphi \iff \Gamma\vdLJ\lnot\lnot\varphi\]
Cette équivalence, vraie en logique propositionnelle (\latinexpr{c.f.}
\cref{exo.glivenko}), n'est plus vraie en logique du premier ordre. A la
place, on donne la traduction de Gödel-Gentzen, qui remplit le rôle qu'a
$\lnot\lnot\varphi$ ci-dessus.

\begin{definition}[Traduction de Gödel-Gentzen \cite{GodelTradNot,Gentzen1936}]
  Pour toute formule $\varphi$, on définit par induction sa traduction de
  Gödel-Gentzen $\varphi^N$ par~:
  \begin{itemize}
  \item si $\varphi$ est atomique, alors $\varphi^N \defeq \lnot\lnot\varphi$.
  \item si $\varphi = \lnot \psi$, alors $\varphi^N\defeq\lnot\psi^N$.
  \item si $\varphi = \psi\land\chi$, alors $\varphi^N\defeq\psi^N\land\chi^N$.
  \item si $\varphi = \psi\lor\chi$, alors
    $\varphi^N\defeq \lnot(\lnot\psi^N\land\lnot\chi^N)$.
  \item si $\varphi = \psi \to \chi$, alors $\varphi^N\defeq\psi^N\to\chi^N$.
  \item si $\varphi = \forall x, \psi$, alors
    $\varphi^N\defeq \forall x, \psi^N$.
  \item si $\varphi = \exists x, \psi$, alors
    $\varphi^N\defeq \lnot(\forall x, \lnot\psi^N)$.
  \end{itemize}
\end{definition}

\begin{theorem}
  Pour tout contexte $\Gamma$ et toute formule $\varphi$, on a
  \[\Gamma\vdLK\varphi \iff \Gamma^N\vdLJ\varphi^N\]
\end{theorem}

\begin{proof}
  On peut facilement vérifier que pour toute formule $\varphi$, on a
  $\varphi\dashv\vdLK\varphi^N$, ce qui permet de déduire directement que si
  $\Gamma^N\vdLJ\varphi^N$, puisqu'alors $\Gamma^N\vdLK\varphi^N$, on a
  $\Gamma\vdLK\varphi$.

  A FAIRE
\end{proof}

\begin{remark}
  La position des doubles négations est en quelque sorte duale de celle des
  modalités $\square$ lorsqu'on définit $\varphi_\square$. Là où les modalités
  $\square$ renforcent les constructeurs négatifs $\to, \forall$ (et $\lnot$
  en conséquence) pour rendre les formules plus difficiles à prouver, ce sont
  les constructeurs positifs $\lor,\exists$ qui sont affaiblis en en prenant
  une formule de dualité de De Morgan. Justement, ce sont ces constructeurs
  positifs qui, en logique intuitionniste, sont plus difficiles à prouver.
  
  Par exemple, prouver $\varphi\lor\psi$ demande d'expliciter laquelle des
  deux formules entre $\varphi$ et $\psi$ est vraie, mais prouver
  $\lnot(\lnot\varphi\land\lnot\psi)$ signifie qu'on se donne des
  contre-preuves de $\varphi$ et de $\psi$, et qu'on veut en déduire une
  contradiction~: l'information est perdue de laquelle des deux formules est
  vraie.
\end{remark}

\begin{exercise}[Théorème de Glivenko \cite{glivenko1929}]\label{exo.glivenko}
  Montrer que dans le cas de formules propositionnelles, l'équivalence
  \[\Gamma\vdLK\varphi\iff\Gamma\vdLJ\lnot\lnot\varphi\]
  est vérifiée.
\end{exercise}

On peut alors en déduire la complétude de $\NK$, en montrant que l'équivalence
$\varphi \leftrightarrow \varphi^N$ est prouvable dans $\NK$. Le choix de la
règle du raisonnement par l'absurde est alors le bon, puisqu'il permet d'obtenir
l'élimination de la double négation, c'est-à-dire le schéma d'axiomes
\[\lnot\lnot\varphi\to\varphi\]
qui permet de passer de $\varphi^N$ à $\varphi$ (l'autre sens peut être montré
directement dans $\NJ$).

\begin{exercise}
  Prouver que $\varphi\vdNJ\varphi^N$ et que $\varphi^N\vdNK\varphi$. En
  déduire la complétude du système $\NK$.
\end{exercise}

\section{Coupures en déduction naturelle}

Dans le calcul des séquents, le \textit{Hauptsatz} permettait de considérer que
les preuves peuvent se réécrire sans coupure. En déduction naturelle, nous
n'avons plus la règle de coupure, mais la coupure elle-même est dérivable~:
\begin{prooftree}
  \def\fCenter{\mbox{\Large$\vdots$}}
  \AxiomC{$\Delta$}
  \AxiomC{$[\varphi]^\alpha$}
  \dashedLine
  \BinaryInf$\fCenter\pi'$
  \noLine
  \UnaryInfC{$\psi$}
  \RightLabel{$\to_\mathrm i,\alpha$}
  \UnaryInfC{$\varphi\to\psi$}
  \AxiomC{$\Gamma$}
  \noLine\UnaryInf$\fCenter\pi$
  \noLine\UnaryInfC{$\varphi$}
  \RightLabel{$\to_\mathrm e$}\BinaryInfC{$\psi$}
\end{prooftree}
On remarque que lorsqu'on a défini la traduction $\tradNJ -$, la traduction de
la règle de coupure était plus simple, en remplaçant la feuille marquée
$[\varphi]^\alpha$ par l'arbre de droite, pour arriver directement à $\psi$.

\`A la place, dans cette dérivation, on introduit une règle qu'on élimine tout
de suite après. Cette situation est ce qu'on appelle, pour la déduction
naturelle, une coupure~: dans le cas de l'implication, on coupe la preuve en
deux pour définir un lemme $\varphi \to \psi$ dont on prouve la prémisse
$\varphi$.

On va donc étudier la suppression de ce type de coupures.

\subsection{L'élimination des coupures}

Commençons par définir formellement une coupure. Précisons que nous ne
travaillons ici que dans $\NJ$.

\begin{definition}[Coupure dans $\NJ$]
  Une coupure dans un arbre de $\NJ$ est une suite de deux règles telle que
  la règle du bas est une règle d'élimination dont la prémisse de gauche est
  la conclusion de la règle du haut, et où cette règle haute est au choix~:
  \begin{itemize}
  \item une règle d'introduction du même constructeur que la règle
    d'élimination, auquel cas on parle de coupure logique~;
  \item la règle $\bot_\mathrm e$, auquel cas on parle de $\bot$-coupure~;
  \item une règle $\lor_\mathrm e$ ou $\exists_\mathrm e$, auquel cas on parle
    de coupure commutative.
  \end{itemize}
\end{definition}

Comme dans le cas du calcul des séquents, la propriété essentielle de notre
système est l'existence de preuves sans coupures, dès lors qu'une preuve pour
un séquent donné existe. On donne directement la réduction permettant d'éliminer
les coupures.

\begin{definition}[Réduction des coupures]
  La réduction des coupures $\reecr{}$ est donnée
  par la plus petite relation compatible contenant l'union des réductions
  logiques $\reecr{L}$, des $\bot$-réductions $\reecr{\bot}$ et des réductions
  commutatives $\reecr{C}$, données respectivement en \cref{tbl.cut.log.NJ},
  en \cref{tbl.cut.bot.NJ} et en \cref{tbl.cut.com.NJ}.
\end{definition}

\begin{table}[t!]
  \centering
  \begin{tabular}{cc}
    \multicolumn{2}{c}{
      %\resizebox{0.5\textwidth}{!}{
      \def\fCenter{\mbox{\Large$\vdots$}}
      \AxiomC{$[\varphi]^\alpha$}
      \noLine
      \UnaryInf$\fCenter\pi$
      \noLine\UnaryInfC{$\psi$}
      \RightLabel{$\to_\mathrm i, \alpha$}
      \UnaryInfC{$\varphi\to\psi$}
      \Axiom$\fCenter\pi'$
      \noLine\UnaryInfC{$\varphi$}
      \RightLabel{$\to_\mathrm e$}
      \BinaryInfC{$\psi$}
      \DisplayProof
      $\reecr{L^\to}$
      \def\fCenter{\mbox{\Large$\vdots$}}
      \Axiom$\fCenter\pi'$
      \noLine\UnaryInfC{$\varphi$}\noLine
      \UnaryInf$\fCenter\pi$
      \noLine\UnaryInfC{$\psi$}
      \DisplayProof
      %}
    }
    \\
    \\
    %\resizebox{0.5\textwidth}{!}{
    \def\fCenter{\mbox{\Large$\vdots$}}
    \Axiom$\fCenter\pi_1$
    \noLine\UnaryInfC{$\varphi_1$}
    \Axiom$\fCenter\pi_2$
    \noLine\UnaryInfC{$\varphi_2$}
    \RightLabel{$\land_\mathrm i$}
    \BinaryInfC{$\varphi_1\land\psi_2$}
    \RightLabel{$\land_\mathrm e^i$}
    \UnaryInfC{$\varphi_i$}
    \DisplayProof
    $\reecr{L^\land_i}$
    \def\fCenter{\mbox{\Large$\vdots$}}
    \Axiom$\fCenter\pi_i$
    \noLine\UnaryInfC{$\varphi_i$}
    \DisplayProof
    %}
    &
    %\resizebox{0.5\textwidth}{!}{
    \def\fCenter{\mbox{\Large$\vdots$}}
    \Axiom$\fCenter\pi$
    \noLine\UnaryInfC{$\varphi_i$}
    \RightLabel{$\lor_\mathrm i^i$}\UnaryInfC{$\varphi_1\lor\varphi_2$}
    \AxiomC{$[\varphi_1]^\alpha$}
    \noLine\UnaryInf$\fCenter\pi_1$
    \noLine\UnaryInfC{$\chi$}
    \AxiomC{$[\varphi_2]^\alpha$}
    \noLine\UnaryInf$\fCenter\pi_2$
    \noLine\UnaryInfC{$\chi$}
    \RightLabel{$\lor_\mathrm e, \alpha$}
    \TrinaryInfC{$\chi$}
    \DisplayProof
    $\reecr{L^\lor_i}$
    \def\fCenter{\mbox{\Large$\vdots$}}
    \Axiom$\fCenter\pi$
    \noLine\UnaryInfC{$\varphi_i$}
    \noLine\UnaryInf$\fCenter\pi_i$
    \noLine\UnaryInfC{$\chi$}
    \DisplayProof
    %}
    \\
    \\
    %\resizebox{0.5\textwidth}{!}{
    \def\fCenter{\mbox{\Large$\vdots$}}
    \Axiom$\fCenter\pi$
    \noLine\UnaryInfC{$\varphi$}
    \RightLabel{$\forall_\mathrm i^\dagger$}
    \UnaryInfC{$\forall x, \varphi$}
    \RightLabel{$\forall_\mathrm e$}
    \UnaryInfC{$\varphi[t/x]$}
    \DisplayProof
    $\reecr{L^\forall}$
    \def\fCenter{\mbox{\Large$\vdots$}}
    \Axiom$\fCenter\pi[t/x]$
    \noLine\UnaryInfC{$\varphi[t/x]$}
    \DisplayProof
    %}
    &
    %\resizebox{0.5\textwidth}{!}{
    \def\fCenter{\mbox{\Large$\vdots$}}
    \Axiom$\fCenter\pi$
    \noLine\UnaryInfC{$\varphi[t/x]$}
    \RightLabel{$\exists_\mathrm i$}
    \UnaryInfC{$\exists x, \varphi$}
    \Axiom$\fCenter\pi'$
    \noLine\UnaryInfC{$\chi$}
    \RightLabel{$\exists_\mathrm e^\dagger$}
    \BinaryInfC{$\chi$}
    \DisplayProof
    $\reecr{L^\exists}$
    \def\fCenter{\mbox{\Large$\vdots$}}
    \Axiom$\fCenter\pi$
    \noLine\UnaryInfC{$\varphi[t/x]$}
    \noLine\UnaryInf$\fCenter\pi'[t/x]$
    \noLine\UnaryInfC{$\chi$}
    \DisplayProof
    %}
  \end{tabular}

  \vspace{0.2cm}
  \hrule

  \vspace{0.4cm}
  \caption{Réductions logiques dans $\NJ$}
  \label{tbl.cut.log.NJ}
  \hrule
\end{table}

\begin{table}[t]
  \centering
  \begin{tabular}{cc}
    \multicolumn{2}{c}{
      %\resizebox{0.5\textwidth}{!}{
      \def\fCenter{\mbox{\Large$\vdots$}}
      \Axiom$\fCenter\pi$
      \noLine\UnaryInfC{$\bot$}
      \RightLabel{$\bot_\mathrm e$}
      \UnaryInfC{$\varphi\to\psi$}
      \Axiom$\fCenter\pi'$
      \noLine\UnaryInfC{$\varphi$}
      \RightLabel{$\to_\mathrm e$}
      \BinaryInfC{$\psi$}
      \DisplayProof
      $\reecr{\bot^\to}$
      \def\fCenter{\mbox{\Large$\vdots$}}
      \Axiom$\fCenter\pi$
      \noLine\UnaryInfC{$\bot$}
      \RightLabel{$\bot$}\UnaryInfC{$\psi$}
      \DisplayProof
      %}
    }
    \\
    \\
    %\resizebox{0.5\textwidth}{!}{
    \def\fCenter{\mbox{\Large$\vdots$}}
    \Axiom$\fCenter\pi$
    \noLine\UnaryInfC{$\bot$}
    \RightLabel{$\bot_\mathrm e$}
    \UnaryInfC{$\varphi_1\land\psi_2$}
    \RightLabel{$\land_\mathrm e^i$}
    \UnaryInfC{$\varphi_i$}
    \DisplayProof
    $\reecr{\bot^\land_i}$
    \def\fCenter{\mbox{\Large$\vdots$}}
    \Axiom$\fCenter\pi$
    \noLine\UnaryInfC{$\bot$}
    \RightLabel{$\bot_\mathrm e$}
    \UnaryInfC{$\varphi_i$}
    \DisplayProof
    %}
    &
    %\resizebox{0.5\textwidth}{!}{
    \def\fCenter{\mbox{\Large$\vdots$}}
    \Axiom$\fCenter\pi$
    \noLine\UnaryInfC{$\bot$}
    \RightLabel{$\bot_\mathrm e$}\UnaryInfC{$\varphi_1\lor\varphi_2$}
    \AxiomC{$[\varphi_1]^\alpha$}
    \noLine\UnaryInf$\fCenter\pi_1$
    \noLine\UnaryInfC{$\chi$}
    \AxiomC{$[\varphi_2]^\alpha$}
    \noLine\UnaryInf$\fCenter\pi_2$
    \noLine\UnaryInfC{$\chi$}
    \RightLabel{$\lor_\mathrm e, \alpha$}
    \TrinaryInfC{$\chi$}
    \DisplayProof
    $\reecr{\bot^\lor_i}$
    \def\fCenter{\mbox{\Large$\vdots$}}
    \Axiom$\fCenter\pi$
    \noLine\UnaryInfC{$\bot$}
    \RightLabel{$\bot_\mathrm e$}
    \UnaryInfC{$\chi$}
    \DisplayProof
    %}
    \\
    \\
    %\resizebox{0.5\textwidth}{!}{
    \def\fCenter{\mbox{\Large$\vdots$}}
    \Axiom$\fCenter\pi$
    \noLine\UnaryInfC{$\bot$}
    \RightLabel{$\bot_\mathrm e$}
    \UnaryInfC{$\forall x, \varphi$}
    \RightLabel{$\forall_\mathrm e$}
    \UnaryInfC{$\varphi[t/x]$}
    \DisplayProof
    $\reecr{\bot^\forall}$
    \def\fCenter{\mbox{\Large$\vdots$}}
    \Axiom$\fCenter\pi$
    \noLine\UnaryInfC{$\bot$}
    \RightLabel{$\bot_\mathrm e$}
    \UnaryInfC{$\varphi[t/x]$}
    \DisplayProof
    %}
    &
    %\resizebox{0.5\textwidth}{!}{
    \def\fCenter{\mbox{\Large$\vdots$}}
    \Axiom$\fCenter\pi$
    \noLine\UnaryInfC{$\bot$}
    \RightLabel{$\bot_\mathrm e$}
    \UnaryInfC{$\exists x, \varphi$}
    \Axiom$\fCenter\pi'$
    \noLine\UnaryInfC{$\chi$}
    \RightLabel{$\exists_\mathrm e^\dagger$}
    \BinaryInfC{$\chi$}
    \DisplayProof
    $\reecr{\bot^\exists}$
    \def\fCenter{\mbox{\Large$\vdots$}}
    \Axiom$\fCenter\pi$
    \noLine\UnaryInfC{$\bot$}
    \RightLabel{$\bot_\mathrm e$}
    \UnaryInfC{$\chi$}
    \DisplayProof
    %}
  \end{tabular}

  \vspace{0.2cm}
  \hrule

  \vspace{0.4cm}
  \caption{$\bot$-réductions dans $\NJ$}
  \label{tbl.cut.bot.NJ}
  \hrule
\end{table}

\begin{table}[t!]
  \centering
  \begin{tabular}{ccc}
    \resizebox{0.4\textwidth}{!}{
      \def\fCenter{\mbox{\Large$\vdots$}}
      \Axiom$\fCenter\pi$
      \noLine\UnaryInfC{$\varphi_1\lor\varphi_2$}
      \AxiomC{$[\varphi_1]^\alpha$}
      \noLine\UnaryInf$\fCenter\pi_1$
      \noLine\UnaryInfC{$\chi\to\xi$}
      \AxiomC{$[\varphi_2]^\alpha$}
      \noLine\UnaryInf$\fCenter\pi_2$
      \noLine\UnaryInfC{$\chi\to\xi$}
      \RightLabel{$\lor_\mathrm e,\alpha$}
      \TrinaryInfC{$\chi\to\xi$}
      \Axiom$\fCenter\pi'$
      \noLine\UnaryInfC{$\chi$}
      \RightLabel{$\to_\mathrm e$}
      \BinaryInfC{$\xi$}
      \DisplayProof}
    & $\reecr{C^\to_\lor}$ &
    \resizebox{0.4\textwidth}{!}{
      \def\fCenter{\mbox{\Large$\vdots$}}
      \Axiom$\fCenter\pi$
      \noLine\UnaryInfC{$\varphi_1\lor\varphi_2$}
      \AxiomC{$[\varphi_1]^\alpha$}
      \noLine\UnaryInf$\fCenter\pi_1$
      \noLine\UnaryInfC{$\chi\to\xi$}
      \Axiom$\fCenter\pi'$
      \noLine\UnaryInfC{$\chi$}
      \RightLabel{$\to_\mathrm e$}
      \BinaryInfC{$\xi$}
      \AxiomC{$[\varphi_2]^\alpha$}
      \noLine\UnaryInf$\fCenter\pi_2$
      \noLine\UnaryInfC{$\chi\to\xi$}
      \Axiom$\fCenter\pi'$
      \noLine\UnaryInfC{$\chi$}
      \RightLabel{$\to_\mathrm e$}
      \BinaryInfC{$\xi$}
      \RightLabel{$\lor_\mathrm e,\alpha$}
      \TrinaryInfC{$\xi$}
      \DisplayProof}
    \\
    \\
    \resizebox{0.3\textwidth}{!}{
      \def\fCenter{\mbox{\Large$\vdots$}}
      \Axiom$\fCenter\pi$
      \noLine\UnaryInfC{$\varphi_1\lor\varphi_2$}
      \Axiom$\fCenter\pi_1$
      \noLine\UnaryInfC{$\psi_1\land\psi_2$}
      \Axiom$\fCenter\pi_2$
      \noLine\UnaryInfC{$\psi_1\land\psi_2$}
      \RightLabel{$\lor_\mathrm e,\alpha$}
      \TrinaryInfC{$\psi_1\land\psi_2$}
      \RightLabel{$\land_\mathrm e^i$}
      \UnaryInfC{$\psi_i$}
      \DisplayProof}
    & $\reecr{C^\land_\lor}$ &
    \resizebox{0.3\textwidth}{!}{
      \def\fCenter{\mbox{\Large$\vdots$}}
      \Axiom$\fCenter\pi$
      \noLine\UnaryInfC{$\varphi_1\lor\varphi_2$}
      \Axiom$\fCenter\pi_1$
      \noLine\UnaryInfC{$\psi_1\land\psi_2$}
      \RightLabel{$\land_\mathrm e^i$}
      \UnaryInfC{$\psi_i$}
      \Axiom$\fCenter\pi_2$
      \noLine\UnaryInfC{$\psi_1\land\psi_2$}
      \RightLabel{$\land_\mathrm e^i$}
      \UnaryInfC{$\psi_i$}
      \RightLabel{$\lor_\mathrm e,\alpha$}
      \TrinaryInfC{$\psi_i$}
      \DisplayProof}
    \\
    \\
    \resizebox{0.4\textwidth}{!}{
      \def\fCenter{\mbox{\Large$\vdots$}}
      \Axiom$\fCenter\pi$
      \noLine\UnaryInfC{$\varphi_1\lor\varphi_2$}
      \AxiomC{$[\varphi_1]^\alpha$}
      \noLine\UnaryInf$\fCenter\pi_1$
      \noLine\UnaryInfC{$\psi_1\lor\psi_2$}
      \AxiomC{$[\varphi_2]^\alpha$}
      \noLine\UnaryInf$\fCenter\pi_2$
      \noLine\UnaryInfC{$\psi_1\lor\psi_2$}
      \RightLabel{$\lor_\mathrm e,\alpha$}
      \TrinaryInfC{$\psi_1\lor\psi_2$}
      \AxiomC{$[\psi_1]^\beta$}
      \noLine\UnaryInf$\fCenter\pi'_1$
      \noLine\UnaryInfC{$\chi$}
      \AxiomC{$[\psi_2]^\beta$}
      \noLine\UnaryInf$\fCenter\pi'_2$
      \noLine\UnaryInfC{$\chi$}
      \RightLabel{$\lor_\mathrm e,\beta$}
      \TrinaryInfC{$\chi$}
      \DisplayProof}
    & $\reecr{C^\lor_\lor}$ &
    \resizebox{0.4\textwidth}{!}{
      \def\fCenter{\mbox{\Large$\vdots$}}
      \Axiom$\fCenter\pi$
      \noLine\UnaryInfC{$\varphi_1\lor\varphi_2$}
      \AxiomC{$[\varphi_1]^\alpha$}
      \noLine\UnaryInf$\fCenter\pi_1$
      \noLine\UnaryInfC{$\psi_1\lor\psi_2$}
      \AxiomC{$[\psi_1]^\beta$}
      \noLine\UnaryInf$\fCenter\pi'_1$
      \noLine\UnaryInfC{$\chi$}
      \AxiomC{$[\psi_2]^\beta$}
      \noLine\UnaryInf$\fCenter\pi'_2$
      \noLine\UnaryInfC{$\chi$}
      \RightLabel{$\lor_\mathrm e,\beta$}
      \TrinaryInfC{$\chi$}
      \AxiomC{$[\varphi_2]^\alpha$}
      \noLine\UnaryInf$\fCenter\pi_2$
      \noLine\UnaryInfC{$\psi_1\lor\psi_2$}
      \AxiomC{$[\psi_1]^\beta$}
      \noLine\UnaryInf$\fCenter\pi'_1$
      \noLine\UnaryInfC{$\chi$}
      \AxiomC{$[\psi_2]^\beta$}
      \noLine\UnaryInf$\fCenter\pi'_2$
      \noLine\UnaryInfC{$\chi$}
      \RightLabel{$\lor_\mathrm e,\beta$}
      \TrinaryInfC{$\chi$}
      \RightLabel{$\lor_\mathrm e, \alpha$}
      \TrinaryInfC{$\chi$}
      \DisplayProof}
    \\
    \\
    \resizebox{0.3\textwidth}{!}{
      \def\fCenter{\mbox{\Large$\vdots$}}
      \Axiom$\fCenter\pi$
      \noLine\UnaryInfC{$\varphi_1\lor\varphi_2$}
      \Axiom$\fCenter\pi_1$
      \noLine\UnaryInfC{$\forall x, \psi$}
      \Axiom$\fCenter\pi_2$
      \noLine\UnaryInfC{$\forall x, \psi$}
      \RightLabel{$\lor_\mathrm e,\alpha$}
      \TrinaryInfC{$\forall x, \psi$}
      \RightLabel{$\forall_\mathrm e$}
      \UnaryInfC{$\psi[t/x]$}
      \DisplayProof}
    & $\reecr{C^\forall_\lor}$ &
    \resizebox{0.3\textwidth}{!}{
      \def\fCenter{\mbox{\Large$\vdots$}}
      \Axiom$\fCenter\pi$
      \noLine\UnaryInfC{$\varphi_1\lor\varphi_2$}
      \Axiom$\fCenter\pi_1$
      \noLine\UnaryInfC{$\forall x, \psi$}
      \RightLabel{$\land_\mathrm e^i$}
      \UnaryInfC{$\psi[t/x]$}
      \Axiom$\fCenter\pi_2$
      \noLine\UnaryInfC{$\forall x, \psi$}
      \RightLabel{$\land_\mathrm e^i$}
      \UnaryInfC{$\psi[t/x]$}
      \RightLabel{$\lor_\mathrm e,\alpha$}
      \TrinaryInfC{$\psi[t/x]$}
      \DisplayProof}
    \\
    \\
    \resizebox{0.3\textwidth}{!}{
      \def\fCenter{\mbox{\Large$\vdots$}}
      \Axiom$\fCenter\pi$
      \noLine\UnaryInfC{$\varphi_1\lor\varphi_2$}
      \AxiomC{$[\varphi_1]^\alpha$}
      \noLine\UnaryInf$\fCenter\pi_1$
      \noLine\UnaryInfC{$\exists x, \psi$}
      \AxiomC{$[\varphi_2]^\alpha$}
      \noLine\UnaryInf$\fCenter\pi_2$
      \noLine\UnaryInfC{$\exists x, \psi$}
      \RightLabel{$\lor_\mathrm e,\alpha$}
      \TrinaryInfC{$\exists x, \psi$}
      \AxiomC{$[\psi]^\beta$}
      \noLine\UnaryInf$\fCenter\pi'$
      \noLine\UnaryInfC{$\chi$}
      \RightLabel{$\exists_\mathrm e^\dagger,\beta$}
      \BinaryInfC{$\chi$}
      \DisplayProof}
    & $\reecr{C^\exists_\lor}$ &
    \resizebox{0.3\textwidth}{!}{
      \def\fCenter{\mbox{\Large$\vdots$}}
      \Axiom$\fCenter\pi$
      \noLine\UnaryInfC{$\varphi_1\lor\varphi_2$}
      \AxiomC{$[\varphi_1]^\alpha$}
      \noLine\UnaryInf$\fCenter\pi_1$
      \noLine\UnaryInfC{$\exists x, \psi$}
      \AxiomC{$[\psi]^\beta$}
      \noLine\UnaryInf$\fCenter\pi'$
      \noLine\UnaryInfC{$\chi$}
      \RightLabel{$\exists_\mathrm e^\dagger,\beta$}
      \BinaryInfC{$\chi$}
      \AxiomC{$[\varphi_2]^\alpha$}
      \noLine\UnaryInf$\fCenter\pi_2$
      \noLine\UnaryInfC{$\exists x, \psi$}
      \AxiomC{$[\psi]^\beta$}
      \noLine\UnaryInf$\fCenter\pi'$
      \noLine\UnaryInfC{$\chi$}
      \RightLabel{$\exists_\mathrm e^\dagger,\beta$}
      \BinaryInfC{$\chi$}
      \RightLabel{$\lor_\mathrm e, \alpha$}
      \TrinaryInfC{$\chi$}
      \DisplayProof}
    \\
    \\
    \resizebox{0.3\textwidth}{!}{
      \def\fCenter{\mbox{\Large$\vdots$}}
      \Axiom$\fCenter\pi$
      \noLine\UnaryInfC{$\exists x, \varphi$}
      \AxiomC{$[\varphi]^\alpha$}
      \noLine\UnaryInf$\fCenter\pi'$
      \noLine\UnaryInfC{$\chi\to\xi$}
      \RightLabel{$\exists_\mathrm e^\dagger,\alpha$}
      \BinaryInfC{$\chi\to\xi$}
      \Axiom$\fCenter\pi''$
      \noLine\UnaryInfC{$\chi$}
      \RightLabel{$\to_\mathrm e$}
      \BinaryInfC{$\xi$}
      \DisplayProof}
    & $\reecr{C^\to_\exists}$ &
    \resizebox{0.3\textwidth}{!}{
      \def\fCenter{\mbox{\Large$\vdots$}}
      \Axiom$\fCenter\pi$
      \noLine\UnaryInfC{$\exists x, \varphi$}
      \AxiomC{$[\varphi]^\alpha$}
      \noLine\UnaryInf$\fCenter\pi'$
      \noLine\UnaryInfC{$\chi\to\xi$}
      \Axiom$\fCenter\pi''$
      \noLine\UnaryInfC{$\chi$}
      \RightLabel{$\to_\mathrm e$}
      \BinaryInfC{$\xi$}
      \RightLabel{$\exists_\mathrm e^\dagger,\alpha$}
      \BinaryInfC{$\xi$}
      \DisplayProof}
    \\
    \\
    \resizebox{0.2\textwidth}{!}{
    \def\fCenter{\mbox{\Large$\vdots$}}
    \Axiom$\fCenter\pi$
    \noLine\UnaryInfC{$\exists x, \varphi$}
    \Axiom$\fCenter\pi'$
    \noLine\UnaryInfC{$\psi_1\land\psi_2$}
    \RightLabel{$\exists_\mathrm e^\dagger,\alpha$}
    \BinaryInfC{$\psi_1\land\psi_2$}
    \RightLabel{$\land_\mathrm e^i$}
    \UnaryInfC{$\psi_i$}
    \DisplayProof
    }
    & $\reecr{C^\land_\exists}$ &
    \resizebox{0.2\textwidth}{!}{
    \def\fCenter{\mbox{\Large$\vdots$}}
    \Axiom$\fCenter\pi$
    \noLine\UnaryInfC{$\exists x, \varphi$}
    \Axiom$\fCenter\pi'$
    \noLine\UnaryInfC{$\psi_1\land\psi_2$}
    \RightLabel{$\land_\mathrm e^i$}
    \UnaryInfC{$\psi_i$}
    \RightLabel{$\exists_\mathrm e^\dagger,\alpha$}
    \BinaryInfC{$\psi_i$}
    \DisplayProof
    }
    \\
    \\
    \resizebox{0.4\textwidth}{!}{
      \def\fCenter{\mbox{\Large$\vdots$}}
      \Axiom$\fCenter\pi$
      \noLine\UnaryInfC{$\exists x, \varphi$}
      \AxiomC{$[\varphi]^\alpha$}
      \noLine\UnaryInf$\fCenter\pi'$
      \noLine\UnaryInfC{$\psi_1\lor\psi_2$}
      \RightLabel{$\exists_\mathrm e^\dagger,\alpha$}
      \BinaryInfC{$\psi_1\lor\psi_2$}
      \AxiomC{$[\psi_1]^\beta$}
      \noLine\UnaryInf$\fCenter\pi'_1$
      \noLine\UnaryInfC{$\chi$}
      \AxiomC{$[\psi_2]^\beta$}
      \noLine\UnaryInf$\fCenter\pi'_2$
      \noLine\UnaryInfC{$\chi$}
      \RightLabel{$\lor_\mathrm e,\beta$}
      \TrinaryInfC{$\chi$}
      \DisplayProof}
    & $\reecr{C^\lor_\exists}$ &
    \resizebox{0.4\textwidth}{!}{
      \def\fCenter{\mbox{\Large$\vdots$}}
      \Axiom$\fCenter\pi$
      \noLine\UnaryInfC{$\exists x, \varphi$}
      \AxiomC{$[\varphi]^\alpha$}
      \noLine\UnaryInf$\fCenter\pi'$
      \noLine\UnaryInfC{$\psi_1\lor\psi_2$}
      \AxiomC{$[\psi_1]^\beta$}
      \noLine\UnaryInf$\fCenter\pi'_1$
      \noLine\UnaryInfC{$\chi$}
      \AxiomC{$[\psi_2]^\beta$}
      \noLine\UnaryInf$\fCenter\pi'_2$
      \noLine\UnaryInfC{$\chi$}
      \RightLabel{$\lor_\mathrm e,\beta$}
      \TrinaryInfC{$\chi$}
      \RightLabel{$\lor_\mathrm e, \alpha$}
      \BinaryInfC{$\chi$}
      \DisplayProof}
    \\
    \\
    \resizebox{0.2\textwidth}{!}{
    \def\fCenter{\mbox{\Large$\vdots$}}
    \Axiom$\fCenter\pi$
    \noLine\UnaryInfC{$\exists x, \varphi$}
    \Axiom$\fCenter\pi'$
    \noLine\UnaryInfC{$\forall y, \psi$}
    \RightLabel{$\exists_\mathrm e^\dagger,\alpha$}
    \BinaryInfC{$\forall y, \psi$}
    \RightLabel{$\forall_\mathrm e$}
    \UnaryInfC{$\psi[t/y]$}
    \DisplayProof
    }
    & $\reecr{C^\forall_\exists}$ &
    \resizebox{0.2\textwidth}{!}{
    \def\fCenter{\mbox{\Large$\vdots$}}
    \Axiom$\fCenter\pi$
    \noLine\UnaryInfC{$\exists x, \varphi$}
    \Axiom$\fCenter\pi'$
    \noLine\UnaryInfC{$\forall y, \psi$}
    \RightLabel{$\land_\mathrm e^i$}
    \UnaryInfC{$\psi[t/y]$}
    \RightLabel{$\exists_\mathrm e^\dagger,\alpha$}
    \BinaryInfC{$\psi[t/y]$}
    \DisplayProof
    }
    \\
    \\
    \resizebox{0.3\textwidth}{!}{
      \def\fCenter{\mbox{\Large$\vdots$}}
      \Axiom$\fCenter\pi$
      \noLine\UnaryInfC{$\exists x, \varphi$}
      \AxiomC{$[\varphi]^\alpha$}
      \noLine\UnaryInf$\fCenter\pi'$
      \noLine\UnaryInfC{$\exists y, \psi$}
      \RightLabel{$\exists_\mathrm e^\dagger,\alpha$}
      \BinaryInfC{$\exists y, \psi$}
      \AxiomC{$[\psi]^\beta$}
      \noLine\UnaryInf$\fCenter\pi''$
      \noLine\UnaryInfC{$\chi$}
      \RightLabel{$\exists_\mathrm e^\dagger,\beta$}
      \BinaryInfC{$\chi$}
      \DisplayProof}
    & $\reecr{C^\exists_\exists}$ &
    \resizebox{0.3\textwidth}{!}{
      \def\fCenter{\mbox{\Large$\vdots$}}
      \Axiom$\fCenter\pi$
      \noLine\UnaryInfC{$\exists x, \varphi$}
      \AxiomC{$[\varphi]^\alpha$}
      \noLine\UnaryInf$\fCenter\pi'$
      \noLine\UnaryInfC{$\exists y, \psi$}
      \AxiomC{$[\psi]^\beta$}
      \noLine\UnaryInf$\fCenter\pi''$
      \noLine\UnaryInfC{$\chi$}
      \RightLabel{$\exists_\mathrm e^\dagger,\beta$}
      \BinaryInfC{$\chi$}
      \RightLabel{$\exists_\mathrm e^\dagger,\alpha$}
      \BinaryInfC{$\chi$}
      \DisplayProof}
  \end{tabular}

  \vspace{0.2cm}
  \hrule

  \vspace{0.4cm}
  \caption{Réductions commutatives dans $\NJ$}
  \label{tbl.cut.com.NJ}
  \hrule

\end{table}

\begin{remark}
  La plupart des coupures logiques (si ce n'est toutes) ressemblent à de pures
  évidences. Une coupure pour $\to$ est par exemple le fait de prouver d'abord
  $\varphi\to\psi$ pour ensuite prouver $\varphi$ et en déduire $\psi$. On
  pourrait penser, sous cette forme, qu'une telle situation n'arrive pas en
  pratique puisqu'il est plus direct de prouver $\psi$ avec la version sans
  coupure. Pourtant, c'est bien la version avec coupure qu'on utilise lorsqu'on
  établit un théorème abstrait, souvent de la forme
  $\forall x, \varphi \to \psi$, pour en récupérer une instance précise
  $\psi[t/x]$.
\end{remark}

Cette réduction conserve les hypothèses et les conclusions, et permet donc de
réécrire un arbre en un arbre prouvant le même séquent, et sans coupure. Il
faut prouver que la réécriture termine, ce que nous verrons dans la
\cref{subsec.ARS} et dans le CHAPITRE LAMBDA CALCUL TYPE.
Pour l'instant, nous nous intéressons aux conséquences de cette élimination des
coupures.

\begin{theorem}[\'Elimination des coupures dans $\NJ$]
  Pour tout arbre $\pi\in\ProofNJ$, il existe un arbre $\pi'$ tel que
  $\pi{\reecr{}}^\star\pi'$ et $\pi'$ est sans coupure.

  Un séquent $\Gamma\vdNJ\varphi$ est prouvable si et seulement s'il est
  prouvable par un arbre de preuve sans coupure.
\end{theorem}

\begin{proof}
  Reportée au CHAPITRE LAMBDA CALCUL TYPE.
\end{proof}

On a alors une notion d'arbre sous une forme d'intérêt, sans coupure. On peut
vérifier que, comme pour le calcul des séquents, on a la propriété de la
sous-formule.

\begin{property}[Sous-formule]
  Pour tout arbre $\pi\in\ProofNJ$ sans coupure et pour $\Gamma\vdNJ\varphi$
  sa conclusion, chaque formule apparaissant dans $\pi$ est une sous-formule
  de $\varphi$ ou de l'une des formules de $\Gamma$.
\end{property}

On peut aussi déduire directement le fait suivant~:

\begin{proposition}
  Tout arbre $\pi\in\ProofNJ$ sans coupure tel que $\pi\concl \vdNJ\varphi$ a
  comme dernière règle une règle d'introduction.
\end{proposition}

\begin{proof}
  Si la dernière règle est une règle d'élimination, alors la propriété de la
  sous-formule n'est pas respectée.
\end{proof}

D'où les propriétés déjà prouvées dans $\LJ$ de la disjonction et du témoin.

\begin{property}[Disjonction]
  Pour toutes formules $\varphi,\psi$, si $\vdNJ\varphi\lor\psi$ alors
  $\vdNJ\varphi$ ou $\vdNJ\psi$.
\end{property}

\begin{proof}
  Si $\vdNJ\varphi\lor\psi$, alors on trouve un arbre $\pi$ sans coupure
  tel que $\pi\concl\vdNJ\varphi\lor\psi$. On sait que $\pi$ a comme dernière
  règle une règle d'introduction, et la seule pouvant s'appliquer à un
  $\lor$ est une des deux règles $\lor_\mathrm i$. On en déduit le résultat.
\end{proof}

\begin{property}[Témoin]
  Pour toute formule $\varphi$, si $\vdNJ\exists x, \varphi$ alors il existe
  un terme $t$ tel que $\vdNJ\varphi[t/x]$.
\end{property}

\begin{proof}
  Si $\vdNJ\exists x, \varphi$, alors on trouve $\pi$ sans coupure tel que
  $\pi\concl\vdNJ\exists x, \varphi$. Comme $\pi$, commence par une règle
  d'introduction, on a le résultat.
\end{proof}

Nous avons vu trois réécriture au fil des derniers chapitres, qui ont permi à
chaque fois de faire apparaître des formes remarquables des arbres de preuves.
Pour abstraire les propriétés et parler de façon plus génériques de systèmes de
réécriture, on introduit les systèmes de réécriture abstraits.

\subsection{Systèmes de réécriture abstraits}\label{subsec.ARS}

L'idée de ce domaine est de considérer une relation $\to$,
représentant une règle de réécriture, pour établir des propriétés d'intérêt et
de souligner les liens entre ces propriétés.

\begin{definition}[Système de réécriture abstrait]
  Un ARS (\foreignexpr{Abstract Rewriting System}), ou système de réécriture
  abstrait, est une paire $(A,\to)$ où $\to$ est une relation binaire sur
  $A$.
\end{definition}

Puisque $\to$ représente une réécriture, l'intérêt de $\to$ est généralement
d'être irréflexif, mais on s'intéresse aussi à sa clôture transitive, ou à
d'autres clôtures construites depuis $\to$.

\begin{definition}[Clôture réflexive, transitive, symétrique]
  Pour une relation binaire $\to$ sur un ensemble $A$, on note
  $\to^{=}$ la plus petite relation réflexive contenant $\to$, on note
  $\to^+$ la plus petite relation transitive contenant $\to$, et on note
  $\to^\star$ la plus petite relation réflexive et transitive contenant $\to$.

  On note aussi $\leftrightarrow$ la plus petite relation symétrique contenant
  $\to$ (ou $R^\leftrightarrow$ si la relation n'est pas notée par un symbole de
  flèche) et $\xrightarrow{\sim}$ la plus petite relation d'équivalence
  contenant $\to$.
\end{definition}

\begin{exercise}
  Montrer que pour chaque opération de clôture $C$, les propriétés suivantes
  sont vérifiées~:
  \begin{align*}
    \forall R,R' \subseteq A^2, &R\subseteq R' \implies C(R) \subseteq C(R') \\
    \forall R \subseteq A^2, &C(C(R)) = C(R) \\
    \forall R \subseteq A^2, &R \subseteq C(R)
  \end{align*}
  Ces propriétés font de $C$ ce qu'on appelle un opérateur de clôture.
\end{exercise}

\begin{exercise}
  Montrer que pour toute relation binaire $to$, on a
  $(\xrightarrow{\sim}) = (\leftrightarrow)^\star$. A-t-on
  $(\xrightarrow{\sim}) = (\to^\star)^\leftrightarrow$ ?
\end{exercise}

\begin{exercise}
  On définit la composée de deux relation $R,S\subseteq A^2$ par
  \[R\circ S \defeq \{(x,z) \in A^2 \mid \exists y \in A, xRy \land yRz\}\]
  On définit aussi les itérées d'une relation par
  \begin{align*}
    R^0 &\defeq \{(x,x) \mid a \in A\} \\
    R^{n+1} &\defeq R \circ R^n
  \end{align*}
  Montrer que $\displaystyle R^\star = \bigcup_{n \in \bN} R^n$.
\end{exercise}

Malgré la situation particulièrement générique dans laquelle on se place, on
souhaite généralement travailler avec des ensembles particuliers, définis
inductivement. Dans ce cadre, on a déjà vu une propriété abstraite importante,
qui est celle de relation compatible.

\begin{definition}[Relation compatible, congruence]
  Soit $A$ un ensemble inductif. Une relation binaire $R\subseteq A^2$ est
  dite compatible si pour tout constructeur $c$ d'arité $n$, pour tous
  termes $a_1,\ldots,a_n,b\in A$ et indice $i$, si
  $a_i R b$ alors
  \[c(a_1,\ldots,a_{i-1},a_i,a_{i+1},\ldots,a_n) R
  c(a_1,\ldots,a_{i-1},b,a_{i+1},\ldots,a_n)\]

  Une relation est dite une congruence si elle est compatible et est une
  relation d'équivalence.
\end{definition}

Dans le cas de $\LK$, de $LJ$ et de $\NJ$, on définit chaque relation $\reecr{}$
comme étant compatible, car il est important de pouvoir réécrire dans les
sous-arbre pour éliminer les coupures. Cependant, la compatibilité entraine un
certain choix dans les coupures possibles. Il arrive de n'avoir qu'une seule
coupure à réduire, mais il est aussi fréquent que plusieurs coupures
apparaissent, et on peut donc se demander si choisir une coupure ou une autre
impacte le résultat.

Cette invariance du résultat par le choix des coupures à réduire peut être
résumé aux propriétés de confluence.

\begin{figure}[ht!]
  \centering
  \hrule

  \vspace{0.2cm}
  \begin{tikzpicture}[node distance = 2cm, on grid, auto]
    \node (q_0) {$a$};
    \node (q_1) [below left = of q_0] {$b$};
    \node (q_2) [below right = of q_0] {$c$};
    \node (q_3) [below right = of q_1] {$d$};
    \node (r_1) [right = of q_2] {$b$};
    \node (r_0) [above right = of r_1] {$a$};
    \node (r_2) [below right = of r_0] {$c$};
    \node (r_3) [below right = of r_1] {$d$};
    \node (s_1) [right = of r_2] {$b$};
    \node (s_0) [above right = of s_1] {$a$};
    \node (s_2) [below right = of s_0] {$c$};
    \node (s_3) [below right = of s_1] {$d$};
    \node (a) at (0,-3.5) {Confluence};
    \node (b) at (5,-3.5) {Confluence locale};
    \node (c) at (10,-3.5) {Propriété du carreau};
    \draw[->,>=latex] (q_0) to node[very near end,below]{$\star$} (q_1);
    \draw[->,>=latex] (q_0) to node[very near end,below]{$\star$} (q_2);
    \draw[->,>=latex,dashed] (q_1) to node[very near end,below]{$\star$} (q_3);
    \draw[->,>=latex,dashed] (q_2) to node[very near end,below]{$\star$} (q_3);
    \draw[->,>=latex] (r_0) -- (r_1);
    \draw[->,>=latex] (r_0) -- (r_2);
    \draw[->,>=latex,dashed] (r_1) to node[very near end,below]{$\star$} (r_3);
    \draw[->,>=latex,dashed] (r_2) to node[very near end,below]{$\star$} (r_3);
    \draw[->,>=latex] (s_0) -- (s_1);
    \draw[->,>=latex] (s_0) -- (s_2);
    \draw[->,>=latex,dashed] (s_1) -- (s_3);
    \draw[->,>=latex,dashed] (s_2) -- (s_3);
  \end{tikzpicture}

  \vspace{0.2cm}
  \hrule

  \vspace{0.4cm}
  \caption{Confluence, confluence locale, propriété du carreau}
  \label{fig.confluence}
  \hrule
\end{figure}

\begin{definition}[Confluence]
  Soit $(A,\to)$ un ARS. On dit que $\to$ est confluente si pour tous
  $a,b,c \in A$, si $a\to^\star b$ et $a\to^\star c$, alors il existe
  $d \in A$ tel que $b\to^\star d$ et $c\to^\star d$.
\end{definition}

\begin{definition}[Confluence locale]
  Soit $(A,\to)$ un ARS. On dit que $\to$ est localement confluente si pour
  tous $a,b,c \in A$, si $a \to b$ et $a \to c$, alors il existe
  $d \in A$ tel que $b \to^\star d$ et $c\to^\star d$.
\end{definition}

\begin{definition}[Propriété du carreau]
  Soit $(A,\to)$ un ARS. On dit que $\to$ a la propriété du carreau
  (on dit aussi propriété du diamant) si pour
  tous $a,b,c \in A$, si $a \to b$ et $a \to c$, alors il existe
  $d \in A$ tel que $b \to d$ et $c \to d$.
\end{definition}

Ces trois propriétés se ressemblent grandement, et diffèrent uniquement de la
position des clôtures $\star$. La \cref{fig.confluence} permet de mieux
apprécier les différences entre chaque propriété.

Le lien entre les trois propriétés n'est pas forcément intuitif~: la propriété
du carreau est la plus forte, suivi par la confluence puis la confluence
locale. Un point important est le fait que la confluence locale n'implique pas
la confluence, comme le montre la \cref{fig.ctrex.confluence}.

\begin{figure}[ht!]
  \centering
  \hrule

  \vspace{0.2cm}
  \begin{tikzpicture}[node distance = 2cm, on grid, auto]
    \node (q_0) {$a$};
    \node (q_1) [right of = q_0] {$b$};
    \node (q_2) [right of = q_1] {$c$};
    \node (q_3) [right of = q_2] {$d$};
    \draw[->,>=latex] (q_1) -- (q_0);
    \draw[->,>=latex] (q_2) -- (q_3);
    \draw[->,>=latex] (q_1) to[bend left] (q_2);
    \draw[->,>=latex] (q_2) to[bend left] (q_1);
  \end{tikzpicture}

  \vspace{0.2cm}
  \hrule
  
  \vspace{0.4cm}
  \caption{Confluence locale sans confluence}
  \label{fig.ctrex.confluence}
  \hrule
\end{figure}

On prouve donc que la propriété du carreau suffit à obtenir la confluence.
En fait, cette propriété assure une encore meilleure propriété, puisqu'elle
conserve en un sens le nombre de réductions.

\begin{proposition}\label{prop.carreau.confluence}
  Soit $(A,\to)$ un ARS. Si $\to$ a la propriété du carreau, alors $\to$ est
  confluente.
\end{proposition}

\begin{figure}[ht!]
  \centering
  \hrule

  \vspace{0.2cm}
  \begin{tikzpicture}[node distance = 1.2cm, on grid, auto]
    \node (a) {$a$};
    \node (a_0) [below left = of a] {$a_0$};
    \node (a_1) [below right = of a] {$a_1$};
    \node (a') [below right = of a_0] {$a'$};
    \node (no) [below left = of a_0] {};
    \node (noo) [below right = of a_1] {};
    \node (b) [below left = of no] {$b$};
    \node (c) [below right = of noo] {$c$};
    \node (b_0) [below right = of b] {$b_0$};
    \node (c_0) [below left = of c] {$c_0$};
    \node (nooo) [below left = of c_0] {};
    \node (d) [below left = of nooo] {$d$};
    \draw[->,>=latex] (a) -- (a_0);
    \draw[->,>=latex] (a) -- (a_1);
    \draw[->,>=latex] (b) -- (b_0);
    \draw[->,>=latex] (c) -- (c_0);
    \draw[->,>=latex,dashed] (a_0) -- (a');
    \draw[->,>=latex,dashed] (a_1) -- (a');
    \draw[->,>=latex] (a_0) to node[very near end,below]{$p$} (b);
    \draw[->,>=latex] (a_1) to node[very near end,below]{$k$} (c);
    \draw[->,>=latex,dashed] (b_0) to node[very near end,below]{$k$} (d);
    \draw[->,>=latex,dashed] (c_0) to node[very near end,below]{$p$} (d);
    \draw[->,>=latex,dashed] (a') to node[very near end,below]{$p$} (b_0);
    \draw[->,>=latex,dashed] (a') to node[very near end,below]{$k$} (c_0);
  \end{tikzpicture}

  \vspace{0.2cm}
  \hrule
  
  \vspace{0.4cm}
  \caption{Preuve de la \cref{prop.carreau.confluence}}
  \label{fig.carreau.confluence}
  \hrule
\end{figure}

\begin{proof}
  On prouve par récurrence sur $\max(p,k)$ que si $a\to^p b$ et $a\to^k c$ alors
  il existe $d$ tel que $b\to^{\leq k} d$ et $c\to^{\leq p} d$. Le cas où $p = 0$
  ou $k = 0$ est évident en prenant respectivement $d = c$ et $d = b$.

  On suppose donc la propriété vraie pour $n$, et on souhaite la
  prouver pour $n + 1$. Soit donc $a \to^{p+1} b$, $a\to^{k+1} c$. Par définition,
  on trouve $a_0,a_1$ tels que $a\to a_0 \to^p b$ et $a\to a_1 \to^k c$.

  Par propriété du diamant, on trouve donc $a'$ tel que
  $a_0 \to a'$ et $a_1 \to a'$. On peut alors appliquer l'hypothèse de
  récurrence à~:
  \begin{itemize}
  \item $a_0 \to^p b$, $a_0 \to a'$ pour trouver $b_0$ tel que
    $b \to b_0$ et $a' \to^{\leq p} b_0$.
  \item $a_1 \to^k c$, $a_1 \to a'$ pour trouver $c_0$ tel que
    $c \to c_0$ et $a' \to^{\leq k} c_0$.
  \end{itemize}
  mais alors, par hypothèse de récurrence sur
  $a' \to^{\leq p} b_0$, $a' \to^{\leq k} c_0$,
  on trouve $d$ tel que $b_0 \to^{\leq k} d$ et $c_0\to^{\leq p} d$, d'où
  $b \to^{\leq k+1} d$ et $c\to^{\leq p+1} d$. D'où le résultat par récurrence.
\end{proof}

\begin{remark}
  La propriété du carreau ne s'applique que sur des ARS qui ont, depuis un
  élément $a$, ou bien une suite infinie de réductions, ou bien aucune
  réduction partant de $a$. En effet, si on a $a \to b$, alors on peut
  considérer $a \to b$, $a \to b$ pour trouver $c$ tel que $b \to c$, puis
  réeffectuer cette construction pour prolonger à chaque fois la suite de
  réductions. En pratique, les relations qui vérifient la propriété du
  carreau sont des relations réflexives, et la suite infinie de réductions est
  donc une suite constante.
\end{remark}

D'après la remarque précédente, la propriété du carreau n'est généralement pas
une propriété qu'on retrouve dans une relation de réécriture, ou de réduction,
puisqu'on veut qu'une telle relation ne crée pas de suite infinie.

On peut cependant utiliser la propriété du carreau pour établir la confluence
d'une autre relation.

\begin{lemma}\label{lem.carreau.confluence.clot}
  Soit $(A,\to)$ un ARS. S'il existe $\to'$ telle que
  \begin{itemize}
  \item $(\to')^\star = \to^\star$
  \item $\to'$ a la propriété du diamant
  \end{itemize}
  alors $\to$ est confluente.
\end{lemma}

\begin{proof}
  Puisque $\to'$ a la propriété du diamant, on sait que $\to'$ est confluente.
  Cela signifie que $(\to')^\star$ a la propriété du diamant, et par
  égalité que $\to^\star$ a la propriété du diamant, ce qui est équivalent à
  dire que $\to$ est confluente.
\end{proof}

Définissons aussi la propriété de Church-Rosser, qui est une propriété analogue
à la confluence mais qui ne se focalise pas sur un élément depuis lequel on
choisit deux réductions.

\begin{definition}[Propriété de Church-Rosser]
  Soit $(A,\to)$ un ARS. On dit que $\to$ a la propriété de Church-Rosser si
  \[\forall a,b \in A, a \xrightarrow{\sim} b \implies \exists c \in A,
  \begin{cases} a \to^\star c \\ b \to^\star c \end{cases}\]
\end{definition}

\begin{proposition}\label{prop.CR.confl}
  La propriété de Church-Rosser est équivalente à la confluence.
\end{proposition}

\begin{figure}[!ht]
  \centering
  \hrule

  \vspace{0.2cm}
  \begin{tikzpicture}[node distance = 2cm, on grid, auto]
    \node (a_0) {$a_0$};
    \node (a_1) [above right = of a_0] {$a_1$};
    \node (a_2) [below right = of a_1] {$a_2$};
    \node (bla) [right = of a_2] {$\cdots$};
    \node (a_n2) [right = of bla] {$a_{n-2}$};
    \node (a_n1) [above right = of a_n2] {$a_{n-1}$};
    \node (a_n) [below right = of a_n1] {$a_n$};
    \draw[->,>=latex] (a_1) -- (a_0);
    \draw[->,>=latex] (a_1) -- (a_2);
    \draw[->,>=latex] (a_n1) -- (a_n2);
    \draw[->,>=latex] (a_n1) -- (a_n);
    \node (b) [below left = of bla] {};
    \draw[->,>=latex,dashed] (a_0) -- (b);
    \draw[->,>=latex,dashed] (a_2) -- (b);
    \node (d) [below right = of b] {};
    \node (c) [below right = of bla] {};
    \draw[->,>=latex,dashed] (a_n2) -- (c);
    \draw[->,>=latex,dashed] (a_n) -- (c);
    \draw[->,>=latex,dashed] (c) -- (d);
    \draw[->,>=latex,dashed] (b) -- (d);
  \end{tikzpicture}

  \vspace{0.2cm}
  \hrule
  
  \vspace{0.4cm}
  \caption{Preuve de la \cref{prop.CR.confl}}
  \label{fig.CR.confl}
  \hrule

\end{figure}

\begin{proof}
  Si $a \to^\star b$ et $a\to^\star c$, alors $a \xrightarrow\sim b$, donc la
  propriété de Church-Rosser implique la confluence.

  Réciproquement, soit $a \xrightarrow\sim b$. On sait qu'il existe une
  suite d'éléments $a_0 = a$, $a_n = b$ et où pour tout $i$,
  $a_i \to^\star a_{i+1}$ ou $a_{i+1} \to^\star a_i$. On procède par induction
  sur la valeur de $n$. On suppose donc, par hypothèse d'induction, que
  $a_{n-1} \to^\star c$ et $a\to^\star c$, et on considère la dernière réduction
  $a_{n-1} (\to^\star)^\leftrightarrow a_n$. Deux cas sont possibles~:
  \begin{itemize}
  \item si $a_n \to^\star a_{n-1}$, alors $a_n \to^\star c$ d'où le résultat.
  \item si $a_{n-1} \to a_n$, alors on sait que $a_{n-1} \to^\star c$ et que
    $a_{n-1} \to^\star a_n$, donc par confluence on trouve un $d$ tel que
    $a_n \to^\star d$ et $c \to^\star d$, donc $a \to^\star d$ et
    $b \to^\star d$.
  \end{itemize}
  Donc, par récurrence, si $\to$ est confluente, alors $\to$ a la propriété de
  Church-Rosser.
\end{proof}

L'intérêt essentiel de la propriété de Church-Rosser est qu'une relation $\to$
qui la vérifie donne lieu à un algorithme simple pour tester l'égalité modulo
$\to$~: on réduit au maximum deux termes, et on vérifie s'ils donnt le même
résultat.

Pour que cela ait un sens, encore faut-il que la réduction aboutisse à un
résultat exploitable, ce qui demande qu'elle se termine. On s'intéresse donc
aux propriétés de terminaison. Tout d'abord, le cas le plus élémentaire de
terminaison est le cas d'un élément ne se réduisant pas.

\begin{definition}[Forme normale]
  Soit $(A,\to)$ un ARS. On dit qu'un élément $a \in A$ est une forme normale
  s'il vérifie
  \[\lnot(\exists b \in A, a\to b)\]
\end{definition}

\begin{exercise}
  Montrer que si $\to$ a la propriété de Church-Rosser, alors pour chaque
  $a \in A$, il existe au plus une forme normale $b$ telle que
  $a \to^\star b$.
\end{exercise}

\begin{notation}
  Lorsque $a \to^\star b$ et $b$ est une forme normale, on notera
  $a\to^! b$.
\end{notation}

Une façon pour une réduction de se terminer est alors d'aboutir à une forme
normale, ou plus généralement de ne s'effectuer qu'un nombre fini de fois. Cela
traduit l'idée de normalisation faible.

\begin{definition}[Normalisation faible]
  Soit $(A,\to)$ un ARS. On dit qu'un élément $a \in A$ est faiblement
  normalisant lorsqu'il existe une suite finie maximale de réduction depuis
  $a$, c'est-à-dire une suite de réductions depuis $a$ dont le dernier élément
  est une forme normale.

  On note $\WN$ l'ensemble des éléments faiblement normalisants. On dit que
  $A$ est faiblement normalisant si $\WN = A$.
\end{definition}

Naturellement, il existe une normalisation forte. Celle-ci traduit l'idée que
la relation $\to$ est bien fondée, c'est-à-dire sans suite infinie.

\begin{definition}[Normalisation forte]
  Soit $(A,\to)$ un ARS. On définit l'ensemble $\SN$ par la règle d'induction
  \begin{prooftree}
    \AxiomC{$\forall b, a \to b \implies b \in \SN$}
    \UnaryInfC{$a \in \SN$}
  \end{prooftree}
  un terme est fortement normalisant lorsqu'il appartient à $\SN$. On dit que
  $A$ est fortement normalisant si $\SN = A$.
\end{definition}

\begin{exercise}
  Montrer qu'une forme normale est fortement normalisante. Montrer qu'un élément
  fortement normalisant est faiblement normalisant.
\end{exercise}

\begin{exercise}
  Soit $(A,\to)$ un ARS tel que $\to$ vérifie la propriété de Church-Rosser,
  et soit $a \in A$ un élément faiblement normalisant.
  Peut-il exister une suite infinie de réductions depuis $a$ ? Si oui, donner
  un exemple.
\end{exercise}

Dans le cas de la normalisation forte, les propriétés de confluence locale et
de confluence sont équivalentes.

\begin{lemma}[Newman]
  Soit $(A,\to)$ un ARS fortement normalisant et localement confluent. Il est
  alors confluent.
\end{lemma}

\begin{proof}
  On raisonne par induction bien fondée sur $\to$~: cette induction est possible
  car $A$ est fortement normalisant, donc l'induction définissant $\SN$
  s'applique à tout élément de $A$.

  On choisit un élément $a \in A$, dont on suppose que $\to$ est confluente
  pour tout $b$ tel que $a \to^+ b$. Soient $b,c \in A$ tels que
  $a \to^\star b, a \to^\star c$. Si $b = a$ ou $c = a$, alors on vérifie
  directement la confluence pour cette instance.

  On suppose donc que $a \to b_0 \to^\star b$ et que $a \to c_0 \to^\star c$.
  Par confluence locale, on trouve $a_0$ tel que $b_0 \to^\star a_0$ et
  $c_0\to^\star a_0$. On peut alors fermer le carré par hypothèse d'induction
  appliquée sur $b_0 \to^\star a_0, b_0\to^\star b$, sur
  $c_0\to^\star a_0,c_0\to^\star c$ puis, en notant $d_0,d_1$ les deux éléments
  obtenus, sur $a_0 \to^\star d_0, a_0\to^\star d_1$, pour obtenir un élément
  $d$ tel que $b \to^\star d$ et $c\to^\star d$.
\end{proof}

Dans le cas des arbres de preuves de $\NJ$, on a les deux propriétés de
confluence et de terminaison.

\begin{theorem}
  L'ARS $(\ProofNJ,\reecr{})$ est fortement normalisant et confluent.
  L'égalité d'arbres de preuves est donc décidable.
\end{theorem}

\begin{proof}
  Reportée au CHAPITRE LAMBDA CALCUL TYPE.
\end{proof}
