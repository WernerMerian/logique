\chapter{Déduction naturelle}
\label{chp.deduc.nat}

\minitoc

\lettrine{D}{ans} son article présentant le calcul des séquents
\cite{Gentzen1935}, Gentzen présente aussi un deuxième formalisme de
démonstration~: la déduction naturelle. C'est ce système qui a été présenté dans
le \cref{chp.logpred}, mais nous allons ici étudier ses aspects structurels.

Comme le relavait déjà Gentzen dans son article original, la déduction naturelle
comporte deux variantes que sont $\NK$ et $\NJ$, respectivement la déduction
naturelle classique et intuitionniste, mais la première est en fait un
formalisme, en un sens, maladroit.

Le chapitre présente donc les deux variantes, mais étudie principalement la
variante intuitionniste. La variante classique peut en fait être construire de
façon plus naturelle, mais ces considérations dépassent le cadre d'une
introduction à la théorie de la démonstration.

Nous donnons ensuite une version adaptée à la déduction naturelle de
l'élimination des coupures, dans laquelle on introduit les notions essentielles
de système de réécriture abstrait. La preuve de la forte normalisation des
arbres de preuve de $\NJ$ est, elle, reportée au CHAPITRE LAMBDA CALCUL TYPE.

\section{Variantes de la déduction naturelle}

\subsection{Déduction naturelle intuitionniste et classique}

On a déjà vu une première présentation de la déduction naturelle dans le
\cref{chp.logpred}. Dans cette présentation, les points importants sont les
suivants~:
\begin{itemize}
\item les séquents des arbres de preuve ressemblent à des séquents
  intuitionnistes, mais contiennent exactement une formule à droite (aucune
  formule à droite dans un séquent intuitionniste pouvant être simulé par la
  formule $\bot$)~;
\item les règles ne sont plus des variantes gauche/droite mais des règles dites
  d'introduction et d'élimination, la première permettant d'introduire un
  constructeur à droite et la seconde permettant d'utiliser une formule du
  constructeur.
\end{itemize}

La première présentation de Gentzen du calcul en est proche, mais n'utilise pas
de séquent. Dans le calcul des séquents, seules existent des formules dans des
arbres, et les hypothèses d'une preuve peuvent être récupérées en considérant
les feuilles de l'arbre. Le second point, lui, reste identique dans cette
présentation.

\begin{table}[t!]
  \centering
  \begin{tabular}{cc}
    \bottomAlignProof
    \AxiomC{$\varphi$}
    \DisplayProof
    &
    \bottomAlignProof
    \AxiomC{$\bot$}
    \RightLabel{$\bot_\mathrm e$}
    \UnaryInfC{$\varphi$}
    \DisplayProof
    \\
    \\
    \bottomAlignProof
    \AxiomC{$[\varphi]^\alpha$}
    \noLine
    \UnaryInfC{$\vdots$}
    \noLine
    \UnaryInfC{$\bot$}
    \RightLabel{$\lnot_\mathrm i,\alpha$}
    \UnaryInfC{$\lnot\varphi$}
    \DisplayProof
    &
    \bottomAlignProof
    \AxiomC{$\lnot \varphi$}
    \AxiomC{$\varphi$}
    \RightLabel{$\lnot_\mathrm e$}
    \BinaryInfC{$\bot$}
    \DisplayProof
    \\
    \\
    \bottomAlignProof
    \AxiomC{$\varphi$}
    \RightLabel{$\lor_\mathrm i^1$}
    \UnaryInfC{$\varphi\lor\psi$}
    \DisplayProof
    \quad
    \bottomAlignProof
    \AxiomC{$\psi$}
    \RightLabel{$\lor_\mathrm i^2$}
    \UnaryInfC{$\varphi\lor\psi$}
    \DisplayProof
    &
    \bottomAlignProof
    \AxiomC{$\varphi\lor\psi$}
    \AxiomC{$[\varphi]^\alpha$}
    \noLine
    \UnaryInfC{$\vdots$}
    \noLine
    \UnaryInfC{$\chi$}
    \AxiomC{$[\psi]^\alpha$}
    \noLine
    \UnaryInfC{$\vdots$}
    \noLine
    \UnaryInfC{$\chi$}
    \RightLabel{$\lor_\mathrm e,\alpha$}
    \TrinaryInfC{$\chi$}
    \DisplayProof
    \\
    \\
    \bottomAlignProof
    \AxiomC{$\varphi$}
    \AxiomC{$\psi$}
    \RightLabel{$\land_\mathrm i$}
    \BinaryInfC{$\varphi\land\psi$}
    \DisplayProof
    &
    \bottomAlignProof
    \AxiomC{$\varphi\land\psi$}
    \RightLabel{$\land_\mathrm e^1$}
    \UnaryInfC{$\varphi$}
    \DisplayProof
    \quad
    \bottomAlignProof
    \AxiomC{$\varphi\land\psi$}
    \RightLabel{$\land_\mathrm e^2$}
    \UnaryInfC{$\psi$}
    \DisplayProof
    \\
    \\
    \bottomAlignProof
    \AxiomC{$[\varphi]^\alpha$}
    \noLine
    \UnaryInfC{$\vdots$}
    \noLine
    \UnaryInfC{$\psi$}
    \RightLabel{$to_\mathrm i,\alpha$}
    \UnaryInfC{$\varphi\to\psi$}
    \DisplayProof
    &
    \bottomAlignProof
    \AxiomC{$\varphi\to\psi$}
    \AxiomC{$\psi$}
    \RightLabel{$\to_\mathrm e$}
    \BinaryInfC{$\psi$}
    \DisplayProof
    \\
    \\
    \bottomAlignProof
    \AxiomC{$\varphi$}
    \RightLabel{$\forall_\mathrm i^\dagger$}
    \UnaryInfC{$\forall x, \varphi$}
    \DisplayProof
    &
    \bottomAlignProof
    \AxiomC{$\forall x, \varphi$}
    \RightLabel{$\forall_\mathrm e$}
    \UnaryInfC{$\varphi[t/x]$}
    \DisplayProof
    \\
    \\
    \bottomAlignProof
    \AxiomC{$\varphi[t/x]$}
    \RightLabel{$\exists_\mathrm i$}
    \UnaryInfC{$\exists x, \varphi$}
    \DisplayProof
    &
    \bottomAlignProof
    \AxiomC{$\exists x, \varphi$}
    \AxiomC{$[\varphi]^\alpha$}
    \noLine
    \UnaryInfC{$\vdots$}
    \noLine
    \UnaryInfC{$\psi$}
    \RightLabel{$\exists_\mathrm e^\dagger,\alpha$}
    \BinaryInfC{$\psi$}
    \DisplayProof
    \\
    \\
    \bottomAlignProof
    \AxiomC{}
    \RightLabel{$=_\mathrm i$}
    \UnaryInfC{$t=t$}
    \DisplayProof
    &
    \bottomAlignProof
    \AxiomC{$t = u$}
    \AxiomC{$\varphi[t/x]$}
    \RightLabel{$=_\mathrm e$}
    \BinaryInfC{$\varphi[u/x]$}
    \DisplayProof
  \end{tabular}
  
  \vspace{0.2cm}
  \hrule
  \caption{Règles de $\NJ$}
  \label{tbl.NJ}
  \hrule
\end{table}

On présente d'abord la variante intuitionniste, car $\NK$ en est un système
dérivé dans lequel on ajoute une règle.

\begin{definition}[Déduction naturelle $\NJ$]
  On définit l'ensemble $\ProofNJ$ des arbres de preuves de $\NJ$ par les
  règles de formation données dans la \cref{tbl.NJ}. Dans un arbre de preuves,
  les formules sur les feuilles de l'arbres sont de deux sortes~:
  \begin{itemize}
  \item les formules marquées, de la forme $[\varphi]^\alpha$ où $\alpha$ est
    appelé une marque~;
  \item les formules non marquées.
  \end{itemize}
  Un arbre $\pi\in \ProofNJ$ a pour conclusion le séquent $\Gamma\vdNJ \varphi$
  si la formule à sa racine est $\varphi$ et si ses feuilles non marquées sont
  des formules de $\Gamma$.

  Les règles avec la condition $\dagger$ demandent que la variable libre $x$
  n'apparaisse pas dans les feuilles non marquées de l'arbre, ou les feuilles
  avec une marque liée à une règle plus bas dans l'arbre que l'instance liant
  $x$.

  On considère uniquement les arbres dont toutes les marques sont liées par
  des règles, c'est-à-dire que pour une formule marquée $[\varphi]^\alpha$
  dans un arbre $\pi$, il existe une règle $\mathrm r_\alpha$ dans $\pi$.

  La relation de prouvabilité dans une théorie $\mathcal T$ est donnée par
  \[\mathcal T \vdNJ \varphi \defeq \exists \Gamma \in \List(\mathcal T),
  \exists \pi \in \ProofNJ, \pi\concl \Gamma\vdNJ\varphi\]
\end{definition}

\begin{remark}
  Dans les règles introduisant les marques telles que $\to_\mathrm i,\alpha$,
  il n'existe pas forcément de formule marquée par $\alpha$, et il peut y en
  avoir plusieurs marquées par le même $\alpha$. L'intérêt de ces marques est
  de supprimer du contexte global des hypothèses en ne les comptant plus comme
  des feuilles d'hypothèses (c'est pour cela qu'on ne considère que les feuilles
  non marquées).
\end{remark}

La déduction naturelle $\NK$ est simplement l'ajout d'une règle de logique
classique. Il existe plusieurs axiomes ou règles qui permettent de passer de la
logique intuitionniste à la logique classique~: l'\cref{exo.EDN.RAA} consiste
par exemple à montrer qu'il est équivalent de considérer l'ajout des axiomes
$\varphi\lor\lnot\varphi$ ou l'ajout des axiomes $\lnot\lnot\varphi\to\varphi$,
les deux pouvant se réécrire comme des règles. On choisit alors le plus simple
en prenant le tiers exclu comme règle pour $\NK$.

\begin{definition}[Déduction naturelle $\NK$]
  L'ensemble $\ProofNK$ des arbres de preuve de déduction naturelle classique
  est donné par les règles de la \cref{tbl.NJ} ainsi que par la règle
  \begin{prooftree}
    \AxiomC{}
    \RightLabel{EM}
    \UnaryInfC{$\varphi\lor\lnot\varphi$}
  \end{prooftree}
  La relation $\vdNK$ est définie de façon analogue à $\vdNJ$.
\end{definition}

\begin{remark}
  \'Etant donné un arbre $\pi\concl \psi_1,\cdots,\psi_n\vdNJ \varphi$ et
  un arbre
  $\pi'\concl \Delta\vdNJ\psi_i$, on peut placer l'arbre $\pi'$ au-dessus des
  feuilles de $\pi$ étiquetées par $\psi_i$ pour obtenir un arbre
  \[\pi \circ_i \pi' \concl \psi_1,\ldots,\psi_{i-1},\Delta,
  \psi_{i+1},\ldots,\psi_n\vdNJ\varphi\]
  Le lecteur ayant vu la notion d'opérade notera des similarités avec ces
  objets.
\end{remark}

\begin{example}
  Donnons un exemple d'arbre de preuve dans $\NJ$. On reprend l'exemple donné
  dans le \cref{chp.logpred}, pour mettre en valeur les différences de style
  avec la première version de déduction naturelle donnée.
  On fixe la signature
  $\Sigma \defeq \{f^1,R^1\}$ où $R$ est un symbole de relation et $f$ un
  symbole de fonction. On veut prouver que
  $\forall x, R(x)\to R(f(x))\vdNJ \forall x, R(x)\to R(f(f(x)))$~:
  \begin{prooftree}
    \AxiomC{$\forall x, R(x)\to R(f(x))$}
    \RightLabel{$\forall_\mathrm e$}
    \UnaryInfC{$R(f(x))\to R(f(f(x)))$}
    \AxiomC{$\forall x, R(x)\to R(f(x))$}
    \RightLabel{$\forall_\mathrm e$}
    \UnaryInfC{$R(x) \to R(f(x))$}
    \AxiomC{$[R(x)]^\alpha$}
    \RightLabel{$\to_\mathrm e$}
    \BinaryInfC{$R(f(x))$}
    \RightLabel{$\to_\mathrm e$}
    \BinaryInfC{$R(f(f(x)))$}
    \RightLabel{$\to_\mathrm i,\alpha$}
    \UnaryInfC{$R(x) \to R(f(f(x)))$}
    \RightLabel{$\forall_\mathrm i^\dagger$}
    \UnaryInfC{$\forall x, R(x)\to R(f(f(x)))$}
  \end{prooftree}
\end{example}

Puisque nous avons donné une nouvelle syntaxe de preuve, à la fois pour la
logique classique et la logique intuitionniste, il convient de montrer la
complétude de ces calculs.

\subsection{Complétude de la déduction naturelle}

Dans le cas de la logique classique, on peut simplement appliquer le théorème de
complétude.

\begin{exercise}
  Montrer que $\NK$ vérifie les hypothèse du \cref{thm.completude.gen}. En
  déduire que $\vdNK = \vDash$.
\end{exercise}

Pour la logique intuitionniste, on peut tenter de construire un théorème de
complétude directement inspiré de celui utilisé pour prouver la complétude de
$\LJ$ pour la sémantique de Kripke. Si cette méthode fonctionne, on propose de
donner une preuve plus algorithmique de l'équivalence entre $\LJ$ et $\NJ$.

\begin{definition}[Construction d'arbre depuis le calcul des séquents]
  On définit une fonction $\tradNJ - : \ProofLJ \to \ProofNJ$ par induction sur
  les arbres, de telle sorte que si $\pi\concl \Gamma\vdLJ \varphi$ alors
  $\tradNJ\pi\concl\Gamma\vdNJ\varphi$ et si $\pi\concl\Gamma\vdLJ$ alors
  $\tradNJ\pi\concl\Gamma\vdNJ\bot$~:
  \begin{itemize}
  \item si $\pi = \bottomAlignProof\AxiomC{}\RightLabel{$ax$}
    \UnaryInfC{$\varphi\vdLJ\varphi$}\DisplayProof$
    alors $\tradNJ\pi \defeq \varphi$.
  \item si $\pi = \bottomAlignProof
    \AxiomC{$\pi_0$}\noLine\UnaryInfC{$\Gamma\vdLJ\varphi$}
    \AxiomC{$\pi_1$}\noLine\UnaryInfC{$\Delta,\varphi\vdLJ[\psi]$}
    \RightLabel{$cut$}\BinaryInfC{$\Gamma,\Delta\vdLJ[\psi]$}
    \DisplayProof$
    alors $\tradNJ\pi \defeq \tradNJ{\pi_1}\circ_i\tradNJ{\pi_0}$ où
    $i$ est la position de $\varphi$ dans $\Delta,\varphi$.
  \item pour les règles structurelles sauf $rw$, la traduction ne change pas
    l'arbre.
  \item si $\pi = \bottomAlignProof
    \AxiomC{$\pi_0$}\noLine\UnaryInfC{$\Gamma\vdLJ$}
    \RightLabel{$rw$}\UnaryInfC{$\Gamma\vdLJ\varphi$}\DisplayProof$
    alors \[\tradNJ\pi \defeq
    \bottomAlignProof
    \AxiomC{$\tradNJ{\pi_0}$}
    \noLine\UnaryInfC{$\vdots$}\noLine\UnaryInfC{$\bot$}
    \RightLabel{$\bot_\mathrm e$}
    \UnaryInfC{$\varphi$}
    \DisplayProof\]
  \item si $\pi = \bottomAlignProof
    \AxiomC{$\pi_0$}\noLine\UnaryInfC{$\vdots$}\noLine
    \UnaryInfC{$\Gamma\vdLJ\varphi$}
    \RightLabel{$l\lnot$}\UnaryInfC{$\Gamma,\lnot\varphi\vdLJ$}\DisplayProof$
    alors
    \[\tradNJ{\pi} \defeq \bottomAlignProof
    \AxiomC{$\lnot\varphi$}
    \AxiomC{$\tradNJ{\pi_0}$}\noLine\UnaryInfC{$\varphi$}
    \RightLabel{$\lnot_\mathrm e$}\BinaryInfC{$\bot$}
    \DisplayProof\]
  \item si $\pi = \bottomAlignProof
    \AxiomC{$\pi_0$}\noLine\UnaryInfC{$\Gamma,\varphi\vdLJ$}
    \RightLabel{$r\lnot$}\UnaryInfC{$\Gamma\vdLJ\lnot\varphi$}
    \DisplayProof$, alors
    \[\tradNJ\pi \defeq \bottomAlignProof
    \AxiomC{$[\varphi]^\alpha$}\noLine\UnaryInfC{$\vdots$}\noLine
    \UnaryInfC{$\tradNJ{\pi_0}$}\noLine\UnaryInfC{$\vdots$}\noLine
    \UnaryInfC{$\bot$}
    \RightLabel{$\lnot_\mathrm i,\alpha$}\UnaryInfC{$\lnot\varphi$}
    \DisplayProof\]
  \item si $\pi = \bottomAlignProof
    \AxiomC{$\pi_0$}\noLine\UnaryInfC{$\Gamma,\varphi\vdLJ[\chi]$}
    \AxiomC{$\pi_1$}\noLine\UnaryInfC{$\Gamma,\psi\vdLJ[\chi]$}
    \RightLabel{$l\lor$}\BinaryInfC{$\Gamma,\varphi\lor\psi\vdLJ[\chi]$}
    \DisplayProof$, alors
    \[\tradNJ\pi\defeq \bottomAlignProof
    \AxiomC{$\varphi\lor\psi$}
    \AxiomC{$[\varphi]^\alpha$}\noLine\UnaryInfC{$\vdots$}\noLine
    \UnaryInfC{$\tradNJ{\pi_0}$}\noLine\UnaryInfC{$\vdots$}\noLine
    \UnaryInfC{$[\chi]$}
    \AxiomC{$[\psi]^\alpha$}\noLine\UnaryInfC{$\vdots$}\noLine
    \UnaryInfC{$\tradNJ{\pi_1}$}\noLine\UnaryInfC{$\vdots$}\noLine
    \UnaryInfC{$[\chi]$}
    \RightLabel{$\lor_\mathrm e,\alpha$}\TrinaryInfC{$[\chi]$}
    \DisplayProof\]
  \item si $\pi = \bottomAlignProof
    \AxiomC{$\pi'$}\noLine\UnaryInfC{$\Gamma\vdLJ\varphi_i$}
    \RightLabel{$r\lor_i$}\UnaryInfC{$\Gamma\vdLJ\varphi_1\lor\varphi_2$}
    \DisplayProof$, alors
    \[\tradNJ\pi\defeq\bottomAlignProof
    \AxiomC{$\tradNJ{\pi'}$}\noLine\UnaryInfC{$\vdots$}\noLine
    \UnaryInfC{$\varphi_i$}
    \RightLabel{$\lor_\mathrm i^i$}\UnaryInfC{$\varphi_1\lor\varphi_2$}
    \DisplayProof\]
  \item si $\pi = \bottomAlignProof
    \AxiomC{$\pi'$}\noLine\UnaryInfC{$\Gamma,\varphi_i\vdLJ [\psi]$}
    \RightLabel{$l\land_i$}
    \UnaryInfC{$\Gamma,\varphi_1\land\varphi_2\vdLJ[\psi]$}\DisplayProof$,
    alors
    \[\tradNJ\pi\defeq\bottomAlignProof
    \AxiomC{$\varphi_1\land\varphi_2$}\RightLabel{$\land_\mathrm e^i$}
    \UnaryInfC{$\varphi_i$}\noLine\UnaryInfC{$\vdots$}\noLine
    \UnaryInfC{$\tradNJ{\pi'}$}\noLine\UnaryInfC{$\vdots$}\noLine
    \UnaryInfC{$[\psi]$}\DisplayProof\]
  \item si $\pi = \bottomAlignProof
    \AxiomC{$\pi_0$}\noLine\UnaryInfC{$\Gamma\vdLJ\varphi$}
    \AxiomC{$\pi_1$}\noLine\UnaryInfC{$\Gamma\vdLJ\psi$}
    \RightLabel{$r\land$}\BinaryInfC{$\Gamma\vdLJ\varphi\land\psi$}
    \DisplayProof$, alors
    \[\tradNJ\pi\defeq \bottomAlignProof
    \AxiomC{$\tradNJ{\pi_0}$}
    \noLine\UnaryInfC{$\vdots$}\noLine\UnaryInfC{$\varphi$}
    \AxiomC{$\tradNJ{\pi_1}$}\noLine\UnaryInfC{$\vdots$}\noLine
    \UnaryInfC{$\psi$}
    \RightLabel{$\land_\mathrm i$}\BinaryInfC{$\varphi\land\psi$}\DisplayProof\]
  \item si $\pi = \bottomAlignProof
    \AxiomC{$\pi_0$}\noLine\UnaryInfC{$\Gamma\vdLJ\varphi$}
    \AxiomC{$\pi_1$}\noLine\UnaryInfC{$\Gamma,\psi\vdLJ[\chi]$}
    \RightLabel{$l\to$}\BinaryInfC{$\Gamma,\varphi\to\psi\vdLJ[\chi]$}
    \DisplayProof$, alors
    \[\tradNJ\pi\defeq\bottomAlignProof
    \AxiomC{$\varphi\to\psi$}
    \AxiomC{$\tradNJ{\pi_0}$}\noLine\UnaryInfC{$\vdots$}\noLine
    \UnaryInfC{$\varphi$}
    \RightLabel{$\to_\mathrm e$}
    \BinaryInfC{$\psi$}
    \noLine\UnaryInfC{$\vdots$}\noLine
    \UnaryInfC{$\tradNJ{\pi_1}$}
    \noLine\UnaryInfC{$\vdots$}\noLine\UnaryInfC{$[\chi]$}
    \DisplayProof\]
  \item si $\pi = \bottomAlignProof
    \AxiomC{$\pi'$}\noLine\UnaryInfC{$\Gamma,\varphi\vdLJ\psi$}
    \RightLabel{$r\to$}\UnaryInfC{$\Gamma\vdLJ\varphi\to\psi$}
    \DisplayProof$, alors
    \[\tradNJ\pi\defeq\bottomAlignProof
    \AxiomC{$[\varphi]^\alpha$}\noLine\UnaryInfC{$\vdots$}\noLine
    \UnaryInfC{$\tradNJ{\pi'}$}\noLine\UnaryInfC{$\vdots$}\noLine
    \UnaryInfC{$\psi$}
    \RightLabel{$\to_\mathrm i,\alpha$}\UnaryInfC{$\varphi\to\psi$}
    \DisplayProof\]
  \item si $\pi = \bottomAlignProof
    \AxiomC{$\pi'$}\noLine\UnaryInfC{$\Gamma,\varphi[t/x]\vdLJ[\psi]$}
    \RightLabel{$l\forall$}\UnaryInfC{$\Gamma,\forall x,\varphi\vdLJ[\psi]$}
    \DisplayProof$, alors
    \[\tradNJ\pi\defeq\bottomAlignProof
    \AxiomC{$\forall x, \varphi$}\RightLabel{$\forall_\mathrm e$}
    \UnaryInfC{$\varphi[t/x]$}\noLine\UnaryInfC{$\vdots$}\noLine
    \UnaryInfC{$\tradNJ{\pi'}$}\noLine\UnaryInfC{$\vdots$}\noLine
    \UnaryInfC{$[\psi]$}
    \DisplayProof\]
  \item si $\pi = \bottomAlignProof
    \AxiomC{$\pi'$}\noLine\UnaryInfC{$\Gamma\vdLJ\varphi$}
    \RightLabel{$r\forall^\dagger$}\UnaryInfC{$\Gamma\vdLJ\forall x, \varphi$}
    \DisplayProof$, alors
    \[\tradNJ\pi\defeq\bottomAlignProof
    \AxiomC{$\tradNJ{\pi'}$}\noLine\UnaryInfC{$\vdots$}\noLine
    \UnaryInfC{$\varphi$}
    \RightLabel{$\forall_\mathrm i^\dagger$}\UnaryInfC{$\forall x, \varphi$}
    \DisplayProof\]
  \item si $\pi = \bottomAlignProof
    \AxiomC{$\pi'$}\noLine\UnaryInfC{$\Gamma,\varphi\vdLJ[\psi]$}
    \RightLabel{$l\exists^\dagger$}
    \UnaryInfC{$\Gamma,\exists x, \varphi\vdLJ[\psi]$}
    \DisplayProof$, alors
    \[\tradNJ\pi\defeq\bottomAlignProof
    \AxiomC{$\exists x, \varphi$}
    \AxiomC{$[\varphi]^\alpha$}\noLine\UnaryInfC{$\vdots$}\noLine
    \UnaryInfC{$\tradNJ{\pi'}$}\noLine\UnaryInfC{$\vdots$}\noLine
    \UnaryInfC{$[\psi]$}\RightLabel{$\exists_\mathrm e^\dagger$}
    \BinaryInfC{$[\psi]$}\DisplayProof\]
  \item si $\pi = \bottomAlignProof
    \AxiomC{$\pi'$}\noLine\UnaryInfC{$\Gamma\vdLJ\varphi[t/x]$}
    \RightLabel{$r\exists$}\UnaryInfC{$\Gamma\vdLJ\exists x, \varphi$}
    \DisplayProof$, alors
    \[\tradNJ\pi\defeq \bottomAlignProof
    \AxiomC{$\tradNJ{\pi'}$}\noLine\UnaryInfC{$\vdots$}\noLine
    \UnaryInfC{$\varphi[t/x]$}
    \RightLabel{$\exists_\mathrm i$}\UnaryInfC{$\exists x, \varphi$}
    \DisplayProof\]
  \item si $\pi = \bottomAlignProof
    \AxiomC{$\pi'$}\noLine
    \UnaryInfC{$\psi_1[u/x],\ldots,\psi_n[u/x]\vdLJ[\varphi[u/x]]$}
    \RightLabel{$l=$}
    \UnaryInfC{$\psi_1[t/x],\ldots,\psi_n[t/x], t = u \vdLJ[\varphi[t/x]]$}
    \DisplayProof$, alors
    \[\tradNJ\pi\defeq\bottomAlignProof
    \AxiomC{$t = u$}
    \AxiomC{}
    \RightLabel{$=_\mathrm i$}
    \UnaryInfC{$t = t$}
    \RightLabel{$=_\mathrm e$}
    \BinaryInfC{$u = t$}
    \AxiomC{$t = u$}
    \AxiomC{$\psi_1[t/x]$}
    \RightLabel{$=_\mathrm e$}
    \BinaryInfC{$\psi_1[u/x]$}
    \AxiomC{$\cdots$}
    \AxiomC{$t = u$}
    \AxiomC{$\psi_n[t/x]$}
    \RightLabel{$=_\mathrm e$}
    \BinaryInfC{$\psi_n[u/x]$}
    \noLine\TrinaryInfC{$\tradNJ{\pi'}$}\noLine
    \UnaryInfC{$[\varphi[u/x]]$}
    \RightLabel{$=_\mathrm e$}
    \BinaryInfC{$[\varphi[t/x]]$}
    \DisplayProof
    \]
  \item si $\pi = \bottomAlignProof\AxiomC{}\RightLabel{$r=$}
    \UnaryInfC{$t =t$}\DisplayProof$, alors
    $\tradNJ{\pi}\defeq \AxiomC{}\RightLabel{$=_\mathrm i$}
    \UnaryInfC{$t = t$}\DisplayProof$.
  \end{itemize}
\end{definition}

\begin{corollary}
  Un séquent intuitionniste est prouvable dans $\NJ$ si et seulement s'il est
  vrai dans tout modèle de Kripke intuitionniste.
\end{corollary}

\begin{proof}
  On vérifie par une induction sans difficulté que le calcul $\NJ$ est correct.
  Si un séquent est vrai dans tout modèle de Kripke intuitionniste, alors il
  est prouvable dans $\LJ$ par complétude, mais pour $\pi$ la preuve ainsi
  obtenue, $\tradNJ\pi$ est un arbre de preuve du même séquent dans $\NJ$.
\end{proof}
