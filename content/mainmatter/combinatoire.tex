\chapter{Combinatoire dans les ordinaux}
\label{chp.combi}

\minitoc

Dans ce chapitre, nous étudions les notions combinatoires liées aux ordinaux.
L'objectif du chapitre est d'aborder les ensembles stationnaires et les clubs,
en en voyant les principales propriétés.

Nous verrons avant tout ce qu'est un ensemble stationnaire, dont l'utilisation
est omniprésente en théorie des ensembles modernes. Nous donnerons ensuite le
lemme de Fodor.

\section{Clubs et ensembles stationnaires}

Pour pouvoir étudier les ensembles stationnaires, il est nécessaire de définir
avant les clubs, qui sont une notion liée à la topologie des ordinaux, que nous
abordons donc en premier lieu.

\subsection{Topologie des ordinaux}

A FAIRE

\subsection{Filtre des clubs}

On a vu dans le chapitre précédent qu'on pouvait considérer des parties
cofinales dans un ordinal $\alpha$. \'Etant données deux parties cofinales,
il est possible de différencier celles qui sont \og suffisament pleines\fg dans
$\alpha$ de celles qui ont de nombreux trous (en un certain sens). On introduit
donc une nouvelle notion, combinant la cofinalité (s'étendre assez haut dans
l'ordinal $\alpha$) et la topologie (être assez non dispersé) pour définir ce
qu'est un club.

\begin{definition}[Club]
  Soit $\alpha$ un ordinal. Un club (abréviation de l'anglais
  \emph{closed unbounded}) est une partie $C\subseteq \alpha$ qui est close dans
  $\alpha$ muni de la topologie de l'ordre, et cofinale dans $\alpha$.
\end{definition}

Un club est donc une grosse partie de $\alpha$. En particulier, en suivant
l'intuition que l'on a donné dans le \cref{chp.ordres}, une telle notion de
grosse partie devrait former un filtre. On peut imaginer une partie contenant
un club mais n'étant pas un club, donc on voudra plutôt considérer le filtre
engendré par cette famille de parties.

Cependant, pour construire ce filtre, il est nécessaire (pour que le filtre ne
soit pas trivial) que les clubs soient stables par intersection finie. En
l'occurrence, ça n'est pas le cas dans le cas général, mais c'est le cas à
partir du moment où la cofinalité de $\alpha$ est indénombrable.

\begin{proposition}
  Soit $\alpha$ un ordinal infini. Alors $\cof(\alpha) > \aleph_0$ si et
  seulement si l'intersection de deux clubs est encore un club.
\end{proposition}

\begin{proof}
  Commençons par le cas où $\cof(\alpha) \leq \aleph_0$. Dans ce cas, on peut
  trouver une suite $(\alpha_n)$ strictement croissante cofinale dans $\alpha$.
  Soient alors $(\alpha_{2n})$ et $(\alpha_{2n+1})$ deux suites extraites. Les
  images $B$ et $C$, respectivement, de $(\alpha_{2n})$ et $(\alpha_{2n+1})$, sont
  cofinales dans $\alpha$ et disjointes. De plus, $B$ et $C$ sont des parties
  fermées de $\alpha$. On a donc trouvé deux parties disjointes de $\alpha$ qui
  sont des clubs.

  Supposons maintenant que $\cof(\alpha) > \aleph_0$. Soient alors $C_1$ et
  $C_2$ deux clubs de $\alpha$. Montrons que $C_1\cap C_2$ est encore un club~:
  \begin{itemize}
  \item c'est une intersection de fermés, donc c'est un fermé.
  \item soit $\beta \in \alpha$, montrons qu'il existe $\gamma \in C_1\cap C_2$
    tel que $\beta \leq \gamma$. Pour cela, on construit une suite $(\gamma_n)$
    d'éléments supérieurs à $\beta$~: comme $C_1$ est cofinal dans $\alpha$,
    on définit $\gamma_{2n+2}$ comme un élément supérieur à $\gamma_{2n+1}$ qui
    est dans $C_1$ et $\gamma_{2n+1}$ comme un élément supérieur à
    $\gamma_{2n}$ qui est dans $C_2$, et on fixe $\gamma_0 = \beta$.

    Comme $\cof(\alpha) > \aleph_0$, $\sup \gamma_n \in \alpha$, donc comme
    $(\gamma_{2n+2})$ est une suite à valeurs dans $C_1$ qui est fermé,
    $\sup \gamma_n \in C_1$ et de même $\sup \gamma_n \in C_2$ car
    $(\gamma_{2n+1})$ est une suite à valeurs dans $C_2$.

    Ainsi $C_1\cap C_2$ est bien cofinal dans $\alpha$.
  \end{itemize}
  On en déduit donc que si $\cof(\alpha) > \aleph_0$, alors l'intersection de
  deux clubs est encore un club.
\end{proof}

Comme la condition porte sur la cofinalité, qui est une notion cardinale, on
parlera à partir de maintenant de cardinaux réguliers. Il est en général
possible de généraliser les résultats au cas d'un ordinal quelconque en
traduisant les conditions sur sa cofinalité, mais parler uniquement de cardinaux
réguliers simplifie les énoncés et suffit en pratique.

\begin{definition}[Filtre des clubs]
  Soit $\kappa \geq \aleph_1$ un cardinal régulier. On définit le filtre propre
  des clubs de $\kappa$ par
  \[\Club_\kappa \defeq \filclose{\{C\subseteq \kappa
    \mid C\text{ est un club}\}}\]
\end{definition}

\begin{proof}
  On a vérifie que l'intersection de deux clubs est encore un clubs, donc
  l'intersection d'un nombre fini de clubs est encore un club. En particulier,
  l'intersection d'un nombre fini de clubs est non vide, donc le filtre engendré
  par la famille des clubs de $\kappa$ est un filtre propre.
\end{proof}

Donnons d'abord une famille d'éléments de ce filtre.

\begin{property}
  Soit $\kappa$ un cardinal régulier indénombrable. L'ensemble $\Club_\kappa$
  contient toutes les parties de la forme
  $\{\beta \in\kappa\mid \alpha\leq\beta\}$ pour $\alpha \in \kappa$.
\end{property}

\begin{proof}
  Soit $\alpha \in \kappa$, montrons que $\{\beta\in\kappa\mid\alpha\leq\beta\}$
  est un club~:
  \begin{itemize}
  \item il est fermé car le complémentaire de l'ouvert $\alpha$.
  \item il est cofinal dans $\kappa$ puisque si $\alpha \leq \beta$, alors
    $\alpha \leq \beta + 1$.
  \end{itemize}
  $\Club_\kappa$ contient donc toutes les parties co-bornées.
\end{proof}

Définissons maintenant les parties non stationnaires et stationnaires.

\begin{definition}[Parties stationnaires, non stationnaires]
  Soit $\kappa$ un cardinal régulier indénombrable. Une partie
  $X\subseteq\kappa$ est dite non stationnaire si elle est le complémentaire
  d'un élément de $\Club_\kappa$. Elle est dite stationnaire si elle n'est pas
  non stationnaire.
\end{definition}

De la même façon que les parties de $\mathbb R$ de complémentaire négligeable
forment un filtre dont le complémentaire forme l'idéal des parties négligeables
(de mesure nulle), on a ici les parties non stationnaires qui sont négligeables
en un sens. Une partie stationnaire est alors l'analogue d'une partie de mesure
non nulle dans $\mathbb R$. On donne une caractérisation des parties
stationnaires, qui peut largement être prise comme définition.

\begin{proposition}
  Une partie $X$ est stationnaire si et seulement si elle intersecte tous les
  clubs de $\kappa$.
\end{proposition}

\begin{proof}
  $X$ est non stationnaire si et seulement s'il $\kappa\setminus X$ contient
  un club $C$, ce qui est équivalent à dire que $X$ est disjoint avec au moins
  un club de $\kappa$~: on en déduit l'équivalence annoncée.
\end{proof}

\subsection{Exemples importants}

Dans cette sous-section, nous traitons des cas élémentaires qu'il est important
de couvrir pour avoir une certaine intuition des clubs et ensembles
stationnaires.

Pour commencer, un exemple important de club est celui des points de fermeture,
que l'on va définir d'abord.

\begin{definition}[Point de fermeture]
  Soit $\alpha$ un ordinal et $f : \alpha \to \alpha$. On dit qu'un point
  $\beta \in \alpha$ est un point de fermeture de $f$ lorsque
  \[\forall \gamma < \beta, f(\gamma) < \beta\]
  ou, de façon équivalence, lorsque $f(\beta) \subseteq \beta$ en considérant
  l'image directe pour $f(\beta)$ (et non l'image de $\beta$). On notera
  $\Clos(f)$ l'ensemble des points de fermeture de $f$.
\end{definition}

\begin{proposition}
  Soit $\kappa$ un ordinal régulier indénombrable et $f : \kappa \to \kappa$,
  alors $\Clos(f)$ est un club de $\kappa$.
\end{proposition}

\begin{proof}
  On vérifie que $\Clos(f)$ est un club~:
  \begin{itemize}
  \item soit $X\subseteq \Clos(f)$, montrons que $\sup X \in \Clos(f)$. Si
    $\sup X \in X$ le résultat est évident. On suppose donc que $\sup X$ est un
    ordinal limite. Soit $\beta \in \sup X$, comme $\sup X$ est limite,
    on trouve $\alpha \in X$ tel que $\beta + 1 \leq \alpha$, et comme
    $\beta < \alpha$ et $\alpha \in \Clos(f)$, on en déduit que
    $f(\beta) < \alpha < \sup X$, donc pour tout
    $\beta \in \sup X, f(\beta) < \sup X$, d'où $X\in \Clos(f)$.
  \item de plus $\Clos(f)$ est cofinal dans $\kappa$. Soit $\beta\in\kappa$, on
    définit la suite
    \[\begin{cases}
    \alpha_0 \defeq \beta + 1 \\
    \alpha_{n+1} \defeq \displaystyle
    \Big(\sup_{\delta \leq \alpha_n} f(\delta)\Big) + 1
    \end{cases}\]
    Comme on a une suite dans un ordinal régulier indénombrable, on en déduit
    que $\sup \alpha_n \in \kappa$ et $\sup \alpha_n > \beta$. Il nous reste à
    montrer que $\sup \alpha_n \in \Clos(f)$. Soit $\gamma < \sup \alpha_n$, on
    trouve $n \in \mathbb N$ tel que $\beta \leq \alpha_n$, dont on déduit par
    définition de $\alpha_{n+ 1}$ que $f(\beta) < \alpha_{n+1}\leq \sup\alpha_n$,
    donc $\sup\alpha_n\in\Clos(f)$.
  \end{itemize}
\end{proof}

Dans cet exemple précis, il est essentiel que $\kappa$ soit régulier. En effet,
en prenant $f : \aleph_{\omega_1} \to \aleph_{\omega_1}$ tel que
$f(\alpha) = \aleph_\alpha$ pour $\alpha < \omega_1$ et $f(\alpha) = 0$ sinon,
on a un exemple de fonction sans point de fermeture.

Un autre exemple important de club est celui des limites. On peut le voir comme
un cas particulier de point de fermeture, pour la fonction $S$, mais cela ne
fonctionne que dans le cas d'un cardinal régulier.

\begin{definition}[Ensemble des limites]
  Soit $\alpha$ un ordinal, on définit l'ensemble des limites de $\alpha$ comme
  \[\Lim_\alpha \defeq
  \left\{\beta \in \alpha \mid \bigcup \beta = \beta\right\}\]
\end{definition}

\begin{proposition}
  Pour $\kappa$ un cardinal régulier indénombrable, $\Lim_\kappa$ est un club
  de $\kappa$.
\end{proposition}

\begin{proof}
  On vérifie que $\Lim_\kappa$ est un club~:
  \begin{itemize}
  \item Soit $X\subseteq \Lim_\kappa$ tel que $\sup X \in \kappa$, alors
    \begin{align*}
      \bigcup (\sup X) &= \bigcup_{\alpha \in \sup X} \alpha \\
      &= \bigcup_{\alpha \in \bigcup X} \alpha \\
      &= \bigcup_{\alpha \in X} \left(\bigcup \alpha\right)\\
      &= \bigcup_{\alpha \in X} \alpha\\
      &= \sup X
    \end{align*}
    donc $\sup X \in \Lim_\kappa$, donc $\Lim_\kappa$ est fermé.
  \item Soit $\alpha \in \kappa$, alors $\alpha + \omega$ est un ordinal limite
    et il est supérieur à $\alpha$, donc $\Lim_\kappa$ est non borné.
  \end{itemize}
  Donc $\Lim_\kappa$ est un club de $\kappa$.
\end{proof}

Parmi ces limites, dans un ordinal donné, on peut faire une distinction sur la
cofinalité. Dans le cas de $\aleph_2$ par exemple, les ordinaux limites peuvent
avoir une cofinalité de $\aleph_0$ ou de $\aleph_1$. Comme toute limite a l'une
des deux cofinalités, il est clair qu'au moins l'ensemble des ordinaux limites
de cofinalité $\aleph_0$ ou celui des ordinaux limites de cofinalité $\aleph_1$
est stationnaire, comme l'union des deux est un club. En réalité, ils sont
tous les deux stationnaires, et on peut généraliser ce résultat, en utilisant
la notation $E_\kappa^\lambda$ introduite dans la \cref{sct.cof}.

\begin{proposition}
  Soient $\kappa$ et $\lambda$ deux cardinaux réguliers infinis tels que
  $\kappa < \lambda$. Alors l'ensemble $E_\kappa^\lambda$ est stationnaire.
\end{proposition}

\begin{proof}
  On va montrer que tout club contient un élément de cofinalité $\kappa$. Soit
  $C$ un club de $\lambda$. On définit $f : \lambda \to C$ par récursion
  transfinie~:
  \begin{itemize}
  \item $f(0) = \min C$
  \item pour tout $\alpha \in \lambda$,
    $f(\alpha + 1) = \min C\setminus f(\alpha)$
  \item pour tout $\alpha \in \Lim_\lambda$,
    $f(\alpha) = \sup_{\beta < \alpha} f(\beta)$
  \end{itemize}

  $f$ est bien définie. En effet, pour $\alpha \in \lambda$, comme $\lambda$ est
  régulier, $f\restr\alpha$ est bornée, et $C$ ne l'est pas donc on peut bien
  trouver un minimum dans $C\setminus f(\alpha)$. Dans le cas limite, comme $C$
  est un club, on en déduit que $\sup f(\beta)$ est effectivement un élément de
  $C$.

  En fait, on peut même montrer que $f$ est une bijection entre $\lambda$ et
  $C$, de façon analogue à la preuve que $\aleph$ établit une bijection entre
  les ordinaux et les cardinaux infinis.

  On voit alors que $f(\kappa)$ est un élément de cofinalité $\kappa$, donc que
  $C\cap E_\kappa^\lambda \neq \varnothing$. $E_\kappa^\lambda$ est donc bien
  stationnaire.
\end{proof}
