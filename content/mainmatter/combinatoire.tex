\chapter{Combinatoire dans les ordinaux}
\label{chp.combi}

\minitoc

\lettrine{D}{ans} ce chapitre, nous étudions les notions combinatoires liées aux
ordinaux. L'objectif du chapitre est d'aborder les ensembles stationnaires et
les clubs, en en voyant les principales propriétés.

Nous verrons avant tout ce qu'est un ensemble stationnaire, dont l'utilisation
est omniprésente en théorie des ensembles modernes. Nous donnerons ensuite le
lemme de Fodor.

\section{Clubs et ensembles stationnaires}

Pour pouvoir étudier les ensembles stationnaires, il est nécessaire de définir
avant les clubs, qui sont une notion liée à la topologie des ordinaux, que nous
abordons donc en premier lieu.

\subsection{Topologie des ordinaux}

La topologie des ordinaux commence par la notion plus générale de topologie de
l'ordre, qui peut se définir pour tout ensemble ordonné.

\begin{definition}[Topologie de l'ordre]
  Soit $(X,\leq)$ un ensemble ordonné. On définit, pour tous $a,b\in X$, le
  segment ouvert
  \[(a,b) \defeq \{x \in X \mid a < x < b\}\]
  La topologie de l'ordre sur $X$ est alors la topologie engendrée par la
  famille de parties $\{(a,b)\}_{a,b\in X\cup\{\pm\infty\}}$, où un segment ouvert de
  la forme $(-\infty,a)$ signifie $\{x \mid x < a\}$ (les autres cas sont
  traités de façon analogue).
\end{definition}

On peut donc définir naturellement une topologie sur un ordinal $\alpha$, en
considérant la topologie de l'ordre pour $\in$ sur $\alpha$.

\begin{example}
  Considérons l'ordinal $\omega + 3$, la partie $\omega$ en est un ouvert
  puisqu'elle peut s'écrire $(-\infty, \omega)$. De plus, $\omega + 1$ est aussi
  un ouvert~: il vaut $(-\infty, \omega+1)$ mais il est aussi un fermé,
  puisque son complémentaire est $(\omega + 1, +\infty)$.
\end{example}

Ces exemples nous montrent que les ordinaux successeurs ont un comportement
relativement dégénéré.

\begin{proposition}
  Soit $\alpha$ un ordinal. Pour tout $\beta \in \alpha$, $\beta + 1$ est un
  point isolé dans $\alpha$, c'est-à-dire que $\{\beta + 1\}$ est un ouvert
  dans $\alpha$. C'est aussi le cas de $0$.
\end{proposition}

\begin{proof}
  Il suffit de voir que $\{\beta + 1\} = (\beta, \beta + 2)$. Pour $0$, on a
  $\{0\} = (-\infty, 1)$.
\end{proof}

\begin{proposition}
  Soit $\alpha$ un ordinal. Pour tout $\beta \in \alpha$, $\beta$ (en tant que
  partie de $\alpha$) est un ouvert de $\alpha$.
\end{proposition}

\begin{proof}
  $\beta$ est simplement $(-\infty, \beta)$.
\end{proof}

On remarque donc que $\omega$ est muni de la topologie discrète. Par contre,
$\omega$ dans $\omega + 1$ par exemple n'est pas un point isolé. En effet,
si on veut qu'un ouvert de base contienne $\omega$, alors il doit contenir un
élément strictement inférieur à $\omega$, qui est donc un élément
$n \in \omega$. On a alors tous les éléments $m > n$ dans notre partie,
puisqu'ils sont entre $n$ et $\omega$. Sachant qu'un ouvert est une union
d'ouverts de base, il sera donc impossible de ne pas contenir au moins un
segment finissant de $\omega$ si on veut contenir $\omega$. Plus généralement,
pour contenir un ordinal limite dans un ouvert, il faut en contenir un segment
finissant.

Cependant, c'est ici la notion de fermé qui nous intéressera. On a vu qu'un
ouvert ne peut pas contenir une limite sans contenir ce qu'il y a en-dessous.
Si on passe au complémentaire, on va voir que cela traduit ce qu'est un fermé
par le fait qu'il est clos par borne supérieur~: c'est un sens très proche du
sens séquentiel, où une partie est fermée si et seulement si toute suite
convergente reste dans la partie. Ici, plutôt qu'une suite, on prend une famille
quelconque, et plutôt que la limite on utilise la borne supérieure.

\begin{proposition}
  Soit $\alpha$ un ordinal. Une partie $C\subseteq \alpha$ est un fermé si et
  seulement si pour toute partie $X\subseteq C$, si $\sup X \in \alpha$ alors
  $\sup X \in C$.
\end{proposition}

\begin{proof}
  Supposons que $C$ est un fermé et soit $X\subseteq C$ tel que
  $\sup X \in \alpha$. On veut montrer que $\sup X$ est dans l'adhérence de $C$
  (qui est $C$ lui-même par hypothèse). Soit $U$ un ouvert contenant
  $\sup X$, sans perte de généralité (quitte à prendre une plus petite partie)
  on peut considérer que $U = (\beta, \gamma)$ où $\beta < \sup X < \gamma$,
  mais alors par définition de $\sup X$, on trouve $\delta \in X$ tel que
  $\beta < \delta < \sup X$, donc $U \cap X \neq\varnothing$, ce qui signifie
  que $\sup X \in X$.

  Réciproquement, supposons que pour tout $X\subseteq C$, si $\sup X \in \alpha$
  alors $\sup X \in C$. On veut montrer que $C$ est fermé. Pour cela, on va
  montrer que tout point adhérent de $X$ est dans $X$. Soit $x \in \adher X$~:
  si $x$ est un successeur ou $0$ alors il est un point isolé, donc il
  appartient directement à $X$ en considérant le fait que $\{x\}\cap X$ est non
  vide. On suppose donc que $x$ est une limite, et on construit une famille
  $\{x_\beta\}$ d'éléments de $X$ dont le sup est $x$~:
  \begin{itemize}
  \item on définit $x_0$ comme un élément de $(-\infty, x)\cap X$, qui est non
    vide car $x$ est adhérent à $X$.
  \item si on a défini $x_\alpha$, alors $x_{\alpha + 1}$ est un élément de
    $(x_\alpha,x)\cap X$.
  \item si $\lambda$ est limite, alors
    $x_{\lambda}=\sup_{\beta < \lambda}x_\beta$, qui est un élément de $X$ car
    tous les $x_\beta$ sont inférieurs à $x$, donc $\sup x_\beta\in \alpha$,
    et en utilisant notre hypothèse sur $X$.
  \end{itemize}
  On a donc créé une suite strictement croissante de $\Ord$ dans
  $(-\infty,x)\cap X$, ce qui est impossible. Notre famille finit donc à un
  certain $\gamma$, et $\sup_{\beta < \gamma} x_\beta = x$, donc en utilisant
  notre hypothèse sur $X$ on en déduit que $x\in X$.
\end{proof}

Enfin, donnons un résultat caractérisant les ordinaux compacts.

\begin{proposition}\label{prop.compac.succ}
  Soit $\alpha$ un ordinal, celui-ci est compact pour la topologie de l'ordre si
  et seulement s'il est un successeur ou $0$.
\end{proposition}

\begin{proof}
  On va d'abord montrer qu'un ordinal compact est successeur ou $0$ par
  contraposée. Supposons que $\alpha$ est un ordinal limite. Alors on considère
  le recouvrement
  \[\mathcal F \defeq \{(-\infty,\beta)\}_{\beta \in \alpha}\]
  Pour toute partie $F\subfin\mathcal F$, on trouve un plus grand $\beta$ tel
  que $\bigcup F = (-\infty,\beta)$, donc $\beta + 1 \notin\bigcup F$. On a
  donc un recouvrement dont on ne peut extraire un recouvrement fini~: $\alpha$
  n'est pas compact.

  Montrons maintenant qu'un ordinal successeur est compact. Par l'absurde,
  supposons qu'il existe un plus petit ordinal $\alpha$ non compact et
  successeur, et soit $\beta$ tel que $\alpha = \beta + 1$. Soit
  \[\mathcal F \defeq \{(\gamma_i,\delta_i)\mid i \in \kappa\}\]
  un recouvrement de $\alpha$ par des ouverts de base, où $\kappa$ est le
  cardinal de $\mathcal F$. Soit $j \in \kappa$ tel que
  $\beta \in (\gamma_j,\delta_j)$ et $\gamma = \min((\gamma_j,\delta_j))$,
  c'est-à-dire $\gamma_j + 1 = \gamma$. Comme on sait que $\mathcal F$ est un
  recouvrement de $\alpha$, on en déduit que
  \[\mathcal F' \defeq \{(\gamma_i,\delta_i) \cap \gamma\mid i \in \kappa\}\]
  est un recouvrement de $\gamma$ constitué d'ouverts de base. Par hypothèse de
  minimalité de $\alpha$, et comme $\gamma < \alpha$
  (puisque $\beta \in (\gamma_j,\delta_j)$) et est un successeur, on en déduit
  qu'il existe un sous-recouvrement $F\subfin \mathcal F'$ de $\gamma$. Alors en
  posant $J\subfin \kappa$ tel que
  $F = \{(\gamma_i,\delta_i)\cap \gamma \mid i \in J\}$, on en déduit que
  $F' = \{(\gamma_i,\delta_i)\mid i \in J\} \cup \{(\gamma_j,\delta_j)\}$ est
  un sous-recouvrement fini de $\mathcal F$. Donc $\alpha$ est compact, ce qui
  contredit notre hypothèse. Par l'absurde, on en déduit que tout ordinal
  successeur est compact.
\end{proof}

\subsection{Filtre des clubs}

On a vu dans le chapitre précédent qu'on pouvait considérer des parties
cofinales dans un ordinal $\alpha$. \'Etant données deux parties cofinales,
il est possible de différencier celles qui sont \og suffisament pleines\fg dans
$\alpha$ de celles qui ont de nombreux trous (en un certain sens). On introduit
donc une nouvelle notion, combinant la cofinalité (s'étendre assez haut dans
l'ordinal $\alpha$) et la topologie (être assez non dispersé) pour définir ce
qu'est un club.

\begin{definition}[Club]
  Soit $\alpha$ un ordinal. Un club (abréviation de l'anglais
  \emph{closed unbounded}) est une partie $C\subseteq \alpha$ qui est fermée
  dans $\alpha$ muni de la topologie de l'ordre, et cofinale dans $\alpha$.
\end{definition}

Un club est donc une grosse partie de $\alpha$. En particulier, en suivant
l'intuition que l'on a donné dans le \cref{chp.ordres}, une telle notion de
grosse partie devrait former un filtre. On peut imaginer une partie contenant
un club mais n'étant pas un club, donc on voudra plutôt considérer le filtre
engendré par cette famille de parties.

Cependant, pour construire ce filtre, il est nécessaire (pour que le filtre ne
soit pas trivial) que les clubs soient stables par intersection finie. En
l'occurrence, ça n'est pas le cas dans le cas général, mais c'est le cas à
partir du moment où la cofinalité de $\alpha$ est indénombrable.

\begin{proposition}
  Soit $\alpha$ un ordinal infini. Alors $\cof(\alpha) > \aleph_0$ si et
  seulement si l'intersection de deux clubs est encore un club.
\end{proposition}

\begin{proof}
  Commençons par le cas où $\cof(\alpha) \leq \aleph_0$. Dans ce cas, on peut
  trouver une suite $(\alpha_n)$ strictement croissante cofinale dans $\alpha$.
  Soient alors $(\alpha_{2n})$ et $(\alpha_{2n+1})$ deux suites extraites. Les
  images $B$ et $C$, respectivement, de $(\alpha_{2n})$ et $(\alpha_{2n+1})$, sont
  cofinales dans $\alpha$ et disjointes. De plus, $B$ et $C$ sont des parties
  fermées de $\alpha$. On a donc trouvé deux parties disjointes de $\alpha$ qui
  sont des clubs.

  Supposons maintenant que $\cof(\alpha) > \aleph_0$. Soient alors $C_1$ et
  $C_2$ deux clubs de $\alpha$. Montrons que $C_1\cap C_2$ est encore un club~:
  \begin{itemize}
  \item c'est une intersection de fermés, donc c'est un fermé.
  \item soit $\beta \in \alpha$, montrons qu'il existe $\gamma \in C_1\cap C_2$
    tel que $\beta \leq \gamma$. Pour cela, on construit une suite $(\gamma_n)$
    d'éléments supérieurs à $\beta$~: comme $C_1$ est cofinal dans $\alpha$,
    on définit $\gamma_{2n+2}$ comme un élément supérieur à $\gamma_{2n+1}$ qui
    est dans $C_1$ et $\gamma_{2n+1}$ comme un élément supérieur à
    $\gamma_{2n}$ qui est dans $C_2$, et on fixe $\gamma_0 = \beta$.

    Comme $\cof(\alpha) > \aleph_0$, $\sup \gamma_n \in \alpha$, donc comme
    $(\gamma_{2n+2})$ est une suite à valeurs dans $C_1$ qui est fermé,
    $\sup \gamma_n \in C_1$ et de même $\sup \gamma_n \in C_2$ car
    $(\gamma_{2n+1})$ est une suite à valeurs dans $C_2$.

    Ainsi $C_1\cap C_2$ est bien cofinal dans $\alpha$.
  \end{itemize}
  On en déduit donc que si $\cof(\alpha) > \aleph_0$, alors l'intersection de
  deux clubs est encore un club.
\end{proof}

Comme la condition porte sur la cofinalité, qui est une notion cardinale, on
parlera à partir de maintenant de cardinaux réguliers. Il est en général
possible de généraliser les résultats au cas d'un ordinal quelconque en
traduisant les conditions sur sa cofinalité, mais parler uniquement de cardinaux
réguliers simplifie les énoncés et suffit en pratique.

\begin{definition}[Filtre des clubs]
  Soit $\kappa \geq \aleph_1$ un cardinal régulier. On définit le filtre propre
  des clubs de $\kappa$ par
  \[\Club_\kappa \defeq \filclose{\{C\subseteq \kappa
    \mid C\text{ est un club}\}}\]
\end{definition}

\begin{proof}
  On a vérifié que l'intersection de deux clubs est encore un clubs, donc
  l'intersection d'un nombre fini de clubs est encore un club. En particulier,
  l'intersection d'un nombre fini de clubs est non vide, donc le filtre engendré
  par la famille des clubs de $\kappa$ est un filtre propre.
\end{proof}

Donnons d'abord une famille d'éléments de ce filtre.

\begin{property}
  Soit $\kappa$ un cardinal régulier indénombrable. L'ensemble $\Club_\kappa$
  contient toutes les parties de la forme
  $\{\beta \in\kappa\mid \alpha\leq\beta\}$ pour $\alpha \in \kappa$.
\end{property}

\begin{proof}
  Soit $\alpha \in \kappa$, montrons que $\{\beta\in\kappa\mid\alpha\leq\beta\}$
  est un club~:
  \begin{itemize}
  \item il est fermé car le complémentaire de l'ouvert $\alpha$.
  \item il est cofinal dans $\kappa$ puisque si $\alpha \leq \beta$, alors
    $\alpha \leq \beta + 1$.
  \end{itemize}
  $\Club_\kappa$ contient donc toutes les parties co-bornées.
\end{proof}

Définissons maintenant les parties non stationnaires et stationnaires.

\begin{definition}[Parties stationnaires, non stationnaires]
  Soit $\kappa$ un cardinal régulier indénombrable. Une partie
  $X\subseteq\kappa$ est dite non stationnaire si elle est le complémentaire
  d'un élément de $\Club_\kappa$. Elle est dite stationnaire si elle n'est pas
  non stationnaire.
\end{definition}

De la même façon que les parties de $\mathbb R$ de complémentaire négligeable
forment un filtre dont le complémentaire forme l'idéal des parties négligeables
(de mesure nulle), on a ici les parties non stationnaires qui sont négligeables
en un sens. Une partie stationnaire est alors l'analogue d'une partie de mesure
non nulle dans $\mathbb R$. On donne une caractérisation des parties
stationnaires, qui peut largement être prise comme définition.

\begin{proposition}
  Une partie $X$ est stationnaire si et seulement si elle intersecte tous les
  clubs de $\kappa$.
\end{proposition}

\begin{proof}
  $X$ est non stationnaire si et seulement si $\kappa\setminus X$ contient
  un club $C$, ce qui est équivalent à dire que $X$ est disjoint avec au moins
  un club de $\kappa$~: on en déduit l'équivalence annoncée.
\end{proof}

\subsection{Exemples importants}

Dans cette sous-section, nous traitons des cas élémentaires qu'il est important
de couvrir pour avoir une certaine intuition des clubs et ensembles
stationnaires.

Pour commencer, un exemple important de club est celui des points de fermeture,
que l'on va définir d'abord.

\begin{definition}[Point de fermeture]
  Soit $\alpha$ un ordinal et $f : \alpha \to \alpha$. On dit qu'un point
  $\beta \in \alpha$ est un point de fermeture de $f$ lorsque
  \[\forall \gamma < \beta, f(\gamma) < \beta\]
  ou, de façon équivalence, lorsque $f(\beta) \subseteq \beta$ en considérant
  l'image directe pour $f(\beta)$ (et non l'image de $\beta$). On notera
  $\Clos(f)$ l'ensemble des points de fermeture de $f$.
\end{definition}

\begin{proposition}
  Soit $\kappa$ un ordinal régulier indénombrable et $f : \kappa \to \kappa$,
  alors $\Clos(f)$ est un club de $\kappa$.
\end{proposition}

\begin{proof}
  On vérifie que $\Clos(f)$ est un club~:
  \begin{itemize}
  \item soit $X\subseteq \Clos(f)$, montrons que $\sup X \in \Clos(f)$. Si
    $\sup X \in X$ le résultat est évident. On suppose donc que $\sup X$ est un
    ordinal limite. Soit $\beta \in \sup X$, comme $\sup X$ est limite,
    on trouve $\alpha \in X$ tel que $\beta + 1 \leq \alpha$, et comme
    $\beta < \alpha$ et $\alpha \in \Clos(f)$, on en déduit que
    $f(\beta) < \alpha < \sup X$, donc pour tout
    $\beta \in \sup X, f(\beta) < \sup X$, d'où $X\in \Clos(f)$.
  \item de plus $\Clos(f)$ est cofinal dans $\kappa$. Soit $\beta\in\kappa$, on
    définit la suite
    \[\begin{cases}
    \alpha_0 \defeq \beta + 1 \\
    \alpha_{n+1} \defeq \displaystyle
    \Big(\sup_{\delta \leq \alpha_n} f(\delta)\Big) + 1
    \end{cases}\]
    Comme on a une suite dans un ordinal régulier indénombrable, on en déduit
    que $\sup \alpha_n \in \kappa$ et $\sup \alpha_n > \beta$. Il nous reste à
    montrer que $\sup \alpha_n \in \Clos(f)$. Soit $\gamma < \sup \alpha_n$, on
    trouve $n \in \mathbb N$ tel que $\beta \leq \alpha_n$, dont on déduit par
    définition de $\alpha_{n+ 1}$ que $f(\beta) < \alpha_{n+1}\leq \sup\alpha_n$,
    donc $\sup\alpha_n\in\Clos(f)$.
  \end{itemize}
\end{proof}

Dans cet exemple précis, il est essentiel que $\kappa$ soit régulier. En effet,
en prenant $f : \aleph_{\omega_1} \to \aleph_{\omega_1}$ tel que
$f(\alpha) = \aleph_\alpha$ pour $\alpha < \omega_1$ et $f(\alpha) = 0$ sinon,
on a un exemple de fonction sans point de fermeture.

Un autre exemple important de club est celui des limites. On peut le voir comme
un cas particulier de point de fermeture, pour la fonction $S$, mais cela ne
fonctionne que dans le cas d'un cardinal régulier.

\begin{definition}[Ensemble des limites]
  Soit $\alpha$ un ordinal, on définit l'ensemble des limites de $\alpha$ comme
  \[\Lim_\alpha \defeq
  \left\{\beta \in \alpha \mid \bigcup \beta = \beta\right\}\]
\end{definition}

\begin{proposition}
  Pour $\kappa$ un cardinal régulier indénombrable, $\Lim_\kappa$ est un club
  de $\kappa$.
\end{proposition}

\begin{proof}
  On vérifie que $\Lim_\kappa$ est un club~:
  \begin{itemize}
  \item Soit $X\subseteq \Lim_\kappa$ tel que $\sup X \in \kappa$, alors
    \begin{align*}
      \bigcup (\sup X) &= \bigcup_{\alpha \in \sup X} \alpha \\
      &= \bigcup_{\alpha \in \bigcup X} \alpha \\
      &= \bigcup_{\alpha \in X} \left(\bigcup \alpha\right)\\
      &= \bigcup_{\alpha \in X} \alpha\\
      &= \sup X
    \end{align*}
    donc $\sup X \in \Lim_\kappa$, donc $\Lim_\kappa$ est fermé.
  \item Soit $\alpha \in \kappa$, alors $\alpha + \omega$ est un ordinal limite
    et il est supérieur à $\alpha$, donc $\Lim_\kappa$ est non borné.
  \end{itemize}
  Donc $\Lim_\kappa$ est un club de $\kappa$.
\end{proof}

Parmi ces limites, dans un ordinal donné, on peut faire une distinction sur la
cofinalité. Dans le cas de $\aleph_2$ par exemple, les ordinaux limites peuvent
avoir une cofinalité de $\aleph_0$ ou de $\aleph_1$. Comme toute limite a l'une
des deux cofinalités, il est clair qu'au moins l'ensemble des ordinaux limites
de cofinalité $\aleph_0$ ou celui des ordinaux limites de cofinalité $\aleph_1$
est stationnaire, comme l'union des deux est un club. En réalité, ils sont
tous les deux stationnaires, et on peut généraliser ce résultat, en utilisant
la notation $E_\kappa^\lambda$ introduite dans la \cref{sct.cof}.

\begin{proposition}
  Soient $\kappa$ et $\lambda$ deux cardinaux réguliers infinis tels que
  $\kappa < \lambda$. Alors l'ensemble $E_\kappa^\lambda$ est stationnaire.
\end{proposition}

\begin{proof}
  On va montrer que tout club contient un élément de cofinalité $\kappa$. Soit
  $C$ un club de $\lambda$. On définit $f : \lambda \to C$ par récursion
  transfinie~:
  \begin{itemize}
  \item $f(0) = \min C$
  \item pour tout $\alpha \in \lambda$,
    $f(\alpha + 1) = \min C\setminus f(\alpha)$
  \item pour tout $\alpha \in \Lim_\lambda$,
    $f(\alpha) = \sup_{\beta < \alpha} f(\beta)$
  \end{itemize}

  $f$ est bien définie. En effet, pour $\alpha \in \lambda$, comme $\lambda$ est
  régulier, $f\restr\alpha$ est bornée, et $C$ ne l'est pas donc on peut bien
  trouver un minimum dans $C\setminus f(\alpha)$. Dans le cas limite, comme $C$
  est un club, on en déduit que $\sup f(\beta)$ est effectivement un élément de
  $C$.

  En fait, on peut même montrer que $f$ est une bijection entre $\lambda$ et
  $C$, de façon analogue à la preuve que $\aleph$ établit une bijection entre
  les ordinaux et les cardinaux infinis.

  On voit alors que $f(\kappa)$ est un élément de cofinalité $\kappa$, donc que
  $C\cap E_\kappa^\lambda \neq \varnothing$. $E_\kappa^\lambda$ est donc bien
  stationnaire.
\end{proof}

\begin{corollary}\label{cor.al2.deuxdis}
  Il existe deux ensembles stationnaires disjoints dans $\aleph_2$.
\end{corollary}

\begin{proof}
  On vient de voir que $E_{\aleph_0}^{\aleph_2}$ et $E_{\aleph_1}^{\aleph_2}$ sont
  stationnaires, et ils sont évidemment disjoints, d'où le résultat.
\end{proof}

Pour revenir sur un point précédent, on sait que dans le cas d'une cofinalité
indénombrable, la famille des clubs est stable par intersection finie. Est-il
possible d'aller plus loin et de montrer une stabilité pour des intersections
plus grandes ? C'est effectivement le cas, comme nous le montrons maintenant.

\begin{proposition}\label{prop.inter.club.k}
  Soit $\kappa$ un cardinal régulier indénombrable. Alors l'intersection d'une
  famille de cardinal $\lambda < \kappa$ de clubs est encore un club.
\end{proposition}

\begin{proof}
  On raisonne par l'absurde. Supposons qu'il existe un cardinal
  $\lambda < \kappa$ tel qu'il existe une famille $\mathcal C$ de cardinal
  $\lambda$ dont les éléments sont des clubs et dont l'intersection n'est pas
  un club. Prenons $\lambda$ minimal, et soit $(C_\alpha)$ une énumération de
  $\mathcal C$~:
  \[\mathcal C \defeq \{C_\alpha \mid \alpha < \lambda\}\]
  On souhaite montrer que $\bigcap \mathcal C$ est un club. Pour cela, on
  définit d'abord la famille suivante~:
  \[D_\alpha \defeq \bigcup_{\beta < \alpha} C_\beta\]
  On remarque que
  $\displaystyle\bigcap_{\alpha < \lambda} D_\alpha = \bigcap\mathcal C$, donc il
  nous suffit de montrer que cette intersection est un club. De plus, on sait
  que chaque $D_\alpha$ est un club, car il est l'intersection
  d'une famille de cardinal strictement inférieur à $\lambda$ de club (et
  $\lambda$ est supposé minimal). Comme chaque $D_\alpha$ est un fermé,
  $\bigcap D_\alpha$ aussi. Il nous reste donc à montrer que $\bigcap D_\alpha$
  est cofinal.

  Soit $\beta < \kappa$, on veut construire une famille $(\beta_\gamma)$ de
  taille $\lambda$ d'éléments supérieurs à $\beta$ par induction transfinie~:
  \begin{itemize}
  \item on sait que $D_0 = C_0$ est cofinal, donc on peut trouver
    $\beta_0 \geq \beta$ un élément de $D_0$.
  \item si on a défini $\beta_\gamma$, alors on peut utiliser le fait que
    $D_{\gamma + 1}$ est un club pour trouver
    $\beta_{\gamma + 1}\in D_{\gamma+ 1}$ supérieur à $\beta_\gamma$.
  \item si on a défini $\beta_\gamma$ pour tout $\gamma < \delta$ et $\delta$
    est un ordinal limite, alors comme $\delta < \cof(\kappa) = \kappa$, on
    sait que $\sup_{\gamma < \delta} \beta_\gamma < \kappa$, donc on peut prendre
    $\beta_\delta \in D_\delta$ supérieur à ce suprémum.
  \end{itemize}
  On considère maintenant
  \[\gamma \defeq \sup_{\gamma < \lambda} \beta_\gamma\]
  Comme $\kappa$ est régulier et $\lambda < \kappa$, on sait que
  $\gamma < \kappa$, et cet élément est dans tous les $D_\alpha$. En effet, il
  est la borne supérieure d'une famille d'éléments de $D_\alpha$, donc comme
  $D_\alpha$ est fermé, $\gamma \in D_\alpha$.

  Ainsi $\displaystyle\bigcap_{\alpha < \lambda} D_\alpha$ est un club, ce qui est
  absurde. On en déduit que l'intersection d'une famille de cardinal
  $\lambda < \kappa$ de clubs est encore un club.
\end{proof}

L'autre question naturelle, se basant sur les $E_\kappa^\lambda$, est de savoir
s'il existe de plus grandes familles de parties stationnaires disjointes, car
$2$ ensembles stationnaires disjoints dans $\aleph_2$ peut sembler peu. On va
ici simplement donner un résultat légèrement plus fort.

\begin{proposition}\label{prop.al1.deuxdis}
  Il existe deux ensembles stationnaires dans $\aleph_1$ qui sont disjoints.
\end{proposition}

\begin{proof}
  Pour commencer, on pose $f : \omega_1 \to [0,1]\setminus \mathbb Q$ une
  injection (qui existe par un simple argument de cardinalité). On va
  construire, pour chaque $u \in \{0,1\}^*$ de taille $n$, un ensemble
  $C_u\subseteq \omega_1$ tel que $f(C_u)$ est de diamètre inférieur à
  $2^{-n}$.

  Pour cela, on pose
  \[\begin{cases}
  C_0 \defeq \{\alpha \mid f(\alpha) < 1/2\}\\
  C_1 \defeq \omega_1\setminus C_0
  \end{cases}\]
  Si $C_u$ a été défini pour $u$ de taille $n$, alors on considère $r$ le
  milieu de $f(C_u)$
  (la moyenne entre la borne supérieure et la borne inférieure) et on pose
  \[\begin{cases}
  C_{u\star 0} \defeq \{\alpha \in C_u \mid f(\alpha) < r\}\\
  C_{u\star 1} \defeq C_u \setminus C_{u0}
  \end{cases}\]
  (où $u\star i$ est le mot constitué de $u$ et de la lettre $i$).
  On a ainsi notre famille de parties $C_u$ pour tout $u \in \mathbb N^*$, dont
  le diamètre est inférieur à $2^{-n}$ pour un mot $u$ de taille $n$.

  On raisonne maintenant par l'absurde~: supposons qu'il n'existe pas deux
  ensembles stationnaires disjoints dans $\omega_1$. On va alors construire une
  suite $g : \omega \to 2$ de telle sorte que $C_{g\restr n}$ contienne toujours
  un club. Ce faisant, $\displaystyle\bigcap_{n\in \mathbb N} C_{g\restr n}$
  contient un club puisque chaque $C_{g\restr n}$ contient un club et que,
  grâce à la \cref{prop.inter.club.k}, on sait qu'une intersection dénombrable
  de clubs de $\omega_1$ est encore un club de $\omega_1$. Comme
  $f(C_{g\restr n})$ est de diamètre inférieur à $2^{-n}$, on sait que
  $\displaystyle f\left(\bigcap_{n \in \mathbb N} C_{g\restr n}\right)$
  ne contient qu'un élément, donc par injectivité de $f$,
  $\displaystyle \bigcap_{n \in \mathbb N} C_{g\restr n}$ ne contient qu'un élément.
  La construction de la suite $g$ mène donc à une contradiction, puisqu'un
  club ne peut pas contenir qu'un seul élément.

  On construit donc $g$ par récurrence~: on sait que $C_{g\restr n\star 0}$ et
  $C_{g\restr n\star 1}$ sont disjoints, et par hypothèse il n'existe
  pas deux ensembles stationnaires dans $\omega_1$ disjoints, donc l'un des
  deux ensembles est non stationnaire~: cela signifie que l'autre ensemble
  contient un club. On choisit alors $i$ tel que $C_{g\restr n \star i}$
  contient un club. A l'étape $0$, on sait que $\omega_1$ entier contient un
  club, nous permettant donc de faire ce raisonnement.

  On en déduit donc, avec les arguments précédents, qu'il existe au moins deux
  ensembles stationnaires disjoints dans $\aleph_1$.
\end{proof}

\section{Lemme de Fodor et applications}

Dans cette section, nous allons étudier plus précisément les ensembles
stationnaires. Notre premier objectif sera de donner un résultat plus général
similaire au \cref{cor.al2.deuxdis} et à la \cref{prop.al1.deuxdis}. En
répondant à la question naturelle de savoir combien on peut trouver d'ensembles
stationnaires disjoints dans un cardinal, nous serons naturellement amenés à
utiliser une version faible du lemme de Fodor, que nous renforcerons ensuite
pour obtenir le vrai lemme de Fodor. Nous verrons dans la deuxième sous-section
une application du lemme de Fodor avec le théorème de Silver, donnant une
condition sur l'exponentiation cardinale dans le cas de cardinaux singuliers.

\subsection{Lemme de Fodor}

Le lemme de Fodor s'applique sur des fonctions dites régressives, commençons
donc par en donner la définition.

\begin{definition}[Fonction régressive]
  Soit $\kappa$ un cardinal, $S\subseteq \kappa$ et $f : S \to \kappa$. On
  dit que $f$ est régressive si la propriété suivante est vérifiée~:
  \[\forall \alpha \in S\setminus\{0\}, f(\alpha) < \alpha\]
\end{definition}

On donne alors la version faible du lemme de Fodor~:

\begin{lemma}[Fodor, version faible]\label{lem.fodor.faible}
  Soit $\lambda$ un cardinal régulier indénombrable et $S\subseteq \lambda$ un
  ensemble stationnaire. Soit $f : S \to \lambda$ régressive, alors $f$ est
  constante sur un ensemble de taille $\lambda$, c'est-à-dire qu'il existe
  $T\subseteq S$ et $\alpha \in \lambda$ tels que
  \[\begin{cases}
  \forall \beta \in T, f(\beta) = \alpha\\
  \Card(T) = \lambda
  \end{cases}\]
\end{lemma}

\begin{proof}
  On procède par l'absurde. Supposons que pour tout $\beta \in \lambda$,
  $f^{-1}(\{\beta\})$ est de cardinal strictement inférieur à $\lambda$. On sait
  donc, comme $\lambda$ est régulier, que $\sup f^{-1}(\{\beta\})\in \lambda$.
  Soit alors
  \[\begin{array}{ccccc}
  g & : & \lambda & \longrightarrow & \lambda\\
  & & \alpha & \longmapsto & \sup f^{-1}(\{\alpha\})
  \end{array}\]
  On va étudier $\Clos(f)$~: si $\alpha \in S$, alors comme pour tout $\beta$,
  $f(\beta) < \beta$, on en déduit que pour tout $\beta < \alpha$,
  $f(\beta) < \beta < \alpha$, donc $g(f(\alpha)) \geq \alpha$, d'où
  $\alpha \notin \Clos(g\circ f)$. Pourtant, on sait que $\Clos(f)$ est un club,
  donc $S$ doit intersecter $\Clos(f)$, ce qui est absurde.

  On en déduit donc qu'il existe $\alpha \in \lambda$ tel que
  $|f^{-1}(\{\alpha\})| = \lambda$, d'où le résultat en prenant ce tel $\alpha$
  et $T = f^{-1}(\{\alpha\})$.
\end{proof}

On va maintenant définir un outil combinatoire de théorie des ensembles~: les
matrices d'Ulam. Celles-ci vous nous servir pour prouver le théorème sur les
ensembles stationnaires disjoints de façon particulièrement élégante.

\begin{definition}[Matrices d'Ulam \cite{Ulam1930}]
  Soit $\kappa$ un cardinal infini et $\lambda = \kappa^+$. Pour chaque
  $\alpha < \lambda$, on se donne une injection $f_\alpha : \alpha \to\kappa$.
  On définit la matrice d'Ulam en $\alpha < \lambda$ et $\beta < \kappa$ par
  \[X_\alpha^\beta \defeq
  \{\xi \mid \alpha < \xi < \lambda, f_\xi(\alpha) = \beta\}\]
  (chaque $X_\alpha^\beta$ est donc une partie de $\lambda$).
\end{definition}

\begin{property}
  Soient $\kappa$ un cardinal infini et $\lambda = \kappa^+$. Alors la matrice
  d'Ulam a les propriétés suivantes~:
  \begin{enumerate}[label=(\roman*)]
  \item\label{enum.ulam1} pour tout $\beta < \kappa$, la famille
    $\{X_\alpha^\beta\}_{\alpha < \lambda}$ est constituée d'ensembles deux à deux
    disjoints.
  \item\label{enum.ulam2} pour tout $\alpha < \lambda$, on a l'identité
    \[\bigcup_{\beta < \kappa} X_\alpha^\beta = \lambda \setminus (\alpha + 1)\]
  \item pour tout $\alpha$ il existe $\beta(\alpha)$ tel que
    $X_\alpha^{\beta(\alpha)}$ est stationnaire.
  \item il existe $\beta < \kappa$ tel que
    $Y = \{\alpha \mid \beta(\alpha) = \beta\}$ est de taille $\lambda$.
  \item pour le $\beta$ ci-dessus, la famille
    $\{X_\alpha^\beta\}_{\alpha \in Y}$ est constituée d'ensembles stationnaires
    deux à deux disjoints.
  \end{enumerate}
\end{property}

\begin{proof}
  On vérifie chaque propriété~:
  \begin{enumerate}[label=(\roman*)]
  \item Soit $\beta < \kappa$ et $\alpha, \gamma \in\lambda$. Si
    $x\in X_\alpha^\beta \cap X_\gamma^\beta$, alors $f_x(\alpha) = f_x(\gamma)$
    donc, comme $f_x$ est une injection, $\alpha = \gamma$.
  \item Soit $\xi \in \lambda \setminus (\alpha + 1)$. Comme $\xi > \alpha$,
    $f_\xi$ est définie en $\alpha$, et $f_\xi : \xi \to \kappa$, donc
    $\xi \in X_\alpha^{f_\xi(\alpha)}$, d'où
    $\displaystyle
    \lambda\setminus(\alpha+1)\subseteq\bigcup_{\beta < \kappa} X_\alpha^\beta$

    Dans le sens réciproque, si $\xi \in X_\alpha^\beta$ pour $\beta <\kappa$,
    alors par définition $\alpha < \xi < \lambda$ donc
    $\xi \in \lambda\setminus(\alpha + 1)$.
  \item Soit $\alpha < \lambda$. Par l'absurde, supposons que tous les
    $X_\alpha^\beta$ pour $\beta < \kappa$ sont non stationnaires. En utilisant
    la \cref{prop.inter.club.k} sous la forme complémentaire (en utilisant des
    ensembles non stationnaires plutôt que des clubs), on en déduit que
    $\displaystyle\bigcup_{\beta < \kappa} X_\alpha^\beta$ est non stationnaire.
    Pourtant, on sait que cette union vaut $\lambda \setminus(\alpha + 1)$ qui
    est, en tant que partie co-bornée de $\lambda$, un club. C'est donc une
    absurdité, donc il existe au moins un indice $\beta(\alpha)$ tel que
    $X_\alpha^{\beta(\alpha)}$ est stationnaire.
  \item On utilise maintenant le \cref{lem.fodor.faible}~: la fonction
    $g : \alpha \mapsto \beta(\alpha)$ est régressive sur
    $\lambda\setminus\kappa$ comme $\beta(\alpha) \in \kappa$, et la partie
    $\lambda\setminus\kappa$ est un club (donc stationnaire). On en déduit
    l'existence de $Y$ de cardinal $\lambda$ et $\beta \in \kappa$ tels que
    \[\forall \alpha \in Y, \beta(\alpha) = \beta\]
  \item On sait que pour tout $\alpha \in Y$, $\beta(\alpha) = \beta$, et
    que $X_\alpha^{\alpha(\beta)}$ est stationnaire, donc la famille
    $\{X_\alpha^\beta\}_{\alpha \in Y}$ est constituée d'ensembles stationnaires. De
    plus, en utilisant \ref{enum.ulam1}, on en déduit que les ensembles sont
    deux à deux disjoints.
  \end{enumerate}
\end{proof}

On en déduit le résultat voulu.

\begin{theorem}
  Soit $\lambda$ un cardinal successeur, il existe une famille de taille
  $\lambda$ de parties stationnaires dans $\lambda$ deux à deux disjointes.
\end{theorem}

\begin{proof}
  Par définition, il existe $\kappa$ tel que $\lambda = \kappa^+$, on prend
  alors comme famille celle précédemment exhibée.
\end{proof}

En fait, on voit que tout le raisonnement sur les matrices d'Ulam peut se faire
relativement à un ensemble stationnaire $S$ donné, en définissant
\[S_\alpha^\beta \defeq \{\xi \in S\mid \alpha < \xi < \lambda,
f_\xi(\alpha) = \beta\}\]
le \cref{enum.ulam2} devient alors
\[\bigcup_{\beta < \kappa} S_\alpha^\beta = (\lambda \setminus (\alpha +1))\cap S\]

On peut alors en déduire une version plus forte du théorème.

\begin{theorem}
  Soit $\lambda$ un cardinal successeur et $S\subseteq\lambda$ un ensemble
  stationnaire. Il existe alors une famille de taille $\lambda$ de parties
  stationnaires de $S$ deux à deux disjointes.
\end{theorem}

Revenons maintenant au lemme de Fodor. Si l'on souhaite naïvement intersecter
$\kappa$ clubs dans un cardinal régulier indénombrable $\kappa$, on peut
se retrouver dans le cas des segments terminaux de $\kappa$ où l'on obtient une
intersection vide. On doit donc définir une intersection plus subtile~: c'est
l'intersection diagonale.

\begin{definition}[Intersection diagonale]
  Soit $\kappa$ un cardinal et $\{C_\alpha\}_{\alpha < \kappa}$ une famille de
  parties de $\kappa$. On définit l'intersection diagonale de $\{C_\alpha\}$ par
  \[\diagcap_{\alpha < \kappa} C_\alpha \defeq \{\beta <\kappa\mid
  \beta\in\bigcap_{\alpha < \beta} C_\alpha\}\]
\end{definition}

\begin{proposition}
  Soit $\kappa$ un cardinal régulier indénombrable et
  $\{C_\alpha\}_{\alpha < \kappa}$ une famille de $\kappa$ clubs. Alors
  $\displaystyle\diagcap_{\alpha < \kappa} C_\alpha$ est un club.
\end{proposition}

\begin{proof}
  Nommons $D$ cette intersection diagonale et vérifions les axiomes d'un club~:
  \begin{itemize}
  \item Soit $X\subseteq D$ borné, montrons que $\sup X \in D$. Soit
    $\beta < \sup X$, par définition pour tout $\delta$ tel que
    $\beta<\delta<\sup X$ et $\delta \in X$, $\delta \in C_\beta$. On en déduit,
    comme $\sup X$ est une limite d'éléments de $X$ supérieurs à $\beta$,
    que $\sup X$ est une limite d'éléments de $C_\beta$, donc un élément de
    $C_\beta$ puisque $C_\beta$ est un club (en particulier un fermé). Donc
    $\sup X\in C_\beta$ pour tout $\beta < \sup X$, donc $\sup X \in D$.
  \item On utilise un argument développé dans la preuve de la
    \cref{prop.inter.club.k} pour montrer que $D$ est cofinal. On pose
    \[D_\alpha \defeq \bigcap_{\beta < \alpha} C_\alpha\]
    Chaque $D_\alpha$ est un club d'après la \cref{prop.inter.club.k}.
    Soit $\beta < \kappa$, on construit une suite $(\alpha_n)$ par
    récurrence~:
    \begin{itemize}
    \item $\alpha_0 = \beta + 1$
    \item pour tout $n \in \mathbb N$, si $\alpha_n$ a été défini, alors on
      sait que $D_{\alpha_n}$ est un club, donc on peut trouver
      $\alpha_{n+1}\in D_{\alpha_n}$ tel que $\alpha_{n+1} > \alpha_n$.
    \end{itemize}
    On définit alors
    \[\alpha \defeq \sup_{n \in \mathbb N} \alpha_n\]
    Alors, pour tout $n \in \mathbb N$, on sait que $\alpha_m \in D_{\alpha_n}$
    pour tout $m \geq n$, donc $\alpha \in D_{\alpha_n}$. Comme
    $\displaystyle D_{\alpha} =\bigcap_{n \in \mathbb N} D_{\alpha_n}$,
    on en déduit que $\alpha \in D_\alpha$, c'est-à-dire
    $\alpha \in \bigcap_{\beta < \alpha} C_\alpha$, donc
    $\alpha \in D$ et $\alpha > \beta$.
  \end{itemize}
  Ainsi $D$ est bien un club.
\end{proof}

On en déduit le lemme de Fodor.

\begin{lemma}[Fodor \cite{Fodor}]
  Soit $\kappa$ un cardinal régulier indénombrable, $S \subseteq \kappa$ une
  partie stationnaire et $f : S \to \kappa$ une fonction régressive. Alors
  $f$ est constante sur un ensemble stationnaire.
\end{lemma}

\begin{proof}
  Supposons que pour tout $\beta \in \kappa$, $f^{-1}(\{\beta\})$ est non
  stationnaire. On trouve alors $C_\beta$ un club disjoint de
  $f^{-1}(\{\beta\})$. Comme $\displaystyle\diagcap_{\beta < \kappa} C_\beta$ est un
  club et $S$ est stationnaire, on peut trouver
  $\gamma \in S\cap \diagcap_{\beta < \kappa} C_\beta$. On sait donc que
  $\gamma \in C_\beta$ pour tout $\beta < \gamma$, c'est-à-dire que
  $f(\gamma) \neq \beta$ pour tout $\beta < \gamma$. On en déduit que
  $f(\gamma) \geq \gamma$, ce qui est impossible car $f$ est supposée
  régressive.
\end{proof}

\subsection{Théorème de Silver}

Dans cette partie, on va montrer le théorème de Silver. On a vu dans le
\cref{chp.ordinaux} que sur les cardinaux réguliers, on disposait de très peu
d'informations sur le comportement de $\kappa \mapsto X^\kappa$. Le théorème de
Silver montre que dans le cas des cardinaux singuliers cela était très
différent. Ce que nous allons montrer est, plus précisément, que si l'hypothèse
généralisée du continue est vraie jusqu'à un certain cardinal singulier
$\lambda$ de cofinalité indénombrable, alors celle-ci est vrai aussi en ce
cardinal.

On va d'abord introduire la notion de suite cofinale continue, qui est un
renforcement naturel de la notion de suite cofinale.

\begin{definition}[Suite cofinale continue]
  Soit $\alpha$ un ordinal et $(\alpha_\beta)$ une suite d'éléments de $\alpha$.
  On dit que $(\alpha_\beta)$ est une suite cofinale continue si c'est une
  suite strictement croissante, que $\sup\alpha_\beta = \alpha$ et que pour
  tout $\gamma \in\Lim_\alpha$, $\sup_{\beta < \gamma} \alpha_\beta = \gamma$.
\end{definition}

\begin{exercise}
  Montrer que si $\alpha$ est un ordinal limite, alors il existe une suite
  cofinale continue de taille $\cof(\alpha)$.
\end{exercise}

On aura aussi besoin de la notion de famille éventuellement différentes, qui
peut se voir comme une généralisation de celle de famille presque disjointe pour
un cardinal autre que $\omega$.

\begin{definition}
  Soit $X$ un ensemble et $\kappa$ un cardinal. Une famille
  $\mathcal F\subseteq\Funct(\kappa,X)$ est dite éventuellement différente si
  la condition suivante est vérifiée~:
  \[\forall f,g \in \mathcal F, f \neq g \implies
  \Card(\{\xi\mid f(\xi) = g(\xi)\}) < \kappa\]
\end{definition}

\begin{notation}
  Dans le suite de cette sous-section, on pose $\lambda$ un cardinal singulier
  de cofinalité indénombrable, $\alpha$ l'ordinal tel que
  $\lambda = \aleph_\alpha$ et $\kappa = \cof(\alpha)$ (avec, donc,
  $\kappa > \omega$). On suppose aussi
  \[\HCG_\lambda \defeq \forall \beta < \lambda, 2^{\aleph_\beta} = \aleph_{\beta + 1}
  \]

  Comme nous allons procéder par l'absurde, on supposera de plus que
  $2^\lambda > \lambda^+$.

  On se fixe une suite cofinale continue $(\alpha_{\beta})_{\beta < \kappa}$ dans
  $\alpha$.
\end{notation}

\begin{proposition}
  Il existe une famille éventuellement différente
  $\mathcal F\subseteq\Funct(\kappa,\lambda)$ de taille $2^\lambda$ telle que
  \[\forall f\in\mathcal F, \forall \beta < \kappa,
  f(\beta) < \aleph_{\alpha_\beta + 1}\]
\end{proposition}

\begin{proof}
  On se donne, pour chaque $\beta < \kappa$, une bijection
  \[\iota_\beta : \powerset(\aleph_{\alpha_\beta}) \xrightarrow[\cong]{}
  \aleph_{\alpha_\beta + 1}\]
  donnée par $\HCG_\alpha$.

  Pour chaque $X\subseteq \lambda$, on définit la fonction
  \[\begin{array}{ccccc}
  f_X & : & \kappa & \longrightarrow & \lambda\\
  & & \beta &\longmapsto & \iota_\beta(X \cap \aleph_{\alpha_\beta})
  \end{array}\]
  Remarquons que cette fonction est bien définie, car
  $X\cap \aleph_{\alpha_\beta}$ est une partie de $\aleph_{\alpha_\beta}$ et elle est
  donc à valeur dans $\aleph_{\beta + 1} < \lambda$. Montrons maintenant que
  $\{f_X\}_{X\subseteq \lambda}$ est une famille éventuellement différente~:

  Soient $X,Y\subseteq \lambda$ tels que $X\neq Y$. On veut montrer que
  $\{\xi \mid f_X(\xi) = f_Y(\xi)\}$ est de cardinal strictement inférieur à
  $\kappa$. Comme $X\neq Y$, on trouve $\delta = \min X \triangle Y$. La suite
  $(\alpha_\beta)$ est cofinale dans $\alpha$, donc on peut trouver
  $\gamma\in \kappa$ tel que $\aleph_{\alpha_\gamma} > \delta$. Comme
  $(\alpha_\beta)$ est strictement croissante, on en déduit que pour tout
  $\gamma' > \gamma, \aleph_{\gamma'} > \delta$.

  En remarquant que
  $(X\triangle Y) \cap \aleph_{\alpha_\beta} =
  (X\cap \aleph_{\alpha_\beta})\triangle (Y\cap \aleph_{\alpha_\beta})$,
  si $\beta \in \kappa$ est tel que $\delta \in \aleph_{\alpha_\beta}$, alors
  $\delta\in(X\cap \aleph_{\alpha_\beta})\triangle (Y\cap \aleph_{\alpha_\beta})$
  donc $X\cap \aleph_{\alpha_\beta}\neq Y\cap \aleph_{\alpha_\beta}$,
  d'où par injectivité de $\iota_\beta$, $f_X(\beta) \neq f_Y(\beta)$.

  On voit donc que l'ensemble $\{\xi\mid f_X(\xi) = f_Y(\xi)\}$ est borné par
  $\gamma$ tel que $\delta \in \aleph_\gamma$, c'est donc un ensemble de
  cardinal inférieur à $\gamma$ et $\gamma < \kappa$, donc $\{f_X\}$ est une
  famille éventuellement différente.
\end{proof}

On définit maintenant une relation entre les parties de $\lambda$.

\begin{definition}
  Pour deux parties $X,Y\subseteq \lambda$, on définit la relation
  $\mathcal R$ par
  \[X \mathcal R Y\defeq \{\gamma\mid f_X(\gamma) < f_Y(\gamma)\}
  \text{ est stationnaire dans }\kappa\]
\end{definition}

\begin{property}
  Soient $X\neq Y$ deux parties de $\lambda$. Alors $X\mathcal R Y$ ou
  $Y\mathcal R X$.
\end{property}

\begin{proof}
  Comme $\{f_X\}$ est une famille éventuelle différente, on peut trouver
  $\kappa' < \kappa$ tel que $\{\xi \mid f_X(\xi) =f_Y(\xi)\}\subseteq\kappa'$
  donc $\{xi\mid f_X(\xi)\neq f_Y(\xi)\}$ contient un club (la partie
  $\kappa\setminus \kappa$). De plus on a l'égalité
  \[\{\xi \mid f_X(\xi) \neq f_Y(\xi)\} = \{\xi\mid f_X(\xi) < f_Y(\xi)\}
  \cup \{\xi\mid f_X(\xi) > f_Y(\xi)\}\]
  donc l'union des deux ensembles est un ensemble stationnaire. On en déduit
  qu'au moins l'un des deux ensembles est stationnaire, ce qui signifie que
  $X\mathcal R Y$ ou $Y\mathcal R X$.
\end{proof}

\begin{proposition}
  Il existe une partie $X\subseteq \lambda$ telle que
  $\Card(\mathcal R^{-1}[X]) \geq \lambda^+$.
\end{proposition}

\begin{proof}
  Soit $A\subseteq \powerset(\lambda)$ tel que $\Card(A) = \lambda^+$. On
  définit alors, pour tout $X\in A$~:
  \[A_X\defeq \mathcal R^{-1}[X]\cup \{X\}\]
  Il y a alors deux possibilités à traiter~:
  \begin{itemize}
  \item s'il existe $X\in A$ tel que $\Card(A_X)\geq \lambda^+$, alors en
    prenant ce tel $X$, on a bien prouvé l'affirmation souhaitée.
  \item sinon, alors on sait que pour tout $X\in A$, $\Card(A_X)< \lambda^+$,
    donc en utilisant le \cref{thm.Konig}, on en déduit que~:
    \[\Card\left(\bigcup_{X\in A}A_X\right) \leq \lambda^+\]
    donc, comme on a supposé que $2^\lambda > \lambda^+$, on peut trouver une
    partie
    \[X\in \powerset(\lambda)\setminus\left(\bigcup_{X\in A}A_X\right)\]
    Comme on sait que pour tous $X\neq Y$, $X\mathcal R Y$ ou $Y\mathcal R X$,
    le fait que $X\notin A_Y$ pour tout $Y\in A$ signifie que
    $\mathcal R^{-1}[X]\supseteq A$, donc que
    $\Card(\mathcal R^{-1}[X]) \geq\lambda^+$.
  \end{itemize}
  Dans les deux cas, on a donc trouvé $X$ tel que
  $\Card(\mathcal R^{-1}[X]) \geq \lambda^+$.
\end{proof}

\begin{notation}
  On fixe maintenant $X$ tel que défini dans la proposition précédente. De plus,
  pour tout $\gamma \in \kappa$, comme
  $f_X(\gamma) \in \aleph_{\alpha_\gamma + 1}$, on se donne une injection
  $g_\gamma : f_X(\gamma) \to \aleph_{\alpha_\gamma}$.

  Pour tout $Y\in\mathcal R^{-1}[X]$, on définit
  $S_Y = \{\gamma\mid f_Y(\gamma) < f_X(\gamma)\}$.
\end{notation}

On va maintenant introduire une variant du lemme de Fodor pour traiter notre
famille éventuellement différente.

\begin{lemma}[Fodor, variante]
  Soit $\alpha_\beta$ une suite cofinale continue dans $\alpha$. Soit
  $S\subseteq\kappa$ une partie stationnaire et $f : S \to \aleph_\alpha$ telle
  que
  \[\forall \beta < \kappa, f(\beta) < \aleph_{\alpha_\beta}\]
  On peut alors trouver $\gamma < \alpha$ et une partie $T\subseteq S$
  stationnaire telle que
  \[\forall \beta \in T, f(\beta) < \aleph_\gamma\]
\end{lemma}

\begin{proof}
  On commence par considérer
  \[S' \defeq S \cap \Lim_\kappa\]
  qui, comme $\Lim_\kappa$ est un club, est encore une partie stationnaire.
  Soit maintenant, pour $\beta \in S'$, l'ensemble
  \[A_\beta \defeq \{\gamma \mid f(\beta) < \aleph_{\alpha_\beta}\}\]
  On sait que pour tout $\beta \in A_\beta$, $f(\beta) < \aleph_{\alpha_\beta}$,
  donc $A_\beta$ est toujours non vide. On définit la fonction
  \[\begin{array}{ccccc}
  g & : & S' & \longrightarrow & \kappa\\
  & & \beta & \longmapsto & \min A_\beta
  \end{array}\]

  Montrons que $g$ est régressive pour pouvoir y appliquer le lemme de Fodor.
  Soit $\beta \in S'$, montrons que $g(\beta) < \beta$. $(\alpha_\beta)$ est
  une suite continue, et $\aleph$ est aussi continue (par définition), donc
  (comme $f(\beta) < \aleph_{\alpha_\beta}$) on en déduit que
  $f(\beta) < \sup_{\gamma < \beta} \aleph_{\alpha_\gamma}$
  On peut donc trouver $\delta \in \beta$ tel que
  $f(\beta) < \aleph_{\alpha_\delta}$, puisque sinon
  $f(\beta) \geq \sup_{\gamma < \beta} \aleph_{\alpha_\gamma}$.
  Il vient donc que $\delta < \beta$ et $\delta \in A_\beta$, donc
  $\min A_\beta < \beta$, d'où $g(\beta) < \beta$.

  On applique donc le lemme de Fodor à $g$~: on trouve un ensemble stationnaire
  $T\subseteq S$ et $\delta \in \kappa$ tel que
  \[\forall \beta \in T, g(\beta) = \delta\]
  ce qui signifie que $f(\beta) < \aleph_{\alpha_\delta}$. En prenant pour
  $\gamma$ notre $\alpha_\delta$ ainsi trouvé, on a bien une partie $T$
  vérifiant la condition attendue.
\end{proof}

En utilisant cette variante, on peut prouver le résultat suivant, qui nous
permettra finalement de conclure le théorème de Silver.

\begin{lemma}
  Pour tout $Y\in\mathcal R^{-1}[X]$, on peut trouver $T_Y\subseteq S_Y$
  stationnaire et $\overline\gamma_Y$ tels que
  \[\forall \gamma \in T_Y, g_\gamma(f_Y(\gamma)) <
  \aleph_{\overline\gamma_Y}\]
\end{lemma}

\begin{proof}
  Soit $Y\in\mathcal R^{-1}[X]$. On applique la variante du lemme de Fodor en
  considérant la suite cofinale continue $(\alpha_\beta)$, l'ensemble
  stationnaire $S_Y$ et la fonction $\gamma\mapsto g_\gamma(f_Y(\gamma))$.

  La fonction
  \[\begin{array}{ccccc}
  F_Y & : & S_Y & \longrightarrow & \lambda\\
  & & \gamma & \longmapsto & g_\gamma(f_Y(\gamma))
  \end{array}\]
  est bien définie car sur $S_Y, f_Y < f_X$. De plus,
  $g_\gamma : f_X(\gamma) \to \aleph_{\alpha_\gamma}$ donc $F_Y$ vérifie bien
  les condition de la variant du lemme de Fodor.

  On trouve donc $\overline\gamma_Y$ et $T_Y\subseteq S_Y$ tels que
  \[\forall \gamma \in T_Y, g_\gamma(f_Y(\gamma)) < \aleph_{\overline\gamma_Y}\]
\end{proof}

On généralise ce lemme pour avoir un $T$ et un $\overline\gamma$ indépendants
de $Y$.

\begin{lemma}
  Il existe $T\subseteq \kappa$ stationnaire et un ordinal $\overline\gamma$
  tels que
  \[\Card\left(\{Y\mid T_Y = T \land \overline\gamma_Y = \overline\gamma\right)
  \geq\lambda^+\]
\end{lemma}

\begin{proof}
  Soit la fonction
  \[\begin{array}{ccccc}
  h & : & \mathcal R^{-1}[X] & \longrightarrow & \powerset(\kappa)\times\alpha\\
  & & Y & \longmapsto & (T_Y,\overline\gamma_Y)
  \end{array}\]
  On peut donc écrire $\mathcal R^{-1}[X]$ comme l'union des
  $h^{-1}(\{(T,\overline\gamma)\})$ pour tous
  $(T,\overline\gamma)\in\powerset(\kappa)\times\alpha$. Par l'absurde,
  supposons que chacun de ces ensembles $h^{-1}(\{(T,\overline\gamma)\})$ sont
  de cardinal strictement inférieur à $\lambda^+$. Comme
  $\Card(\powerset(\kappa)) < \lambda^+$ et $\Card(\alpha) < \lambda^+$, on
  peut utiliser le \cref{thm.Konig} pour en déduire que
  \[\Card\left(
  \bigcup_{\genfrac{}{}{0pt}{2}{T\subseteq \kappa}{\overline\gamma\in \alpha}}
  h^{-1}(\{(T,\overline\gamma)\})\right) < \lambda^+\]
  ce qui contredit le fait que cette union vaut $\mathcal R^{-1}[X]$, qui est
  de cardinal supérieur ou égal à $\lambda^+$.

  On en déduit donc qu'il existe un certain $(T,\overline\gamma)$ tel que
  $h^{-1}(\{(T,\overline\gamma)\})$ est de taille $\lambda^+$, nous donnant le
  résultat attendu.
\end{proof}

\begin{theorem}[Silver \cite{Silver1981-SILOTS}]
  Soit $\lambda$ un cardinal singulier de cofinalité indénombrable. Si
  $\HCG_\lambda$ est vraie, alors $2^\lambda = \lambda^+$.
\end{theorem}

\begin{proof}
  On suppose donc que $2^\lambda > \lambda$ et on veut arriver à une
  contradiction, on reprend les constructions et notations précédentes.

  Soit
  \[\mathcal Y \defeq \{Y\subseteq \lambda\mid T_Y = T \land
  \overline\gamma_Y = \overline\gamma\}\]

  On commence par trouver $Y_1\neq Y_2$ deux éléments de $\mathcal Y$ tels que
  \[\forall \gamma \in T, g_\gamma(f_{Y_1}(\gamma)) = g_\gamma(f_{Y_2}(\gamma))\]

  Un argument de dénombrement suffit. En effet, la condition précédente revient
  à dire que $F_{Y_1} = F_{Y_2}$, et ces deux fonctions sont des éléments de
  $\Funct(T,\aleph_{\overline\gamma})$. On sait que
  $\Card(T) < \aleph_{\overline\gamma}$, donc
  \begin{align*}
    \Card(\Funct(S,\aleph_{\overline\gamma})) &
    \leq \aleph_{\overline\gamma}^{\aleph_{\overline\gamma}}\\
    &\leq 2^{\aleph_{\overline\gamma}} \\
    &\leq \aleph_{\overline\gamma + 1}
  \end{align*}
  Comme $\Card(\mathcal Y) = \lambda^+$, on en déduit qu'il existe au moins
  deux parties $Y_1,Y_2$ telles que $F_{Y_1} = F_{Y_2}$ et $Y_1\neq Y_2$.

  Maintenant, comme on sait que chaque $g_\gamma$ est injective, et qu'on sait
  que
  \[\forall \gamma\in T, g_\gamma(f_{Y_1}(\gamma)) = g_\gamma(f_{Y_2}(\gamma))\]
  on en déduit que $f_{Y_1}$ et $f_{Y_2}$ coïncident sur $T$. Pourtant, on sait
  que $\{\xi\mid f_{Y_1}(\xi) = f_{Y_2}(\xi)\}$ est de cardinal strictement
  inférieur à $\kappa$, et est donc non stationnaire. C'est absurde, donc
  notre hypothèse de départ, que $2^\lambda > \lambda^+$, est fausse. Comme
  $2^\lambda \leq \lambda^+$, on en déduit que $2^\lambda = \lambda^+$.
\end{proof}
