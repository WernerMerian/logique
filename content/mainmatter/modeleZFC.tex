\chapter{Modèles de ZFC}
\label{chp.modZFC}

\minitoc

Nous avons vu dans les chapitres précédents combien $\ZFC$ était une théorie
riche en terme d'expressivité, au point même où l'ensemble des mathématiques
peut se formaliser dans $\ZFC$. Il devient alors assez technique de parler de
modèle de $\ZFC$, puisqu'un tel modèle est en quelque sorte un modèle de toutes
les mathématiques que l'on connait.

Une première conséquence de cette expressivité est un certain flou entre ce que
l'on exprime dans $\ZFC$ et dans les maths usuelles. Puisque tout peut se
définir en terme d'ensembles, il semble que tout ce dont nous avons parlé jusque
là puisse se formaliser dans $\ZFC$. C'est le cas, mais il faut faire attention
à ce qu'on manipule.

En particulier, un modèle de $\ZFC$, nommons-le $\mathcal M$, contient lui-même
un ensemble $\Formula(\Sigma)$ pour une signature $\Sigma$ donnée. On pourrait
alors se dire que, comme on peut formaliser nos résultats de logique en terme
d'ensembles, on peut travailler au sein de $\ZFC$ pour faire de la logique.
Malheureusement, les propositions de $\Formula(\Sigma)$ ne sont pas les mêmes
que celles que nous connaissons~: on peut facilement interpréter une proposition
habituelle par un élément de $\Formula(\Sigma)$, mais l'inverse n'a pas de
raison d'être vrai.

Cela mène à deux résultats essentiels pour décrire les limitations de cette
approche formelle de la logique~: les théorèmes d'incomplétudes de Gödel et le
théorème d'indéfinissabilité de la vérité de Tarski, que nous verrons dans la
PARTIE TROIS. Ceux-ci ont pour conséquence que si l'on considère un codage
$\ceil -$ qui à une formule au sens habituel associe une formule de notre
modèle $\mathcal M$, alors il n'existe pas de formule $\varphi_\top$ telle que
\[\forall \psi \in \Formula(\Sigma), \mathcal M \models \psi
\iff \mathcal M\models \varphi_\top(\ceil \psi)\]

Une conséquence de cette limitation est que l'on ne peut pas à strictement
parler écrire une formule comme $\mathcal U\models \ZFC$ pour notre univers
ensembliste $\mathcal U$. Par contre, il est possible d'écrire par exemple
$\ZFC\vdash \varphi$, qui en utilisant le théorème de complétude permet de dire
que $\mathcal U \models \varphi$~: ainsi, l'écriture de $\mathcal U\models\ZFC$
se traduit en fait par une infinité de théorèmes de la forme
$\ZFC\vDash \varphi$ pour chaque axiome $\varphi \in \ZFC$.

S'il est assez clair que $\ZFC\vDash\ZFC$ est un constat assez vide, cela nous
permet ensuite d'introduire la relativisation, qui permet d'exprimer la notion
de modèle classe de $\ZFC$.

Ce chapitre s'attachera à l'étude des outils basiques à propos des modèles
classe. Nous verrons d'abord les définitions et conséquences de cette notion,
puis nous aborderons l'absoluité, qui permet d'étudier la conservation entre des
modèles de $\ZFC$.

Nous verrons ensuite la théorie des modèles intérieurs, et son exemple le plus
connu~: l'univers constructible de Gödel $\mathbb L$.

\section{Premiers pas}

\subsection{Relativisation}

On commence par introduire la notion de relativisation d'une formule. Comme on
l'a dit, il est impossible de représenter de façon interne à un modèle de $\ZFC$
que celui-ci vérifie une formule~: il faut donc parler de façon externe au
modèle. Pour exprimer que $\varphi$ est vraie dans l'univers ambiant, il suffit
alors de travailler normalement, en utilisant les axiomes de $\ZFC$. Cependant,
cette approche est très limitée, et on aimerait s'autoriser à parler d'autres
classes que l'univers $\mathcal U$ tout entier. Pour ce faire, comme on doit
procéder de façon purement syntaxique, on modifie la formule même qu'on veut
prouver en y incluant l'information de la classe qu'on utilise.

Disons qu'on veuille montrer qu'une formule $\varphi$ est vraie dans la classe
des ordinaux, l'idée est simplement de modifier tous les quantificateurs pour
que ceux-ci ne parlent que d'éléments de $\Ord$. La relativisation est
précisément ce procédé de réduire la portée des quantificateurs par une formule,
exprimant en général l'appartenance à une classe qui nous intéresse.

\begin{definition}[Relativisation d'une formule]
  Soient $\varphi,\psi\in\Formula(\mathcal L_{\ZF})$. On appelle relativisation
  de $\varphi$ à $\psi$ la formule $\varphi^\psi$ définie par induction sur
  $\varphi$ par~:
  \begin{itemize}
  \item si $\varphi = x \in y$ ou $\varphi = x \in y$ alors
    $\varphi^\psi = \varphi$.
  \item si $\varphi = \top$ alors $\varphi^\psi = \top$.
  \item si $\varphi = \bot$ alors $\varphi^\psi = \bot$.
  \item si $\varphi = \lnot \chi$ alors $\varphi^\psi = \lnot \chi^\psi$.
  \item si $\varphi = \chi \lor \xi$ alors
    $\varphi^\psi = \chi^\psi \lor \xi^\psi$.
  \item si $\varphi = \chi \land \xi$ alors
    $\varphi^\psi = \chi^\psi \land \xi^\psi$.
  \item si $\varphi = \chi \to \xi$ alors
    $\varphi^\psi = \chi^\psi \to \xi^\psi$.
  \item si $\varphi = \exists x, \chi$ alors
    $\varphi^\psi = \exists x, \psi(x) \land \chi^\psi$.
  \item si $\varphi = \forall x, \chi$ alors
    $\varphi^\psi = \forall x, \psi(x) \to \chi^\psi$.
  \end{itemize}
\end{definition}

Le théorème central pour l'étude des modèles classes est que la cohérence est
préservée par ceux-ci.

\begin{lemma}
  Pour $\Gamma\in\List(\Formula(\Sigma))$ et $\varphi,\psi\in\Formula(\Sigma)$,
  si $\Gamma\vdash \varphi$ alors $\Gamma^\psi \vdash \varphi^\psi$ où
  $\Gamma^\psi$ l'application de la relativisation à $\psi$ de tous les éléments
  de $\Gamma$.
\end{lemma}

\begin{proof}
  A FAIRE
\end{proof}

\begin{theorem}[Préservation de la cohérence]
  Soit $\mathcal T$ et $\mathcal S$ deux théories sur $\mathcal L_{\ZF}$ et
  $\psi(x)$ une formule à une variable libre dans ce même langage telles que
  pour toute formule
  $\varphi \in \mathcal S$, $\mathcal T\vdash \varphi^\psi$. Alors
  \[\Coher(\mathcal T) \implies \Coher(\mathcal S)\]
\end{theorem}

\begin{proof}
  On donne deux preuves, l'une syntaxique et l'autre sémantique.
  \begin{itemize}
  \item supposons $\lnot\Coher(\mathcal S)$, cela signifie donc qu'il existe une
    preuve $\mathcal S\vdash \bot$, et comme cette preuve est finie on peut
    extraire une partie finie $F\subfin \mathcal S$ telle que $F\vdash \bot$.
    En utilisant le lemme précédent, il vient donc que $F^\psi\vdash \bot^\psi$
    mais $\bot^\psi = \bot$, donc on en déduit que $F^\psi\vdash \bot$. Comme
    $\mathcal T\vdash \varphi^\psi$ pour chaque $\varphi \in F$, on en déduit
    que $\mathcal T\vdash \bot$, donc que $\lnot\Coher(\mathcal T)$. Par
    contraposée, on en déduit l'implication voulue.
  \item supposons $\Coher(\mathcal T)$. On trouve donc un modèle $\mathcal M$
    de $\mathcal T$. Soit $\mathcal N$ le sous-modèle de $\mathcal M$ défini
    par
    \[|\mathcal N|\defeq \{x \mid x\in |\mathcal M|,
    \mathcal M\models \psi(x)\}\]
    Par hypothèse, on sait que $\mathcal M\models \mathcal S^\psi$, donc
    $\mathcal N\models \mathcal S$, d'où $\Coher(\mathcal S)$.
  \end{itemize}

  Les deux approches nous donnent donc que
  $\Coher(\mathcal T) \implies \Coher(\mathcal S)$.
\end{proof}

On peut par exemple appliquer ce théorème à $\ZFC - \AxF$ et $\ZFC$ pour dire
que s'il existe une classe $\mathbb V$ telle que
$\ZFC - \AxF\vdash \ZFC^\mathbb V$, alors les deux théories sont équicohérentes
(puisque dans l'autre sens, il est évident qu'enlever des axiomes ne change pas
la cohérence).

\subsection{L'univers bien fondé de Von Neumann}

On peut maintenant introduire l'univers $\mathbb V$, qui est d'une certaine
façon l'univers ensembliste au sens intuitif~: il est un modèle de $\ZFC$
présent dans tout modèle de $\ZFC - \AxF$.

\begin{definition}[L'univers bien fondé]
  On définit la hiérarchie d'ensembles $\mathbb V_{\alpha}$, pour
  $\alpha \in \Ord$, par induction transfinie~:
  \begin{itemize}
  \item $\mathbb V_0 = \varnothing$.
  \item $\mathbb V_{\alpha + 1} = \powerset(\mathbb V_{\alpha})$
  \item pour tout ordinal limite $\lambda$,
    $\displaystyle\mathbb V_\lambda = \bigcup_{\beta < \lambda}\mathbb V_\beta$
  \end{itemize}

  On définit maintenant la classe $\mathbb V$ par
  \[\mathbb V \defeq \bigcup_{\alpha \in \Ord} \mathbb V_\alpha\]
\end{definition}

\begin{remark}
  Par une union indicée par $\alpha \in \Ord$, il faut en fait comprendre que le
  prédicat $\mathbb V(x)$ est défini comme
  \[\mathbb V(x) \defeq \exists \alpha \in \Ord, \mathbb V(\alpha,x)\]
  On préfère une approche sémantique que syntaxique ici, car celle-ci est bien
  plus facile à appréhender.
\end{remark}

Notre premier théorème à propos de l'univers bien fondé est le suivant~: il est
un modèle de $\ZFC$.

\begin{theorem}
  Pour tout $\varphi \in \ZFC$, on a $\ZFC - \AxF \vdash \varphi^\mathbb V$.
\end{theorem}

\begin{proof}
  A FAIRE
\end{proof}

On a donc notre premier modèle classe de $\ZFC$. Un point important lorsque l'on
ne travaille pas \emph{a priori} avec l'axiome de fondation est que la classe
$\mathbb V$ correspond exactement à la classe des éléments bien fondés de
l'univers ambiant.

\begin{proposition}
  Soit un ensemble $x$, alors on a l'équivalence suivante dans $\ZFC-\AxF$~:
  \[x\in \mathbb V \iff \exists y \in \trcl(x), \trcl(x) \cap y = \varnothing\]
\end{proposition}

\begin{proof}
  A FAIRE
\end{proof}

On voit donc que dans le cas d'un univers ensembliste $\mathcal U$, celui-ci
vérifie $\AxF$ si et seulement si $\mathbb V = \mathcal U$. On utilisera donc en
général $\mathbb V$ pour parler de l'univers ensembliste.
