\chapter{Modèles de ZFC}
\label{chp.modZFC}

\minitoc

\lettrine{N}{ous} avons vu dans les chapitres précédents combien $\ZFC$ était
une théorie riche en terme d'expressivité, au point même où l'ensemble des
mathématiques peut se formaliser dans $\ZFC$. Il devient alors assez technique
de parler de modèle de $\ZFC$, puisqu'un tel modèle est en quelque sorte un
modèle de toutes les mathématiques que l'on connait.

Une première conséquence de cette expressivité est un certain flou entre ce que
l'on exprime dans $\ZFC$ et dans les maths usuelles. Puisque tout peut se
définir en terme d'ensembles, il semble que tout ce dont nous avons parlé jusque
là puisse se formaliser dans $\ZFC$. C'est le cas, mais il faut faire attention
à ce qu'on manipule.

En particulier, un modèle de $\ZFC$, nommons-le $\mathcal M$, contient lui-même
un ensemble $\Formula(\Sigma)$ pour une signature $\Sigma$ donnée. On pourrait
alors se dire que, comme on peut formaliser nos résultats de logique en terme
d'ensembles, on peut travailler au sein de $\ZFC$ pour faire de la logique.
Malheureusement, les propositions de $\Formula(\Sigma)$ ne sont pas les mêmes
que celles que nous connaissons~: on peut facilement interpréter une proposition
habituelle par un élément de $\Formula(\Sigma)$, mais l'inverse n'a pas de
raison d'être vrai.

Cela mène à deux résultats essentiels pour décrire les limitations de cette
approche formelle de la logique~: les théorèmes d'incomplétudes de Gödel et le
théorème d'indéfinissabilité de la vérité de Tarski, que nous verrons dans la
PARTIE QUATRE. Ceux-ci ont pour conséquence que si l'on considère un codage
$\ceil -$ qui à une formule au sens habituel associe une formule de notre
modèle $\mathcal M$, alors il n'existe pas de formule $\varphi_{\mathcal M}$
telle que
\[\forall \psi \in \Formula(\Sigma), \mathcal M \models \psi
\iff \mathcal \varphi_{\mathcal M}(\ceil \psi)\]

Cette limitation implique que l'on ne peut pas à strictement
parler écrire une formule comme $\mathcal U\models \ZFC$ pour notre univers
ensembliste $\mathcal U$. Par contre, il est possible d'écrire par exemple
$\ZFC\vdash \varphi$, qui en utilisant le théorème de complétude permet de dire
que $\mathcal U \models \varphi$~: ainsi, l'écriture de $\mathcal U\models\ZFC$
se traduit en fait par une infinité de théorèmes de la forme
$\ZFC\vDash \varphi$ pour chaque axiome $\varphi \in \ZFC$.

S'il est assez clair que $\ZFC\vDash\ZFC$ est un constat assez vide, cela nous
permet ensuite d'introduire la relativisation, qui permet d'exprimer la notion
de modèle classe de $\ZFC$.

Ce chapitre s'attachera à l'étude des outils basiques à propos des modèles
classe. Nous verrons d'abord les définitions et conséquences de cette notion,
puis nous aborderons l'absoluité, qui permet d'étudier la conservation entre des
modèles de $\ZFC$.

Nous verrons ensuite la théorie des modèles intérieurs, et son exemple le plus
connu~: l'univers constructible de Gödel $\mathbb L$.

\section{Premiers pas}

\subsection{Relativisation}

On commence par introduire la notion de relativisation d'une formule. Comme on
l'a dit, il est impossible de représenter de façon interne à un modèle de $\ZFC$
que celui-ci vérifie une formule~: il faut donc parler de façon externe au
modèle. Pour exprimer que $\varphi$ est vraie dans l'univers ambiant, il suffit
alors de travailler normalement, en utilisant les axiomes de $\ZFC$. Cependant,
cette approche est très limitée, et on aimerait s'autoriser à parler d'autres
classes que l'univers $\mathcal U$ tout entier. Pour ce faire, comme on doit
procéder de façon purement syntaxique, on modifie la formule même qu'on veut
prouver en y incluant l'information de la classe qu'on utilise.

Disons qu'on veuille montrer qu'une formule $\varphi$ est vraie dans la classe
des ordinaux, l'idée est simplement de modifier tous les quantificateurs pour
que ceux-ci ne parlent que d'éléments de $\Ord$. La relativisation est
précisément ce procédé de réduire la portée des quantificateurs par une formule,
exprimant en général l'appartenance à une classe qui nous intéresse.

\begin{definition}[Relativisation d'une formule]
  Soient $\varphi,\psi\in\Formula(\mathcal L_{\ZF})$ où $\psi$ possède une
  variable libre. On appelle relativisation de $\varphi$ à $\psi$ la formule
  $\varphi^\psi$ définie par induction sur $\varphi$ par~:
  \begin{itemize}
  \item si $\varphi$ est atomique alors $\varphi^\psi = \varphi$.
  \item si $\varphi = \top$ alors $\varphi^\psi = \top$.
  \item si $\varphi = \bot$ alors $\varphi^\psi = \bot$.
  \item si $\varphi = \lnot \chi$ alors $\varphi^\psi = \lnot \chi^\psi$.
  \item si $\varphi = \chi \lor \xi$ alors
    $\varphi^\psi = \chi^\psi \lor \xi^\psi$.
  \item si $\varphi = \chi \land \xi$ alors
    $\varphi^\psi = \chi^\psi \land \xi^\psi$.
  \item si $\varphi = \chi \to \xi$ alors
    $\varphi^\psi = \chi^\psi \to \xi^\psi$.
  \item si $\varphi = \exists x, \chi$ alors
    $\varphi^\psi = \exists x, \psi(x) \land \chi^\psi$.
  \item si $\varphi = \forall x, \chi$ alors
    $\varphi^\psi = \forall x, \psi(x) \to \chi^\psi$.
  \end{itemize}
\end{definition}

Le théorème central pour l'étude des modèles classes est que la cohérence est
préservée par ceux-ci.

\begin{lemma}\label{lem.relat.deduc}
  Pour $\Gamma\in\List(\Formula(\mathcal L_{\ZF}))$ et
  $\varphi,\psi\in\Formula(\mathcal L_{\ZF})$, si $\Gamma\vdash \varphi$ alors
  $\Gamma^\psi \vdash \varphi^\psi$ où $\Gamma^\psi$ est l'application de la
  relativisation à $\psi$ de tous les éléments de $\Gamma$, augmentée de la
  formule $\psi(x)$ pour toute variable $x\in \VL(\Gamma)\cup \VL(\varphi)$.
\end{lemma}

\begin{proof}
  On procède par induction sur la relation $\Gamma \vdash \varphi$, en
  considérant une variant de $\vdash$ où l'on tient compte des variables libres
  dans les séquents~:
  \begin{itemize}
  \item cas Ax~: si $\varphi \in \Gamma$, alors $\varphi^\psi \in \Gamma^\psi$,
    d'où le résultat.
  \item cas $\top$~: peu importe $\Gamma$, on a toujours
    $\Gamma^\psi \vdash \top$.
  \item cas $\bot_\mathrm c$~: on suppose que
    $\Gamma^\psi, \lnot A^\psi \vdash \bot$, alors on peut appliquer
    $\bot_\mathrm c$ pour en déduire que $\Gamma^\psi \vdash A^\psi$.
  \item cas $\lor_\mathrm i^\mathrm g$~: on suppose que
    $\Gamma^\psi \vdash A^\psi$, donc $\Gamma^\psi \vdash A^\psi\lor B^\psi$,
    c'est-à-dire $\Gamma^\psi \vdash (A\lor B)^\psi$.
  \item le cas $\lor_\mathrm i^\mathrm d$ est analogue.
  \item cas $\lor_\mathrm e$~: on suppose que
    $\Gamma^\psi\vdash A^\psi\lor B^\psi$, $\Gamma^\psi, A^\psi \vdash C^\psi$ et
    $\Gamma^\psi,B^\psi\vdash C^\psi$, alors on peut appliquer la règle
    $\lor_\mathrm e$ pour en déduire que $\Gamma^\psi \vdash C^\psi$.
  \item cas $\land_\mathrm i$~: on suppose que $\Gamma^\psi\vdash A^\psi$ et
    $\Gamma^\psi \vdash B^\psi$, alors $\Gamma^\psi \vdash A^\psi \land B^\psi$.
  \item cas $\land_\mathrm e^\mathrm g$~: on suppose que
    $\Gamma^\psi\vdash (A\land B)^\psi$, c'est-à-dire que
    $\Gamma^\psi\vdash A^\psi \land B^\psi$, alors on peut en déduire que
    $\Gamma^\psi \vdash A^\psi$.
  \item le cas $\land_\mathrm e^\mathrm d$ est analogue.
  \item cas $\to_\mathrm i$~: si $\Gamma^\psi,A^\psi\vdash B^\psi$ alors
    $\Gamma^\psi\vdash A^\psi \to B^\psi$, et par définition
    $A^\psi\to B^\psi = (A\to B)^\psi$, d'où le résultat.
  \item cas $\to_\mathrm e$~: si $\Gamma^\psi \vdash (A\to B)^\psi$ et
    $\Gamma^\psi \vdash A^\psi$, alors $\Gamma^\psi\vdash B^\psi$.
  \item cas $\forall_\mathrm i$~: si $\Gamma^\psi\vdash A[v/x]^\psi$, où $v$ est
    libre dans $\Gamma$ et $A$, alors par affaiblissement on sait que
    $\Gamma^\psi, \psi(v)\vdash A[v/x]^\psi$, donc que
    $\Gamma^\psi\vdash \psi(v)\to A[v/x]^\psi$, d'où par la règle
    $\forall_\mathrm i$ le fait que $\Gamma^\psi \vdash (\forall x, A)^\psi$.
  \item cas $\forall_\mathrm e$~: si $\Gamma^\psi \vdash (\forall x,A)^\psi$, et
    pour une variable $y \in \VL(\Gamma)\cup \VL(A)$ (puisque tout terme dans
    $\mathcal L_{\ZF}$ est une variable et que la relativisation n'ajoute pas
    de variable libre), on peut spécialiser notre séquent en
    $\Gamma^\psi \vdash \psi(y)\to A[y/x]$, puis utiliser l'hypothèse
    $\psi(y)\in\Gamma^\psi$ pour en déduire que $\Gamma^\psi \vdash A[t/x]$.
  \item cas $\exists_\mathrm i$~: si $\Gamma^\psi \vdash A[y/x]$ alors on sait
    que $\Gamma^\psi\vdash \psi(y)$, donc
    $\Gamma^\psi\vdash \psi(y)\land A[y/x]$, d'où
    $\Gamma^\psi \vdash (\exists x, A)^\psi$.
  \item cas $\exists_\mathrm e$~: si $\Gamma^\psi \vdash (\exists x, A)^\psi$ et
    $\Gamma^\psi, \psi(x) \land A^\psi \vdash B^\psi$, alors la règle
    $\lor_\mathrm e$ nous donne directement $\Gamma^\psi \vdash B^\psi$. On
    remarque qu'on peut effectivement utiliser l'hypothèse d'induction sur
    $\Gamma^\psi, \psi(x)\land A^\psi\vdash B^\psi$ car ce séquent est équivalent
    à $\Gamma^\psi, \psi(x), A^\psi \vdash B^\psi$, qui contient bien $\psi(a)$
    pour chaque $a \in \VL(\Gamma)\cup\VL(A)$.
  \item cas $=_\mathrm i$~: si $t$ est un terme, alors $\Gamma^\psi\vdash t = t$
  \item cas $=_\mathrm e$~: si $\Gamma^\psi \vdash t = u$ et
    $\Gamma^\psi \vdash A[t/x]^\psi$, alors on peut vérifier par induction sur
    $A$ que $(A[t/x])^\psi = A^\psi [t/x]$, donc on peut directement appliquer
    $=_\mathrm e$ pour obtenir le résultat.
  \end{itemize}
  Ainsi, par induction, on sait que $\Gamma^\psi \vdash \varphi^\psi$.
\end{proof}

\begin{remark}
  La preuve ci-dessus n'est pas parfaitement rigoureuse, principalement à cause
  du choix fait plus tôt de ne pas détailler les règles de déduction naturelle
  prenant en compte des variables. Pour être plus précis, étant donné un séquent
  $\Gamma\vdash \varphi$ bien typé en considérant une liste de variables
  $x_1,\ldots,x_n$, on veut montrer que
  $\Gamma^\psi, \psi(x_1),\ldots,\psi(x_n) \vdash \varphi^\psi$. Sans gestion des
  variables du premier ordre, ce lemme serait faux~: en prenant pour $\psi$ un
  prédicat toujours faux, tel que $x\neq x$, on peut prouver
  $\vdash (\forall x, A)\to (\exists x, A)$ mais pas
  $\vdash (\forall x,A)^\psi \to (\exists x, A)^\psi$ puisque le $x$ invoqué par
  la première formule ne vérifie pas $\psi$. Dans un tel cas, en ajoutant la
  gestion des variables, on a une hypothèse absurde donc la conclusion est
  forcément vraie.
\end{remark}

\begin{remark}
  Il est possible de généraliser ce lemme à une signature $\Sigma$ quelconque.
  Cependant, pour cela, il est nécessaire que $\psi$ vérifie une propriété de
  stabilité~: il faut que pour chaque symbole de fonction $f$ d'arité $n$,
  on puisse prouver
  \[\vdash\forall x_1,\ldots,x_n, \left(\bigwedge_{i = 1}^n \psi(x_i)\right)
  \to \psi(f(x_1,\ldots,x_n))\]
  Ce faisant, on peut montrer que tout terme vérifie $\psi$ (en supposant que
  les variables vérifient $\psi$), et ainsi vérifier les parties de l'induction
  liées à l'élimination du $\forall$ et l'introduction du $\exists$.
\end{remark}

\begin{theorem}[Préservation de la cohérence]
  Soit $\mathcal T$ et $\mathcal S$ deux théories sur $\mathcal L_{\ZF}$ et
  $\psi(x)$ une formule à une variable libre dans ce même langage telles que
  pour toute formule
  $\varphi \in \mathcal S$, $\mathcal T\vdash \varphi^\psi$. Alors
  \[\Coher(\mathcal T) \implies \Coher(\mathcal S)\]
\end{theorem}

\begin{proof}
  On donne deux preuves, l'une syntaxique et l'autre sémantique.
  \begin{itemize}
  \item supposons $\lnot\Coher(\mathcal S)$, cela signifie donc qu'il existe une
    preuve $\mathcal S\vdash \bot$, et comme cette preuve est finie on peut
    extraire une partie finie $F\subfin \mathcal S$ telle que $F\vdash \bot$.
    En utilisant le \cref{lem.relat.deduc}, il vient donc que
    $F^\psi\vdash \bot^\psi$ (comme les énoncés sont clos, on n'a pas à rajouter
    de $\psi(x)$ pour des variables $x$). Or, $\bot^\psi = \bot$, donc on en
    déduit que $F^\psi\vdash \bot$. Comme $\mathcal T\vdash \varphi^\psi$ pour
    chaque $\varphi \in F$, on en déduit que $\mathcal T\vdash \bot$, donc que
    $\lnot\Coher(\mathcal T)$. Par contraposée, on en déduit l'implication
    voulue.
  \item supposons $\Coher(\mathcal T)$. On trouve donc un modèle $\mathcal M$
    de $\mathcal T$. Soit $\mathcal N$ le sous-modèle de $\mathcal M$ défini
    par
    \[|\mathcal N|\defeq \{x \mid x\in |\mathcal M|,
    \mathcal M\models \psi(x)\}\]
    Par hypothèse, on sait que $\mathcal M\models \mathcal S^\psi$, donc
    $\mathcal N\models \mathcal S$, d'où $\Coher(\mathcal S)$.
  \end{itemize}

  Les deux approches nous donnent donc que
  $\Coher(\mathcal T) \implies \Coher(\mathcal S)$.
\end{proof}

On peut par exemple appliquer ce théorème à $\ZFC - \AxF$ et $\ZFC$ pour dire
que s'il existe une classe $\mathbb V$ telle que
$\ZFC - \AxF\vdash \ZFC^\mathbb V$, alors les deux théories sont équicohérentes
(puisque dans l'autre sens, il est évident qu'enlever des axiomes ne change pas
la cohérence).

\subsection{L'univers bien fondé de Von Neumann}

On peut maintenant introduire l'univers $\mathbb V$, qui est d'une certaine
façon l'univers ensembliste au sens intuitif~: il est un modèle de $\ZFC$
présent dans tout modèle de $\ZFC - \AxF$.

Historiquement, même si le nom de cet univers est attribué à Von Neumann,
il semble (d'après \cite{moore2012zermelo}) que celui-ci date des travaux de 
Zermelo, notamment \cite{Zermelo1930}.

\begin{definition}[L'univers bien fondé]
  On définit la hiérarchie d'ensembles $\mathbb V_{\alpha}$, pour
  $\alpha \in \Ord$, par induction transfinie~:
  \begin{itemize}
  \item $\mathbb V_0 = \varnothing$.
  \item $\mathbb V_{\alpha + 1} = \powerset(\mathbb V_{\alpha})$
  \item pour tout ordinal limite $\lambda$,
    $\displaystyle\mathbb V_\lambda = \bigcup_{\beta < \lambda}\mathbb V_\beta$
  \end{itemize}

  On définit maintenant la classe $\mathbb V$ par
  \[\mathbb V \defeq \bigcup_{\alpha \in \Ord} \mathbb V_\alpha\]
\end{definition}

\begin{remark}
  Par une union indicée par $\alpha \in \Ord$, il faut en fait comprendre que le
  prédicat $\mathbb V(x)$ est défini comme
  \[\mathbb V(x) \defeq \exists \alpha \in \Ord, \mathbb V(\alpha,x)\]
  On préfère une approche sémantique que syntaxique ici, car celle-ci est bien
  plus facile à appréhender.
\end{remark}

Un premier résultat important~: chaque $\mathbb V_\alpha$ est transitif, et
$\mathbb V$ tout entier l'est donc aussi.

\begin{property}
  Pour chaque $\alpha\in\Ord$, $\mathbb V_\alpha$ est un ensemble transitif. De
  plus, $\mathbb V$ est une classe transitive.
\end{property}

\begin{proof}
  On raisonne par induction transfinie~:
  \begin{itemize}
  \item si $x\in \varnothing$, alors $x\subseteq\varnothing$.
  \item supposons que $\mathbb V_\alpha$ est transitif. Soit
    $x\in\mathbb V_{\alpha+1}$. Par définition, cela signifie que
    $x\subseteq \mathbb V_\alpha$. Soit $y\in x$~: par inclusion,
    $y\in \mathbb V_\alpha$, donc (comme $\mathbb V_\alpha$ est transitif)
    $y\subseteq \mathbb V_\alpha$, c'est-à-dire $y\in \mathbb V_{\alpha + 1}$,
    donc $x\subseteq \mathbb V_{\alpha + 1}$.
  \item si pour tous $\beta < \lambda$, $\mathbb V_\beta$ est transitif, alors
    pour $x\in \mathbb V_\lambda$, on trouve $\alpha < \lambda$ tel que
    $x\in V_\alpha$, donc $x\subseteq V_\alpha$. De plus,
    $V_\alpha \subseteq V_\lambda$ donc $x\subseteq V_\lambda$.
  \end{itemize}
  Ainsi tous les $\mathbb V_\alpha$ sont transitifs. Le fait que $\mathbb V$ est
  une classe transitive est une conséquence directe.
\end{proof}

Une conséquence première est que la hiérarchie est cumulative, ce qui signifie
que les ensembles sont de plus en plus grands.

\begin{property}
  Si $\alpha \leq \beta$, alors $\mathbb V_\alpha\subseteq \mathbb V_\beta$.
\end{property}

\begin{proof}
  On prouve ce résultat par induction sur l'ordinal $\gamma$ tel que
  $\alpha + \gamma = \beta$~:
  \begin{itemize}
  \item pour tout $\alpha$, $\mathbb V_\alpha \subseteq \mathbb V_\alpha$.
  \item supposons que $\mathbb V_\alpha \subseteq \mathbb V_\beta$, alors
    montrons que $\mathbb V_\alpha \subseteq \mathbb V_\beta$~: il nous suffit
    pour cela de montrer que pour tout $\beta$,
    $\mathbb V_\beta \subseteq \mathbb V_{\beta + 1}$. Si $x \in \mathbb V_\beta$,
    alors comme $\mathbb V_\beta$ est transitif, $x\subseteq \mathbb V_\beta$,
    donc $x\in \mathbb V_{\beta + 1}$ par définition.
  \item supposons que pour tout $\delta < \gamma$ et $\gamma$ ordinal limite,
    $\mathbb V_\alpha \subseteq \mathbb V_{\alpha + \delta}$. Alors
    comme
    \[\mathbb V_{\alpha + \gamma} =
    \bigcup_{\delta < \gamma} \mathbb V_{\alpha + \delta}\]
    on en déduit que $\mathbb V_\alpha \subseteq \mathbb V_{\alpha + \gamma}$.
  \end{itemize}
  D'où le résultat par induction transfinie.
\end{proof}

Notre premier objectif est de montrer que $\mathbb V$ constitue un premier
modèle classe de $\ZFC$. Pour cela, on donne d'abord un lemme essentiel pour
prouver de tels axiomes.

\begin{lemma}\label{lem.axZFC}
  Soit $M$ une classe, alors on a les propriétés suivantes~:
  \begin{itemize}
  \item si $M$ est une classe transitive, alors $M$ vérifie l'axiome
    d'extensionalité.
  \item si $M$ est clos par sous-ensemble, c'est-à-dire si pour tout ensemble
    $x \in M$ et $y \subseteq x$, on a $y \in M$, alors $M$ vérifie l'axiome
    de compréhension.
  \item si pour tout $x,y \in M$, $\{x,y\} \in M$ alors $M$ vérifie l'axiome de
    la paire.
  \item si pour tout $x \in M, \bigcup x \in M$ alors $M$ vérifie l'axiome de la
    réunion.
  \item si $M$ est une classe transitive vérifiant la proposition suivante~:
    pour toute fonction partielle $f$ (non obligatoirement dans $M$) telle que
    $\dom(f) \in M$ et $\im(f) \subseteq M$, on a $\im(f) \in M$ ; alors $M$
    vérifie l'axiome de remplacement.
  \end{itemize}
\end{lemma}

\begin{proof}
  On prouve chaque propriété~:
  \begin{itemize}
  \item soient $x,y \in M$ deux ensembles tels que
    $\forall z \in M, z \in x \iff z \in y$. Alors pour tout $z \in x$, comme
    $M$ est transitive, $z \in M$, donc par notre hypothèse $z \in y$, donc
    $x \subseteq y$. L'argument fonctionne symétriquement pour nous assurer que
    $y \subseteq x$, donc $x = y$.
  \item supposons que $M$ est clos par sous-ensemble. Soient $a_1,\ldots,a_n,a$
    des éléments de $M$ et $\psi(x_1,\ldots,x_n,a)$ une formule sur
    $\mathcal L_{\ZF}$. Alors on sait que
    $\{ y \in a \mid \psi(a_1,\ldots,a_n,y)\} \subseteq a \in M$,
    donc il existe bien dans $M$ un ensemble $A$ tel que
    $y \in A \iff \psi(a_1,\ldots,a_n,y)$.
  \item soient $x,y \in M$, alors $\{x,y\} \in M$, donc il existe dans $M$ un
    ensemble $A$ tel que $z \in A \iff z = x \lor z = y$.
  \item de même, si $M$ est stable par réunion, on a directement un candidat
    pour être l'union d'un ensemble $x$, donné par $\bigcup x$ lui-même~: on
    voit qu'un élément $z$ est dans $\bigcup x$ si et seulement s'il existe
    $y$ tel que $z \in y \in x$.
  \item soit une formule fonctionnelle $\psi^M(x_1,\ldots,x_n,x,y)$,
    $a_1,\ldots,a_n,a \in M$ des paramètrs. Par l'axiome de remplacement (dans
    l'univers ambiant), on sait qu'il existe un ensemble $b$, image de $a$
    par $\psi^M(a_1,\ldots,a_n)$, et une fonction $f : a \to b$ (partielle).
    Alors, par le \cref{lem.relat.deduc}, pour tout $y \in b$, comme on sait que
    $\psi^M(a_1,\ldots,a_n,x,y)$ pour un certain $x \in a$ (donc $x \in M$
    d'après le fait que $M$ est transitif), on en déduit que $y \in M$.
    Ainsi, $f : a \to b$, dont le domaine est $a \in M$, a une image
    $b \subseteq M$. Par notre hypothèse, $b \in M$, donc l'axiome de
    remplacement est vérifié.
  \end{itemize}
  D'où les propriétés énoncées.
\end{proof}

\begin{theorem}\label{thm.V.ZFCAF}
  Pour tout $\varphi \in \ZFC$, on a $\ZFC - \AxF \vdash \varphi^\mathbb V$.
\end{theorem}

\begin{proof}
  On vérifie chaque axiome relativisé à l'aide de \cref{lem.axZFC}~:
  \begin{itemize}
  \item vérifions l'axiome d'extensionalité. Soient $x,y\in \mathbb V$ tels que
    $\forall z\in \mathbb V, z\in x\iff z \in y$. Comme $\mathbb V$ est une
    classe transitive, pour tous $z\in x, z\in \mathbb V$ et de même pour les
    éléments de $y$. On en déduit que $\forall z, z\in x \iff z \in y$, donc
    l'axiome d'extensionalité peut s'appliquer directement pour conclure que
    $x = y$.
  \item vérifions l'axiome de la réunion~: si $x\in \mathbb V$ alors on trouve
    $\alpha$ tel que $x \in \mathbb V_\alpha$. On sait que $\mathbb V_\alpha$ est
    transitif, donc pour tout $y \in x$, $y \in \mathbb V_\alpha$ et, pour tout
    $z \in y \in x$, $z \in \mathbb V_\alpha$~: on en déduit que $\bigcup x$
    est une partie de $\mathbb V_\alpha$, donc
    $\bigcup x\in \mathbb V_{\alpha + 1}$.
  \item vérifions l'axiome de l'ensemble des parties~: soit $x\in \mathbb V$,
    on trouve $\alpha$ tel que $x \in\mathbb V_\alpha$. Alors pour tous
    $y\subseteq x$, $y\subseteq \mathbb V_\alpha$ donc
    $y\in\mathbb V_{\alpha +1}$. On en déduit que pour tous $y\in \powerset(x)$,
    $y\in \mathbb V$, donc $\powerset(x)\in\mathbb V$.
  \item vérifions l'axiome de remplacement~: soit $f$ une fonction partielle
    dont le domaine est un ensemble $x \in \mathbb V_\alpha$, et dont l'image
    est $y \subseteq \mathbb V$. On peut alors considérer pour chaque $z \in y$
    un $\beta_z$ tel que $z \in \mathbb V_{\beta_z}$. Alors en prenant
    $\beta = \sup_{z\in y} \beta_z$, on a $y \subseteq \mathbb V_\beta$, donc
    $y\in \mathbb V_{\beta + 1}$ par définition, donc $y \in \mathbb V$.
  \item vérifions l'axiome de l'infini~: on sait que $\Ord\subseteq\mathbb V$,
    donc il existe bien un ensemble $\omega^\mathbb V \in \mathbb V$
    contenant $0$ et stable par successeur, c'est $\omega$ lui-même.
  \item vérifions l'axiome de fondation~: par l'absurde, supposons qu'il existe
    un élément $x \in \mathbb V$ non vide tel que pour tout $y \in x$
    $x \cap y \neq \varnothing$. Soit $\alpha$ l'ordinal minimal tel que
    $x \in \mathbb V_\alpha$. On sait que pour tout $y \in x$, il existe
    $\beta < \alpha$ tel que $y \in \mathbb V_\beta$~: prenons le $\beta$ minimal
    contenant au moins un $y \in x$ et $y$ un tel élément de $x$ associé. On
    sait alors, par hypothèse, qu'il existe $z \in x \cap y$. On en déduit,
    comme $z \in y$, qu'il existe $\gamma < \beta$ tel que
    $z \in \mathbb V_\gamma$, mais comme $z \in x$, cela contredit la minimalité
    de $\beta$. C'est donc absurde, donc $x$ est bien fondé pour $\in$.
  \item vérifions l'axiome du choix~: soit un ensemble $X$ non vide, dont les
    éléments sont non vides et deux à deux disjoints. Soit $\alpha$ tel que
    $X \in \mathbb V_\alpha$, on sait donc que tous les éléments de $X$ sont
    des éléments de $\mathbb V_\alpha$. Par l'axiome du choix, on trouve
    $C$ tel que $C \cap x$ est un singleton pour tout $x \in X$. Quitte à
    restreindre $C$ aux éléments d'éléments de $X$, on a donc un ensemble
    dont les éléments sont des éléments de $\mathbb V_\alpha$, donc
    $C\in\mathbb V_{\alpha + 1}$, d'où le réstultat.
  \end{itemize}
  D'où le résultat.
\end{proof}

On a donc notre premier modèle classe de $\ZFC$. Un point important lorsque l'on
ne travaille pas \emph{a priori} avec l'axiome de fondation est que la classe
$\mathbb V$ correspond exactement à la classe des éléments bien fondés de
l'univers ambiant.

\begin{proposition}
  Soit un ensemble $x$, alors on a l'équivalence suivante dans $\ZFC-\AxF$~:
  \[x\in \mathbb V \iff
  \forall y \in \trcl(x), \exists z \in y, y \cap z = \varnothing\]
\end{proposition}

\begin{proof}
  L'implication directe est une conséquence du fait que $\mathbb V\models \AxF$.
  Pour le sens réciproque, on raisonne par induction bien fondée sur $\in$~:
  si pour tout $x \in X$, $x \in \mathbb V$, alors en considérant
  \[\alpha_x \defeq \min\{\alpha \in \Ord\mid x \in \mathbb V_\alpha\}\]
  puis
  \[\alpha \defeq \sup_{x \in X} \alpha_x\]
  on sait que $x\in \mathbb V_\alpha$ pour tout $x \in X$, donc
  $X\subseteq \mathbb V_\alpha$, donc $X \in \mathbb V_{\alpha + 1}$.
  Ainsi, pour tout ensemble $X$ tel que $\in$ est bien fondé sur $X$,
  $X \in \mathbb V$.
\end{proof}

On voit donc que dans le cas d'un univers ensembliste $\mathcal U$, celui-ci
vérifie $\AxF$ si et seulement si $\mathbb V = \mathcal U$. On utilisera donc en
général $\mathbb V$ pour parler de l'univers ensembliste.

\section{Vers le théorème de réflexion}

Dans le reste du chapitre, on suppose l'axiome de fondation~: on considère donc
que $\mathcal U = \mathbb V$. Notre premier objectif est de montrer le théorème
de réflexion. Pour ce faire, nous introduisons d'abord la notion de rang,
permettant d'étudier plus en détail la hiérarchie des $\mathbb V_\alpha$, et les
ensembles $H_\kappa$. Nous verrons ensuite l'absoluité, permettant de rendre plus
formel ce qu'on attend d'une proposition qui passe d'un modèle de $\ZFC$ à un
autre.

\subsection{Rang et parties héréditaires}

Le rang indique à quelle étape de la construction des $\mathbb V_\alpha$ notre
ensemble a été construit.

\begin{definition}[Rang]
  On définit le rang d'un ensemble $x\in\mathbb V$ par~:
  \[\rk(x) \defeq \min \{\alpha \in \Ord\mid x\subseteq \mathbb V_\alpha\}\]
\end{definition}

Le rang d'un ensemble peut se voir comme la profondeur d'accolades imbriquées
lorsque l'on définit l'ensemble en question. Ce rang se comporte bien vis à vis
des opérations ensemblistes usuelles, comme nous allons le voir.

\begin{proposition}\label{prop.cara.rk}
  Soit un ensemble $X\in\mathbb V$, on peut caractériser son rang par
  \[\rk(X) = \sup_{x \in X} (\rk(x) + 1)\]
\end{proposition}

\begin{proof}
  Supposons que $\forall x \in X, \rk(x) \leq \alpha$ pour un certain
  $\alpha\in\Ord$. Soit $x\in X$. Par définition de $\rk(x)$, on déduit que
  $x\subseteq \mathbb V_\alpha$, donc $x\in \mathbb V_{\alpha + 1}$, donc
  $X = \{x\}_{x\in X} \subseteq \mathbb V_{\alpha + 1}$. Donc $\rk(X)$ est
  un majorant de $\{\rk(x) + 1 \mid x \in X\}$.

  Montrons que ce majorant est une borne supérieure. Soit $\beta\in\Ord$ tel que
  $\forall x \in X, \rk(x) + 1 \leq \beta$. Pour tout
  $x\in X$, $x\in \mathbb V_\beta$ donc $X\subseteq \mathbb V_\beta$, donc
  $\rk(X) \leq \beta$. Ce majorant est donc une borne supérieure, d'où
  l'égalité.
\end{proof}

\begin{lemma}\label{lem.rk.surj}
  Soit $x$ un ensemble et $\alpha$ son rang. Alors pour tout $\beta < \alpha$,
  il existe $y$ tel que $\beta = \rk(y)$ et $y \in \trcl(x)$.
\end{lemma}

\begin{proof}
  Par l'absurde, en fixant $\alpha$, supposons qu'il existe un ensemble $x$,
  minimal pour $\in$, tel que $\rk(x) > \alpha$ et il n'existe aucun élément
  $y\in\trcl(x)$ tel que $\rk(y) = \alpha$. Comme on sait que $\alpha < \rk(x)$
  et que $\displaystyle \rk(x) = \sup_{y \in x} (\rk(y) + 1)$,
  alors on peut trouver $y \in x$ tel que $\rk(y) + 1 > \alpha$. Si
  $\rk(y) = \alpha$, alors cela contredit notre hypothèse sur $x$. Si
  $\rk(y) > \alpha$, alors par $\in$-minimalité de $x$, on en déduit qu'il
  existe $z \in \trcl(y)$ tel que $\rk(z) = \alpha$, mais alors
  $z \in \trcl(x)$, ce qui est absurde. Ainsi, dans tous les cas, on a une
  absurdité. On en déduit que pour tout $x$ et $\alpha < \rk(x)$, il existe
  $y \in \trcl(x)$ tel que $\rk(y) = \alpha$.
\end{proof}

\begin{property}\label{prop.rg.oper}
  Les propriétés suivantes sont vérifiées~:
  \begin{itemize}
  \item $\forall x,y\in\mathbb V, x\in y \implies \rk(x) + 1 \leq \rk(y)$
  \item $\forall x,y\in\mathbb V, x\subseteq y \implies \rk(x) \leq \rk(y)$
  \item $\displaystyle \forall x\in\mathbb V,
    \rk\left(\bigcup x\right) + 1 \leq \rk(x)$
  \item $\forall x,y\in\mathbb V, \rk(x\cup y) = \max(\rk(x),\rk(y))$
  \item $\forall x,y\in\mathbb V, \rk(\{x,y\}) = \max(\rk(x),\rk(y)) + 1$
  \item $\forall x\in\mathbb V, \rk(\powerset(x)) = \rk(x) + 1$
  \item $\forall x\in\mathbb V, \rk(\trcl(x)) = \rk(x)$
  \item $\displaystyle \forall \alpha \in \Ord, \rk(\alpha) = \alpha$
  \end{itemize}
\end{property}

\begin{proof}
  Vérifions chaque propriété~:
  \begin{itemize}
  \item l'inégalité vient directement de la \cref{prop.cara.rk}.
  \item soient $x,y$ tels que $x\subseteq y$. Si $y\subseteq\mathbb V_\alpha$,
    comme $x\subseteq y$, on en déduit que $x\subseteq\mathbb V_\alpha$, donc
    $\rk(x)\leq \rk(y)$.
  \item soit $y\in \bigcup x$, on trouve $z\in x$ tel que $y\in z \in x$. En
    utilisant l'inégalité précédente, on en déduit que $\rk(y) + 1 \leq \rk(z)$
    et que $\rk(z) + 1 \leq \rk(x)$, donc $\rk(y) + 2 \leq \rk(x)$. On en détuit
    en utilisant la \cref{prop.cara.rk} que $\rk(\bigcup x) + 1 \leq \rk(x)$.
  \item soient $x,y$. On utilise directement la \cref{prop.cara.rk}~:
    \begin{align*}
      \rk(x\cup y) &= \sup_{z \in x \cup y} (\rk(z) + 1)\\
      &= \max(\sup_{z \in x} (\rk(z) + 1), \sup_{z\in y} (\rk(z) + 1))\\
      &= \max(\rk(x),\rk(y))
    \end{align*}
  \item pour deux éléments $x,y$, il est clair que le sup est un max, d'où
    l'égalité.
  \item soit un ensemble $x$. comme $x\in \mathcal P(x)$, on en déduit que
    $\rk(x) + 1 \leq \mathcal (\powerset (x))$. Pour l'autre inclusion, si
    $y\in \mathcal (\powerset(x))$ alors $y\subseteq x$, donc
    $\rk(y) \leq \rk(x)$ par un point précédent. On en déduit que
    $\sup_{y\in \powerset(x)} (\rk(y) + 1) \leq \rk(x) + 1$ d'où
    l'égalité.
  \item comme chaque $\mathbb V_\alpha$ est transitif, si
    $x\subseteq \mathbb V_\alpha$ alors $\trcl(x)\subseteq \mathbb V_\alpha$. On
    en déduit que $\rk(\trcl(x)) \leq \trcl(x)$, et il est clair que l'autre
    inégalité tient toujours puisqu'on rajoute des éléments.
  \item on prouve cela par induction transfinie~:
    \begin{itemize}
    \item $\varnothing \subseteq \varnothing$ donc $\rk(0) = 0$.
    \item on sait que $\alpha + 1 = \alpha \cup \{\alpha\}$. En utilisant les
      précédentes égalités, $\rk(\{\alpha\}) = \rk(\alpha) + 1$ donc
      $\rk(\alpha + 1) = \alpha + 1$.
    \item si $\lambda$ est un ordinal limite, alors
      $\rk(\lambda) = \sup_{\beta < \lambda} (\rk(\beta) + 1)$ et par hypotèse
      d'induction, $\rk(\beta) = \beta$, donc $\rk(\lambda) = \lambda$.
    \end{itemize}
    Ainsi pour tout $\alpha \in \Ord, \rk(\alpha) = \alpha$.
  \end{itemize}
  Les propriétés sont donc vérifiées.
\end{proof}

On peut alors affiner certains résultats liés au fait que
$\mathbb V \models \ZFC$. En effet, puisque
$\mathbb V_\alpha \subseteq\mathbb V$, on peut chercher ce que chaque
$\mathbb V_\alpha$ vérifie. Chaque $\mathbb V_\alpha$ vérifie l'axiome de
fondation.

On voit dans la \cref{prop.rg.oper} que la plupart des opérations ne font pas
exploser le rang. Cela signifie que, dans le cas d'un ordinal limite, où
$\alpha \in \lambda \implies \alpha + 1 \in \lambda$, beaucoup de constructions
sont stables par $\mathbb V_\lambda$.

\begin{property}
  Soit $\lambda$ un ordinal limite. Alors $\mathbb V_\lambda$ vérifie les
  axiomes suivants~:
  \begin{itemize}
  \item l'axiome de la paire
  \item l'axiome de l'union
  \item l'axiome de l'ensemble des parties
  \item l'axiome de compréhension
  \item l'axiome du choix
  \end{itemize}

  De plus, si $\lambda > \omega$ alors $\mathbb V_\lambda$ vérifie l'axiome de
  l'infini.
\end{property}

\begin{proof}
  Vérifions chaque axiome, en utilisant le \cref{lem.axZFC}~:
  \begin{itemize}
  \item axiome de la paire~: soient $x,y\in \mathbb V_\lambda$, donc tels que
    $\rk(x) <\lambda$ et $\rk(y)<\lambda$. On a alors  l'inégalité
    $\rk(\{x,y\}) = \max(\rk(x),\rk(y)) + 1 < \lambda$, donc
    $\{x,y\}\in\mathbb V_\lambda$.
  \item axiome de l'union: soit $x\in \mathbb V_\lambda$, alors
    $\rk(x) < \lambda$, donc $\rk(\bigcup x) < \lambda$ donc
    $\bigcup x \in \mathbb V_\lambda$.
  \item axiome de l'ensemble des parties~: soit $x\in \mathbb V_\lambda$. Comme
    $\rk(\powerset(x)) = \rk(x) + 1$, que $\rk(x) < \lambda$ et que $\lambda$
    est limite, on en déduit que $\rk(\powerset(x)) < \lambda$ donc que
    $\powerset(x) \in \mathbb V_\lambda$.
  \item axiome de compréhension~: on sait que si $x\subseteq y$ alors
    $\rk(x) \leq \rk(y)$, donc $\mathbb V_{\lambda}$ est stable par
    sous-ensemble.
  \item soit $X$ un ensemble non vide, d'éléments deux à deux disjoints. Alors
    par l'axiome du choix, il existe $C$ tel que $\forall x \in X,|C\cap x|=1$.
    Quitte à prendre $C \cap \bigcup X$, $C\subseteq \bigcup X$ donc
    $\rk(C) < \lambda$, donc $C\in \mathbb V_\lambda$.
  \end{itemize}

  Si $\lambda > \omega$, alors $\omega \in \mathbb V_\lambda$ donc il existe
  bien dans $V_\lambda$ un élément contenant l'ensemble vide et stable par
  successeur. On peut préciser que l'ensemble vide est bien dans
  $\mathbb V_\lambda$, donnant un sens à la définition.
\end{proof}

On peut se demander alors si l'axiome de remplacement est vérifié. En fait il
est évident que ça n'est pas le cas, en partant du deuxième théorème
d'incomplétude de Gödel~: si c'était le cas, alors on trouverait un modèle de
$\ZFC$, impliquant que $\ZFC\vdash \Coher(\ZFC)$ (ou plutôt
$\ZFC\vdash \Coher(\lceil \ZFC\rceil)$).

Les $\mathbb V_\lambda$ nous donnent donc une approximation assez fidèle d'un
modèle de $\ZFC$ s'il s'agit de mathématiques usuelles, mais ces ensembles
perdent tout leur intérêt pour de la théorie des ensembles~: le schéma d'axiomes
de remplacement est utilisé pour la plupart de nos constructions, dès
l'utilisation de la récursion transfinie.

On introduit une autre hiérarchie d'ensembles qui, eux, vérifieront le schéma
d'axiomes de remplacement~: ce sont les ensembles héréditaires.

\begin{definition}[Ensemble héréditaire]
  Soit $\kappa \in \Card$. On appelle ensemble des ensembles héréditairement de
  cardinal $\kappa$ la classe
  \[\mathbb H_\kappa \defeq \{x \mid \Card(\trcl(x)) < \kappa\}\]
\end{definition}

Pour deux cardinaux $\kappa,\lambda$, on définit le cardinal
\[\lambda^{<\kappa} \defeq \sup_{\kappa' < \kappa} \lambda^{\kappa'}\]
On peut voir ce cardinal comme le cardinal de l'ensemble des fonctions de
$\kappa$ dans $\lambda$ dont le support (ensemble des points de $\kappa$ où la
fonction n'est pas nulle) est strictement inférieur à $\kappa$.

On montre maintenant que, plus que des classes, les $\mathbb H_\kappa$ sont des
ensembles.

\begin{proposition}\label{prop.Hcard}
  Soit $\kappa\in\Card$ infini. Alors $\mathbb H_\kappa\subseteq\mathbb V_\kappa$
  et $|\mathbb H_\kappa| = 2^{<\kappa}$.
\end{proposition}

\begin{proof}
  Soit un ensemble $x$ et $\alpha$ son rang. Par le \cref{lem.rk.surj}, pour
  tout $\beta < \alpha$, on trouve $x_\beta \in \trcl(x)$ tel que
  $\rk(x_\beta) = \beta$. On a donc une injection de $\rk(x)$ vers $\trcl(x)$,
  donc $\Card(\trcl(x)) \geq \Card(\rk(x))$, d'où $x \in \mathbb V_\kappa$.

  Pour la deuxième proposition, procédons par double inégalité. Tout d'abord,
  pour $\lambda < \kappa$, on sait que
  $\powerset(\lambda) \subseteq \mathbb H_\kappa$, donc
  $|\mathbb H_\kappa| \geq 2^\lambda$, d'où $|\mathbb H_\kappa| \geq 2^{<\kappa}$.

  Pour l'autre inégalité, on va construire pour chaque $x \in \mathbb H_\kappa$
  une relation $r_x \subseteq \lambda \times \lambda$ pour un certain
  $\lambda < \kappa$, de telle sorte que l'association soit injective.
  Soit donc $x \in \mathbb H_\kappa$, on pose $\lambda = |\trcl(x)\cup \{x\}|$
  (donc $\lambda < \kappa$) et on pose $r_x\subseteq \lambda \times \lambda$
  une relation telle que
  \[(\lambda,r_x)\cong (\trcl(x)\cup \{x\}, \in)\]
  (cette relation existe puisqu'on a une bijection entre $\trcl(x)\cup\{x\}$ et
  $\lambda$). On veut maintenant prouver que pour tout autre $y$ et
  $\lambda' = |\trcl(y)\cup\{y\}|$, si $(\lambda, r_x) = (\lambda',r_y)$, alors
  $x = y$. On suppose donc que $(\lambda, r_x) = (\lambda',r_y)$. Par le
  \cref{lem.most}, on sait (puisque $\trcl(x)\cup \{x\}$ et $\trcl(x)\cup \{y\}$
  sont des ensembles transitifs) que le seul isomorphisme existant entre
  $(\trcl(x)\cup \{x\},\in)$ et $(\trcl(y)\cup\{y\},\in)$ est l'identité. Or,
  on sait que $(\lambda,r_x) = (\lambda',r_y)$, donc on en conclut que
  $\trcl(x)\cup \{x\} = \trcl(y)\cup \{y\}$ par unicité de l'isomorphisme.
  Il vient ensuite que $x = y$ puisque $x$ (respectivement $y$) est l'unique
  élément $\in$-maximal dans $\trcl(x)\cup \{x\}$ (respectivement dans
  $\trcl(y)\cup\{y\}$). On a donc une injection de $\mathbb H_\kappa$ dans
  $\displaystyle\bigcup_{\lambda < \kappa}\lambda \times 2^{\lambda \times \lambda}$,
  d'où $|\mathbb H_\kappa| \leq 2^{<\kappa}$.
\end{proof}

Comme $\mathbb H_\kappa$ est inclus dans un ensemble, c'est lui-même un
ensemble.

On veut maintenant montrer les cas d'égalité entre $\mathbb H$ et $\mathbb V$.

\begin{property}
  Les propriétés suivantes sont vérifiées~:
  \begin{itemize}
  \item $\mathbb V_\omega = \mathbb H_\omega$
  \item pour tout $\alpha\in\Ord$, $|\mathbb V_{\omega + \alpha}| = \beth_\alpha$,
    donc en particulier pour $\alpha \geq \omega^2$,
    $|\mathbb V_\alpha| = \beth_\alpha$.
  \item si $\kappa > \omega$, alors $\mathbb H_\kappa = \mathbb V_\kappa$
    si et seulement si $\kappa = \beth_\kappa$.
  \end{itemize}
\end{property}

\begin{proof}
  Prouvons chaque propriété~:
  \begin{itemize}
  \item on voit que pour tout $n$, $\mathbb V_n$ ne contient que des ensembles
    finis (c'est le cas de $\mathbb V_0$, et $\powerset(X)$ ne contient que des
    ensembles finis si $X$ est fini), donc
    $\mathbb V_n \subseteq \mathbb H_\omega$, d'où
    $\mathbb V_\omega \subseteq \mathbb H_\omega$. Réciproquement, si
    $x \notin \mathbb V_\omega$, alors $\trcl(x)$ est au moins dénombrbale,
    puisque $x$ a un élément de chaque rang $n \in \omega$~: par contraposée,
    si $x\in \mathbb H_\omega$ alors $x\in \mathbb V_\omega$.
  \item on sait que $|\mathbb V_\omega| = \omega$ car on a une union
    dénombrable d'ensembles finis, que
    $|\mathbb V_{\alpha + 1}| = 2^{\mathbb V_\alpha}$ et que
    $|\mathbb V_\lambda | = \sup_{\beta < \lambda} |\mathbb V_\beta|$ pour
    $\lambda$ limite, donc $\mathbb V_{\omega + \alpha}$ coïncide avec
    $\beth_\alpha$. Si $\alpha \geq \omega^2$, alors $\omega + \alpha = \alpha$,
    d'où le résultat.
  \item on prouve l'équivalence par double implication~:
    \begin{itemize}
    \item soit $\kappa > \omega$ tel que $\mathbb H_\kappa = \mathbb V_\kappa$.
      Soit $\alpha$ tel que $\omega^2\leq \alpha < \kappa$. On sait que
      $\mathbb V_\alpha \in \mathbb V_\kappa$, donc
      $\mathbb V_\alpha \in \mathbb H_\kappa$, donc en utilisant la propriété
      prouvée précédemment, $\beth_\alpha < \kappa$. En prenant alors la borne
      supérieure de ces égalités pour $\alpha < \kappa$ et par continuité de
      $\beth$, on en déduit que $\beth_\kappa \leq \kappa$. Comme on sait que
      $\kappa \leq \beth_\kappa$ pour n'importe quel $\kappa$, on en déduit que
      $\kappa = \beth_\kappa$.
    \item supposons que $\kappa = \beth_\kappa$. Soit $x \in \mathbb V_\kappa$,
      on peut trouver $\alpha$ tel que $\omega^2\leq \alpha < \kappa$ et
      $x \in \mathbb V_\alpha$, car $\kappa$ est un ordinal limite et un
      cardinal indénombrable. Comme $x \subseteq \mathbb V_\alpha$, on en
      déduit que $|\trcl(x)| \leq |\mathbb V_\alpha|$, or $\mathbb V_\alpha$ est
      de cardinal $\beth_\alpha$, et $\beth_\alpha < \beth_\kappa = \kappa$. On
      en déduit donc que $x \in \mathbb H_\kappa$. L'inclusion réciproque est
      directement due à la \cref{prop.Hcard}.
    \end{itemize}
    d'où le résultat.
  \end{itemize}
  Ainsi, on a prouvé les trois propriétés.
\end{proof}

On peut maintenant prouver que $\mathbb H_\kappa$ vérifie les axiomes de
$\ZFC - P$ pour un cardinal régulier $\kappa$, c'est-à-dire les axiomes de
$\ZFC$ sauf celui de l'ensemble des parties.

\begin{property}
  Soit $\kappa$ un cardinal régulier indénombrable. Alors $\mathbb H_\kappa$
  vérifie~:
  \begin{itemize}
  \item l'axiome d'extensionalité
  \item l'axiome de la paire
  \item l'axiome de la réunion
  \item l'axiome de remplacement
  \item l'axiome de l'infini
  \item l'axiome de fondation
  \item l'axiome du choix
  \end{itemize}
\end{property}

\begin{proof}
  On vérifie chaque axiome, en utilisant le \cref{lem.axZFC}~:
  \begin{itemize}
  \item axiome de la paire~: comme
    $\trcl(\{x,y\}) = \trcl(x) \cup \trcl(y) \cup \{x,y\}$, si
    $\trcl(x)$ et $\trcl(y)$ sont de cardinal inférieur strictement à $\kappa$,
    alors $|\trcl(\{x,y\})| < \kappa$.
  \item axiome de la réunion~: si $X\in \mathbb H_\kappa$, alors
    $|\trcl(X)| < \kappa$. Or, on peut remarquer que
    \[\trcl\left(\bigcup X\right)\subseteq \trcl(X)\]
    puisque $\trcl(X)$ contient déjà $\bigcup X$ et est transitif. Il vient donc
    que $\displaystyle\left|\trcl\left(\bigcup X\right)\right| < \kappa$.
  \item axiome de remplacement~: soit $f$ une fonction partielle d'un
    ensemble $x \in \mathbb H_\kappa$ vers un ensemble
    $y \subseteq \mathbb H_\kappa$. Alors
    \[\trcl(y) = \bigcup_{z \in y} (\trcl(z)\cup \{z\})\]
    mais comme chaque $z \in y$ est un élément de $\mathbb H_\kappa$, et qu'on
    sait que $x \in \mathbb H_\kappa$, on en déduit que $\trcl(y)$ est une
    union de cardinal inférieur strictement à $\kappa$ d'ensembles de cardinal
    inférieur strictement à $\kappa$. Ainsi, d'après le \cref{thm.Konig},
    $|\trcl(y)| < \kappa$, donc $y \in \mathbb H_\kappa$.
  \item axiome de l'infini~: puisque $\kappa$ est indénombrable, il contient
    $\omega$.
  \item axiome de fondation~: $\mathbb H_\kappa \subseteq \mathbb V_\kappa$ donc
    tous les éléments de $\mathbb H_\kappa$ sont bien fondés pour $\in$.
  \item axiome du choix~: si un ensemble $x \in \mathbb H_\kappa$ est non vide
    et tous ses éléments sont non vide et deux à deux disjoints, alors par
    l'axiome du choix on trouve $C$ tel que $C \cap y$ est un singleton pour
    tout $y \in x$. Quitte à intersecter $C$ avec $\bigcup x$, on sait que
    $C\subseteq \bigcup x$, donc comme $\mathbb H_\kappa$ est stable par
    union et par sous-ensembles, $C \in \mathbb H_\kappa$.
  \end{itemize}
  Ainsi $\mathbb H_\kappa$ vérifie $\ZF - P$.
\end{proof}

On voit qu'ici, pour que $\mathbb H_\kappa$ vérifie tout $\ZFC$, il lui suffit
que pour tout $\kappa' \in \kappa$, $2^{\kappa'} \in \kappa$. C'est ce qu'on
appelle un cardinal fortement inaccessible.

\begin{definition}[Cardinal fortement inaccessible]
  On dit qu'un cardinal $\lambda$ est fortement inaccessible (on dit aussi, plus
  simplement, inaccessible) si $\lambda > \omega$ et
  \[\forall \kappa \in \lambda, 2^\kappa \in \lambda\]
\end{definition}

Il vient directement qu'un tel cardinal ne peut exister dans tout modèle de
$\ZFC$.

\begin{proposition}
  Il n'est pas prouvable dans $\ZFC$ qu'il existe un cardinal inaccessible.
\end{proposition}

\begin{proof}
  S'il existait un $\kappa$ inaccessible, alors $\mathbb H_\kappa$ serait un
  modèle ensembliste de $\ZFC$, ce qui contredit le second théorème
  d'incomplétude de Gödel.
\end{proof}

Enfin, nous allons introduire l'univers constructible $\mathbb L$ de Gödel, que
nous étudierons dans la section prochaine. Nous aurons cependant besoin de
notions liées à sa construction.

\begin{definition}[Partie définissable]
  Soit un ensemble $X$, on dit qu'une partie $Y\subseteq X$ est définissable
  dans $X$ s'il existe une formule $\varphi(x_1,\ldots,x_n,x)$ à $n+1$ variables
  libres et des ensembles $a_1,\ldots,a_n \in X$ tels que
  \[\forall x \in X, x \in Y \iff X\models \varphi(a_1,\ldots,a_n,x)\]

  On appelle ensemble des parties définissables d'un ensemble $X$ l'ensemble
  \begin{multline*}
    \powerdef(X) \defeq \{Y \subseteq X \mid \exists
    \varphi(x_1,\ldots,x_n,x)\in\Formula(\mathcal L_{\ZF}),\\
    \exists a_1,\ldots,a_n \in X,
    \forall x \in X, x \in Y \iff X \models \varphi(a_1,\ldots,a_n,x)\}
  \end{multline*}
\end{definition}

\begin{remark}
  Il peut être bon de préciser, ici, de quel type de formule on parle. Dans le
  cas de la notion de définissabilité, on considère les propositions internes
  au modèle de $\ZFC$, c'est-à-dire les objets ensemblistes représentant
  intérieurement nos propositions. Cela explique l'utilisation de
  \og $X\models \varphi(x)$\fg plutôt que \og$\varphi(x)$\fg directement.
\end{remark}

L'univers constructible est alors l'itération de cette opérations à la place de
l'ensemble des parties.

\begin{definition}[Univers constructible de Gödel\cite{GödelL}]\label{def.L}
  On définit par induction transfinie la hiérarchie $\mathbb L_\alpha$~:
  \begin{itemize}
  \item $\mathbb L_0 = \varnothing$
  \item $\mathbb L_{\alpha + 1} = \powerdef (\mathbb L_\alpha)$
  \item si $\lambda$ est limite, alors
    $\displaystyle\mathbb L_\lambda = \bigcup_{\beta < \lambda} \mathbb L_\beta$
  \end{itemize}
  L'univers constructible de Gödel est alors
  \[\mathbb L \defeq \bigcup_{\alpha \in \Ord} \mathbb L_\alpha\]
\end{definition}

\subsection{Absoluité et réflexion}

On va maintenant s'intéresser à une question assez naturelle à se poser, dans
les conditions de notre étude des modèles de $\ZFC$. Supposons donnés deux
modèles (potentiellement des classes) $(\mathcal M,\in)$ et $(\mathcal N,\in)$
et $\mathcal M \subseteq \mathcal N$, quand une formule $\varphi$ à paramètres
dans $\mathcal M$ garde-t-elle la même valeur de vérité que l'on l'étudie dans
$\mathcal M$ ou dans $\mathcal N$ ?

Prenons un exemple~: on a vu que $\mathcal V_\alpha$ étant un ensemble
transitif, celui-ci vérifiait l'extensionalité. La preuve montre assez
clairement que la proposition $\forall z, z \in x \iff z \in y$ en elle-même
ne dépend pas du point de vue choisi, tant que celui-ci se fait dans un
ensemble transitif. Ainsi si on imagine un surmodèle de notre univers
ensembliste, ou une partie seulement, la vérité de
$\forall z, z \in x \iff z \in y$ est presque constante (encore une fois, tant
que l'univers considéré est transitif). Une formule absolue est donc une
formule dont la valeur de vérité ne changera pas suivant les modèles.

\begin{definition}[Formule absolue]
  Soient deux classes $(\mathcal M,\in)$, $(\mathcal N,\in)$ telles que
  $\mathcal M \subseteq \mathcal N$.
  Soit $\varphi(x_1,\ldots,x_n) \in \mathcal L_{\ZF}$ une formule à $n$
  paramètres. On dit que $\varphi$ est absolue entre $\mathcal M$ et
  $\mathcal N$ lorsque
  \[\mathcal M \preccurlyeq_\varphi \mathcal N \defeq
  \forall a_1,\ldots,a_n \in \mathcal M, \varphi^{\mathcal M}(a_1,\ldots,x_n)
  \iff \varphi^{\mathcal N}(a_1,\ldots,a_n)\]

  On dirait que $\varphi$ est absolue pour une classe $\mathcal M$ si elle est
  absolue entre $\mathcal M$ et $\mathbb V$.
\end{definition}

L'exemple précédent sur l'extensionalité peut en fait se généraliser. La
propriété exploitée est le fait que si l'on étudie
$\forall x, z \in x \iff z \in y$ dans un modèle transitif, alors on est obligé
de contenir tous les éléments de $x$ et $y$. Cette propriété est en quelque
sorte locale~: tout ce qui est nécessaire pour évaluer la vérité de cette
proposition est la donnée de tous les éléments de $x$ et de $y$.

Au contraire, une propriété comme \og être un cardinal\fg n'est pas du tout
locale, puisqu'elle repose sur la non existence d'une injection dans l'univers
ambiant. Imaginons qu'on puisse ajouter une telle injection dans notre univers,
alors un élément qui était un cardinal peut ne plus l'être.

Pour étudier une proposition qui présente cette propriété locale mentionnée
ci-dessus, on introduit la hiérarchie de Lévy.

\begin{definition}[Hiérarchie de Lévy \cite{LevyH}]
  On définit par induction trois familles d'ensembles de formules sur
  $\mathcal L_{\ZF}, (\Delta^{\ZFC}_n)_{n\in\mathbb N},
  (\Sigma^{\ZFC}_n)_{n\in\mathbb N}$
  et $(\Pi^{\ZFC}_n)_{n \in \mathbb N}$~:
  \begin{itemize}
  \item $\Delta^{\ZFC}_0=\Sigma^{\ZFC}_0=\Pi^{\ZFC}_0$ est l'ensemble des formules
    prouvablement équivalentes dans $\ZFC$ à une formule dont les
    quantificateurs sont tous bornés, c'est-à-dire que toute quantification
    $Q x, \psi$ est de la forme $Q x \in t, \psi$ où $t$ est un terme ne
    contenant pas $x$.
  \item $\Sigma^{\ZFC}_{n+1}$ est l'ensemble des formules prouvablement
    équivalentes
    dans $\ZFC$ à une formule de la forme
    $\exists x, \psi$ où $\psi$ est une formule $\Pi^{\ZFC}_n$.
  \item $\Pi^{\ZFC}_{n+1}$ est l'ensemble des formules prouvablement équivalentes
    dans $\ZFC$ à une formule de la forme $\forall x, \psi$ où $\psi$ est une
    formule $\Sigma^{\ZFC}_n$.
  \item $\Delta^{\ZFC}_n$ est l'ensemble des formules qui sont à la fois
    $\Sigma^{\ZFC}_n$ et $\Pi^{\ZFC}_n$.
  \end{itemize}

  On notera en général simplement $\Delta_n$ (respectivement $\Sigma_n$ et
  $\Pi_n$) pour $\Delta^{\ZFC}_n$.
\end{definition}

\begin{remark}
  Comme on peut toujours prendre un produit fini comme un nouvel ensemble, il
  est équivalent de prendre un seul quantificateur $\exists$ (respectivement
  $\forall$) et une suite finie de quantificateurs $\exists$ (respectivement
  $\forall$), d'où le fait qu'on se limite dans la hiérachie aux alternances
  de quantificateurs.
\end{remark}

On peut maintenant énoncer la généralisation de notre preuve de l'axiome
d'extensionalité pour toute formule $\Delta_0$.

\begin{proposition}
  Soit $\mathcal M$ une classe transitive. Alors toute formule
  $\varphi\in\Delta_0$ est absolue pour $\mathcal M$.
\end{proposition}

\begin{proof}
  On prouve le résultat par induction sur la formule $\psi$ qu'on considère~:
  \begin{itemize}
  \item dans le cas d'une formule atomique, puisque $\in$ est la même relation
    interprétée dans $\mathcal M$ et dans $\mathbb V$, on a le résultat (de
    même pour $=$).
  \item dans le cas de $\lor,\land,\to,\lnot,\top,\bot$, le résultat se fait
    simplement en écrivant les définitions.
  \item dans le cas de $\exists$, supposons donc que
    $\psi(x_1,\ldots,x_n,z) = \exists y \in z,\varphi(x_1,\ldots,x_n,y,z)$
    (les termes dans $\mathcal L_{\ZF}$ sont uniquement des variables). Par
    hypothèse d'induction, on suppose donc que $\varphi$ est absolue pour
    $\mathcal M$. Soient des paramètres $a_1,\ldots,a_n$ et $c$ dans
    $\mathcal M$. On veut donc prouver que
    \[\exists b^{\mathcal M}, b \in c \land \varphi^{\mathcal M}(a_1,\ldots,a_n,b,c)
    \iff \exists b \in c, \varphi(a_1,\ldots,a_n,b,c)\]
    Le sens direct se vérifie facilement. Pour le sens réciproque, on
    remarque que comme $\mathcal M$ est transitive, $b \in c$ implique que
    $b \in \mathcal M$. Alors, par absoluité de $\varphi$ pour $\mathcal M$,
    on en déduit que $\varphi(a_1,\ldots,a_n,b,c)$.
  \item on peut prouver le cas de $\forall$ en passant par les loi de De Morgan.
  \end{itemize}
  Ainsi, par induction, on sait que $\psi \in \Delta_0$ est absolue pour
  $\mathcal M$.
\end{proof}

Donnons des exemples de formules $\Delta_0$~:

\begin{example}
  Les formules suivantes sont $\Delta_0$~:
  \begin{itemize}
  \item $x\in y$
  \item $x\subseteq y$, car cela s'écrit $\forall z \in x, z \in y$
  \item $x = \{y,z\}$
  \item $x = \bigcup y$
  \item $x$ est transitif, car cela s'écrit $\forall y \in x, y\subseteq x$
  \item $x$ est un couple, car cela s'écrit
    $\exists y,\exists z, x = \{\{y,z\},\{y\}\}$
  \item $f$ est une fonction, car cela s'écrit
    \begin{multline*}
      \forall \gamma \in f, \gamma\text{ est un couple }\land
      \forall x,y \in \bigcup \bigcup f,\\
      (\exists z \in \bigcup\bigcup f,
      (x,z) = \gamma \land (y,z) = \gamma) \implies x = y
    \end{multline*}
  \item $\alpha \in \Ord$, car être un ordinal est la conjonction de~:
    \begin{itemize}
    \item être transitif, dont on vient de voir que la proposition était
      $\Delta_0$.
    \item que $\in$ est transitif sur $\alpha$, ce qui s'écrit
      \[\forall x \in \alpha, \forall y \in \alpha, \forall z \in \alpha,
      x\in y \land y \in z \implies x \in z\] qui est clairement $\Delta_0$.
    \item que $\in$ est total, ce qui s'écrit
      \[\forall x \in \alpha, \forall y \in \alpha, x \neq y \implies
      x \in y \lor y \in x\] qui est là encore clairement $\Delta_0$.
    \item comme on suppose l'axiome de fondation, l'irréflexivité de $\in$ et sa
      bonne fondation sont forcément vérifiés.
    \end{itemize}
  \item être égal à $\omega$, car \og $X = \omega$\fg peut s'écrire
    \[X\in \Ord \land (\forall z \in X, S\;z \in X) \land
    (\forall y \in X, \lnot (\forall z \in y, S\;z \in y))\]
  \end{itemize}
\end{example}

\begin{exercise}
  Montrer qu'être un ordinal successeur (respectivement limite) peut s'exprimer
  par une formule $\Delta_0$. Montrer qu'être une bijection peut s'exprimer
  par une formule $\Delta_0$.
\end{exercise}

On voit donc une famille importante de formules qui sont absolues entre les
classes transitives (qui sont presque les seules classes d'intérêt).

On peut aussi étudier les formules $\Sigma_1$ et $\Pi_1$, qui ne sont pas
absolues mais transmettent leur vérité d'une classe à une sur-classe
(respectivement à une sous-classe).

\begin{proposition}
  Soit $\mathcal M$, $\mathcal N$ deux classes transitives, avec
  $\mathcal M \subseteq \mathcal N$. Alors pour un énoncé
  $\varphi\in\Formula(\mathcal L_{\ZF})$, on a les implications suivantes~:
  \begin{itemize}
  \item si $\varphi \in \Sigma_1$ alors
    $\varphi^{\mathcal M} \implies \varphi^{\mathcal N}$
  \item si $\varphi \in \Pi_1$ alors
    $\varphi^{\mathcal N} \implies \varphi^{\mathcal M}$
  \end{itemize}
  et toute formule $\Delta_1$ est absolue entre $\mathcal M$ et $\mathcal N$.
\end{proposition}

\begin{proof}
  On montre d'abord les deux implications~:
  \begin{itemize}
  \item supposons que $\varphi \in \Sigma_1$ et $\varphi^{\mathcal M}$. On
    suppose sans perte de généralité que $\varphi$ est de la forme
    $\exists x, \psi$ où $\psi\in\Delta_0$. On sait donc qu'il existe un
    ensemble $a\in\mathcal M$ tel que $\psi[a/x]$ est vraie. Comme
    $\psi[a/x]$ est $\Delta_0$, elle est absolue entre ces deux classes
    transitives, et $\mathcal M \subseteq\mathcal N$ donc $\psi[a/x]$
    permet de déduire $(\exists x, \psi)^{\mathcal N}$.
  \item supposons que $\varphi \in \Pi_1$ et $\varphi^{\mathcal N}$. On
    suppose sans perte de généralité que $\varphi$ est de la forme
    $\forall x, \psi$ où $\psi\in\Delta_0$. On sait que donc pour tout
    ensemble $a\in \mathcal N$, $\psi[a/x]$ est vraie. Ainsi, pour tout
    ensemble $a\in \mathcal M$, $\psi[a/x]$ est vérifiée aussi, par inclusion.
    Donc $(\forall x, \psi)^{\mathcal M}$.
  \end{itemize}

  Si $\varphi\in\Delta_1$, alors $\varphi$ est $\Sigma_1$ et $\Pi_1$, ce qui
  signifie que $\varphi^{\mathcal M}\iff \varphi^{\mathcal N}$, donc que
  $\mathcal M\preccurlyeq_\varphi \mathcal N$.
\end{proof}

Donnons maintenant le théorème de réflexion, qui est un théorème majeur de la
théorie des ensembles. Celui-ci nous permet de montrer que pour toute hiérarchie
telle $\mathbb V$ ou $\mathbb L$ et toute proposition $\varphi$, il existe des
ordinaux arbitrairement élevés tels que $\varphi$ est absolue entre la
hiérarchie et sa construction partielle à l'ordinal donné.

Ce théorème est plutôt un schéma de théorème~: pour chaque $\varphi$, il existe
un théorème correspondant. La démonstration, sans surprise, s'effectuera par
induction sur $\varphi$. Cependant, on remarque un point important~: pour une
formule sans quantificateur, $\varphi^\mathcal M = \varphi$. On n'a donc qu'à
traiter le cas des quantifications, et en utilisant la mise sous forme prénexe
des formules, déduire le résultat pour toute formule. De plus, en utilisant les
lois de De Morgan, on sait qu'on peut se contenter de prouver ce fait pour un
quantificateur (on choisit $\exists$) et de traiter le cas de la négation~:
celui-ci est direct, puisque si on sait que $\varphi \iff \psi$ alors
$\lnot\varphi \iff \lnot \psi$.

Pour montrer le résultat dans le cas de $\exists$, on a besoin d'un lemme sur
l'absoluité.

\begin{lemma}\label{lem.refl1}
  Soit une formule $Q_1\; x_1,\ldots,Q_k\;x_k, \psi$ où les $Q_i$ sont des
  quantificateurs et $\psi$ est sans quantificateurs, $\mathcal X$ une classe,
  $(X_n)_{n\in \mathbb N}$ une suite croissante d'ensembles inclus dans
  $\mathcal X$, et
  \[X \defeq \bigcup_{n \in \mathbb N} X_n\]
  Si pour tout $i \leq k$ et $n \in \mathbb N$, on a
  \[X_n \preccurlyeq_{Q_i\;x_i,\ldots, Q_k\;x_k,\psi} \mathcal X\]
  alors on a aussi pour tout $i \leq k$
  \[X \preccurlyeq_{Q_i\; x_i,\ldots, Q_k\;q_k, \psi}\mathcal X\]
\end{lemma}

\begin{proof}
  On raisonne par induction sur $k$, en ajoutant des négations ou des
  quantificateurs $\exists$~:
  \begin{itemize}
  \item si la formule est simplement $\psi$, sans quantificateurs, alors comme
    $\psi^\mathcal X = \psi^X$, le résultat est automatiquement vrai.
  \item si la formule est $\lnot \varphi$ où $\varphi$ est une formule absolue
    entre $X$ et $\mathcal X$, comme on sait que $\chi \iff \theta$ est
    équivalente à $\lnot\chi \iff \lnot \theta$ pour toutes formules
    $\chi,\theta$, on en déduit à partir de l'hypothèse d'induction que
    $\lnot \varphi$ est absolue entre $X$ et $\mathcal X$ (et de même pour
    toutes les formules partielles obtenues à partir de $\varphi$ en en
    supprimant des quantificateurs en tête).
  \item si la formule est $\exists x, Q_1\;x_1,\ldots,Q_k\;x_k,\psi$, alors
    en posant
    \[\varphi \defeq Q_1\;x_1,\ldots,Q_k\;x_k, \psi\]
    on sait par hypothèse d'induction que chaque
    $Q_j\;x_j,\ldots,Q_k\;x_k,\psi$ est absolue entre $X$ et $\mathcal X$.
    On souhaite donc maintenant montrer que $\exists x, \varphi$ est absolue
    entre $X$ et $\mathcal X$. Soient des paramètres $a_1,\ldots,a_p \in X$,
    montrons
    \[\exists x\in X, \varphi^X(a_1,\ldots,a_p)
    \iff \exists x \in \mathcal X, \varphi^\mathcal X(a_1,\ldots,a_p)\]
    par double implication~:
    \begin{itemize}
    \item supposons que $\exists x \in X,\varphi^X(a_1,\ldots,a_p)$.
      On trouve alors $a \in X$ vérifiant la formule
      $\varphi^X(a,a_1,\ldots,a_p)$. Par hypothèse d'induction, comme $\varphi$
      est absolue entre $X$ et $\mathcal X$, on en déduit que
      $\varphi^\mathcal X(a,a_1,\ldots,a_p)$, donc il existe bien
      $x \in \mathcal X$ tel que $\varphi^\mathcal X(a_1,\ldots,a_p)$.
    \item réciproquement, si
      $\exists x\in\mathcal M, \varphi^\mathcal X(a_1,\ldots,a_p)$,
      en prenant un tel $a$, on trouve $j$ tel que $a_1,\ldots,a_p \in X_j$~:
      par absoluité entre $X_j$ et $\mathcal X$, on sait donc qu'il existe
      $a \in X_j$, donc $a \in X$, tel que $\varphi^\mathcal X(a,a_1,\ldots,a_p)$
      donc, par absoluité de $\varphi$ entre $X$ et $\mathcal X$, on en déduit
      que $\exists x \in X, \varphi^X(a_1,\ldots,a_p)$.
    \end{itemize}
    On en déduit donc que $\exists x, \varphi$ est absolue entre $\mathcal X$
    et $X$.
  \end{itemize}
  D'où le résultat, par induction sur $k$.
\end{proof}

On peut maintenant montrer le théorème à proprement parler.

\begin{theorem}[Réflexion]\label{thm.reflexion}
  Soit une fonction $W : \Ord \to \mathbb V$ vérifiant~:
  \begin{itemize}
  \item pour tous $\alpha, \beta \in \Ord$,
    $\alpha \leq \beta \implies W_\alpha \subseteq W_\beta$.
  \item pour tout $\lambda \in \Ord$ limite,
    $\displaystyle W_\lambda = \bigcup_{\beta < \lambda} W_\beta$
  \end{itemize}
  et $\displaystyle \mathcal W = \bigcup_{\alpha \in \Ord} W_\alpha$.
  alors pour toute proposition $\varphi\in\Formula(\mathcal L_{\ZF})$ et tout
  ordinal $\alpha \in \Ord$, il existe un ordinal $\beta > \alpha$ tel que
  $W_\beta\preccurlyeq_\varphi \mathcal W$. On peut de plus prendre $\beta$
  limite.
\end{theorem}

\begin{proof}
  On montre le résultat uniquement sur les formules sous formes prénexes, mais
  comme toute formule est équivalente à une formule sous forme prénexe, cela
  suffit à montrer le résultat souhaité. On procède donc par récurrence sur
  le nombre de quantificateur de la formule, en ajoutant des $\exists$ et des
  négations~:
  \begin{itemize}
  \item comme dans le lemme précédent, le cas sans quantificateur se traite
    directement puisque la relativisation ne change pas la formule.
  \item dans le cas d'une négation, comme $\varphi \iff \psi$ et
    $\lnot\varphi\iff\lnot\psi$ sont équivalentes, on a aussi le résultat
    directement.
  \item on suppose donc que la formule pour laquelle on souhaite prouver notre
    énoncé est $\exists x, \varphi(x,x_1,\ldots,x_n)$. Par hypothèse
    d'induction, on sait déjà que pour tout $\alpha \in \Ord$, on peut trouver
    $\beta > \alpha$ limite, tel que
    $W_\beta \preccurlyeq_{\varphi} \mathcal W$.

    On définit maintenant une fonction (entre classes) $F$, à $n$ arguments,
    donnée par le fait que $F(x_1,\ldots,x_n)$ est l'ensemble des
    $x \in \mathcal W$ tels que $\varphi(x,x_1,\ldots,x_n)$ et dont le
    $\mathcal W$-rang (au sens de plus petit $\alpha$ tel que l'élément est dans
    $W_\alpha$) est minimal. Ainsi, quitte à choisir uniquement les éléments de
    $\mathcal W$-rang minimal, on a l'équivalence suivante, pour toute formule
    $\psi$~:
    \[\exists x \in \mathcal W, \psi(x,x_1,\ldots,x_n) \iff
    \exists x \in F(x_1,\ldots,x_n), \psi(x,x_1,\ldots,x_n)\]

    Soit $\alpha \in \Ord$. On décide maintenant de définir une suite
    $(\beta_n)$ d'ordinaux limites supérieurs à $\alpha$, par récursion d'ordre
    2~:
    \begin{itemize}
    \item on choisit pour $\beta_0$ le premier $\beta > \alpha$ tel que
      $W_\beta \preccurlyeq_\varphi \mathcal W$.
    \item pour un rang $k$ fixé, on définit $\beta_{2k + 1}$ comme le plus
      petit ordinal strictement supérieur à $\beta_{2k}$ tel que
      $W_{\beta_{2k + 1}}$ contienne $F(a_1,\ldots,a_n)$ pour tous
      $a_1,\ldots,a_n \in W_{\beta_{2k}}$ (il nous suffit de considérer l'union de
      tous ces $F(a_1,\ldots,a_n)$ pour obtenir un ensemble, dont les
      $\mathcal W$-rangs sont donc majorés, et de prendre un ordinal supérieur
      à cette majoration).
    \item on définit ensuite $\beta_{2k + 2}$ comme le plus petit ordinal
      supérieur strictement à $\beta_{2k + 1}$ tel que
      $W_{\beta_{2k + 2}}\preccurlyeq_\varphi \mathcal W$.
    \end{itemize}
    On définit alors $\beta = \sup \beta_n$~: c'est un ordinal limite car
    les $\beta_k$ sont des ordinaux limites. D'après le \cref{lem.refl1}, et
    comme les $W_{\beta_k}$ forment une suite croissante et incluse dans
    $\mathcal W$ par hypothèse, on sait donc que $\varphi$ est absolue entre
    $W_\beta$ et $\mathcal W$ (puisque $\beta$ est le sup des $\beta_{2k}$). On
    montre maintenant que $\exists x, \varphi$ est absolue entre $W_\beta$ et
    $\mathcal W$. Soient $a_1,\ldots,a_k \in W_\beta$, ce que l'on peut
    réécrire par $a_1,\ldots,a_k \in W_{2k}$ pour un certain $k \in \mathbb N$
    (que l'on fixe désormais). Montrons par double implication que
    \[\exists x\in W_\beta, \varphi^{W_\beta}(a_1,\ldots,a_n) \iff
    \exists x \in \mathcal W, \varphi^{\mathcal W}(a_1,\ldots,a_n)\]
    \begin{itemize}
    \item si on trouve $x \in W_\beta$ vérifiant l'énoncé, alors en utilisant
      le fait que $\varphi$ est absolue entre $W_\beta$ et $\mathcal W$ et
      le fait que $W_\beta \subseteq \mathcal W$, on obtient le résultat.
    \item réciproquement, si l'on trouve $a \in \mathcal W$ tel que
      $\varphi^\mathcal W(a,a_1,\ldots,a_n)$ alors on peut aussi le trouver de
      $\mathcal W$-rang minimal. Cela signifie donc que
      $a \in F(a_1,\ldots,a_n)$, donc on sait que $a \in W_{\beta_{2k + 1}}$, donc
      $a \in W_\beta$. On en déduit, comme $\varphi$ est absolue entre
      $W_\beta$ et $\mathcal W$, qu'il existe $a\in W_\beta$
      tel que $\varphi^{W_\beta}(a,a_1,\ldots,a_n)$, d'où le résultat.
    \end{itemize}
  \end{itemize}
  On a donc prouvé notre résultat par induction sur les formules sous forme
  prénexe.
\end{proof}

\begin{remark}
  La preuve a été légèrement allégée pour être plus digeste~: pour vérifier les
  hypothèses du \cref{lem.refl1}, il nous faut établir comme énoncé à montrer
  par induction que le $\beta$ que l'on trouve à chaque fois est vrai non
  seulement pour $\varphi$, mais aussi pour $\varphi$ auquel on enlève une
  partie des quantificateurs de tête. Remarquons aussi que, comme le
  \cref{lem.refl1} est lui-même un schéma de lemme, et que l'on veut prouver
  un schéma de théorèmes, il est bon de vérifier un fait~: pour une formule
  $\varphi$ donnée, il nous suffit d'utiliser notre schéma de lemmes un nombre
  fini de fois pour atteindre la conclusion.
\end{remark}

\begin{remark}
  Comme on peut toujours représenter un ensemble fini de propositions $F$ par
  $\bigwedge F$, le théorème de réflexion s'étend directement à un ensemble fini
  de propositions.
\end{remark}

Une première conséquence est un résultat assez peu surprenant, mais qu'il est
intéressant d'avoir.

\begin{proposition}
  La théorie $\ZFC$ n'est pas finiment axiomatisable.
\end{proposition}

\begin{proof}
  Si $\ZFC$ était finiment axiomatisable, alors on trouverait un certain
  $\mathbb V_\alpha$ tel que $\ZFC$ est absolu pour $\mathbb V_\alpha$, nous
  donnant donc un modèle ensembliste de $\ZFC$.
\end{proof}

\section{L'univers constructible de Gödel}

Nous avons introduit cet univers dans la \cref{def.L}, mais ne l'avons pas
encore étudié. Nous le verrons, c'est un univers ensembliste se comportant
particulièrement bien~: l'axiome du choix y est vrai dans une version
particulièrement forte, l'hypothèse du continu généralisée y est vraie\ldots
Nous allons donc étudier plus en profondeur l'univers $\mathbb L$.

\subsection{Premiers pas}

Tout d'abord, intéressons-nous à l'opération $\powerdef$. Celle-ci nous donne
une forme plus faible d'ensemble des parties, mais on peut remarquer que son
action sur le cardinal n'est pas du tout la même.

\begin{proposition}
  Les assertions suivantes sont vérifiées~:
  \begin{itemize}
  \item si $X$ est un ensemble fini, alors $\powerdef(X) = \powerset(X)$.
  \item si $X$ est un ensemble infini, alors $|\powerdef(X)| = |X|$.
  \end{itemize}
\end{proposition}

\begin{proof}
  Prouvons ces deux assertions~:
  \begin{itemize}
  \item La première inclusion, $\powerdef(X)\subseteq\powerset(X)$, est directe.
    Supposons que $X$ est fini et soit $Y\subseteq X$. Décomposons
    $Y = \{y_1,\ldots,y_p\}$. On peut alors définir la proposition
    \[\varphi(x) \defeq \bigvee_{i = 1}^p (x = y_i)\]
    dont il est évident que $x \in Y \iff X \models \varphi(x)$. Ainsi
    $Y\in \powerdef(X)$, d'où l'égalité.
  \item Supposons que $X$ est infini. Pour énumérer $\powerdef(X)$, il nous
    suffit d'énumérer $\Formula(\mathcal L_{\ZF})$, qui est dénombrable, et
    l'ensemble des parties finies de $X$~: $|X^{<\omega}| = |X|$ car $X$ est
    au moins dénombrable. Ainsi il y a $|\omega \times X^{<\omega}| = |X|$
    tuples $(\varphi,a_1,\ldots,a_n)$ définissant une partie définissable de
    $X$, donc $|\powerdef(X)|\leq |X|$. De plus, on peut injecter $X$ dans
    $\powerdef(X)$ par $x\mapsto \{x\}$, dont la représentation est la
    formule
    \[\psi(y) \defeq x = y \]
    donc $|\powerdef(X)| = |X|$.
  \end{itemize}
\end{proof}

On montre aussi, de façon analogue au travail effectué pour $\mathbb V$, que
$(\mathbb L_\alpha)_{\alpha \in \Ord}$ est une hiérarchie cumulative et transitive.

\begin{proposition}
  Pour tout $\alpha,\beta \in \Ord$, si $\alpha \leq \beta$ alors
  $\mathbb L_{\alpha} \subseteq \mathbb L_\beta$. De plus chaque
  $\mathbb L_\alpha$ est un ensemble transitif.
\end{proposition}

\begin{proof}
  On prouve d'abord que chaque $\mathbb L_\alpha$ est un ensemble transitif,
  par induction sur $\alpha$~:
  \begin{itemize}
  \item $\mathbb L_0 = \varnothing$ est transitif.
  \item supposons que $\mathbb L_\alpha$ est transitif. Soit alors
    $x \in \mathbb L_{\alpha + 1}$. Pour tout $y \in x$, comme $\mathbb L_\alpha$
    est transitif, $y \in \mathbb L_\alpha$. Mais alors on peut définir
    $y$ par la formule
    \[\varphi_y(x) \defeq x \in y\]
    à paramètre ($y$) dans $\mathbb L_\alpha$. Ainsi
    $y \in \mathbb L_{\alpha + 1}$, donc $x \subseteq \mathbb L_{\alpha + 1}$.
  \item pour le cas limite, l'union d'ensembles transitifs est un ensemble
    transitif.
  \end{itemize}

  On prouve maintenant que la hiérarchie est cumulative, par induction sur
  la différence entre $\beta$ et $\alpha$~:
  \begin{itemize}
  \item $\mathbb L_\alpha \subseteq \mathbb L_\alpha$
  \item on montre que $\mathbb L_{\beta}\subseteq \mathbb L_{\beta + 1}$, ce qui
    suffit à montrer que $\mathbb L_\alpha \subseteq \mathbb L_\beta \implies
    \mathbb L_\alpha \subseteq \mathbb L_{\beta + 1}$.
    Si $x \in \mathbb L_{\beta}$, alors on peut définir $x$ par
    \[\varphi_x(a) \defeq a \in x\]
    dans $\mathbb L_{\beta + 1}$, donc $x \in \mathbb L_{\beta + 1}$.
  \item le cas où la différence est un ordinal limite est direct.
  \end{itemize}
  Ainsi, par induction transfinie, $\mathbb L_\alpha$ est une hiérarchie
  cumulative.
\end{proof}

Il est donc clair que $\powerdef(X)\neq \powerset(X)$ lorsque $X$ est infini.
On peut donc montrer que $\mathbb L$ ne coïncide avec $\mathbb V$ que jusqu'à
$\omega$.

\begin{proposition}
  Pour tout $\alpha \leq \omega$, $\mathbb L_\alpha = \mathbb V_\alpha$.
  Cependant, pour tout $\alpha > \omega$, si $|\alpha| \neq \beth_\alpha$ alors
  $\mathbb L_\alpha \subsetneq \mathbb V_\alpha$.
\end{proposition}

\begin{proof}
  On procède par induction transfinie~:
  \begin{itemize}
  \item $\mathbb L_0 = \mathbb V_0$
  \item pour tout $n \in \omega$, si $\mathbb L_n = \mathbb V_n$, alors comme
    les deux ensembles sont finis~:
    \begin{align*}
      \mathbb L_{n+1} &= \powerdef(\mathbb L_n)\\
      &= \powerset(\mathbb L_n)\\
      &= \powerset(\mathbb V_n)\\
      &= \mathbb V_{n+1}
    \end{align*}
  \item si $\forall n < \omega, \mathbb L_n = \mathbb V_n$, alors
    $\mathbb L_\omega = \bigcup \mathbb L_n = \mathbb V_\omega$.
  \item comme $\mathbb L_\omega$ est infini,
    $\mathbb L_{\omega+1} \subsetneq \powerset(\mathbb L_\omega)$, donc
    $\mathbb L_{\omega+1} \subsetneq\mathbb V_{\omega+1}$.
  \item comme on sait que $|\powerdef(X)| = |X|$ pour $X$ infini, on en déduit
    que $|\mathbb L_{\alpha}| = |\alpha|$ pour $\alpha > \omega$. En comparaison,
    $|\mathbb V_\alpha| = \beth_\alpha$ à partir de $\alpha^2$, donc l'égalité
    des deux ensembles implique que $|\alpha| = \beth_\alpha$, d'où l'inclusion
    stricte par contraposée.
  \end{itemize}
\end{proof}

Cependant, $\mathbb L$ contient bien tous les ordinaux.

\begin{proposition}
  Pour tout $\alpha \in \Ord$, $\alpha \in \mathbb L_{\alpha + 1}$.
\end{proposition}

\begin{proof}
  On raisonne par induction transfinie~:
  \begin{itemize}
  \item $\varnothing \in \mathbb L$.
  \item si $\alpha \in \mathbb L_{\alpha + 1}$,
    alors $\alpha \in \mathbb L_{\alpha + 2}$ et on définit la formule
    \[\varphi(x,\alpha) \defeq x = \alpha \lor x \in \alpha\]
    qui définit exactement la partie
    $\alpha + 1 \subseteq \mathbb L_{\alpha + 1}$, donc
    $\alpha + 1 \in \mathbb L_{\alpha + 2}$.
  \item si $\lambda$ est un ordinal limite, que pour tout $\beta < \lambda$,
    $\beta \in \mathbb L_\beta$, alors on peut définir $\lambda$ par
    \[\varphi(x,\lambda) \defeq \exists \beta\in \Ord, x = \beta\]
  \end{itemize}
  Ainsi pour tout $\alpha \in \Ord, \alpha \in \mathbb L_{\alpha + 1}$.
\end{proof}

Il y a donc toujours plus d'éléments dans $\mathbb V_\alpha$ que dans
$\mathbb L_\alpha$. Cependant, on pourrait imaginer qu'en prenant par exemple
$\alpha' = \beth_\alpha$, on puisse rattraper $\mathbb V_\alpha$ en considérant
$\mathbb L_{\alpha'}$. Le fait que l'on puisse toujours rattraper la hiérarchie
$(\mathbb V_{\alpha})_{\alpha\in\Ord}$ avec $\mathbb L$ est indépendant de $\ZFC$,
d'où l'introduction de l'axiome suivant~:

\begin{axiom}[V = L]\label{ax.VeqL}
  On définit l'axiome $\VeqL$ par la formule suivante~:
  \[\VeqL \defeq \forall x, \exists \alpha \in \Ord, x \in \mathbb L_\alpha\]
\end{axiom}

Nous étudions maintenant les propriétés de $\ZFC+\VeqL$. Pour cela, cherchons
d'abord à situer dans la hiérarchie de Lévy les éléments importants de l'univers
constructible.

\begin{property}
  La définition d'une fonction $H$-inductive est $\Delta_0$ si $H$ est
  $\Delta_0$.
\end{property}

\begin{proof}
  En effet, dire que $f : \alpha \to \mathbb V$ est $H$-inductive signifie que~:
  \begin{itemize}
  \item $\forall \beta < \alpha, f\restr{\beta}\subseteq \dom(H)$
  \item $\forall \beta < \alpha, f(\beta) = H(f\restr{\beta})$
  \end{itemize}
  Il nous suffit donc de montrer~:
  \begin{itemize}
  \item qu'être le domaine de $f$ est un prédicat $\Delta_0$, et on peut
    exprimer \og $X$ est le domaine de $f$\fg par
    \[\forall (x,y)\in f, x\in X \land \forall x \in X, \exists (z,y)\in f,
    x = z\]
    donc $\dom(f)$ est défini par un prédicat $\Delta_0$.
  \item qu'être une restriction de $f$ est un prédicat $\Delta_0$, ce qui est
    direct en voyant que
    \[g = f\restr{X} \iff \dom(g)\subseteq\dom(f)\land
    \forall x \in \dom(g), g(x)=f(x)\]
  \end{itemize}
  Comme $H$ est supposé $\Delta_0$, il vient qu'être $H$-inductif est un
  prédicat $\Delta_0$. De plus, si $H$ est $\Delta_n$ (respectivement
  $\Sigma_n$, $\Pi_n$) alors être $H$-inductif l'est aussi.
\end{proof}

Ainsi exprimer, pour $\alpha$ un ordinal, le prédicat d'être $\mathbb L_\alpha$,
est à un niveau aussi élevé que le prédicat utilisé pour sa définition, qui est
clairement la complexité de $\powerdef$.

C'est précisément ici que l'on voit que $\mathbb L$ se comporte beaucoup mieux
que $\mathbb V$~: là où $\powerset$ est une fonction $\Pi_1$, $\powerdef$ n'est
que $\Delta_0$.

\begin{property}\label{prop.Form.Delta}
  Le prédicat $Y = \Formula(\mathcal L_{\ZF})$ est $\Delta_0$.
\end{property}

\begin{proof}
  La définition est par induction, et la relation permettant de construire
  de nouvelles formules à partir d'anciennes est clairement $\Delta_0$.
\end{proof}

\begin{property}
  Le prédicat $Y = \powerdef(X)$ est $\Delta_0$.
\end{property}

\begin{proof}
  On sait grâce à la \cref{prop.Form.Delta} que l'on peut quantifier
  sur $\Formula(\mathcal L_{\ZF})$ sans souci. La définition de
  $X\models \varphi$ est elle aussi simplement une induction sur
  $\Formula(\mathcal L_{\ZF})$. La quantification sur les éléments de $X$ est,
  elle aussi, $\Delta_0$. Ainsi
  \begin{multline*}
    \exists \varphi(x_1,\ldots,x_n)\in\Formula(\mathcal L_{\ZF}),
    \exists a_1,\ldots,a_n\in X, \forall x\in X, \\
    x\in Y \iff X\models \varphi(a_1,\ldots,a_n,x)
  \end{multline*}
  est bien $\Delta_0$, donc $Y\in \powerdef(X)$ est un prédicat $\Delta_0$.
  Maintenant, dire que $Y = \powerdef(X)$ signifie
  \begin{multline*}
    (\forall y \in Y, y \in \powerdef(X)) \land
    \forall \varphi\in\Formula(\mathcal L_{\ZF}),
    \forall a_1,\ldots,a_n\in X, \exists y \in Y, \forall x \in X,\\
    x \in y \iff X\models \varphi(a_1,\ldots,a_n,x)
  \end{multline*}
  donc $Y = \powerdef(X)$ est un prédicat $\Delta_0$.
\end{proof}

\begin{corollary}
  Pour tout ordinal $\alpha$ et toute classe transitive $\mathcal M$, le
  prédicat $x\in \mathbb L_{\alpha}$ est absolu sur $\mathcal M$.
\end{corollary}

On peut donc placer l'axiome de constructibilité dans la hiérarchie de Lévy.

\begin{proposition}
  $\VeqL \in \Pi_2$.
\end{proposition}

\begin{proof}
  On peut énoncer $\VeqL$ par
  $\forall x, \exists \alpha \in \Ord, x\in \mathbb L_\alpha$ qui est clairement
  $\Pi_2$ avec les résultats précédents.
\end{proof}

Une conséquence intéressante est que $\VeqL$ est un axiome vérifié dans
$\mathbb L$. Cela parait assez évident, mais il faut rappeler
qu'\textit{a priori}, on pourrait obtenir plus de parties en regardant
$\powerdef$ dans $\mathbb V$. L'avantage justement ici est que toutes nos
fonctions sont absolues pour $\mathbb L$. 

\begin{theorem}
  $\ZFC\models (\VeqL)^{\mathbb L}$
\end{theorem}

\begin{proof}
  On sait que le prédicat $x\in \mathbb L_{\alpha}$ est vrai dans
  $\mathbb L$ si et seulement s'il est vrai dans $\mathbb V$. Par définition,
  on a
  \[(\VeqL)^{\mathbb L} =
  \forall x \in \mathbb L,\exists \alpha \in \Ord\cap \mathbb L,
  x\in \mathbb L_\alpha\]
  mais on a vu que $\Ord\subseteq \mathbb L$, donc il nous reste à prouver que
  $\forall x \in \mathbb L, \exists \alpha \in \Ord, x \in \mathbb L_\alpha$
  ce qui est vrai dans $\mathbb V$ par définition, et donc dans
  $\mathbb L$ par absoluité.
\end{proof}

On montre aussi que $\mathbb L$ est un modèle de $\ZF$. On donne d'abord un
lemme, similaire à \cref{lem.axZFC}, mais utilisant des conditions plus fortes.

\begin{lemma}\label{lem.axZFC2}
  Soit $\mathcal M$ une classe transitive vérifiant l'axiome de compréhension
  et telle que pour tout $x \subseteq \mathcal M$, si $x$ est un ensemble, alors
  il existe $y \in \mathcal M$ tel que $x \subseteq y$. Dans ce cas,
  $\mathcal M$ est un modèle de $\ZF$.
\end{lemma}

\begin{proof}
  On traite les différents axiomes~:
  \begin{itemize}
  \item axiome d'extensionalité~: $\mathcal M$ est une classe transitive.
  \item axiome de fondation~: vrai puisque $\mathcal M$ est une partie de
    $\mathbb V$.
  \item axiome de la paire~: on voit que si $x, y \in \mathcal M$ alors
    $\{x,y\} \subseteq \mathcal M$, puis on applique l'axiome de compréhension
    à l'ensemble contenant $\{x,y\}$ et appartenant à $\mathcal M$ qu'on trouve
    par hypothèse sur $\mathcal M$.
  \item axiome de la réunion~: si $x \in \mathcal M$ alors par transitivité de
    $\mathcal M$, $\bigcup x \subseteq \mathcal M$, d'où le résultat par
    compréhension et hypothèse sur $\mathcal M$.
  \item axiome de compréhension~: vrai par hypothèse.
  \item axiome de remplacement~: supposons que $f$ est une fonction dont le
    domaine est $x \in \mathcal M$ et l'image $y \subseteq \mathcal M$. Alors
    par hypothèse on trouve $z \in \mathcal M$ contenant $y$, et on utilise
    l'axiome de compréhension sur $z$ avec la formule
    \[\varphi(b) \defeq \exists a \in x, f(x) = b\]
    pour en déduire que $y \in \mathcal M$.
  \item axiome de l'ensemble des parties~: soit $x \in \mathcal M$. On
    considère
    \[\mathcal P \defeq \{ y \in \mathcal M \mid y \subseteq x\}\]
    par hypothèse sur $\mathcal M$, on trouve $z \in \mathcal M$ tel que
    $\mathcal P \subseteq z$, et par compréhension on peut restreindre $z$
    aux éléments contenus dans $x$, nous redonnant $\mathcal P$. Ainsi pour
    tout $y \in \mathcal M, y \in \mathcal P \iff y \subseteq x$.
  \item axiome de l'infini~: comme $\mathcal M$ est clos par parties finies
    grâce à l'axiome de la paire et de la réunion, et puisque
    $\varnothing \in \mathcal M$ par compréhension, on en déduit que pour tout
    $n \in \omega, V_n \subseteq \mathcal M$, donc
    $\mathbb V_\omega \subseteq \mathcal M$. On applique alors notre hypothèse
    pour trouver $z$ contenant $\mathbb V_\omega$ et donc $\omega$, puis on
    utilise la compréhension.
  \end{itemize}
  On a ainsi montré notre critère pour avoir un modèle de $\ZF$.
\end{proof}

On peut ainsi prouver que $\mathbb L$ est un modèle (classe) de $\ZF$.

\begin{theorem}
  $\ZF\models (\ZF)^{\mathbb L}$.
\end{theorem}

\begin{proof}
  On vérifie les hypothèses du \cref{lem.axZFC2}. On sait déjà que $\mathbb L$
  est transitif. Pour la dernière hypothèse, si $x$ est un ensemble inclus dans
  $\mathbb L$, alors pour tout $y \in x$ on trouve $\alpha_y$ tel que
  $y \in \mathbb L_{\alpha_y}$. On sait alors que pour
  \[\alpha \defeq \sup_{y \in x} \alpha_y\]
  on a $x \subseteq \mathbb L_\alpha \in \mathbb L$.

  On veut maintenant prouver l'axiome de compréhension. Soit donc
  $\psi^{\mathbb L}(x_1,\ldots,x_n,x,y)$ une formule sur $\mathcal L_{\ZF}$, et
  $a_1,\ldots,a_n,a$ des paramètres dans $\mathbb L$. Quitte à prendre une
  borne supérieure, soit $\alpha$ tel que
  $a_1,\ldots,a_n,a \in \mathbb L_\alpha$. Par le \cref{thm.reflexion}, on
  trouve $\beta > \alpha$ tel que $\psi^{\mathbb L}(x_1,\ldots,x_n,x,y)$ est
  absolue entre $\mathbb L_\beta$ et $\mathbb L$. On a alors
  \[\{ b \in \mathbb L_\beta \mid b \in a \land
  \psi^{\mathbb L}(a_1,\ldots,a_n,a,b)\} \in \mathbb L_{\beta + 1}\]
  et un élément de $\mathbb L$ vérifie $\psi^{\mathbb L}$ si et seulement s'il
  le vérifie dans $\mathbb L_{\beta}$, si et seulement s'il appartient à $b$.
  On a donc vérifié l'axiome de compréhension.

  Ainsi $\mathbb L \models \ZF$.
\end{proof}

Il en découle donc que $\VeqL$ est relativement cohérent~: on peut ajouter cet
axiome sans risque puisque si ajouter $\VeqL$ mène à une incohérence, alors
cette incohérence était déjà dans $\ZF$.

\begin{corollary}
  Si $\Coher(\ZF)$ alors $\Coher(\ZF + \VeqL)$.
\end{corollary}

\begin{remark}
  On verra qu'en fait $\VeqL$ implique une version forte de l'axiome du choix,
  donc on a aussi une preuve de $\Coher(\ZF)\implies \Coher(\ZFC)$.
\end{remark}

\subsection{Les propriétés de l'univers constructible}

On montre dans cette sous-section que l'univers $\mathbb L$ vérifie des
résultats particulièrement forts~: l'axiome du choix global et l'hypothèse du
continu généralisée.

Présentons donc l'axiome du choix global. Plutôt qu'un simple axiome, celui-ci
a besoin d'autres données pour être exprimé. Nous considérons que nous sommes
munis d'un symbole $F$ de fonction unaire.

\begin{axiom}[Axiome du choix global]
  L'axiome du choix global, pour le symbole de fonction $F$, s'exprime par
  \[\forall x, x\neq \varnothing \implies F(x)\in x\]
\end{axiom}

Dans notre cas, puisqu'on cherche à représenter des fonctions par des prédicats
binaires fonctionnels, on peut à la place considérer une formule $\varphi(x,y)$
fonctionnelle. L'axiome du choix global pour $\varphi$ (supposée fonctionnelle)
est alors
\[\forall x, x \neq \varnothing \implies (\forall y, \varphi(x,y)\implies
y\in x)\]

On cherche donc à construire une telle formule $\varphi$.

En fait, on voit avec les constructions précédentes à propos de l'axiome du
choix qu'il nous suffit de pouvoir bien ordonner l'univers, c'est-à-dire
construire un prédicat $\leL$ vérifiant les axiomes suivants~:
\begin{itemize}
\item $\forall x, \lnot(x \leL x)$
\item $\forall x,y,z, (x \leL y \land y \leL z) \implies x \leL z$
\item $\forall x,y, x\neq y \implies x \leL y \lor y \leL x$
\item $\forall x, x\neq \varnothing \implies
  \exists y, y \in x \land (\forall z \in x, \lnot (z \leL y))$
\end{itemize}
La fonction est alors simplement
\[\varphi(x,y) \defeq y \in x \land \forall z \in x, \lnot(z \leL y)\]
qui est évidemment fonctionnelle totale pour $x\neq \varnothing$ puiqu'il existe
un tel $y$ par le dernier point, et les autres axiomes assurent que ce $y$ est
unique.

Il nous faut donc construire $\leL$. Pour cela, remarquons deux choses~:
\begin{itemize}
\item $\Formula(\mathcal L_{\ZF})$ peut être bien ordonné.
\item si $\mathbb L_\alpha$ peut être bien ordonné, alors l'ensemble des
  suites finies $\mathbb L_\alpha^{<\omega}$ à valeurs dans $\mathbb L_\alpha$
  peut l'être aussi.
\end{itemize}
En utilisant ces deux faits, comme un élément de $\mathbb L_{\alpha + 1}$
peut être associé à un tuple $(\varphi,x_1,\ldots,x_n)$ où
$\varphi\in\Formula(\mathcal L_{\ZF})$ et $x_1,\ldots,x_n \in\mathcal L_{\alpha}$,
on peut bien ordonner $\mathbb L_{\alpha + 1}$. Remarquons cependant que
plusieurs tuples peuvent correspondre à une seule partie définissable~: il faut
donc prendre à chaque fois le tuple le plus petit possible.

Enfin, la question se pose du recollement des ordres~: si on a défini un ordre
sur $\mathbb L_\alpha$ il faut que celui défini sur $\mathbb L_{\alpha + 1}$
contienne l'ordre défini juste avant. Cela nous mène à l'énoncé que l'on veut
donner.

\begin{theorem}
  Il existe une fonction $\leL : \Ord \to \mathbb V$ telle que que
  $\leL(\alpha)$ est un bon ordre sur $\mathbb L_\alpha$ pour tout
  $\alpha\in \Ord$ et telle que pour tous $\alpha < \beta$,
  $\leL(\alpha)$ est un segment initial de $\leL(\beta)$.
\end{theorem}

\begin{proof}
  Notre démonstration est une induction transfinie~:
  \begin{itemize}
  \item pour $\varnothing$, le résultat est direct.
  \item supposons que $\leL(\alpha)$ existe. On utilise ce qu'on a dit
    précédemment~: on se fixe un bon ordre $<$ sur
    $\Formula(\mathcal L_{\ZF})\times \mathbb L_{\alpha}^{<\omega}$, soit
    $\beta$ le type de cet ordre. On peut associer à chaque partie
    $X\in\mathbb L_{\alpha + 1}\setminus\mathbb L_\alpha$ un élément de $\beta$
    comme étant le plus petit tuple $(\varphi,a_1,\ldots,a_n)$ tel que
    $x\in X \iff \mathbb L_{\alpha}\models \varphi(a_1,\ldots,a_n,x)$.
    Cette association est clairement injective, vu que la partie définie par
    un tuple est unique. On a donc une injection de
    $\mathbb L_{\alpha + 1}\setminus \mathbb L_\alpha$ vers un ordinal, donc un
    bon ordre sur $\mathbb L_{\alpha + 1} \setminus \mathbb L_\alpha$.
    En prenant simplement la somme ordinale avec l'ordre défini sur
    $\mathbb L_\alpha$, on trouve ainsi un bon ordre sur
    $\mathbb L_{\alpha + 1}$ tel que l'ordre sur $\mathbb L_\alpha$ en est
    un segment initial.
  \item si $\leL(\beta)$ a été défini pour tout ordinal $\beta < \lambda$,
    avec $\lambda$ un ordinal limite, alors on peut définir $\leL(\lambda)$
    comme l'union des $\leL(\beta)$. Comme tous les bons ordres sont
    compatibles, l'union est encore un bon ordre.
  \end{itemize}
  On a donc construit notre fonction $\leL : \Ord \to \mathbb V$ telle que
  $\leL(\alpha)$ est un bon ordre sur $\mathbb L_\alpha$ pour tout
  $\alpha\in\Ord$.
\end{proof}

On en déduit l'axiome du choix global.

\begin{corollary}
  Il existe une relation $\leL$ sur $\mathbb L$ qui est un bon ordre au sens
  défini plus tôt, donc $\ZF\models (\AxC)^{\mathbb L}$.
\end{corollary}

On veut maintenant montrer que $\mathbb L$ vérifie $\HCG$.

Pour cela, on va montrer d'abord que, sous l'hypothèse $\VeqL$, la hiérarchie
$\mathbb H$ coïncide avec $\mathbb L$ sur les cardinaux infinis.

On remarque que, comme $\mathbb H_\lambda$ et $\mathbb L_\lambda$ peuvent
s'exprimer comme l'union des $\mathbb H_\beta$ (respectivement $\mathbb L_\beta$)
pour $\beta < \lambda$ lorsque $\lambda$ est limite, il nous suffit de nous
occuper du cas successeur.

L'inclusion qui nous demandera du travail est
$\mathbb H_\kappa \subseteq\mathbb L_\kappa$, nous allons donc l'étudier à part.

\begin{lemma}
  Soit $\kappa$ un cardinal infini successeur, $\kappa = \lambda^+$. Si
  $\VeqL$, alors $\mathbb H_\kappa \subseteq \mathbb L_\kappa$.
\end{lemma}

\begin{proof}
  Soit $x \in \mathbb H_\kappa$. On pose $y = \trcl(\{x\})$, on sait donc que
  $x \in y$ et $|y| \leq \lambda < \lambda^+$. On définit $\rho(y)$ comme étant
  le rang minimal tel que $y\in\mathbb L_{\rho(y)}$. On peut trouver un cardinal
  régulier indénombrable $\theta > \rho(y)$ dans $\mathbb V$, donc dans
  $\mathbb L$ puisque $\VeqL$. On peut vérifier qu'alors,
  $\mathbb L_{\theta}\models \ZF - \mathrm{P} + (\VeqL)$. On utilise alors le
  théorème de Löwenheim-Skolem pour trouver un modèle $M$ vérifiant les mêmes
  énoncés de $\mathcal L_{\ZF}$ que $\mathbb L_{\theta}$, contenant
  $y$ et tel que $|M| \leq \lambda$ (ce qui est possible car $|y|\leq \lambda$).

  En utilisant alors le \cref{lem.most}, on trouve un ensemble $N$ et un
  isomorphisme $\pi : (M,\in)\cong (N,\in)$. Comme $y\in M$ est transitif, il
  est un point fixe par l'effondrement, donc pour tout $z\in y, \pi(z) = z$. En
  particulier, $\pi(x) = x$ et $x\in N$.

  On veut alors prouver que $N = \mathbb L_\beta$ avec $\beta = N\cap \Ord$.

  Soit $\alpha < \beta$, comme $\alpha \mapsto \mathbb L_\alpha$ est absolu
  pour les modèles de $\ZF - \mathrm{P}$, l'axiome de remplacement nous donne
  que $\mathbb L_\alpha \in N$. Comme, de plus, $N$ est transitif,
  $\mathbb L_\alpha \subseteq N$ donc $\mathbb L_\beta \subseteq N$.

  Réciproquement, comme $N\models \VeqL$, et par absoluité, on a
  \[(\VeqL)^N \iff \forall x \in N, \exists \alpha \in \Ord\cap N,
  x\in \mathbb L_\alpha\]
  Puisque $\beta = \Ord \cap N$, on en déduit que
  $\forall x \in N, \exists \alpha < \beta, x\in \mathbb L_\alpha$ donc
  $N\subseteq\mathbb L_\beta$.

  Ainsi $|\beta| = |N| = |M| = \lambda < \lambda^+$, et $x\in N$, donc
  $x\in L_\beta$ avec $\beta < \lambda^+$, donc $x\in L_\kappa$.
\end{proof}

Il nous reste maintenant à vérifier l'autre inclusion.

\begin{lemma}
  Soit $\kappa$ un cardinal infini, on suppose de plus que $\VeqL$. Alors
  $\mathbb H_\kappa = \mathbb L_\kappa$.
\end{lemma}

\begin{proof}
  Comme on l'a dit, le cas des cardinaux limite est direct par continuité de
  $\mathbb H_\alpha$ et $\mathbb L_\alpha$. Il nous reste donc à prouver que
  $\mathbb L_\kappa\subseteq \mathbb H_\kappa$ pour $\kappa = \lambda^+$.

  Si $x\in \mathbb L_\kappa$, alors on trouve $\alpha < \kappa$ tel que
  $x\in \mathbb L_\alpha$, mais alors $\trcl(x)\subseteq \mathbb L_\alpha$ et, de
  plus, $|\mathbb L_\alpha| = |\alpha| < \kappa$.
  Donc $\trcl(x)\in\mathbb H_\kappa$.

  Ainsi $\mathbb H_\kappa = \mathbb L_\kappa$.
\end{proof}

On en déduit $\HCG$ dans $\VeqL$.

\begin{theorem}
  Si $\VeqL$, alors $\HCG$ est vérifiée.
\end{theorem}

\begin{proof}
  Soit un cardinal infini $\lambda$, alors
  $\powerset(\lambda) \subseteq \mathbb H_{\lambda^+} = \mathbb L_{\lambda^+}$,
  donc $|\powerset(\lambda)| \leq \lambda^+$.
\end{proof}

Nous concluons donc ce chapitre sur ce résultat important~:
\[\Coher(\ZFC) \implies \Coher(\ZFC + \HCG)\]

$\HCG$ ne peut donc pas être prouvée faux dans $\ZFC$. Pour montrer que ce
résultat est indépendant de $\ZFC$, il nous reste alors à montrer que
$\Coher(\ZFC)\implies\Coher(\ZFC+\lnot\HC)$, ce qui nécessite la technique du
forcing, introduite par Cohen.
