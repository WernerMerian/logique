\chapter{Modèles de ZFC}
\label{chp.modZFC}

\minitoc

Nous avons vu dans les chapitres précédents combien $\ZFC$ était une théorie
riche en terme d'expressivité, au point même où l'ensemble des mathématiques
peut se formaliser dans $\ZFC$. Il devient alors assez technique de parler de
modèle de $\ZFC$, puisqu'un tel modèle est en quelque sorte un modèle de toutes
les mathématiques que l'on connait.

Une première conséquence de cette expressivité est un certain flou entre ce que
l'on exprime dans $\ZFC$ et dans les maths usuelles. Puisque tout peut se
définir en terme d'ensembles, il semble que tout ce dont nous avons parlé jusque
là puisse se formaliser dans $\ZFC$. C'est le cas, mais il faut faire attention
à ce qu'on manipule.

En particulier, un modèle de $\ZFC$, nommons-le $\mathcal M$, contient lui-même
un ensemble $\Formula(\Sigma)$ pour une signature $\Sigma$ donnée. On pourrait
alors se dire que, comme on peut formaliser nos résultats de logique en terme
d'ensembles, on peut travailler au sein de $\ZFC$ pour faire de la logique.
Malheureusement, les propositions de $\Formula(\Sigma)$ ne sont pas les mêmes
que celles que nous connaissons~: on peut facilement interpréter une proposition
habituelle par un élément de $\Formula(\Sigma)$, mais l'inverse n'a pas de
raison d'être vrai.

Cela mène à deux résultats essentiels pour décrire les limitations de cette
approche formelle de la logique~: les théorèmes d'incomplétudes de Gödel et le
théorème d'indéfinissabilité de la vérité de Tarski, que nous verrons dans la
PARTIE TROIS. Ceux-ci ont pour conséquence que si l'on considère un codage
$\ceil -$ qui à une formule au sens habituel associe une formule de notre
modèle $\mathcal M$, alors il n'existe pas de formule $\varphi_\top$ telle que
\[\forall \psi \in \Formula(\Sigma), \mathcal M \models \psi
\iff \mathcal M\models \varphi_\top(\ceil \psi)\]

Une conséquence de cette limitation est que l'on ne peut pas à strictement
parler écrire une formule comme $\mathcal U\models \ZFC$ pour notre univers
ensembliste $\mathcal U$. Par contre, il est possible d'écrire par exemple
$\ZFC\vdash \varphi$, qui en utilisant le théorème de complétude permet de dire
que $\mathcal U \models \varphi$~: ainsi, l'écriture de $\mathcal U\models\ZFC$
se traduit en fait par une infinité de théorèmes de la forme
$\ZFC\vDash \varphi$ pour chaque axiome $\varphi \in \ZFC$.

S'il est assez clair que $\ZFC\vDash\ZFC$ est un constat assez vide, cela nous
permet ensuite d'introduire la relativisation, qui permet d'exprimer la notion
de modèle classe de $\ZFC$.

Ce chapitre s'attachera à l'étude des outils basiques à propos des modèles
classe. Nous verrons d'abord les définitions et conséquences de cette notion,
puis nous aborderons l'absoluité, qui permet d'étudier la conservation entre des
modèles de $\ZFC$.

Nous verrons ensuite la théorie des modèles intérieurs, et son exemple le plus
connu~: l'univers constructible de Gödel $\mathbb L$.

\section{Premiers pas}

\subsection{Relativisation}

On commence par introduire la notion de relativisation d'une formule. Comme on
l'a dit, il est impossible de représenter de façon interne à un modèle de $\ZFC$
que celui-ci vérifie une formule~: il faut donc parler de façon externe au
modèle. Pour exprimer que $\varphi$ est vraie dans l'univers ambiant, il suffit
alors de travailler normalement, en utilisant les axiomes de $\ZFC$. Cependant,
cette approche est très limitée, et on aimerait s'autoriser à parler d'autres
classes que l'univers $\mathcal U$ tout entier. Pour ce faire, comme on doit
procéder de façon purement syntaxique, on modifie la formule même qu'on veut
prouver en y incluant l'information de la classe qu'on utilise.

Disons qu'on veuille montrer qu'une formule $\varphi$ est vraie dans la classe
des ordinaux, l'idée est simplement de modifier tous les quantificateurs pour
que ceux-ci ne parlent que d'éléments de $\Ord$. La relativisation est
précisément ce procédé de réduire la portée des quantificateurs par une formule,
exprimant en général l'appartenance à une classe qui nous intéresse.

\begin{definition}[Relativisation d'une formule]
  Soient $\varphi,\psi\in\Formula(\mathcal L_{\ZF})$. On appelle relativisation
  de $\varphi$ à $\psi$ la formule $\varphi^\psi$ définie par induction sur
  $\varphi$ par~:
  \begin{itemize}
  \item si $\varphi = x \in y$ ou $\varphi = x \in y$ alors
    $\varphi^\psi = \varphi$.
  \item si $\varphi = \top$ alors $\varphi^\psi = \top$.
  \item si $\varphi = \bot$ alors $\varphi^\psi = \bot$.
  \item si $\varphi = \lnot \chi$ alors $\varphi^\psi = \lnot \chi^\psi$.
  \item si $\varphi = \chi \lor \xi$ alors
    $\varphi^\psi = \chi^\psi \lor \xi^\psi$.
  \item si $\varphi = \chi \land \xi$ alors
    $\varphi^\psi = \chi^\psi \land \xi^\psi$.
  \item si $\varphi = \chi \to \xi$ alors
    $\varphi^\psi = \chi^\psi \to \xi^\psi$.
  \item si $\varphi = \exists x, \chi$ alors
    $\varphi^\psi = \exists x, \psi(x) \land \chi^\psi$.
  \item si $\varphi = \forall x, \chi$ alors
    $\varphi^\psi = \forall x, \psi(x) \to \chi^\psi$.
  \end{itemize}
\end{definition}

Le théorème central pour l'étude des modèles classes est que la cohérence est
préservée par ceux-ci.

\begin{lemma}
  Pour $\Gamma\in\List(\Formula(\Sigma))$ et $\varphi,\psi\in\Formula(\Sigma)$,
  si $\Gamma\vdash \varphi$ alors $\Gamma^\psi \vdash \varphi^\psi$ où
  $\Gamma^\psi$ l'application de la relativisation à $\psi$ de tous les éléments
  de $\Gamma$.
\end{lemma}

\begin{proof}
  A FAIRE
\end{proof}

\begin{theorem}[Préservation de la cohérence]
  Soit $\mathcal T$ et $\mathcal S$ deux théories sur $\mathcal L_{\ZF}$ et
  $\psi(x)$ une formule à une variable libre dans ce même langage telles que
  pour toute formule
  $\varphi \in \mathcal S$, $\mathcal T\vdash \varphi^\psi$. Alors
  \[\Coher(\mathcal T) \implies \Coher(\mathcal S)\]
\end{theorem}

\begin{proof}
  On donne deux preuves, l'une syntaxique et l'autre sémantique.
  \begin{itemize}
  \item supposons $\lnot\Coher(\mathcal S)$, cela signifie donc qu'il existe une
    preuve $\mathcal S\vdash \bot$, et comme cette preuve est finie on peut
    extraire une partie finie $F\subfin \mathcal S$ telle que $F\vdash \bot$.
    En utilisant le lemme précédent, il vient donc que $F^\psi\vdash \bot^\psi$
    mais $\bot^\psi = \bot$, donc on en déduit que $F^\psi\vdash \bot$. Comme
    $\mathcal T\vdash \varphi^\psi$ pour chaque $\varphi \in F$, on en déduit
    que $\mathcal T\vdash \bot$, donc que $\lnot\Coher(\mathcal T)$. Par
    contraposée, on en déduit l'implication voulue.
  \item supposons $\Coher(\mathcal T)$. On trouve donc un modèle $\mathcal M$
    de $\mathcal T$. Soit $\mathcal N$ le sous-modèle de $\mathcal M$ défini
    par
    \[|\mathcal N|\defeq \{x \mid x\in |\mathcal M|,
    \mathcal M\models \psi(x)\}\]
    Par hypothèse, on sait que $\mathcal M\models \mathcal S^\psi$, donc
    $\mathcal N\models \mathcal S$, d'où $\Coher(\mathcal S)$.
  \end{itemize}

  Les deux approches nous donnent donc que
  $\Coher(\mathcal T) \implies \Coher(\mathcal S)$.
\end{proof}

On peut par exemple appliquer ce théorème à $\ZFC - \AxF$ et $\ZFC$ pour dire
que s'il existe une classe $\mathbb V$ telle que
$\ZFC - \AxF\vdash \ZFC^\mathbb V$, alors les deux théories sont équicohérentes
(puisque dans l'autre sens, il est évident qu'enlever des axiomes ne change pas
la cohérence).

\subsection{L'univers bien fondé de Von Neumann}

On peut maintenant introduire l'univers $\mathbb V$, qui est d'une certaine
façon l'univers ensembliste au sens intuitif~: il est un modèle de $\ZFC$
présent dans tout modèle de $\ZFC - \AxF$.

\begin{definition}[L'univers bien fondé]
  On définit la hiérarchie d'ensembles $\mathbb V_{\alpha}$, pour
  $\alpha \in \Ord$, par induction transfinie~:
  \begin{itemize}
  \item $\mathbb V_0 = \varnothing$.
  \item $\mathbb V_{\alpha + 1} = \powerset(\mathbb V_{\alpha})$
  \item pour tout ordinal limite $\lambda$,
    $\displaystyle\mathbb V_\lambda = \bigcup_{\beta < \lambda}\mathbb V_\beta$
  \end{itemize}

  On définit maintenant la classe $\mathbb V$ par
  \[\mathbb V \defeq \bigcup_{\alpha \in \Ord} \mathbb V_\alpha\]
\end{definition}

\begin{remark}
  Par une union indicée par $\alpha \in \Ord$, il faut en fait comprendre que le
  prédicat $\mathbb V(x)$ est défini comme
  \[\mathbb V(x) \defeq \exists \alpha \in \Ord, \mathbb V(\alpha,x)\]
  On préfère une approche sémantique que syntaxique ici, car celle-ci est bien
  plus facile à appréhender.
\end{remark}

Un premier résultat important~: la hiérarchie est cumulative, ce qui signifie
que les ensembles sont de plus en plus grands.

\begin{property}
  Si $\alpha \leq \beta$, alors $\mathbb V_\alpha\subseteq \mathbb V_\beta$.
\end{property}

\begin{proof}
  A FAIRE
\end{proof}

Une conséquence de ce résultat est la transitivité de chaque
$\mathbb V_\alpha$.

\begin{property}
  Pour chaque $\alpha\in\Ord$, $\mathbb V_\alpha$ est un ensemble transitif. De
  plus, $\mathbb V$ est une classe transitive.
\end{property}

\begin{proof}
  On raisonne par induction transfinie~:
  \begin{itemize}
  \item si $x\in \varnothing$, alors $x\subseteq\varnothing$.
  \item supposons que $\mathbb V_\alpha$ est transitif. Soit
    $x\in\mathbb V_{\alpha+1}$. Par définition, cela signifie que
    $x\subseteq \mathbb V_\alpha$. Soit $y\in x$, par inclusion cela signifie
    que $y\in \mathbb V_\alpha$, donc comme la hiérarchie est cumulative, on
    en déduit que $y\in \mathbb V_{\alpha +1}$, donc
    $x\subseteq \mathbb V_{\alpha +1}$.
  \item si pour tous $\beta < \lambda$, $\mathbb V_\beta$ est transitif, alors
    pour $x\in \mathbb V_\lambda$, on trouve $\alpha < \lambda$ tel que
    $x\in V_\alpha$, donc $x\subseteq V_\alpha$. De plus,
    $V_\alpha \subseteq V_\lambda$ donc $x\subseteq V_\lambda$.
  \end{itemize}
  Ainsi tous les $\mathbb V_\alpha$ sont transitif. Le fait que $\mathbb V$ est
  une classe transitive est une conséquence directe.
\end{proof}

Notre premier objectif est de montrer que $\mathbb V$ constitue un premier
modèle classe de $\ZFC$. Ce résultat est assez long à montrer directement. A la
place, nous allons introduire la notion de rang. Celui-ci va nous permettre de
montrer plus facilement que les axiomes sont vérifiés.

Le rang indique à quelle étape de la construction des $\mathbb V_\alpha$ notre
ensemble a été construit.

\begin{definition}[Rang]
  On définit le rang d'un ensemble $x\in\mathbb V$ par~:
  \[\rk(x) \defeq \min \{\alpha \in \Ord\mid x\subseteq \mathbb V_\alpha\}\]
\end{definition}

Le rang d'un ensemble peut se voir comme la profondeur d'accolades imbriquées
lorsque l'on définit l'ensemble en question. Ce rang se comporte bien vis à vis
des opérations ensemblistes usuelles, comme nous allons le voir.

\begin{proposition}\label{prop.cara.rk}
  Soit un ensemble $X\in\mathbb V$, on peut caractériser son rang par
  \[\rk(X) = \sup_{x \in X} (\rk(x) + 1)\]
\end{proposition}

\begin{proof}
  A FAIRE
\end{proof}

\begin{property}
  Les propriétés suivantes sont vérifiées~:
  \begin{itemize}
  \item $\forall x,y\in\mathbb V, x\in y \implies \rk(x) + 1 \leq \rk(y)$
  \item $\forall x,y\in\mathbb V, x\subseteq y \implies \rk(x) \leq \rk(y)$
  \item $\displaystyle \forall x\in\mathbb V,
    \rk\left(\bigcup x\right) + 1 \leq \rk(x)$
  \item $\forall x,y\in\mathbb V, \rk(x\cup y) = \max(\rk(x),\rk(y))$
  \item $\forall x,y\in\mathbb V, \rk(\{x,y\}) = \max(\rk(x),\rk(y)) + 1$
  \item $\forall x\in\mathbb V, \rk(\powerset(x)) = \rk(x) + 1$
  \item $\forall x\in\mathbb V, \rk(\trcl(x)) = \rk(x)$
  \item $\displaystyle \forall \alpha \in \Ord, \rk(\alpha) = \alpha$
  \end{itemize}
\end{property}

\begin{proof}
  Vérifions chaque propriété~:
  \begin{itemize}
  \item l'inégalité vient directement de la \cref{prop.cara.rk}.
  \item soient $x,y$ tels que $x\subseteq y$. Si $y\subseteq\mathbb V_\alpha$,
    comme $x\subseteq y$, on en déduit que $x\subseteq\mathbb V_\alpha$, donc
    $\rk(x)\leq \rk(y)$.
  \item soit $y\in \bigcup x$, on trouve $z\in x$ tel que $y\in z \in x$. En
    utilisant l'inégalité précédente, on en déduit que $\rk(y) + 1 \leq \rk(z)$
    et que $\rk(z) + 1 \leq \rk(x)$, donc $\rk(y) + 2 \leq \rk(x)$. On en détuit
    en utilisant la \cref{prop.cara.rk} que $\rk(\bigcup x) + 1 \leq \rk(x)$.
  \item soient $x,y$. On utilise directement la \cref{prop.cara.rk}~:
    \begin{align*}
      \rk(x\cup y) &= \sup_{z \in x \cup y} (\rk(z) + 1)\\
      &= \max(\sup_{z \in x} (\rk(z) + 1), \sup_{z\in y} (\rk(z) + 1))\\
      &= \max(\rk(x),\rk(y))
    \end{align*}
  \item pour deux éléments $x,y$, il est clair que le sup est un max, d'où
    l'égalité.
  \item soit un ensemble $x$. comme $x\in \mathcal P(x)$, on en déduit que
    $\rk(x) + 1 \leq \mathcal (\powerset (x))$. Pour l'autre inclusion, si
    $y\in \mathcal (\powerset(x))$ alors $y\subseteq x$, donc
    $\rk(y) \leq \rk(x)$ par un point précédent. On en déduit que
    $\sup_{y\in \powerset(x)} (\rk(y) + 1) \leq \rk(x) + 1$ d'où
    l'égalité.
  \item comme chaque $\mathbb V_\alpha$ est transitif, si
    $x\subseteq \mathbb V_\alpha$ alors $\trcl(x)\subseteq \mathbb V_\alpha$. On
    en déduit que $\rk(\trcl(x)) \leq \trcl(x)$, et il est clair que l'autre
    inégalité tient toujours puisqu'on rajoute des éléments.
  \item on prouve cela par induction transfinie~:
    \begin{itemize}
    \item $\varnothing \subseteq \varnothing$ donc $\rk(0) = 0$.
    \item on sait que $\alpha + 1 = \alpha \cup \{\alpha\}$. En utilisant les
      précédentes égalités, $\rk(\{\alpha\}) = \rk(\alpha) + 1$ donc
      $\rk(\alpha + 1) = \alpha + 1$.
    \item si $\lambda$ est un ordinal limite, alors
      $\rk(\lambda) = \sup_{\beta < \lambda} \rk(\beta)$ et par hypotèse
      d'induction, $\rk(\beta) = \beta$, donc $\rk(\lambda) = \lambda$.
    \end{itemize}
    Ainsi pour tout $\alpha \in \Ord, \rk(\alpha) = \alpha$.
  \end{itemize}
  Les propriétés sont donc vérifiées.
\end{proof}

\begin{theorem}
  Pour tout $\varphi \in \ZFC$, on a $\ZFC - \AxF \vdash \varphi^\mathbb V$.
\end{theorem}

\begin{proof}
  On vérifie chaque axiome relativisé~:
  \begin{itemize}
  \item vérifions l'axiome d'extensionalité. Soient $x,y\in \mathbb V$ tels que
    $\forall z\in \mathbb V, z\in x\iff z \in y$. Comme $\mathbb V$ est une
    classe transitive, pour tous $z\in x, z\in \mathbb V$ et de même pour les
    éléments de $y$. On en déduit que $\forall z, z\in x \iff z \in y$, donc
    l'axiome d'extensionalité peut s'appliquer directement pour conclure que
    $x = y$.
  \item vérifions l'axiome de la réunion~: si $x\in \mathbb V$ alors on trouve
    $\alpha$ tel que $\rk(x) = \alpha$. A FAIRE
  \item vérifions l'axiome de l'ensemble des parties~: soit $x\in \mathbb V$,
    on trouve $\alpha$ tel que $\rk(x) = \alpha$. Alors pour tous
    $y\subseteq x$, $y\subseteq \mathbb V_\alpha$ donc
    $y\in\mathbb V_{\alpha +1}$. On en déduit que pour tous $y\in \powerset(x)$,
    $y\in \mathbb V$, donc $\powerset(x)\in\mathbb V$.
  \item vérifions l'axiome de remplacement~: A FAIRE
  \item vérifions l'axiome de l'infini~: on sait que $\Ord\subseteq\mathbb V$,
    donc il existe bien un ensemble $\omega^\mathbb V \in \mathbb V$
    contenant $0$ et stable par successeur, c'est $\omega$ lui-même.
  \item vérifions l'axiome de fondation~: A FAIRE
  \item vérifions l'axiome du choix~: A FAIRE
  \end{itemize}
  D'où le résultat.
\end{proof}

On a donc notre premier modèle classe de $\ZFC$. Un point important lorsque l'on
ne travaille pas \emph{a priori} avec l'axiome de fondation est que la classe
$\mathbb V$ correspond exactement à la classe des éléments bien fondés de
l'univers ambiant.

\begin{proposition}
  Soit un ensemble $x$, alors on a l'équivalence suivante dans $\ZFC-\AxF$~:
  \[x\in \mathbb V \iff \exists y \in \trcl(x), \trcl(x) \cap y = \varnothing\]
\end{proposition}

\begin{proof}
  L'implication directe est une conséquence du fait que $\mathbb V\models \AxF$.
  Soit $x$ tel que $\exists y \in \trcl(x), \trcl(x) \cap y = \varnothing$. On
  en déduit que $\in$ est une relation bien fondée sur $x$, 
\end{proof}

On voit donc que dans le cas d'un univers ensembliste $\mathcal U$, celui-ci
vérifie $\AxF$ si et seulement si $\mathbb V = \mathcal U$. On utilisera donc en
général $\mathbb V$ pour parler de l'univers ensembliste.

\section{Vers le théorème de réflexion}

A FAIRE

