\chapter{Fonctions récursives, machines de Turing}
\label{chp.recur}

\minitoc

\lettrine{C}{e} chapitre est celui commençant à traiter la calculabilité à
proprement parler. Jusque là, nous avons vu des versions simples d'automates, et
de langages traités par ces automates, pour introduire donner une meilleure
connaissance du traitement des mots par des automates.

Les langages algébriques (et encore plus les langages rationnels) restent
restreints. Par exemple, un langage tel que $\{a^nb^nc^n\mid n \in \mathbb N\}$
n'est pas algébrique, bien qu'il existe clairement un algorithme permettant de
le reconnaître. On souhaite donc trouver une notion plus forte et expressive de
fonction.

Cette notion plus forte est celle de fonction calculable, que nous allons voir
dans ce chapitre à travers les fonctions récursives en premier lieu, puis par
les machines de Turing.

\section{Fonctions primitives récursives}

Avant d'introduire les fonctions récursives, nous en donnons une version faible
mais très importante~: les fonctions primitives récursives (abrégées en
fonctions RP). Si l'on imagine une fonction calculable comme un programme
C ou Python, une fonction RP est un programme n'employant comme seule boucle que
la boucle \texttt{for}.

\subsection{Définitions et premières fonctions}

Donnons dès maintenant la définition de la classe RP.

\begin{definition}[Fonctions primitives récursives]
  On définit la classe
  $\RecP\subseteq \bigcup_{n \in \mathbb N}\Funct(\mathbb N^n,\mathbb N)$
  comme la plus petite classe telle que~:
  \begin{itemize}
  \item soient les projections
    \[\begin{array}{ccccc}
    \pi_k^n &:& \mathbb N^n &\longrightarrow& \mathbb N\\
    & & (x_1,\ldots,x_n) & \longmapsto & x_k
    \end{array}\]
    on a pour tout $n,k\in \mathbb N, \pi_k^n \in \RecP$.
  \item pour tout $n \in \mathbb N$, on note
    $\mathrm k_0^n : \mathbb N^p \to \mathbb N$ la fonction nulle. Pour tout
    $n \in \mathbb N$, $\mathrm k_0^n \in \RecP$.
  \item la fonction $n \mapsto n + 1$ est primitive récursive~: $\sucs\in\RecP$.
  \item si $f_1,\ldots,f_n : \mathbb N^k \to \mathbb N$ sont des fonctions RP
    et $g : \mathbb N^n \to \mathbb N$ est une fonction RP, alors la fonction
    \[\begin{array}{ccccc}
    g \circ (f_1,\ldots,f_n) & : & \mathbb N^k &\longrightarrow & \mathbb N\\
    & & (x_1,\ldots,x_k) & \longmapsto &
    g(f_1(x_1,\ldots,x_k),\ldots,f_n(x_1,\ldots,x_k))
    \end{array}\]
    est elle aussi RP.
  \item si $f : \mathbb N^k \to \mathbb N$ et
    $g : \mathbb N^{k+2} \to \mathbb N$ sont des fonctions RP, alors la fonction
    \[\begin{array}{ccccc}
    \rec(f,g) & : & \mathbb N^{k+1} &\longrightarrow & \mathbb N \\
    & & (x_1,\ldots,x_k,0) &\longmapsto & f(x_1,\ldots,x_k)\\
    & & (x_1,\ldots,x_k,y+1) & \longmapsto &
    g(x_1,\ldots,x_k,y,\rec(f,g)(x_1,\ldots,x_k,y))
    \end{array}\]
    est elle aussi RP. On appelle cette opération la récursion primitive.
  \end{itemize}
\end{definition}

\begin{remark}
  Comme toujours, on identifie $\mathbb N^0 \to \mathbb N$ avec $\mathbb N$.
  Ainsi on peut définir une fonction $\mathbb N \to \mathbb N$ à partir d'une
  constante $x_0 \in \mathbb N$ et d'une fonction
  $f_s : \mathbb N^2 \to \mathbb N$.
\end{remark}

\begin{remark}
  Toute fonction RP est une fonction totale, puisque les opérations permettant
  de construire de nouvelles fonctions RP conservent la totalité, et les
  fonctions RP de base sont totales.
\end{remark}

On peut donc, dès lors, donner plusieurs fonctions RP élémentaires.

\begin{proposition}
  Les fonctions suivantes sont RP~:
  \begin{itemize}
  \item pour tout $n \in \mathbb N$ et tout $p \in \mathbb N$, la fonction
    constante
    \[\begin{array}{ccccc}
    \mathrm k_n^p &:& \mathbb N^p&\longrightarrow &\mathbb N\\
    & & (x_1,\ldots,x_p) &\longmapsto & n
    \end{array}\]
  \item $+ : \mathbb N^2 \to \mathbb N$
  \item $\times : \mathbb N^2 \to \mathbb N$
  \item $\exp : \mathbb N^2 \to \mathbb N$
  \end{itemize}
\end{proposition}

\begin{proof}
  Pour les fonctions constantes, il nous suffit de composer $n$ fois la fonction
  $\sucs$ à la fonction nulle $\mathrm k_0^n$.
  
  On sait que les autres fonctions souhaitées sont définies par ces équations~:
  \begin{itemize}
  \item $\forall n \in \mathbb N, n+0 = n$
  \item $\forall n,m \in \mathbb N, n + \sucs(m) = \sucs(n+m)$
  \item $\forall n \in \mathbb N, n \times 0 = 0$
  \item $\forall n,m \in \mathbb N, n \times \sucs(m) = n + (n\times m)$
  \item $\forall n \in \mathbb N, \exp(n,0) = 1$
  \item $\forall n,m\in \mathbb N, \exp(n,\sucs(m)) = n \times \exp(n,m)$
  \end{itemize}
  On définit les fonctions suivantes~:
  \begin{itemize}
  \item $+ \defeq \rec(\pi_1^1,\sucs\circ \pi_3^3)$
  \item $\times \defeq \rec(k_0^1,+ \circ (\pi_2^3,\pi_3^3))$
  \item $\exp \defeq \rec(k_1^1,\times\circ (\pi_2^3,\pi_3^3))$
  \end{itemize}
  Il est direct de vérifier que ces fonctions vérifient les équations attendues.
\end{proof}

A partir de ces fonctions élémentaires, on peut construire les fonctions de
somme et de produit.

\begin{proposition}
  Les fonctions suivantes sont RP~:
  \[
  \begin{array}{ccccc}
    \Sigma^p &: & \mathbb N^p &\longrightarrow & \mathbb N \\
    & & (x_1,\ldots,x_p) &\longmapsto & \sum_{i = 1}^p x_i
  \end{array}\qquad
  \begin{array}{ccccc}
    \Pi^p &: & \mathbb N^p &\longrightarrow & \mathbb N\\
    & & (x_1,\ldots,x_p) &\longmapsto & \prod_{i = 1}^p x_i
  \end{array}
  \]
\end{proposition}

\begin{proof}
  On prouve par récurrence sur $p$ que chaque $\Sigma^p,\Pi^p$ est une fonction
  RP~:
  \begin{itemize}
  \item dans le cas où $p = 0$, on a simplement deux constantes, respectivement
    $0$ et $1$, qui sont RP.
  \item supposons que $\Sigma^p$ est RP, alors on peut définir $\Sigma^{p+1}$
    par
    \[\Sigma^{p+1} \defeq +\circ
    (\Sigma^p \circ (\pi_1^{p+1},\ldots,\pi_p^{p+1}),
    \pi_{p+1}^{p+1})\]
    et $\Pi^{p+1}$ par
    \[\Pi^{p+1}\defeq \times\circ
    (\Pi^p \circ (\pi_1^{p+1},\ldots,\pi_p^{p+1}),
    \pi_{p+1}^{p+1})\]
  \end{itemize}
  Les deux fonctions sont donc RP.
\end{proof}

\begin{exercise}
  Soit une fonction RP $f : \mathbb N^n \to \mathbb N$. Montrer que la
  fonction
  \[\begin{array}{ccccc}
  \Sigma^f &: &\mathbb N^{n+1} &\longrightarrow & \mathbb N\\
  & & (x_1,\ldots,x_n,p) &\longmapsto & \sum_{i = 1}^p f(x_1,\ldots,x_n,i)
  \end{array}\]
  est une fonction RP. Montrer la même chose pour la fonction $\Pi^f$ dans
  laquelle la somme est remplacée par le produit.
\end{exercise}

D'autres fonctions arithmétiques parmi les plus élémentaires sont facilement
prouvables comme étant RP.

\begin{proposition}
  Les fonctions $\min,\max : \mathbb N^2 \to \mathbb N$, la fonction
  $d : \mathbb N^2 \to \mathbb N$ définie par $d(n,m) = | n - m |$,
  ainsi que la fonction
  \[\begin{array}{ccccc}
  - &: & \mathbb N^2 &\longrightarrow & \mathbb N\\
  & & (n,0) & \longmapsto & n\\
  & & (0,\sucs(m)) &\longmapsto & 0\\
  & & (\sucs(n),\sucs(m)) &\longmapsto & n - m
  \end{array}\]
  sont RP.
\end{proposition}

\begin{proof}
  On commence par prouver que $-$ est RP. Pour cela, on définit d'abord la
  fonction prédécesseur, qui associe $0$ à $0$ et $\sucs(n)$ à $n$~:
  \[p \defeq \rec(\mathrm k_0^0,\pi_1^2)\]
  A partir de cette fonction $p$, on définit la fonction $-$ comme son
  itération~:
  \[- \defeq \rec(\pi_1^1,p \circ \pi_3^3)\]
  On voit qu'alors, $n - 0 = n$ pour tout $n \in \mathbb N$. On montre ensuite
  par récurrence sur $m$ que $0 - m = 0$~:
  \begin{itemize}
  \item $0 - 0 = 0$
  \item si $0 - m = 0$, alors $0 - \sucs(m) = p(0-m) = p(0) = 0$
  \end{itemize}
  Il nous reste alors à montrer que $\sucs(n) - \sucs(m) = n - m$. On raisonne
  par induction sur $m$~:
  \begin{itemize}
  \item $\sucs(n) - \sucs(0) = p(\sucs(n) - 0) = p(\sucs(n)) = n = n - 0$
  \item supposons que $\sucs(n) - \sucs(m) = n - m$, alors
    \[
    \sucs(n) - \sucs(\sucs(m)) = p(\sucs(n) - \sucs(m)) = p(n - m)
    = n - \sucs(m)
    \]
  \end{itemize}
  Donc $-$ est RP.

  A partir de cette définition de $-$, on peut prouver que $n-m = \max(0,n-m)$
  (nous considérons la preuve suffisamment directe pour la laisser en
  exercice). On peut alors définir $d(n,m) = (n - m) + (m - n)$.

  On peut vérifier que pour tous $n,m$, on a les deux équations suivantes~:
  \[\min(n,m) = \frac{1}{2}(n + m - |n - m|)\qquad
  \max(n,m) = \frac{1}{2}(n + m + |n - m|)\]
  donc les fonctions $\min$ et $\max$ s'écrivent comme
  \begin{align*}
    \min &\defeq \mathrm{demi}\circ - \circ (+,d)\\
    \max &\defeq \mathrm{demi}\circ + \circ (+,d)
  \end{align*}
  On en déduit que les fonctions souhaitées sont RP.
\end{proof}

\begin{exercise}
  Montrer que les fonctions $\min$ et $\max$ sur un $n$-uplet sont encore des
  fonctions RP.
\end{exercise}

\begin{exercise}
  Vérifier que la fonction
  \[\begin{array}{ccccc}
  \mathrm{demi} & : & \mathbb N &\longrightarrow & \mathbb N\\
  & & n &\longmapsto & \displaystyle\left\lfloor \frac{n}{2}\right\rfloor
  \end{array}\]
  est bien une fonction RP.
\end{exercise}

\begin{exercise}
  Montrer que la fonction $q : \mathbb N^2 \to \mathbb N$ et la fonction
  $r : \mathbb N^2 \to \mathbb N$, renvoyant respectivement le quotient et le
  reste de la division euclidienne d'un entier par un autre, sont des fonctions
  RP.
\end{exercise}

Attardons-nous maintenant sur la notion de prédicat RP. On a vu qu'une fonction
RP était une certaine fonction $\mathbb N^n \to \mathbb N$, mais il est aussi
intéressant de considérer à la place de fonctions des relations
$R\subseteq \mathbb N^n$. On adapte donc notre notion de RP à ces relations.

\begin{definition}[Relation RP]
  On dit qu'une relation $R \subseteq \mathbb N^n$ est RP si sa fonction
  caractéristique $\chi_R : \mathbb N^n \to \btwo$ est une fonction RP.
\end{definition}

\begin{remark}
  On parlera aussi d'ensemble RP ou de partie RP, en général lorsque l'on parle
  de partie de $\mathbb N$.
\end{remark}

Les relations RP se comportent bien, au sens de la proposition suivante.

\begin{proposition}
  Pour tout $n \in \mathbb N$, l'ensemble des relations RP sur $\mathbb N^n$
  est une sous-algèbre de Boole de $\powerset(\mathbb N^n)$.
\end{proposition}

\begin{proof}
  On cherche donc à prouver que les relations RP sont stables par intersection,
  union et complément, et que $\mathbb N^n$ et $\varnothing$ sont des relations
  RP. Les deux derniers points sont directs puisque ces deux relations
  correspondent respectivement à $\mathrm k_1^n$ et $\mathrm k_0^n$.

  Soient $R,S$ deux relations RP, alors~:
  \[\forall n_1,\ldots,n_p \in \mathbb N,
  \begin{cases}
    \chi_{R\cap S}(n_1,\ldots,n_p) =
    \min(\chi_R(n_1,\ldots,n_p),\chi_S(n_1,\ldots,n_p))\\
    \chi_{R\cup S}(n_1,\ldots,n_p) =
    \max(\chi_R(n_1,\ldots,n_p),\chi_S(n_1,\ldots,n_p))
  \end{cases}
  \]
  donc $R\cap S$ et $R\cup S$ sont RP, en composant par $\min$ et $\max$,
  respectivement.

  De plus,
  \[\forall n_1,\ldots,n_p \in \mathbb N,
  \chi_{\mathbb N^p\setminus R}(n_1,\ldots,n_p) = 1 - \chi_R(n_1,\ldots,n_p)\]
  donc $\mathbb N^p \setminus R$ est aussi RP, en composant par
  $x \mapsto 1 - x$.

  Ainsi les relations RP forment une algèbre de Boole pour les opérations
  ensemblistes usuelles.
\end{proof}

\begin{proposition}
  Les relations $\triangle \defeq \{(x,x) \mid x \in \mathbb N^n\}$ et
  $\leq \defeq \{(n,m) \mid n \leq m\}$ sont RP.
\end{proposition}

\begin{proposition}
  On prouve d'abord que $\leq$ est RP. Pour cela, on remarque que $n \leq m$
  exactement lorsque $n - m \leq 0$, c'est-à-dire lorsque $n - m = 0$.
  En prenant alors $(n,m) \mapsto 1 - (n - m)$, on a bien une fonction RP à
  valeurs dans $\btwo$ et qui coïncide avec $\chi_\leq$. Pour montrer que
  la relation $\triangle$ est aussi RP, il suffit de remarquer que
  \[(n,m) \in \triangle \iff (n \leq m) \land (m \leq n)\]
\end{proposition}

\begin{exercise}
  Montrer que $\triangle^n \defeq
  \{(n,\ldots,n) \in \mathbb N^n\mid n \in \mathbb N\}$ est
  une relation RP.
\end{exercise}

Grâce à ces propriétés de base, on sait déjà que des formules sans
quantification utilisant l'égalité et l'inégalité comme symbole de relation
sont, en s'interprétant dans la structure $\mathbb N$, des relations RP.

On verra qu'il n'est pas possible de rajouter n'importe quelle quantification,
mais le résultat suivant est déjà un résultat particulièrement fort~: une
quantification bornée sur une relation RP reste une relation RP.

\begin{proposition}
  Soit une relation RP $R\subseteq \mathbb N^{n+1}$. On définit les fonctions
  \[\begin{array}{ccccc}
  \exists R&:&\mathbb N^{n+1} & \longrightarrow & \btwo\\
  & & (x_1,\ldots,x_n,y) &\longmapsto &
  \begin{cases}
    1 \text{ si } \exists z < y, R(x_1,\ldots,x_n,z)\\
    0 \text{ sinon }
  \end{cases}
  \end{array}\]
  \[\begin{array}{ccccc}
  \forall R & : & \mathbb N^{n+1} & \longrightarrow & \btwo\\
  & & (x_1,\ldots,x_n,y) &\longmapsto &
  \begin{cases}
    1 \text{ si } \forall z < y, R(x_1,\ldots,x_n,z)\\
    0 \text{ sinon }
  \end{cases}
  \end{array}\]
  Ces deux fonctions sont RP.
\end{proposition}

\begin{proof}
  On construit les deux fonctions par récursion primitive~:
  \[\exists R \defeq \rec(\mathrm k_0^n,\max\circ
  (\pi_{n+2}^{n+2},R(\pi_1^{n+2},\ldots,\pi_{n+1}^{n+1}))\]
  \[\forall R \defeq \rec(\mathrm k_1^n,\min\circ
  (\pi_{n+2}^{n+2},R(\pi_1^{n+2},\ldots,\pi_{n+1}^{n+1}))\]
  On montre seulement le fait que la fonction $\exists R$ ainsi définie
  vérifie la propriété voulue, le cas de $\forall$ étant parfaitement analogue.

  On remarque d'abord qu'il n'existe aucun $z < 0$, donc dans le cas de
  $(x_1,\ldots,x_n,0)$ la valeur retournée est $0$. Dans le cas où l'on appelle
  la fonction sur $(x_1,\ldots,x_n,\sucs(y))$, alors on voit qu'il existe
  $z < \sucs (y)$ tel que $R(x_1,\ldots,x_n,z)$ si et seulement s'il existe
  $z < y$ tel que $R(x_1,\ldots,x_n,z)$ ou si $R(x_1,\ldots,x_n,y)$, ce qui
  correspond à l'équation vérifiée dans le cas récursif.
\end{proof}

\begin{exercise}
  Montrer que l'ensemble des nombres premiers est un ensemble RP.
\end{exercise}

On peut aussi construire des instructions conditionnelles sur des relations RP.

\begin{proposition}
  Soit $R \subseteq \mathbb N^n$ une fonction RP et deux fonctions
  $f,g : \mathbb N^n \to \mathbb N$ deux fonctions RP. Alors la fonction
  \[\begin{array}{ccccc}
  \ifrm(R,f,g)& : & \mathbb N^n &\longrightarrow & \mathbb N\\
  & & (x_1,\ldots,x_n) &\longmapsto &
  \begin{cases}
    f(x_1,\ldots,x_n) \text{ si } (x_1,\ldots,x_n) \in R\\
    g(x_1,\ldots,x_n) \text{ sinon }
  \end{cases}
  \end{array}\]
\end{proposition}

\begin{proof}
  On définit $\ifrm$ de la façon suivante~:
  \[\ifrm(R,f,g) \defeq \rec(g,f\circ(\pi_1^{n+2},\ldots,\pi_n^{n+2}))\circ
  (\pi_1^n,\ldots,\pi_n^n,R)\]
  Lorsque $R(x_1,\ldots,x_n) = 0$, la fonction vaut
  $(g\circ (\pi_1^n,\ldots,\pi_n^n)) (x_1,\ldots,x_n)$, c'est-à-dire
  $g(x_1,\ldots,x_n)$. Lorsque $R(x_1,\ldots,x_n) = 1$, la fonction
  vaut $(f\circ (\pi_1^n,\ldots,\pi_n^n))(x_1,\ldots,x_n)$, c'est-à-dire
  $f(x_1,\ldots,x_n)$. D'où le résultat.
\end{proof}

\begin{remark}
  La fonction $\ifrm(R,f,g)$ peut s'appliquer avec une fonction $h$ à la place
  de $R$. La condition sera alors que la valeur retournée par $h$ est non nulle.
\end{remark}

On voit qu'il est possible de passer facilement d'une fonction à un ensemble,
en considérant la fonction caractéristique. Réciproquement, on souhaite avoir
un moyen étant donné un prédicat d'en extraire une fonction~: c'est la notion de
minimisation qui nous sera utile.

\begin{definition}[Minimisation, minimisation bornée]
  Soit $R \subseteq \mathbb N^{n+1}$ une relation. On définit la fonction
  partielle de minimisation comme suit~:
  \[\begin{array}{ccccc}
  \mu R & : & \mathbb N^n & \longrightarrow & \mathbb N\\
  & & (x_1,\ldots,x_n) & \longmapsto & \min
  \{ y \in \mathbb N\mid (x_1,\ldots,x_n,y)\in R\}
  \end{array}\]
  où la fonction n'est pas définie lorsque l'ensemble sur lequel est pris le
  minimum est vide.

  On appelle minimisation bornée la fonction totale suivante~:
  \[\begin{array}{ccccc}
  \mu_B R & : & \mathbb N^{n+1} & \longrightarrow &\mathbb N\\
  & & (x_1,\ldots,x_n,y) & \longmapsto & \min
  \{ z < y \mid (x_1,\ldots,x_n,z) \in R\}
  \end{array}\]
  où la fonction vaut $0$ lorsque l'ensemble sur lequel est pris le minimum est
  vide.
\end{definition}

La minimisation en général n'est pas RP, mais la minimisation bornée l'est. De
plus, le fait que la fonction n'est pas définie dans le cas de $\mu$ est dû
au fait que si l'on songe à nos fonctions comme à des algorithmes, et à $R$
comme à un algorithme de décision, il est impossible de savoir (en temps fini)
s'il existe ou non un certain $y$ tel que $R(x_1,\ldots,x_n,y)$. Au contraire,
dans le cas de la minimisation bornée, il est possible de le vérifier en temps
fini, et on peut donc donner une valeur dans le cas où l'ensemble est vide.

\begin{proposition}
  Soit $R\subseteq \mathbb N^{n+1}$ une relation RP. La fonction $\mu_B R$ est
  une fonction RP.
\end{proposition}

\begin{proof}
  On définit $\mu_B R$ par récursion primitive. Dans le cas où $y = 0$, alors
  l'ensemble $\{ z < 0 \mid (x_1,\ldots,x_n,z) \in R\}$ est vide donc la
  fonction retourne $0$. Dans le cas inductif, lorsque $y = \sucs(y')$, le
  minimum de l'ensemble est soit le minimum de
  $\{ z < y' \mid (x_1,\ldots,x_n,z)\in R\}$ si cet ensemble est non vide, soit
  $y'$ si $(x_1,\ldots,x_n,y') \in R$ si cet ensemble est non vide, soit $0$. On
  pourrait se contenter de cet algorithme, qui nous fait recalculer en
  permanence l'habitation des ensembles. A la place, on va construire une
  fonction qui retourne $\sucs(m)$ où $m$ est le minimum s'il existe, et $0$
  sinon~:
  \[f_0 \defeq \rec(\km_0^n,\ifrm(\pi_{n+2}^{n+2},\pi_{n+2}^{n+2},
  \times\circ(R\circ(\pi_1^{n+2},\ldots,\pi_{n+1}^{n+1},
  \sucs\circ \pi_{n+1}^{n+1}))))\]
  On définit alors $\mu_B R$ simplement par $p\circ f_0$, où $p$ est la
  fonction précédesseur définie plus haut.
\end{proof}

Avec les outils présents, on peut montrer que la bijection de Cantor introduite
dans la \cref{def.bij.Cantor} est une fonction RP.

\begin{proposition}
  La bijection de Cantor $\alpha : \mathbb N \times \mathbb N \to \mathbb N$
  est une fonction RP. Les deux fonctions
  $\pi_1',\pi_2' : \mathbb N \to \mathbb N$ telles que
  $(\pi_1'(\alpha(n,m)),\pi_2'(\alpha(n,m))) = (n,m)$ sont elles aussi RP.
\end{proposition}

\begin{proof}
  On sait que la fonction $f_1 : k \mapsto \sum_{i = 1}^k i$ est RP. On peut
  alors définir $\alpha$~:
  \[\alpha \defeq + \circ (f_1\circ +,\pi_2^2)\]
  Ainsi $\alpha$ est bien une fonction RP.

  Comme la situation est symétrique entre $\pi_1'$ et $\pi_2'$, on montre
  seulement que $\pi_1'$ est RP. Soit $n,m\in \mathbb N$ et $a = \alpha(n,m)$.
  Le prédicat $k = \pi_1'(a)$ revient à $\exists p \leq a,\alpha(k,p) = a$ car
  $p \leq \alpha(p,i)$ pour tout $i\in\mathbb N$. De plus, pour tout
  $a\in \mathbb N$, il existe effectivemnet $n,m$ tels que $a = \alpha(n,m)$,
  donc $\{k \leq a \mid \exists p\leq a, \alpha(k,p) = a\}$
  contient exactement un élément, que l'on peut récupérer par minimisation.
  On en déduit donc
  \[\pi_1' \defeq \mu_B (\exists(\triangle\circ
  (\alpha\circ (\pi_2^3,\pi_3^3), \pi_1^3)))\circ (\pi_1^1,\pi_1^1)\]
  Donc $\pi_1'$ est une fonction RP.
\end{proof}

\begin{exercise}
  Montrer que chaque $\alpha_n$ est aussi une fonction RP, et que chaque
  projection $\varpi_i^n$ telle que $\varpi_i^n(\alpha_n(x_1,\ldots,x_n)) = n$,
  est une fonction RP.
\end{exercise}

Avec ces données, on se rend compte que le fait d'avoir des fonctions
$\mathbb N^n \to \mathbb N$ n'apporte pas réellement plus d'expressivité que
d'avoir simplement des fonctions $\mathbb N \to \mathbb N$, puisque toutes les
fonctions RP $f : \mathbb N^k \to \mathbb N$ peuvent se réécrire en une
composée de fonctions
$n \mapsto (\varpi_1^n(n),\ldots,\varpi_n^n(n)) \overset{f}{\mapsto}$ elle aussi
RP. Bien sûr, il est toujours nécessaire de définir les fonctions avec plusieurs
coordonnées de par la définition même des fonctions RP (puisqu'en particulier
la récursion primitive utilise un appel sur une seule coordonnée en laissant les
autres statiques).
