\chapter{Formaliser les mathématiques}

Pour commencer notre étude, nous allons étudier l'une des propriétés les plus
connues de la théorie des ensembles : la capacité d'expression suffisante pour
formaliser toutes les mathématiques usuelles. Ce chapitre se concentrera donc sur
l'étude des axiomes de ZFC et de leur utilisation pour construire les objets
mathémtiques que nous connaissons habituellement.

Nous devons ici faire quelques éclaircissements d'ordre philosophique. Tout
d'abord, sur la méta-théorie : nous considérons que l'univers ambiant dans lequel
nous pratiquons les mathématiques est lui-même un modèle de la théorie ZFC. Plus
précisément, en notant $\mathcal U$ l'objet mathématique qui sera la collection
de tous les objets mathématiques, ZFC peut être vu comme une théorie donnant une
approximation du comportement de $\mathcal U$.

En particulier, $\mathcal U$ contient des collections, que nous appelons
d'habitude ensembles, et si un objet $X$ apparait dans une collection $C$, on
écrit $X\in C$. Cependant, les objets que nous manipulerons ne seront pas ces
collections, et la relation $\in$ utilisée dans ZFC ne sera pas \og être un objet
mathématique apparaissant dans cette collection intuitive\fg{} : nous utiliserons
des outils purement formels pour étudier ZFC, et devons donc les distinguer des
objets intuitifs associés. Aussi nous appellerons \og ensemble\fg{} un objet de
la théorie ZFC, et \og collection\fg{} un ensemble au sens intuitif, dans
$\mathcal U$. De même, les objets formels seront associés à la notion
d'appartenance, et les collections à la notion d'occurrence.

Remarquons cependant qu'un ensemble donne de fait naissance à une
collection : l'ensemble $X$ définit la collection
$\{ x \in\mathcal U\mid x \in X \}$.

Enfin, nous traiterons régulièrement des classes. S'il est possible de ne se
restreindre qu'aux ensembles pour l'étude de ZFC, nous verrons qu'il est bien
plus pratique d'énoncer certaines constructions en terme de classes. Une classe
est une collection qui ne peut pas être représentée par un ensemble (par exemple
la classe des ensembles, \textit{cf}. le paradoxe de Russell), mais que l'on peut
décrire par un prédicat. Ainsi, une classe peut se représenter par le prédicat
lui correspondant. Si une classe $C$ est décrite par un prédicat $\varphi(x)$,
alors $x\in C$ doit se lire comme une reformulation plus lisible de $\varphi(x)$.
Un léger défaut vient avec cette approche : certains théorèmes, se référant aux
classes, doivent se lire comme des schémas de théorèmes, c'est-à-dire des
théorèmes paramétrés par le prédicat $\varphi$ définissant une classe.

\section{Axiomes de ZFC}

\subsection[Premiers axiomes]{Extensionalité, paire, union, ensemble des parties}

Commençons par étudier la théorie ZFC en en listant les axiomes. Un premier point
important est de définir le langage sur lequel nous travaillerons : nous
utiliserons $\mathcal L_{\mathrm{ZF}} = \{ \in^2\}$. En effet, tous les énoncés
seront écrits seulement à l'aide du symbole de relation binaire $\in$ et de la
relation $=$, symbole logique déjà inclus dans tout langage.

\begin{remark}
  En réalité, nous le verrons, notre langage sera bien plus riche : nous
  ajouterons des symboles de fonctions, de constantes, et diverses relations.
  Cependant, tous ces ajouts doivent être considérés comme des aides à la lecture
  : toute proposition formulée dans l'enrichissement que nous donnerons au fur
  et à mesure doit pouvoir se formuler dans $\mathrm L_{\mathrm{ZF}}$, mais avec un
  nombre possiblement bien plus grand de symboles.
\end{remark}

\begin{axiom}[Extensionalité]
  L'axiome d'extensionalité exprime que deux ensembles sont égaux exactement
  lorsqu'ils possèdent les mêmes éléments :
  \[\forall x \forall y, \qquad x = y \iff (\forall z, z \in x\iff z \in y)\]
\end{axiom}

Cet axiome définit ce que signifie l'égalité dans le monde des ensembles. Cela
permet directement de voir que l'ordre ou le nombre d'occurrences n'importent pas
dans un ensemble, contrairement par exemple au cas des listes. C'est un axiome au
statut particulier car il ne permet pas de construire de nouvel ensemble. Les
autres axiomes, eux, donneront principalement de nouvelles méthodes pour, à
partir d'ensembles déjà construits, définir de nouveaux ensembles.

\begin{axiom}[Paire]\label{ax.ZF.pair}
  L'axiome de la pair exprime que si deux ensembles $x$ et $y$ ont été
  construits, alors l'ensemble $\{x,y\}$ peut être construit à partir d'eux :
  \[\forall x \forall y \exists z, \qquad \forall a, a \in z \iff
  a = x \lor a = y\]
\end{axiom}

\begin{exercise}
  Montrer que pour tous ensembles $x,y$, il existe un unique ensemble $z$ tel que
  décrit dans l'\cref{ax.ZF.pair}. Cela justifie donc la notation $\{x,y\}$
  pour cet ensemble, puisqu'il n'y a pas d'ambigüité sur lequel il est.
\end{exercise}

En utilisant l'axiome de la paire, on peut ainsi construire des collections
finies telles que $\{x,\{y,z\}\}$, mais on ne peut pas par exemple définir
$\{x,y,z\}$ : il nous faut un axiome permettant d'accéder à l'intérieur d'un
ensemble pour en construire un nouveau.

\begin{axiom}[Union]
  Pour tout ensemble $x$, on peut construire l'ensemble $\bigcup x$ contenant
  les éléments des éléments de $x$ :
  \[\forall x \exists y,\qquad \forall z, z\in y \iff
  (\exists a, z \in a \land a \in x)\]
\end{axiom}

\begin{notation}
  De la même façon qu'avec l'axiome de la paire, l'ensemble $y$ défini par
  l'axiome est unique, on l'écrira donc $\bigcup x$.
\end{notation}

\begin{exercise}
  Soient $x_1,\ldots,x_n$ des ensembles, montrer par récurrence qu'il existe
  l'ensemble $\{x_1,\ldots,x_n\}$.
\end{exercise}

Enfin, l'axiome de l'ensemble des parties permet de considérer comme un ensemble
la collection des parties d'un ensemble. En un sens, il permet de faire croître
considérablement la taille de ce que l'on peut construire.

\begin{notation}
  On définit le prédicat binaire $\subseteq$, appelé l'inclusion, par
  \[x \subseteq y \defeq \forall z, z \in x \implies z \in y\]
\end{notation}

\begin{axiom}[Ensemble des parties]
  Pour tout ensemble $x$, il existe l'ensemble $\powerset (x)$ dont les éléments
  sont exactement les ensembles inclus dans $x$ :
  \[\forall x\exists y, \qquad \forall z, z \in y \iff z \subseteq x\]
\end{axiom}

\begin{notation}
  Comme le $y$ défini plus haut est unique, on peut là encore lui donner une
  notation, qui est bien sûr $\powerset(x)$.
\end{notation}

\subsection{Les schémas d'axiomes}

Les axiomes précédemment donnés constituent les briques de base pour construire
des ensembles, mais sont en général trop grossières. L'intérêt de la théorie des
ensembles est de pouvoir construire des ensembles correspondant à des
collections, ce qui manque pour l'instant à notre système. C'est pour cela que
nous ajoutons l'axiome de compréhension : étant donnée une formule $\varphi$ et
un ensemble $X$, on peut construire l'ensemble $\{x \in X\mid \varphi(x)\}$ des
éléments de $x$ vérifiant $\varphi$.

La restriction de la compréhension à des parties d'un ensemble s'explique par le
paradoxe de Russell : si l'on pouvait construire un ensemble pour chaque
formule, on pourrait construire $\{ x \mid x\notin x\}$, qui appartient à
lui-même si et seulement s'il n'appartient pas à lui-même.

Un autre problème doit être contourné, et il est la raison pour laquelle on parle
de schéma d'axiomes et non d'axiome. Si l'on voulait définir le schéma avec ce
qui a été dit, celui-ci commencerait moralement par $\forall \varphi$ : cela
n'est pas une quantification du premier ordre, et il n'est donc pas possible de
donner un axiome pour toutes les formules. A la place, on introduit un schéma
d'axiomes, c'est-à-dire une infinité d'axiomes ayant tous la même forme et
dépendant d'un paramètre que l'on quantifie sur les formules.

\begin{axiom}[Schéma de compréhension]
  Soit $\varphi(x_0,\ldots,x_n)$ une phrase mathématique à $n$ variables libres.
  Alors pour tous ensembles $X$ et $a_1,\ldots,a_n$, il existe l'ensemble
  $\{x\in X \mid \varphi(x,a_1,\ldots,a_n)\}$, ce qui s'écrit formellement
  \[\forall X\forall a_1\;\cdots\;a_n\exists y,\qquad
    \forall x, x \in y \iff (x\in X\land \varphi(x,a_1,\ldots,a_n))\]
\end{axiom}

\begin{notation}
  On définit donc la notation $\{x\in X\mid \varphi(x,a_1,\ldots,a_n)\}$ pour
  l'ensemble précédent.
\end{notation}

\begin{exercise}
  Soit un ensemble $x$ non vide, montrer qu'il existe l'ensemble $\bigcap x$ des
  éléments qui sont dans tous les éléments de $x$.
\end{exercise}

\begin{exercise}\label{exo.ZF.prod}
  Soient $x$ et $y$ deux ensembles. On définit le couple $(x,y)$ par
  \[(x,y) \defeq \{\{x,y\},\{x\}\}\]
  Montrer que pour tous $x,y,x',y'$, $(x,y) = (x',y')$ si et seulement si
  $x=x'$ et $y=y'$.

  Construire un prédicat $\varphi(x,y,z)$ tel que $\varphi(x,y,z)$ est vrai
  si et seulement si $z = (x,y)$. En déduire en considérant une partie bien
  choisie de $\powerset (\powerset(x\cup y))$ que
  \[x\times y \defeq \{(a,b)\mid a \in x, b \in y\}\]
  est un ensemble bien défini.
\end{exercise}

Le second axiome, le schéma d'axiomes de remplacement, peut être vu comme une
version plus forte de la compréhension. Plutôt que de s'intéresser à filtrer des
éléments dans un ensemble plus gros, le but de ce schéma d'axiomes est de
construire un ensemble par une fonction. Comme la notion de fonction n'est pas
encore définie, nous utilisons à la place la notion de relation fonctionnelle.

\begin{definition}[Relation fonctionnelle]
  Une relation binaire est ici une formule à deux variables libres. On dit
  qu'une relation $R(x,y)$ est fonctionnelle si pour chaque $x$, il existe au
  plus un $y$ tel que $R(x,y)$. On écrira pour raccourcir
  \[\Funct(R) \defeq \forall x\forall y\forall z, R(x,y)\land R(x,z)\implies
  y = z\]

  Pour une relation fonctionnelle $R(x,y)$, on définit la collection du domaine
  de $R$ :
  \[\dom(R)(x)\defeq \exists y. R(x,y)\]
  et la collection de l'image de $R$ :
  \[\im(R)(y)\defeq \exists x. R(x,y)\]
\end{definition}

\begin{axiom}[Schéma de remplacement]
  Pour toute formule $R(x_0,\ldots,x_{n+1})$, pour tous ensembles
  $X,a_1,\ldots,a_n$, l'ensemble image de $R(a_1,\ldots,a_n)$ par $X$ est aussi
  un ensemble :
  \[\Funct(R)\implies \forall X \forall a_1\;\cdots\;a_n\exists y,\qquad
  \forall x, x \in y \iff (\exists z, z\in X \land R(a_1,\ldots,a_n,z,x))\]
\end{axiom}

\begin{exercise}
  En remarquant qu'une relation fonctionnelle peut représenter une fonction
  partielle, montrer que le schéma d'axiomes de compréhension peut se déduire du
  schéma d'axiomes de remplacement.
\end{exercise}

\begin{exercise}
  En réutilisant le prédicat $\varphi$ de l'\cref{exo.ZF.prod}, montrer grâce au
  schéma d'axiomes de remplacement que pour tous ensembles $x,y$, l'ensemble
  $x\times y$ est bien défini même sans l'axiome de l'ensemble des parties.
\end{exercise}

Donnons aussi l'axiome le plus évident.

\begin{axiom}[Univers non vide]
  Il existe un ensemble.
\end{axiom}

\begin{exercise}
  Montrer que l'axiome précédent est équivalent à l'existe de l'ensemble vide
  $\varnothing$ défini par \[\forall z, z\notin \varnothing\]
  (où $x\notin y$ signifie $\lnot(x\in y)$) 
\end{exercise}

\subsection{L'axiome de l'infini et les entiers}

Pour introduire l'axiome suivant, il nous faut d'abord motiver l'utilisation de
ses éléments constitutifs. L'ensemble mathématique le plus élémentaire que l'on
est amené à étudier est certainement $\mathbb N$, l'ensemble des entiers
naturels. Une formalisation habituelle de cet ensemble demande en général les
trois constituants suivant:
\begin{itemize}
\item l'élément $0$
\item la fonction unaire $S$, correspondant à $n \mapsto n + 1$
\item le principe de récurrence, que l'on peut encoder dans les ensembles par le
  fait que si $F\subseteq\mathbb N$, $0 \in F$ et
  $\forall n, n\in F \implies S\;n \in F$ alors $F = \mathbb N$.
\end{itemize}

Chercher à définir $\mathbb N$ dans ZFC nous demande donc de définir ces
éléments. Un candidat naturel à $0$ est $\varnothing$ : les deux sont les objets
nuls par excellence, et $\varnothing$ est le premier ensemble que l'on peut
construire. La question de savoir ce qu'est $S\;x$ pour un ensemble $x$ est alors
naturelle : pour cela, nous utilisons le codage de Von Neumann consistant à coder
l'entier naturel $n$ par $\{0,\ldots,n-1\}$. En effet, cela nous offre une
définition naturelle à la fonction $S$ :
\[S\;x\defeq x\cup \{x\}\]

Remarquons que l'on peut déjà construire tous les entiers que l'on souhaite :
on peut construire $0$ et itérer la fonction $S$. Malheureusement, rien ne nous
dit que la collection $\{S^n\;0\mid n \in \mathbb N\}$ est bien elle-même un
ensemble (où le $\mathbb N$ apparaissant dans la définition est l'ensemble des
entiers naturels de notre méta-théorie). Pour palier ce problème, et pour éviter
d'utiliser notre méta-théorie explicitement, on va à la place définir $\mathbb N$
comme le plus petit ensemble contenant $0$ et stable par la fonction $S$. Il nous
reste à savoir qu'un ensemble contenant $0$ et stable par $S$ existe bien.

\begin{axiom}[Infini]
  Il existe un ensemble contenant $\varnothing$ et stable par $S$ :
  \[\exists x, \qquad \varnothing \in x \land
  \forall y, y \in x \implies S\;y\in x\]
\end{axiom}

\begin{remark}
  On peut en fait se passer de l'axiome de l'ensemble vide en prenant l'axiome de
  l'infini (et en adaptant sa définition pour ne pas appeler explicitement
  l'ensemble vide).
\end{remark}

\begin{definition}[Entiers naturels]
  Soit $X$ l'ensemble défini par l'axiome de l'infini. On définit alors
  $\mathbb N$ comme
  \[\mathbb N \defeq \{x\in X \mid \forall Y, Y\subseteq X\land
  \varnothing \in Y \land (\forall a, a \in Y \implies S\;a\in Y) \implies
  x\in Y\}\]
\end{definition}

On vérifie alors le principe de récurrence.

\begin{theorem}[Récurrence]
  Soit $F$ une partie de $\mathbb N$ telle que $\varnothing \in F$ et
  $\forall n, n \in F \implies S\;n \in F$, alors $F = \mathbb N$.
\end{theorem}

\begin{proof}
  Comme $F$ est une partie de $\mathbb N$, il nous suffit de montrer que
  $\mathbb N \subseteq F$. Par transitivité de l'inclusion,
  $F\subseteq X$ pour $X$ l'ensemble à partir duquel $\mathbb N$ a été défini.
  On sait de plus que
  \[x \in \mathbb N \implies \forall Y, Y\subseteq X \land
  \varnothing \in Y \land (\forall a, a \in Y \implies S\;a \in Y) \implies
  x\in Y\]
  d'où, en spécialisant $Y$ en $F$, et sachant que $\varnothing\in F$ et
  $\forall n, n \in F \implies S\;n \in F$, il vient que
  \[x\in \mathbb N \implies x \in F\] ce qui est exactement
  $\mathbb N\subseteq F$, d'où le résultat.
\end{proof}

\begin{notation}
  A partir de maintenant, pour fluidifier l'écriture, on adoptera un style plus
  laxiste sur l'écriture des propositions. Par exemple on se permettra d'écrire
  $\forall x \in X, \psi$ pour $\forall x, x\in X \implies \psi$ et tous les
  légers abus de notations du même genre.
\end{notation}

\begin{exercise}
  Montrer que l'on peut définir des fonctions par récursion sur $\mathbb N$,
  c'est-à-dire montrer que si l'on a un ensemble $X$, un élément $x_0\in X$ et
  une fonction $f_S : \mathbb N \times X \to X$ alors il existe une unique
  fonction $f : \mathbb N \to X$ telle que
  \[\begin{cases}
  f(0) = x_0\\
  \forall n \in \mathbb N, f(S\;n) = f_S(n,f(n))
  \end{cases}\]
\end{exercise}
