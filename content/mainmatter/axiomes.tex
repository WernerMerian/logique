\chapter{Formaliser les mathématiques}
\label{chp.axiomes}

Pour commencer notre étude, nous allons étudier l'une des propriétés les plus
connues de la théorie des ensembles : la capacité d'expression suffisante pour
formaliser toutes les mathématiques usuelles. Ce chapitre se concentrera donc
sur l'étude des axiomes de ZFC et de leur utilisation pour construire les objets
mathémtiques que nous connaissons habituellement.

Nous devons ici faire quelques éclaircissements d'ordre philosophique. Tout
d'abord, sur la méta-théorie : nous considérons que l'univers ambiant dans
lequel nous pratiquons les mathématiques est lui-même un modèle de la théorie
ZFC. Plus précisément, en notant $\mathcal U$ l'objet mathématique qui sera la
collection de tous les objets mathématiques, ZFC peut être vu comme une théorie
donnant une approximation du comportement de $\mathcal U$.

En particulier, $\mathcal U$ contient des collections, que nous appelons
d'habitude ensembles, et si un objet $X$ apparait dans une collection $C$, on
écrit $X\in C$. Cependant, les objets que nous manipulerons ne seront pas ces
collections, et la relation $\in$ utilisée dans ZFC ne sera pas \og être un
objet mathématique apparaissant dans cette collection intuitive\fg{} : nous
utiliserons des outils purement formels pour étudier ZFC, et devons donc les
distinguer des objets intuitifs associés. Aussi nous appellerons \og ensemble\fg
un objet de la théorie ZFC, et \og collection\fg un ensemble au sens intuitif,
dans $\mathcal U$. De même, les objets formels seront associés à la notion
d'appartenance, et les collections à la notion d'occurrence.

Remarquons cependant qu'un ensemble donne de fait naissance à une
collection : l'ensemble $X$ définit la collection
$\{ x \in\mathcal U\mid x \in X \}$.

Enfin, nous traiterons régulièrement des classes. S'il est possible de ne se
restreindre qu'aux ensembles pour l'étude de ZFC, nous verrons qu'il est bien
plus pratique d'énoncer certaines constructions en terme de classes. Une classe
est une collection qui ne peut pas être représentée par un ensemble (par exemple
la classe des ensembles, \textit{cf}. le paradoxe de Russell), mais que l'on
peut décrire par un prédicat. Ainsi, une classe peut se représenter par le
prédicat lui correspondant. Si une classe $C$ est décrite par un prédicat
$\varphi(x)$, alors $x\in C$ doit se lire comme une reformulation plus lisible
de $\varphi(x)$. Un léger défaut vient avec cette approche~: certains théorèmes,
se référant aux classes, doivent se lire comme des schémas de théorèmes,
c'est-à-dire des théorèmes paramétrés par le prédicat $\varphi$ définissant une
classe.

\section{Axiomes de ZFC}

\subsection[Premiers axiomes]{Extensionalité, paire, union, ensemble de parties}

Commençons par étudier la théorie ZFC en en listant les axiomes. Un premier
point important est de définir le langage sur lequel nous travaillerons : nous
utiliserons $\mathcal L_{\mathrm{ZF}} = \{ \in^2\}$. En effet, tous les énoncés
seront écrits seulement à l'aide du symbole de relation binaire $\in$ et de la
relation $=$, symbole logique déjà inclus dans tout langage.

\begin{remark}
  En réalité, nous le verrons, notre langage sera bien plus riche : nous
  ajouterons des symboles de fonctions, de constantes, et diverses relations.
  Cependant, tous ces ajouts doivent être considérés comme des aides à la
  lecture~: toute proposition formulée dans l'enrichissement que nous donnerons
  au fur et à mesure doit pouvoir se formuler dans $\mathrm L_{\mathrm{ZF}}$, mais
  avec un nombre possiblement bien plus grand de symboles.
\end{remark}

\begin{axiom}[Extensionalité]\label{ax.ZF.ext}
  L'axiome d'extensionalité exprime que deux ensembles sont égaux exactement
  lorsqu'ils possèdent les mêmes éléments :
  \[\forall x \forall y, \qquad x = y \iff (\forall z, z \in x\iff z \in y)\]
\end{axiom}

Cet axiome définit ce que signifie l'égalité dans le monde des ensembles. Cela
permet directement de voir que l'ordre ou le nombre d'occurrences n'importent
pas dans un ensemble, contrairement par exemple au cas des listes. C'est un
axiome au statut particulier car il ne permet pas de construire de nouvel
ensemble. Les autres axiomes, eux, donneront principalement de nouvelles
méthodes pour, à partir d'ensembles déjà construits, définir de nouveaux
ensembles.

\begin{axiom}[Paire]\label{ax.ZF.pair}
  L'axiome de la pair exprime que si deux ensembles $x$ et $y$ ont été
  construits, alors l'ensemble $\{x,y\}$ peut être construit à partir d'eux :
  \[\forall x \forall y \exists z, \qquad \forall a, a \in z \iff
  a = x \lor a = y\]
\end{axiom}

\begin{exercise}
  Montrer que pour tous ensembles $x,y$, il existe un unique ensemble $z$ tel
  que décrit dans l'\cref{ax.ZF.pair}. Cela justifie donc la notation $\{x,y\}$
  pour cet ensemble, puisqu'il n'y a pas d'ambigüité sur lequel il est.
\end{exercise}

En utilisant l'axiome de la paire, on peut ainsi construire des collections
finies telles que $\{x,\{y,z\}\}$, mais on ne peut pas par exemple définir
$\{x,y,z\}$ : il nous faut un axiome permettant d'accéder à l'intérieur d'un
ensemble pour en construire un nouveau.

\begin{axiom}[Union]\label{ax.ZF.union}
  Pour tout ensemble $x$, on peut construire l'ensemble $\bigcup x$ contenant
  les éléments des éléments de $x$ :
  \[\forall x \exists y,\qquad \forall z, z\in y \iff
  (\exists a, z \in a \land a \in x)\]
\end{axiom}

\begin{notation}
  De la même façon qu'avec l'axiome de la paire, l'ensemble $y$ défini par
  l'axiome est unique, on l'écrira donc $\bigcup x$.
\end{notation}

\begin{exercise}
  Soient $x_1,\ldots,x_n$ des ensembles, montrer par récurrence qu'il existe
  l'ensemble $\{x_1,\ldots,x_n\}$.
\end{exercise}

\begin{notation}
  Etant donnée une famille $x_1,\ldots,x_n$ d'ensembles, on définit
  $\displaystyle \bigcup_{i = 1}^n x_i$ comme $\bigcup\{x_1,\ldots,x_n\}$. En
  particulier, $x \cup y$ est $\bigcup\{x,y\}$.
\end{notation}

Enfin, l'axiome de l'ensemble des parties permet de considérer comme un ensemble
la collection des parties d'un ensemble. En un sens, il permet de faire croître
considérablement la taille de ce que l'on peut construire.

\begin{notation}
  On définit le prédicat binaire $\subseteq$, appelé l'inclusion, par
  \[x \subseteq y \defeq \forall z, z \in x \implies z \in y\]
\end{notation}

\begin{axiom}[Ensemble des parties]\label{ax.ZF.pow}
  Pour tout ensemble $x$, il existe l'ensemble $\powerset (x)$ dont les éléments
  sont exactement les ensembles inclus dans $x$ :
  \[\forall x\exists y, \qquad \forall z, z \in y \iff z \subseteq x\]
\end{axiom}

\begin{notation}
  Comme le $y$ défini plus haut est unique, on peut là encore lui donner une
  notation, qui est bien sûr $\powerset(x)$.
\end{notation}

\subsection{Les schémas d'axiomes}

Les axiomes précédemment donnés constituent les briques de base pour construire
des ensembles, mais sont en général trop grossières. L'intérêt de la théorie des
ensembles est de pouvoir construire des ensembles correspondant à des
collections, ce qui manque pour l'instant à notre système. C'est pour cela que
nous ajoutons l'axiome de compréhension : étant donnée une formule $\varphi$ et
un ensemble $X$, on peut construire l'ensemble $\{x \in X\mid \varphi(x)\}$ des
éléments de $x$ vérifiant $\varphi$.

La restriction de la compréhension à des parties d'un ensemble s'explique par le
paradoxe de Russell : si l'on pouvait construire un ensemble pour chaque
formule, on pourrait construire $\{ x \mid x\notin x\}$, qui appartient à
lui-même si et seulement s'il n'appartient pas à lui-même.

Un autre problème doit être contourné, et il est la raison pour laquelle on
parle de schéma d'axiomes et non d'axiome. Si l'on voulait définir le schéma
avec ce qui a été dit, celui-ci commencerait moralement par $\forall \varphi$~:
cela n'est pas une quantification du premier ordre, et il n'est donc pas
possible de donner un axiome pour toutes les formules. A la place, on introduit
un schéma d'axiomes, c'est-à-dire une infinité d'axiomes ayant tous la même
forme et dépendant d'un paramètre que l'on quantifie sur les formules.

\begin{axiom}[Schéma de compréhension]\label{ax.ZF.compre}
  Soit $\varphi(x_0,\ldots,x_n)$ une phrase mathématique à $n$ variables libres.
  Alors pour tous ensembles $X$ et $a_1,\ldots,a_n$, il existe l'ensemble
  $\{x\in X \mid \varphi(x,a_1,\ldots,a_n)\}$, ce qui s'écrit formellement
  \[\forall X\forall a_1\;\cdots\;a_n\exists y,\qquad
    \forall x, x \in y \iff (x\in X\land \varphi(x,a_1,\ldots,a_n))\]
\end{axiom}

\begin{notation}
  On définit donc la notation $\{x\in X\mid \varphi(x,a_1,\ldots,a_n)\}$ pour
  l'ensemble précédent.
\end{notation}

\begin{exercise}
  Soit un ensemble $x$ non vide, montrer qu'il existe l'ensemble $\bigcap x$ des
  éléments qui sont dans tous les éléments de $x$.
\end{exercise}

\begin{notation}
  On définit des notations pour l'intersection analogues à celles pour l'union :
  $\displaystyle\bigcap_{i = 1}^n x_i$ et $x \cap y$.
\end{notation}

\begin{exercise}\label{exo.ZF.prod}
  Soient $x$ et $y$ deux ensembles. On définit le couple $(x,y)$ par
  \[(x,y) \defeq \{\{x,y\},\{x\}\}\]
  Montrer que pour tous $x,y,x',y'$, $(x,y) = (x',y')$ si et seulement si
  $x=x'$ et $y=y'$.

  Construire un prédicat $\varphi(x,y,z)$ tel que $\varphi(x,y,z)$ est vrai
  si et seulement si $z = (x,y)$. En déduire en considérant une partie bien
  choisie de $\powerset (\powerset(x\cup y))$ que
  \[x\times y \defeq \{(a,b)\mid a \in x, b \in y\}\]
  est un ensemble bien défini.
\end{exercise}

Le second axiome, le schéma d'axiomes de remplacement, peut être vu comme une
version plus forte de la compréhension. Plutôt que de s'intéresser à filtrer des
éléments dans un ensemble plus gros, le but de ce schéma d'axiomes est de
construire un ensemble par une fonction. Comme la notion de fonction n'est pas
encore définie, nous utilisons à la place la notion de relation fonctionnelle.

\begin{definition}[Relation fonctionnelle]
  Une relation binaire est ici une formule à deux variables libres. On dit
  qu'une relation $R(x,y)$ est fonctionnelle si pour chaque $x$, il existe au
  plus un $y$ tel que $R(x,y)$. On écrira pour raccourcir
  \[\Funct(R) \defeq \forall x\forall y\forall z, R(x,y)\land R(x,z)\implies
  y = z\]

  Pour une relation fonctionnelle $R(x,y)$, on définit la collection du domaine
  de $R$ :
  \[\dom(R)(x)\defeq \exists y. R(x,y)\]
  et la collection de l'image de $R$ :
  \[\im(R)(y)\defeq \exists x. R(x,y)\]
\end{definition}

\begin{axiom}[Schéma de remplacement]\label{ax.ZF.repl}
  Pour toute formule $R(x_0,\ldots,x_{n+1})$, pour tous ensembles
  $X,a_1,\ldots,a_n$, l'ensemble image de $R(a_1,\ldots,a_n)$ par $X$ est aussi
  un ensemble :
  \[\Funct(R)\implies \forall X \forall a_1\;\cdots\;a_n\exists y,\qquad
  \forall x, x \in y \iff (\exists z, z\in X \land R(a_1,\ldots,a_n,z,x))\]
\end{axiom}

\begin{exercise}
  En remarquant qu'une relation fonctionnelle peut représenter une fonction
  partielle, montrer que le schéma d'axiomes de compréhension peut se déduire du
  schéma d'axiomes de remplacement.
\end{exercise}

\begin{exercise}\label{exo.ZF.prod2}
  En réutilisant le prédicat $\varphi$ de l'\cref{exo.ZF.prod}, montrer grâce au
  schéma d'axiomes de remplacement que pour tous ensembles $x,y$, l'ensemble
  $x\times y$ est bien défini même sans l'axiome de l'ensemble des parties.
\end{exercise}

Donnons aussi l'axiome le plus évident.

\begin{axiom}[Univers non vide]\label{ax.ZF.nonEmpty}
  Il existe un ensemble.
\end{axiom}

\begin{exercise}
  Montrer que l'axiome précédent est équivalent à l'existe de l'ensemble vide
  $\varnothing$ défini par \[\forall z, z\notin \varnothing\]
  (où $x\notin y$ signifie $\lnot(x\in y)$) 
\end{exercise}

\subsection{L'axiome de l'infini et les entiers}

Pour introduire l'axiome suivant, il nous faut d'abord motiver l'utilisation de
ses éléments constitutifs. L'ensemble mathématique le plus élémentaire que l'on
est amené à étudier est certainement $\mathbb N$, l'ensemble des entiers
naturels. Une formalisation habituelle de cet ensemble demande en général les
trois constituants suivant:
\begin{itemize}
\item l'élément $0$
\item la fonction unaire $S$, correspondant à $n \mapsto n + 1$
\item le principe de récurrence, que l'on peut encoder dans les ensembles par le
  fait que si $F\subseteq\mathbb N$, $0 \in F$ et
  $\forall n, n\in F \implies S\;n \in F$ alors $F = \mathbb N$.
\end{itemize}

Chercher à définir $\mathbb N$ dans ZFC nous demande donc de définir ces
éléments. Un candidat naturel à $0$ est $\varnothing$ : les deux sont les objets
nuls par excellence, et $\varnothing$ est le premier ensemble que l'on peut
construire. La question de savoir ce qu'est $S\;x$ pour un ensemble $x$ est
alors naturelle : pour cela, nous utilisons le codage de Von Neumann consistant
à coder l'entier naturel $n$ par $\{0,\ldots,n-1\}$. En effet, cela nous offre
une définition naturelle à la fonction $S$ :
\[S\;x\defeq x\cup \{x\}\]

Remarquons que l'on peut déjà construire tous les entiers que l'on souhaite :
on peut construire $0$ et itérer la fonction $S$. Malheureusement, rien ne nous
dit que la collection $\{S^n\;0\mid n \in \mathbb N\}$ est bien elle-même un
ensemble (où le $\mathbb N$ apparaissant dans la définition est l'ensemble des
entiers naturels de notre méta-théorie). Pour palier ce problème, et pour éviter
d'utiliser notre méta-théorie explicitement, on va à la place définir
$\mathbb N$ comme le plus petit ensemble contenant $0$ et stable par la fonction
$S$. Il nous reste à savoir qu'un ensemble contenant $0$ et stable par $S$
existe bien.

\begin{axiom}[Infini]\label{ax.ZF.infini}
  Il existe un ensemble contenant $\varnothing$ et stable par $S$ :
  \[\exists x, \qquad \varnothing \in x \land
  \forall y, y \in x \implies S\;y\in x\]
\end{axiom}

\begin{remark}
  On peut en fait se passer de l'axiome de l'ensemble vide en prenant l'axiome
  de l'infini (et en adaptant sa définition pour ne pas appeler explicitement
  l'ensemble vide).
\end{remark}

\begin{definition}[Entiers naturels]
  Soit $X$ l'ensemble défini par l'axiome de l'infini. On définit alors
  $\mathbb N$ comme
  \[\mathbb N \defeq \{x\in X \mid \forall Y, Y\subseteq X\land
  \varnothing \in Y \land (\forall a, a \in Y \implies S\;a\in Y) \implies
  x\in Y\}\]
\end{definition}

On vérifie alors le principe de récurrence.

\begin{theorem}[Récurrence]
  Soit $F$ une partie de $\mathbb N$ telle que $\varnothing \in F$ et
  $\forall n, n \in F \implies S\;n \in F$, alors $F = \mathbb N$.
\end{theorem}

\begin{proof}
  Comme $F$ est une partie de $\mathbb N$, il nous suffit de montrer que
  $\mathbb N \subseteq F$. Par transitivité de l'inclusion,
  $F\subseteq X$ pour $X$ l'ensemble à partir duquel $\mathbb N$ a été défini.
  On sait de plus que
  \[x \in \mathbb N \implies \forall Y, Y\subseteq X \land
  \varnothing \in Y \land (\forall a, a \in Y \implies S\;a \in Y) \implies
  x\in Y\]
  d'où, en spécialisant $Y$ en $F$, et sachant que $\varnothing\in F$ et
  $\forall n, n \in F \implies S\;n \in F$, il vient que
  \[x\in \mathbb N \implies x \in F\] ce qui est exactement
  $\mathbb N\subseteq F$, d'où le résultat.
\end{proof}

\begin{notation}
  A partir de maintenant, pour fluidifier l'écriture, on adoptera un style plus
  laxiste sur l'écriture des propositions. Par exemple on se permettra d'écrire
  $\forall x \in X, \psi$ pour $\forall x, x\in X \implies \psi$ et tous les
  légers abus de notations du même genre.
\end{notation}

\subsection{Axiome du choix et fonctions}

Pour introduire l'axiome du choix, le mieux est de parler de ce qu'on voudrait
faire mais ne peut pas faire sans lui. Prenons $X_1,\ldots,X_n$ des ensembles
finis non vides. On peut prouver qu'il existe
$(x_1,\ldots,x_n)\in X_1\times\cdots\times X_n$ (en codant le produit itéré et
les tuples de taille arbitraire). Pour cela, il suffit de raisonner par
récurrence sur $n$ : pour passer de l'étape $n$ à l'étape $n+1$ il suffit de
remarquer que $X_{n+1}$ est non vide, et prendre alors un élément $x\in X_{n+1}$
pour l'ajouter au tuple déjà construit.

Le problème ici est que, si l'on peut obtenir ce résultat sur un nombre fini
d'ensembles non vides, ce processus ne fonctionne plus lorsque l'on passe au cas
infini. On ne peut pas faire une infinité de choix, puisque l'on n'utilise que
des méthodes finies. C'est pourquoi l'axiome du choix existe : il permet de
choisir dans des familles infinies d'ensembles non vides. Donnons d'abord la
définition de cet axiome.

\begin{axiom}[Choix]\label{ax.ZF.AC}
  Soit $X$ un ensemble dont les éléments sont non vides et deux à deux
  disjoints. Alors il existe un ensemble $C$ tel que pour tout $x\in X$,
  $C\cap x$ est un singleton. \'Ecrit formellement :
  \[
  \forall X, \varnothing\neq X \land \varnothing\notin X \land
  (\forall x,y \in X, x \cap y \neq
  \varnothing \implies x = y)
  \implies \exists C, \forall x \in X, \exists y, C \cap x = \{y\}
  \]
\end{axiom}

Il existe plusieurs reformulation naturelles de l'axiome du choix. Pour
celles-ci, nous allons introduire le formalisme pour traiter des fonctions.

\begin{definition}[Fonction]
  Soient $X,Y$ deux ensembles. Une fonction $f : X \to Y$ est une partie
  $\Gamma_f\subseteq X \times X$ (appelée graphe de $f$)  telle que le prédicat
  $(x,y)\in \Gamma$, défini pour $x\in X$ et $y\in Y$, est une relation
  fonctionnelle. On désigne par $\dom(f)$ et $\im(f)$ les ensembles associés à
  la relation fonctionnelle. On dit que $f$ est partielle si $\dom(f) \neq X$,
  et totale si $\dom(f) = X$.

  Pour deux ensembles $X,Y$, on note par $\Funct(X,Y)$ l'ensemble des fonctions
  de $X$ dans $Y$ :
  \[\Funct(X,Y) \defeq \{f : X \to Y\}\]
\end{definition}

\begin{exercise}
  Soient $X$ et $Y$ deux ensembles. Montrer que $\Funct(X,Y)$ est bien un
  ensemble.
\end{exercise}

\begin{remark}
  Dire que $f$ est partielle, dans ce cadre, n'est pas bien défini. En effet, on
  peut toujours prendre une fonction $f : X \to Y$ et l'étendre en
  $f : S\;X \to Y$ : la fonction est la même mais cette propriété change. Pour
  éviter cela, une fonction consistera souvent plutôt en un triplet
  $(X,Y,\Gamma_f)$, mais nous nous concentrerons simplement sur $\Gamma_f$. Il y
  a donc un abus de notation ici lorsque nous construirons par exemple une
  fonction en construisant juste son graphe, mais celui-ci reste léger car on
  connait toujours grâce au contexte les deux autres composantes de la fonction.
\end{remark}

\begin{notation}
  Soit une fonction $f : X \to Y$. On s'autorise à ajouter $f$ comme symbole de
  fonction dans notre langage, dont l'argument est un élément de $X$. Cela n'est
  pas autorisé par défaut, et nous devons donc le coder dans notre langage
  élémentaire (de la même manière que l'on convient qu'il est possible de
  réécrire $x\cup y$ uniquement grâce à $\in$ et $=$). Pour cela, on définit la
  traduction d'une phrase $\varphi$ faisant potentiellement intervenir
  l'expression $f(x)$ en une phrase $\varphi'$ sans $f(x)$, par induction sur
  la structure de $\varphi$ :
  \begin{itemize}
  \item si $\varphi = R(f(x))$ où $R$ est une proposition atomique (une relation
    avec potentiellement d'autres termes que $f(x)$), alors
    \[\varphi' \defeq \forall y\in Y, (x,y)\in \Gamma_f \implies R(y)\]
  \item si $\varphi = \lnot \psi$, alors $\varphi' = \lnot \psi'$.
  \item si $\varphi = \psi \lor \chi$, alors $\varphi' = \psi'\lor \chi'$.
  \item si $\varphi = \psi \land \chi$, alors $\varphi' = \psi'\land \chi'$.
  \item si $\varphi = \exists a, \psi$, alors $\varphi' = \exists a, \psi'$.
  \item si $\varphi = \forall a, \psi$, alors $\varphi' = \forall a, \psi'$.
  \end{itemize}

  On considère donc qu'introduire la notation $f(x)$ est une facilité
  d'écriture.
\end{notation}

\begin{remark}
  Cet abus de notation est aussi possible dans le cas d'une relation
  fonctionnelle qui n'est pas donnée par une fonction (au sens ensembliste).
\end{remark}

On définit de plus les notions d'injectivité, de surjectivité et de bijectivité.

\begin{definition}[Injectivité, surjectivité, bijectivité]
  Une fonction $f : A \to B$ est dite injective si chaque élément $b\in B$ a au
  plus un antécédent, c'est-à-dire si la proposition suivante est vérifiée :
  \[\forall x,y\in A, f(x) = f(y)\implies x=y\]

  Une fonction $f : A \to B$ est dite surjective si tout élément $b\in B$ a au
  moins un antécédent, c'est-à-dire si la proposition suivante est vérifiée :
  \[\forall y\in B, \exists x \in A, f(x)=y\]

  Une fonction $f : A \to B$ est dite bijective si elle est à la fois injective
  et surjective.
\end{definition}

Donnons un exercice pour manipuler la notion de fonction et celle de récurrence.

\begin{exercise}
  Montrer que l'on peut définir des fonctions par récursion sur $\mathbb N$,
  c'est-à-dire montrer que si l'on a un ensemble $X$, un élément $x_0\in X$ et
  une fonction $f_S : \mathbb N \times X \to X$ alors il existe une unique
  fonction $f : \mathbb N \to X$ telle que
  \[\begin{cases}
  f(0) = x_0\\
  \forall n \in \mathbb N, f(S\;n) = f_S(n,f(n))
  \end{cases}\]
\end{exercise}

L'exercice suivant permet de définir des fonctions depuis un produit cartésien
en considérant seulement ses projections.

\begin{exercise}
  Soient $X,Y,Z$ des ensembles. Montrer qu'il existe une bijection entre
  $\Funct(X\times Y,Z)$ et $\Funct(X,\Funct(Y,Z))$ donnée par
  $f\mapsto (x\mapsto y \mapsto f(x,y))$.
\end{exercise}

Donnons maintenant deux nouvelles formulations de l'axiome du choix.

\begin{proposition}
  L'axiome du choix est équivalent à la proposition suivante~:
  \[\forall x, \varnothing\neq x \land \varnothing\notin x \implies
  \exists f : x \to \bigcup x, \forall y \in x, f(y) \in y\]
\end{proposition}

\begin{proof}
  Supposons l'axiome du choix. On considère un ensemble $x$ dont tous les
  éléments sont non vides. On définit alors l'ensemble
  \[C_0\defeq \{\{(y,z)\mid z \in y\}\mid y \in x\}\]
  Comme $x$ n'a pas d'élément vide, chaque élément de $C_0$ est non vide. De
  plus, si $a \neq b \in C_0$, alors
  $a = \{(a,y)\mid y \in a\}$ et $b = \{(b,y)\mid y \in b\}$, donc
  $a\cap b = \varnothing$. On peut donc appliquer l'axiome du choix sur
  $C_0$ pour trouver $C$ tel que pour tout $y\in x$,
  \[\{(y,z)\mid z \in y\}\cap C = \{(y,f(y))\}\]
  donc $C\cap \bigcup C_0$ définit exactement une relation fonctionnelle $f$
  (à $y$ on n'associe qu'une seule image), et par définition de $C_0$ on sait
  que $f(y)\in y$.

  Supposons qu'on possède une fonction de choix pour tout ensemble sans élément
  vide. Soit $X$ un ensemble dont les éléments sont non vides et deux à deux
  disjoints. On peut alors trouver $f : X \to \bigcup X$ telle que $f(x)\in x$
  pour tout $x\in X$. Soit alors $C = \im(f)$. Comme $f(x)\in x$ et tous les
  éléments de $X$ sont deux à deux disjoints, on en déduit que $C\cap x$
  contient au plus un élément, pour chaque $x\in X$. Comme de plus
  $f(x) = x$, $C\cap x$ contient au moins un élément. Donc $C\cap x$ est un
  singleton.
\end{proof}

\begin{proposition}
  L'axiome du choix est équivalent à la proposition suivante~:
  \[\forall x, x\neq\varnothing\implies
  \exists f : \powerset (x) \backslash \{\varnothing\} \to x,
  \forall y \in \powerset(x)\backslash \{\varnothing\}, f(y)\in y\]
\end{proposition}

\begin{proof}
  Cette formulation est un cas particulier de la formulation précédente, puisque
  $powerset(x)\setminus\{\varnothing\}$ est non vide et ne contient que des
  éléments non vides.

  Réciproquement, supposons qu'on a $x$ non vide donc les éléments sont non
  vides. On a alors
  \[x\subseteq \powerset\bigg(\bigcup x\bigg)\setminus\{\varnothing\}\]
  d'où le résultat en spécialisant une fonction de choix donnée par hypothèse
  sur $x$.
\end{proof}

Une autre formulation importante nécessite la notion de produit, que l'on peut
simplement considérer comme une généralisation de l'opération $\times$.

\begin{definition}[Produit cartésien quelconque]
  Soit $X$ un ensemble, on définit $\prod X$ par
  \[\prod X \defeq \Big\{f : X \to \bigcup X \;\Big|\; \forall x \in X, f(x)\in
  x\Big\}\]

  Si $\{X_i\}_{i\in I}$ est une famille de fonctions (c'est-à-dire une fonction
  $g : i \mapsto X_i$) alors on considère que
  \[\prod_{i\in I} X_i \defeq \Big\{ f : I \to \bigcup_{i\in I} X_i\;\Big|\;
  \forall i \in I, f(i) \in X_i\Big\}\]
\end{definition}

\begin{proposition}
  L'axiome du choix est équivalent à la proposition suivante :
  \[\forall x, x\neq \varnothing \land (\forall y \in x, y \neq \varnothing)
  \implies \prod x \neq \varnothing\]
\end{proposition}

\begin{proof}
  Par définition, $\prod X$ est non vide si et seulement si on a une fonction
  $f : X \to \bigcup X$ telle que $\forall x \in X, f(x)\in x$. C'est un énoncé
  précédent de l'axiome du choix.
\end{proof}

\subsection{Axiome de fondation}

Le dernier axiome de ZFC est l'axiome de fondation. Selon les auteurs, il est ou
non ajouté à ZFC : ici, nous considérerons que ZFC le contient. Cet axiome
empêche l'existence de cycles d'appartenance, par exemple d'ensemble $x$ tel que
$x\in x$. Formellement, l'axiome énonce que la relation $\in$ sur l'univers
ensembliste $\mathcal U$ est bien fondée.

\begin{axiom}[Fondation]\label{ax.ZF.AF}
  Pour tout ensemble $x$ non vide, il existe $y\in x$ tel que
  $x\cap y = \varnothing$. Formellement, cela s'écrit :
  \[\forall x, x\neq\varnothing \implies \exists y \in x, x\cap y =
  \varnothing\]
\end{axiom}

Si l'on voit un ensemble $x$ comme un arbre, où chaque n\oe ud est un ensemble
et où l'on place une arête entre $x$ et $y$ si $x\in y$, alors l'axiome de
fondation stipule que pour chaque ensemble, l'arbre ainsi construit n'a pas de
branche infinie (ni de cycle).

Remarquons que l'arbre que nous avons construit contient non seulement les
éléments de $x$ mais aussi les éléments des éléments de $x$, et ainsi de suite.
Cette idée nous permet d'introduire la notion d'ensemble transitif, qui est un
ensemble tel que l'arbre ainsi généré ne contient que des éléments de l'ensemble
(c'est-à-dire que tous les éléments des éléments de $x$ sont aussi éléments de
$x$).

\begin{definition}[Ensemble transitif]
  Soit un ensemble $x$. On dit que $x$ est transitif si tout élément d'un
  élément de $x$ est aussi élément de $x$. Formellement, cela s'écrit
  \[\trans(x) \defeq \forall y\in x, \forall z \in y, z\in x\]
\end{definition}

Remarquons qu'étant donné un ensemble, on peut toujours ajouter ses éléments,
les éléments de ses éléments et ainsi de suite pour obtenir à la fin un ensemble
transitif.

\begin{proposition}
  Soit $x$ un ensemble. Alors il existe un plus petit ensemble $\trcl(x)$
  transitif contenant $x$.
\end{proposition}

\begin{proof}
  Pour construire $\trcl(x)$, on construit la suite d'ensembles
  $(x_i)_{i\in\mathbb N}$ suivante :
  \begin{itemize}
  \item $x_0 = x$
  \item $x_{i+1} = \bigcup x_i$
  \end{itemize}
  L'ensemble $\trcl(x)$ est alors $\displaystyle\bigcup_{n\in \mathbb N} x_n$.

  Vérifions que $\trcl(x)$ est bien transitif et contient $x$. Comme
  $x_0\subseteq \trcl(x)$ par définition de l'union, $\trcl(x)$ contient $x$.
  Soient $y\in \trcl(x)$ et $z\in y$. Par définition, il existe $n\in \mathbb N$
  tel que $y\in x_n$. Alors $z\in x_{n+1}$ puisque $z\in y \in x_n$ et
  $x_{n+1}=\bigcup x_n$. Ainsi $z\in \trcl(x)$.

  Soit un ensemble transitif $y$ contenant $x$. Alors par récurrence sur $n$,
  $x_n\subseteq y$ :
  \begin{itemize}
  \item par hypothèse, $x\subseteq y$.
  \item si $x_n\subseteq y$, alors soit $z\in x_{n+1}$, par définition on trouve
    $a\in x_n$ tel que $z\in a \in x_n$, et comme $x_n\subseteq y$, $a\in y$.
    Mais $y$ est transitif, donc $z\in y$ : on en déduit que
    $x_{n+1}\subseteq y$.
  \end{itemize}
  Par récurrence, on en déduit que $\forall n\in\mathbb N, x_n\subseteq y$,
  d'où $\trcl(x)\subseteq y$.
\end{proof}

\begin{exercise}
  Montrer qu'un ensemble $x$ est transitif si et seulement si la propriété
  suivante est vérifiée :
  \[\forall y\in x, y\subseteq x\]
\end{exercise}

Les deux propriétés de bonne fondation de $\in$ et de transitivité mènent au
lemme d'effondrement de Mostowski. Celui-ci montre que toute classe munie d'une
relation se comportant suffisamment comme la relation $\in$ peut en fait se
simuler par une unique classe transitive bien fondée où la relation est
directement $\in$.

\begin{lemma}[Effondrement de Mostowski]\label{lem.most}
  Soit $\mathcal M$ une classe et $R$ une relation vérifiant :
  \begin{itemize}
  \item $R$ définit des ensembles, c'est-à-dire que pour tout $x\in \mathcal M$,
    la collection définie par $R^{-1}[x] = \{y\mid y R x\}$ est un ensemble.
  \item $R$ est bien fondée, c'est-à-dire :
    \[\forall X\subseteq \mathcal M, \exists x \in X, R^{-1}[x]\cap X =
    \varnothing\]
  \item $R$ est extensionnelle, c'est-à-dire :
    \[\forall x,y\in \mathcal M, R^{-1}[x] = R^{-1}[y] \implies x = y\]
  \end{itemize}

  Alors il existe une unique classe $\mathcal N$ et un unique isomorphisme
  $\phi : \mathcal M\to \mathcal N$ tels que
  $(\mathcal M,R) \overset{\phi}{\cong} (\mathcal N,\in)$.
\end{lemma}

\begin{remark}
  Nous n'avons pas formellement défini ce qu'est un isomorphisme entre classes.
  Cette définition est assez naturelle : étant données deux classes $\mathcal M$
  et $\mathcal N$ avec chacune une relation $R$ (respectivement $R'$), un
  isomorphisme $\phi : (\mathcal M,R)\cong (\mathcal N,R')$ est une proposition
  $\phi$ à deux variables libres telle que, pour une proposition $\psi$
  définissant $\mathcal M$ et une proposition $\chi$ définissant $\mathcal N$,
  les propositions suivantes peuvent être prouvées :
  \begin{itemize}
  \item $\forall x,\forall y,\forall z, \psi(x)\land \chi(y)\land\chi(z)\land
    \phi(x,y)\land\phi(x,z) \implies y = z$
  \item $\forall x,\psi(x)\implies \exists y, \chi(y)\land \phi(x,y)$
  \item $\forall x,\forall y,\forall z, \psi(x)\land\psi(y)\land\chi(z)\land
    \phi(x,z)\land\phi(y,z)\implies x = y$
  \item $\forall y,\chi(y)\implies \exists x, \psi(x)\land \phi(x,y)$
  \item $\forall x,\forall x',\forall y, \forall y', \psi(x)\land\psi(x')\land
    \chi(y)\land\chi(y')\land\phi(x,y)\land\phi(x',y')\implies
    (xR x'\iff yR' y')$
  \end{itemize}

  Cependant, nous allons simplement travailler sur les classes comme sur les
  ensembles, puisque les manipulations syntaxiques correspondant à nous
  arguments ensemblistes habituels sont les mêmes que celles que nous allons
  faire sur nos classes. Remarquons simplement que ce lemme est, puisqu'il parle
  de classe, un schéma de lemmes paramétré par les propositions définissant
  $\mathcal M$ et $R$.
\end{remark}

\begin{proof}
  Montrons d'abord que si $M_1$ et $M_2$ sont deux classes transitives et que
  $\theta : M_1\to M_2$ est un isomorphisme pour la rleation $\in$, alors
  $\theta$ est l'identité.

  Par l'absurde, supposons que $\theta$ n'est pas l'identité. L'ensemble
  $C = \{x\mid \theta(x)\neq x\}$ n'est donc pas vide. Par l'axiome de
  fondation, on trouve un élément minimal $x\in C$, et soit $y = \theta(x)$. On
  peut alors remarquer que par minimalité de $x$, pour tous $z\in x$,
  $\theta(z)=z$, et comme $\theta$ est un isomorphisme on en déduit que
  $z\in y$. Ainsi $x\subseteq y$. Il nous reste à montrer que $y\subseteq x$~:
  si $z\in y$, alors $z\in M_2$ comme $M_2$ est une classe transitive, donc on
  trouve $a\in M_1$ tel que $\theta(a)=z$, mais comme $\theta(a)\in \theta(x)$,
  on en déduit que $a\in x$, et puisque $x$ est minimal dans $C$, $\theta(a)=a$,
  c'est-à-dire $z = a$. Donc $y\subseteq x$. On a donc une absurdité étant donné
  que $x = \theta(x)$ : ainsi $C$ est vide, donc $\theta$ est l'identité.

  Reprenons maintenant notre classe $\mathcal M$. On définit alors la classe
  $\mathcal N$ par l'image de la fonction $\pi$ définie comme suit :
  \[\begin{array}{rcccl}
  \pi & : & \mathcal M & \longrightarrow & \mathcal U\\
  & & x & \longmapsto & \{\pi(y)\mid y \in R^{-1}[x]\}
  \end{array}\]
  c'est-à-dire que la relation $\pi$ est définie par la proposition
  \[\pi(y) \defeq \{\pi(z)\mid z \in R^{-1}[x]\}\]
  où l'on prouve par l'absurde que cette proposition vérifie les propriétés d'un
  isomorphisme entre $(\mathcal M,R)$ et $(\mathcal N,\in)$, en utilisant un
  principe analogue à la preuve précédente. La classe $\mathcal N$ est définie
  par la proposition
  \[y \in \mathcal N \defeq \exists x\in\mathcal M, \pi(x,y)\]

  L'unicité découle alors du fait que s'il existe une autre telle classe
  $\mathcal N'$, alors l'isomorphisme entre elle et $\mathcal N$ vérifie les
  hypothèses du résultat intermédiaire, et est donc l'identité. D'où
  $\mathcal N = \mathcal N'$.
\end{proof}

\begin{remark}
  Notre définition de $\pi$ peut sembler illégitime, étant donné que le
  \cref{thm.recur.bf} ne couvre que le cas où le domaine est un ensemble (et
  qu'il est un ordre bien fondé, mais il est assez clair que cela se déroule de
  la même façon pour une relation bien fondée qui n'est pas forcément un ordre).
  Cependant, on voit que pour définir $\pi(y)$, il suffit de le définir sur
  un ensemble, celui des éléments inférieurs (comme $R$ définit des ensembles,
  cela fonctionne bien).

  Ainsi la version purement formelle de $\pi$ devrait être, avec $\varphi$ le
  prédicat définissant $\mathcal M$~:
  \begin{multline*}
    \pi(x,y) \defeq \exists X,
    (\forall a \in X, \varphi(a) \land \forall b, b R a \implies b \in X)
    \land \exists f : X \to \mathcal U, \\(\forall z\in X,
    \forall Y, (\forall a, a \in Y \iff a R z) \implies
    f(z) = \{f(a) \mid a R z\})\\
    \land f(x) = y
  \end{multline*}

  Et l'utilisation du \cref{thm.recur.bf} permet de conclure que cette relation
  est bien fonctionnelle.
\end{remark}

\subsection{Résumé~: les fragments de ZFC}

Concluons cette section en redonnant les différents axiomes de ZFC. En fait,
en enlevant les doublons, il nous suffit d'un petit fragment de ce que nous
avons vu. Donnons d'abord les axiomes de ZF~:
\begin{itemize}
\item l'\cref{ax.ZF.ext} (extensionalité)
\item l'\cref{ax.ZF.union} (réunion)
\item l'\cref{ax.ZF.pow} (ensemble des parties)
\item l'\cref{ax.ZF.repl} (schéma de remplacement)
\item l'\cref{ax.ZF.infini} (infini)
\item l'\cref{ax.ZF.AF} (fondation)
\end{itemize}
ZFC est alors obtenue comme $\ZF + \AxC$. On peut aussi définir
$\ZF - \AxF$ par exemple. De plus, la théorie Z, de Zermelo, est la théorie
contenant le schéma d'axiomes de compréhension mais pas de remplacement,
ne contenant pas l'axiome de fondation et contenant l'axiome de la paire (et les
axiomes d'extensionalité, de réunion, d'ensemble des parties et de l'infini).

Dans la suite du chapitre, nous n'aurons besoin que de $\ZF - \AxF$.

\section{Construction des autres ensembles de nombres}

Nous avons construit $\mathbb N$ directement grâce à l'axiome de l'infini, mais
celui-ci n'est pas le seul ensemble de nombres important à construire. Dans
cette section, nous allons vois la construction de $\mathbb Z, \mathbb Q$ et
$\mathbb R$, ainsi que leurs propriétés essentielles. Nous allons commencer
notre étude par celle de $\mathbb N$, puisque nous avons simplement défini cet
ensemble et le principe de récurrence.

\subsection{Les entiers naturels}

Rappelons les points essentiels à propos de $\mathbb N$ : cet ensemble contient
$0$, une fonction $S$ unaire, et le principe de récurrence. De plus, on peut
définir une fonction par récurrence. Nous pouvons donc définir les fonctions
$+$ et $\times$.

\begin{definition}[Addition, multiplication]
  On définit la fonction $+ : \mathbb N \times \mathbb N \to \mathbb N$ et la
  fonction $\times : \mathbb N \times \mathbb N \to \mathbb N$ par récurrence
  sur leur premier argument :
  \begin{itemize}
  \item $\forall n \in \mathbb N, n + 0 = n$
  \item $\forall n,m\in \mathbb N, n + (S\;m) = S(n + m)$
  \item $\forall n \in \mathbb N, n \times 0 = 0$
  \item $\forall n,m\in\mathbb N, n \times (S\;m) = n\times m + n$
  \end{itemize}
\end{definition}

\begin{notation}
  On définit $1 = S\;0$, $2 = S\;S\;0$ et ainsi de suite. Plutôt que $S\;n$,
  nous utiliserons $n + 1$.
\end{notation}

\begin{property}
  Les propriétés suivantes sont vérifiées :
  \begin{itemize}
  \item $0$ est un élément neutre pour $+$ : pour tout $n\in \mathbb N$,
    $0 + n = n$ et $n + 0 = n$.
  \item $+$ est commutatif : pour tous $n,m\in\mathbb N$, $n+m=m+n$.
  \item $+$ est associatif : pour tous $n,m,p\in\mathbb N$,
    $n+(m+p) = (n+m)+p$.
  \item $+$ est régulier : pour tous $n,p,m\in\mathbb N$, si $n+m = n+p$ alors
    $m=p$.
  \item pour tous $n,m\in \mathbb N$, si $n + m = 0$ alors $n=0$ et $m = 0$.
  \item $+$ distribue sur $\times$ : pour tous $n,m,p\in\mathbb N$,
    $n\times(m+p) = n\times m + n \times p$ et
    $(n+m)\times p = n\times p + m \times p$.
  \item $1$ est un élément neutre pour $\times$.
  \item $0$ est absorbant pour $\times$ : pour tout $n\in \mathbb N$,
    $0\times n = 0$.
  \item $\times$ est commutatif.
  \item $\times$ est associatif.
  \item La structure est intègre : pour tous $n,m\in \mathbb N$, si
    $n\times m = 0$ alors $n = 0$ ou $m = 0$.
  \end{itemize}
\end{property}

\begin{proof}
  A FAIRE
\end{proof}

La relation $\leq$ est aussi définissable grâce à notre langage ensembliste.

\begin{definition}[Inégalité]
  On définit la relation $\leq\subseteq \mathbb N \times \mathbb N$ par :
  \[n \leq m \defeq \exists k \in \mathbb N, m = n + k\]
\end{definition}

\begin{property}
  La relation $\leq$ est une relation d'ordre.
\end{property}

\begin{proof}
  Prouvons que la relation est réfléxive, antisymétrique et transitive :
  \begin{itemize}
  \item soit $n\in \mathbb N$, alors $n = n + 0$, d'où $n\leq n$.
  \item soient $n,m\in \mathbb N$, supposons que $n \leq m$ et $m \leq n$. On
    trouve alors $k,k'$ tels que $n = m + k$ et $m = n + k'$. En substituant
    la deuxième inégalté dans la première, on obtient alors
    \begin{align*}
      n &= (n + k') + k\\
      &= n + (k' + k)\\
      n + 0 &= n + (k' + k)\\
      0 &= k + k'\qquad\text{par régularité de }+\\
    \end{align*}
    d'où $0 = k$, donc $n = m$.
  \item soient $n,m,p\in \mathbb N$ tels que $n\leq m$ et $m\leq p$. On trouve
    donc $k,k'$ tels que $m = n + k$ et $p = m + k'$. Par substitution, on en
    déduit que $p = (n + k) + k'$, et par transitivité $p = n + (k + k')$, d'où
    $n \leq p$.
  \end{itemize}
  Ainsi, $\leq$ est une relation d'ordre.
\end{proof}
