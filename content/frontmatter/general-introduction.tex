\chapter[Introduction générale]{Introduction générale}
\markboth{Introduction générale}{Introduction générale}

Il est intéressant, en étudiant la logique, de voir combien ce domaine est
difficile à situer : à l'origine une part de la philosophie, puis devenue plus
tard une branche mathématique, elle est de nos jours omniprésente en
informatique. Autant dire que ce livre est bien trop court pour vous donner,
cher lecteur, une vision réunissant tous ces domaines et se voulant complète.
Les auteurs étant principalement rodés à la logique mathématique et à
l'inforamtique, ce livre portera largement sur ces deux visions : la logique
sera un outil formel d'analyse du discours mathématique en premier lieu.
Cependant, il est important de rappeler que ce domaine n'est pas exempt de
controverses, loin s'en faut. Ces controverses sont rarement de nature
mathématique, et encore moins de nature informatique : elles appartiennent
pleinement à la philosophie.

Il est donc nécessaire d'accepter dès le début de ce livre que, dans un objectif
de pédagogie, et puisque les controverses peuvent nuire dans un premier temps à
la compréhension de certaines notions, des choix d'ordre philosophique seront
régulièrement pris au long de cet ouvrage. Lorsque cela arrivera, la position
qui sera tenue sera argumentée si possible, et le lecteur est libre de ne pas
adhérer à l'interprétation qui sera donnée de certains phénomènes. Le contenu du
livre, lui, se trouve avant tout dans la compréhension des objets étudiés et
dans les résultats démontrés.

Pour commencer, qu'est-ce que la logique ? Au sens philosophique, cela désigne
l'étude du raisonnement, mais son utilisation sur les mathématiques permet
d'être plus précis. La logique est l'étude du langage mathématique. C'est donc
la branche qui s'intéresse en premier lieu à comment parler des mathématiques.

Cela mène à une première distinction importante : qu'est-ce que n'est pas la
logique ? Elle n'est pas, du moins dans le cadre donné dans ce livre, une
recherche d'une vérité pré-existante aux mathématiques. Au contraire, la logique
mathématique commence par l'acceptation des mathématiques, pour mieux les
étudier. Cela peut apparaître comme un raisonnement circulaire : quelle valeur
prend une étude d'un système se basant sur le système lui-même ? N'a-t-on pas un
raisonnement erroné à partir du moment où nous utilisons les mathématiques pour
parler des mathématiques ? La réponse que nous adopterons ici est la suivante :
la logique mathématique, utilisant les mathématiques pour étudier le langage
mathématique, est un procédé empirique, et les conclusions qu'elle tire ne sont
à proprement parler que des résultats portant sur des objets mathématiques.
Cependant, de la même manière qu'une mesure d'intensité électrique fait penser à
un électricien que des électrons sont en mouvement alors que c'est la théorie
électrique elle-même qui permet de supposer la pertinence de cette mesure, nos
résultats mathématiques nous donnent à croire que quelque chose arrive, au-delà
d'un simple fait mathématique.

Ainsi, quand nous aurons prouvé qu'il ne peut exister de preuve dans ZFC que ZFC
est cohérente, où ZFC est la théorie des ensembles dans laquelle toutes les
mathématiques usuelles peuvent se faire, nous en extrpolons largement qu'il
n'existe pas de preuve de la cohérence de ZFC. Pourtant, les deux expressions
ne signifient pas strictement la même chose, mais il est cette conviction forte
chez nombre de logiciens que cette étude des mathématiques par les mathématiques
nous apprend des choses sur leur nature.

Beaucoup d'auteurs, pour distinguer ces mathématiques usuelles qui sont celles
que nous utilisons lorsque l'on fait de la logique des théories formelles
utilisées au sein de la logique pour représenter les mathématiques usuelles,
utilisent l'expression \og méta-théorie\fg{} pour la première. Ainsi la logique
se place dans une méta-théorie pour étudier des théories. En utilisant ce terme,
la thèse des paragraphes précédents est que l'étude des théories nous renseigne
sur la méta-théorie.

Notons particulièrement la différence de traitement entre les deux : nous
étudierons la théorie, tandis que la méta-théorie sera considérée comme acquise.
Ainsi la théorie ZFC permet d'imaginer un langage formel pour parler des
ensembles, mais elle se formule elle-même dans un univers que l'on considère
comme pré-existant et vérifiant beaucoup de propriétés qu'on ne saurait écrire
au sein de ce même univers. Face au trilemme d'Agrippa, le choix est donc fait
de prendre une posture dogmatique initiale, en gardant l'esprit ouvert sur ce
que l'étude logique peut nous faire réviser sur cet univers mathématique.

Cet ouvrage est principalement basé sur un semestre de cours suivi par les deux
auteurs, qui couvrait les 4 thèmes principaux de la logique :
\begin{itemize}
\item la théorie des ensembles
\item la théorie des modèles
\item la calculabilité
\item la théorie de la démonstration
\end{itemize}

C'est donc naturellement que ce cours sera structuré suivant ces quatres
parties, mais en ajoutant une partie préliminaire présentant les outils de base
qui seront utilisés en logique : l'induction, la logique propositionnelle et le
calcul des prédicats ainsi que la théorie des ensembles ordonnés. Nous pensons
en effet que ces prérequis méritent d'être traités à part, à la fois pour leur
importance dans toutes les autres parties et pour pouvoir s'attarder plus
longuement sur des éléments qui ne sont pas toujours approfondis dans un thème
donné.

Nous remercions nos professeurs Arnaud Durand, Thomas Ibarlucia, Thierry Joly et
Alessandro Vignati, ainsi que Daniel Hirschkoff, Pascal Koiran, Natacha Portier
et Colin Riba qui ont été nos professeurs à l'\'ENS de Lyon et dont les cours
ont été de magnifiques portes d'entrées vers le monde de la logique.
