\chapter[Sommaire]{Sommaire}
\markboth{Sommaire}{Sommaire}

Cet ouvrage est constitué de $5$ parties~: la partie préliminaire et les parties
fondamentales. La partie préliminaire cherche à donner des bases suffisantes
pour aborder sereinement la logique en général, et les parties fondamentales du
reste du livre en particulier. Les parties fondamentales, elles, donnent un
aperçu des domaines traités en logique, autour de $4$ axes~:
\begin{itemize}
\item la théorie des ensembles, qui étudie la théorie ZFC et ses implications.
\item la théorie des modèles, qui étudie les propriétés des classes de
  structures et des théories logiques du premier ordre.
\item la calculabilité, qui ajoue aux mathématiques la notion de calcul et
  cherche à répondre à des questions telles que \og comment peut-on concrètement
  décider un problème ?\fg
\item la théorie de la démonstration, qui formalise et étudie les propriétés des
  systèmes de démonstration.
\end{itemize}

Mis à part la première partie, qui est supposée connue pour aborder les autres,
toutes les parties sont principalement indépendantes. On fera parfois référence
à une autre partie dans l'une d'elle, mais seulement pour des éléments
sporadiques. Par exemple, la partie sur la théorie des modèles utilisera la
notion de cardinal, qui est définie dans la partie de théorie des ensembles,
mais n'utilisera pas de notion poussée à son propos (on n'aura pas à maîtriser
la cofinalité d'un cardinal ou à construire des ensembles stationnaires pour
apprendre la théorie des modèles).

Décrivons plus précisément les différentes parties.

\paragraph{Préliminaires}
La première partie contient ce qui est considéré par les auteurs comme essentiel
pour aborder sereinement la suite de l'ouvrage. Cette partie est un mélange de
différents cours~: les cours que les auteurs ont suivi à l'\'ENS de Lyon et les
cours de mise à niveau proposés avant la rentrée au master LMFI. Elle naît
justement de l'observation que les cours de mise à niveau, cherchant à donner
les bases de la logique du premier ordre en deux semaines de cours, abordent
parfois trop rapidement les sujets (faute de temps, certainement pas faute
d'investissement du professeur en charge). Au contraire, à l'\'ENS, les cours
proposés, particulièrement en L3, permettent d'aborder bien plus en détails les
structures les plus élémentaires qui sont utiles en logique. Ces chapitres sont
ainsi une tentative de synthèse des cours de l'\'ENS dans l'objectif d'être
directement utilisables pour aborder le reste de la logique, en suivant le
plan proposé lors des cours de mise à niveau du LMFI.

Les différents chapitres sont structurés de la façon suivante~:
\begin{itemize}
\item Le \cref{chp.induction} s'attarde sur les structure inductives. Celles-ci
  sont en quelque sorte l'exemple paradigmatique des objets syntaxiques,
  c'est-à-dire des objets que l'on peut traiter formellement. Il est en effet
  difficile d'imaginer sans autre information comment traiter des ensembles
  infinis (et que dire des ensembles infinis indénombrables), là où l'idée de
  manipuler un arbre binaire, ou une suite de $0$ et de $1$, est intuitivement
  acceptable.

  L'approche que nous développons ici utilise des ensembles d'arbres étiquetés~:
  pour définir un ensemble inductif, la structure même de l'ensemble nous permet
  d'utiliser nos objets sans ambigüité. Cette approche a l'avantage d'être plus
  intuitive, mais le désavantage d'être moins convaincante d'un point de vue
  formel. L'autre possibilité, que l'on peut retrouver chez différents autours,
  est de considérer les objets syntaxiques comme des mots particuliers,
  c'est-à-dire comme des suites finies de lettres, sur un alphabet spécifié par
  avance.

  Nous voyons dans ce chapitre comment raisonner sur des structures inductives,
  particulièrement comment définir des fonctions par récursion et prouver des
  propriétés par induction.
\item Le \cref{chp.logprop} décrit la logique propositionnelle comme une porte
  d'entrée vers la logique du premier ordre. D'une certaine façon, cette logique
  peut être vue comme une logique d'ordre $0$, et nous pensons que l'étude de ce
  cas particulier et restreint permet de faciliter l'étude qui suit, du calcul
  des prédicats. Cela nous permet de montrer différentes propriétés dans un
  contexte où la technicité est réduite, pour mieux en aborder l'essence.
\item Le \cref{chp.logpred} est la continuité du deuxième, et aborde le calcul
  des prédicats de la logique du premier ordre. Il introduit aussi des thèmes
  majeurs de la suite de l'ouvrage~: les modèles et les systèmes de
  démonstration du premier ordre. Ce chapitre peut être vu comme le plus
  indispensable de la partie préliminaire, puisque tous les chapitres qui
  suivent ont besoin de celui-ci. Les formules du premier ordre constituent le
  c\oe ur de la logique mathématique, d'où le soin particulier apporté à leur
  introduction et à leur traitement dans un chapitre dédié.
\item Le \cref{chp.ordres} porte sur la théorie des ensembles ordonnés. Il
  permet d'introduire deux notions essentielles~: les structures ordonnées et
  les filtres. La première n'est pas directement utile dans les parties
  fondamentales, mais offre un point de vue riche pour étudier de nombreux
  phénomènes liés à la logique. Elle permet aussi, de façon plus pragmatique, de
  démontrer une version forte du théorème de Knaster-Tarski, déjà utilisé dans
  le premier chapitre. La deuxième notion, les filtres, est utile dans la
  première moitié de l'ouvrage. Elle joue un rôle dès le chapitre qui suit et
  permet d'interpréter de façon féconde certains aspects de théorie des
  ensembles, notamment pour traiter les clubs, mais elle est avant tout
  vitale pour la théorie des modèles, où les ultraproduits sont l'une des
  constructions les plus présentes.
\item Le \cref{chp.topo} aborde la topologie. Le langage topologique est
  effectivement important en logique, car il est éclairant voire directement
  utile. Un exemple est celui de la topologie sur les ordinaux, dont
  l'importance se révèlera en théorie des ensembles, mais qui est bien plus
  fluide en ayant déjà étudié les bases de la topologie générale.

  Nous y verrons les définitions élémentaires de topologie générale pour traiter
  de façon abstraite les espaces topologiques, d'une façon proche de la théorie
  des ensembles. Cela nous donnera l'occasion de voir le théorème de Tychonov,
  incontournable de la topologie générale. Nous traiterons aussi de la dualité
  de Stone, dans une version adaptée pour ne pas demander de connaissances en
  théorie des catégories~: comme pour la théorie des ordres, cette étude est une
  occasion de voir une correspondance pour sa profondeur plus que pour un simple
  intérêt direct, car elle permet de montrer que des phénomènes topologiques
  peuvent correspondre à des phénomènes logiques. Enfin, ce chapitre comporte un
  rappel des notions de topologie métriques nécessaires à la théorie descriptive
  des ensembles.

  Relevons cependant que ce chapitre n'est en aucun cas un cours de topologie
  pour un élève débutant, et cherche avant tout à fixer le vocabulaire et à
  offrir un point de vue différent, plus algébrique, à un lecteur ayant vu de la
  topologie métrique principalement, et sous un prisme plus analytique.
\end{itemize}

\paragraph{Théorie des ensembles}
La théorie des ensembles, née des travaux de Cantor, est connue pour permettre
d'y formaliser l'ensemble des mathématiques. Même la partie préliminaire, donnée
avant cette partie, peut en fait y être exprimée. En ce sens, elle introduit
donc des outils particulièrement généraux qui peuvent être utiles dans d'autres
domaines. Nous mettons donc, dans un premier temps, l'accent sur ces notions
incontournables de la théorie des ensembles qui peuvent être utilisées dans
d'autres domaines, et donnons des bases pour tout lecteur voulant acquérir un
bagage minimal pour parler de théorie des ensembles. La deuxième partie, plus
spécialisée, décrit trois parties plus isolées de la théorie des ensembles pour
donner un panorama de ce que l'on peut étudier grâce à cette théorie.

La structure des chapitres est la suivante~:
\begin{itemize}
\item Le \cref{chp.axiomes} se concentre sur l'aspect le plus pratique de la
  théorie des ensembles, en montrant comment il est possible de reconstruire
  toutes les mathématiques en utilisant la théorie ZFC. Le début de ce chapitre
  est donc une description des axiomes de ZFC, pour pouvoir les manipuler.
  Plutôt que de simplement donner une liste de ces axiomes, nous les
  introduisons les uns après les autres pour permettre au lecteur de se
  familiariser avec chacun et avec ses quelques conséquences directes. Cela nous
  permet de donner par exemple le lemme d'effondrement de Mostovski, résultat
  important en théorie des ensembles. Le reste du chapitre est dédié à la
  construction des ensembles les plus basiques que nous manipulons, des entiers
  jusqu'aux réels.
\item Le \cref{chp.ordinaux} décrit les ordinaux, les cardinaux et la
  cofinalité. Ces notions sont en quelque sorte les points de départ de la
  théorie des ensembles à proprement parler. Nous y aborderons aussi l'intérêt
  de l'axiome du choix, qui est énoncé dans le chapitre précédent mais dont
  l'importance apparait grâce au lemme de Zorn, se basant sur la récursion
  transfinie. L'axiome du choix est ensuite utilisé dans l'ensemble du chapitre,
  car celui-ci est nécessaire pour assurer un bon comportement à la hiérarchie
  cardinale. Nous utilisons les cardinaux comme moyen d'aborder l'hypothèse du
  continu et le problème de définir $2^\kappa$ en tant que fonction cardinale,
  motivant l'étude de la cofinalité. Le chapitre se conclut alors qur les
  conditions d'Easton pour définir une fonction correspondant à l'exponentielle
  cardinale.
\item Le \cref{chp.combi} étudie la combinatoire des ordinaux. Le premier
  objectif est d'introduire la notion d'ensemble stationnaire et d'offrir
  quelques applications élémentaires permettant de comprendre leur
  fonctionnement. Nous y étudions en préliminaire la topologie de l'ordre pour
  les ordinaux, puis voyons la notion de club ainsi que les ensembles
  stationnaires et non stationnaires.

  Une fois la notion de club introduite, les questions d'intérêt du chapitre
  sont de chercher à quel point le filtre des clubs est stable par des
  intersections de grandes familles et combien on peut construire d'ensembles
  stationnaires disjoints. L'étude de ces questions est l'occasion d'introduire
  le lemme de Fodor sur les fonctions régressives.

  Nous voyons pour finir le chapitre une application du lemme de Fodor à un
  point important du \cref{chp.ordinaux}~: la détermination de l'exponentielle
  cardinale. Plus précisément, nous y voyons une démonstration combinatoire du
  théorème de Silver, décrivant le comportement de la fonction
  $\kappa\mapsto 2^\kappa$ pour des cardinaux singuliers.
\item Le CH9 étudie les modèles de ZFC.
\end{itemize}

\paragraph{Théorie des modèles}
A FAIRE

\paragraph{Calculabilité}
A FAIRE

\paragraph{Théorie de la démonstration}
A FAIRE

Ce livre est un livre de cours se voulant autonome. Pour permettre au lecteur de
travailler et d'approfondir ce cours, nous proposons donc des exercices. Ces
exercices sont de deux types~: les exercices au sein d'un chapitre et les
exercices à la fin d'un chapitre.

Les exercices au sein d'un chapitre sont des exercices relativement simples,
souvent calculatoires, cherchant à faire pratiquer les bases des définitions.
Certains résultats dont la démonstration est en exercice sont utilisés pour des
démonstrations dans la suite, mais ce sont des résultats jugés suffisamment
naturels pour ne pas être remis en question même sans démonstration. Un lecteur
n'ayant donc pas l'envie de faire ces exercices n'aura pas de gros souci à
suivre le reste, mais il n'aura moins l'occasion de vérifier qu'il maîtrise les
bases du cours.

Les exercices en fin de chapitre sont en général plus techniques. Certains sont
même des problèmes entiers. Ceux-ci sont tirés de différentes sources, en
premier lieu de TD ou de DM vus pendant le cursus des auteurs.

Cher lecteur, il ne nous reste qu'à vous souhaiter une bonne lecture.
