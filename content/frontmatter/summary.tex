\chapter[Sommaire]{Sommaire}
\markboth{Sommaire}{Sommaire}

Cet ouvrage est constitué de $5$ parties~: la partie préliminaire et les parties
fondamentales. La partie préliminaire cherche à donner des bases suffisantes
pour aborder sereinement la logique en général, et les parties fondamentales du
reste du livre en particulier. Les parties fondamentales, elles, donnent un
aperçu des domaines traités en logique, autour de $4$ axes~:
\begin{itemize}
\item la théorie des ensembles, qui étudie la théorie ZFC et ses implications.
\item la théorie des modèles, qui étudie les propriétés des classes de
  structures et des théories logiques du premier ordre.
\item la calculabilité, qui ajoue aux mathématiques la notion de calcul et
  cherche à répondre à des questions telles que \og comment peut-on concrètement
  décider un problème ?\fg
\item la théorie de la démonstration, qui formalise et étudie les propriétés des
  systèmes de démonstration.
\end{itemize}

Mise à part la première partie, qui est supposée connue pour aborder les autres,
toutes les parties sont principalement indépendantes. On fera parfois référence
à une autre partie dans l'une d'elle, mais seulement pour des éléments
sporadiques. Par exemple, la partie sur la théorie des modèles utilisera la
notion de cardinal, qui est définie dans la partie de théorie des ensembles,
mais n'utilisera pas de notion poussée à son propos (on n'aura pas à maîtriser
la cofinalité d'un cardinal ou à construire des ensembles stationnaires pour
apprendre la théorie des modèles).

Décrivons plus précisément les différentes parties.

\paragraph{Préliminaires}
La première partie contient ce qui est considéré par les auteurs comme essentiel
pour aborder sereinement la suite de l'ouvrage. Cette partie est un mélange de
différents cours~: les cours que les auteurs ont suivi à l'\'ENS de Lyon et les
cours de mise à niveau proposés avant la rentrée au master LMFI. Elle naît
justement de l'observation que les cours de mise à niveau, cherchant à donner
les bases de la logique du premier ordre en deux semaines de cours, abordent
parfois trop rapidement les sujets (faute de temps, certainement pas faute
d'investissement du professeur en charge). Au contraire, à l'\'ENS, les cours
proposés, particulièrement en L3, permettent d'aborder bien plus en détails les
structures les plus élémentaires qui sont utiles en logique. Ces chapitres sont
ainsi une tentative de synthèse des cours de l'\'ENS dans l'objectif d'être
directement utilisables pour aborder le reste de la logique, en suivant le
plan proposé lors des cours de mise à niveau du LMFI.

Les différents chapitres sont structurés de la façon suivante~:
\begin{itemize}
\item Le \cref{chp.induction} s'attarde sur les structures inductives. Celles-ci
  sont en quelque sorte l'exemple paradigmatique des objets syntaxiques,
  c'est-à-dire des objets que l'on peut traiter formellement. Il est en effet
  difficile d'imaginer sans autre information comment traiter des ensembles
  infinis (et que dire des ensembles infinis indénombrables), là où l'idée de
  manipuler un arbre binaire, ou une suite de $0$ et de $1$, est intuitivement
  acceptable.

  L'approche que nous développons ici utilise des ensembles d'arbres étiquetés~:
  pour définir un ensemble inductif, la structure même de l'ensemble nous permet
  d'utiliser nos objets sans ambigüité. Cette approche a l'avantage d'être plus
  intuitive, mais le désavantage d'être moins convaincante d'un point de vue
  formel. L'autre possibilité, que l'on peut retrouver chez différents auteurs,
  est de considérer les objets syntaxiques comme des mots particuliers,
  c'est-à-dire comme des suites finies de lettres, sur un alphabet spécifié par
  avance.

  Nous voyons dans ce chapitre comment raisonner sur des structures inductives,
  particulièrement comment définir des fonctions par récursion et prouver des
  propriétés par induction. Les définitions par inductions et le théorème de
  Knaster-Tarski sont en partie tirés de \cite{winskell1996formal}.
\item Le \cref{chp.logprop} décrit la logique propositionnelle comme une porte
  d'entrée vers la logique du premier ordre. D'une certaine façon, cette logique
  peut être vue comme une logique d'ordre $0$, et nous pensons que l'étude de ce
  cas particulier et restreint permet de faciliter l'étude qui suit, du calcul
  des prédicats. Cela nous permet de montrer différentes propriétés dans un
  contexte où la technicité est réduite, pour mieux en aborder l'essence.

  La logique propositionnelle est abordée dans la plupart des ouvrages traitant
  de la logique, puisque sujet incontournable~: citons par exemple
  \cite{cori2003logique} et \cite{raffalli:hal-00396917}.
\item Le \cref{chp.logpred} est la continuité du deuxième, et aborde le calcul
  des prédicats de la logique du premier ordre. Il introduit aussi des thèmes
  majeurs de la suite de l'ouvrage~: les modèles et les systèmes de
  démonstration du premier ordre. Ce chapitre peut être vu comme le plus
  indispensable de la partie préliminaire, puisque tous les chapitres qui
  suivent ont besoin de celui-ci. Les formules du premier ordre constituent le
  c\oe ur de la logique mathématique, d'où le soin particulier apporté à leur
  introduction et à leur traitement dans un chapitre dédié.

  Comme pour le chapitre précédent, le sujet est abordé dans la majorité des
  ouvrages.
\item Le \cref{chp.ordres} porte sur la théorie des ensembles ordonnés. Il
  permet d'introduire deux notions essentielles~: les structures ordonnées et
  les filtres. La première n'est pas directement utile dans les parties
  fondamentales, mais offre un point de vue riche pour étudier de nombreux
  phénomènes liés à la logique. Elle permet aussi, de façon plus pragmatique, de
  démontrer une version forte du théorème de Knaster-Tarski, déjà utilisé dans
  le premier chapitre. La deuxième notion, les filtres, est utile dans la
  première moitié de l'ouvrage. Elle joue un rôle dès le chapitre qui suit et
  permet d'interpréter de façon féconde certains aspects de théorie des
  ensembles, notamment pour traiter les clubs, mais elle est avant tout
  vitale pour la théorie des modèles, où les ultraproduits sont l'une des
  constructions les plus présentes.

  Pour l'étude des treillis et des notions connexes, \cite{cori2003logique}
  donne un point de vue succinct mais suffisant pour étudier les bases de la
  logique, tandis que \cite{lattice} fournit une étude approfondit de ces
  sujets. Citons aussi \cite{Rutherford1967-RUTITL} pour une introduction dédiée
  aux treillis et structures ordonnées.
\item Le \cref{chp.topo} aborde la topologie. Le langage topologique est
  effectivement important en logique, car il est éclairant voire directement
  utile. Un exemple est celui de la topologie sur les ordinaux, dont
  l'importance se révèlera en théorie des ensembles, mais qui est bien plus
  fluide en ayant déjà étudié les bases de la topologie générale.

  Nous y verrons les définitions élémentaires de topologie générale pour traiter
  de façon abstraite les espaces topologiques, d'une façon proche de la théorie
  des ensembles. Cela nous donnera l'occasion de voir le théorème de Tychonov,
  incontournable de la topologie générale. Nous traiterons aussi de la dualité
  de Stone, dans une version adaptée pour ne pas demander de connaissances en
  théorie des catégories~: comme pour la théorie des ordres, cette étude est une
  occasion de voir une correspondance pour sa profondeur plus que pour un simple
  intérêt direct, car elle permet de montrer que des phénomènes topologiques
  peuvent correspondre à des phénomènes logiques. Enfin, ce chapitre comporte un
  rappel des notions de topologie métriques nécessaires à la théorie descriptive
  des ensembles.

  Relevons cependant que ce chapitre n'est en aucun cas un cours de topologie
  pour un élève débutant, et cherche avant tout à fixer le vocabulaire et à
  offrir un point de vue différent, plus algébrique, à un lecteur ayant vu de la
  topologie métrique principalement, et sous un prisme plus analytique.

  La référence principale sur la topologie générale et l'utilisation des filtres
  dans son étude est \cite{bourbaki1971topologie}. A propos de la dualité de
  Stone, le sujet est traité dans \cite{DBLP:books/daglib/0093287} (avec une
  présentation des ensembles ordonnés qui peut aussi être utilisée pour
  approfondir le \cref{chp.ordres}). On peut aussi se référer à l'article
  original \cite{StoneDuality}. A propos de la topologie métrique, un cours de
  licence classique de topologie suffit, par exemple \cite{hassan2021topologie}
  qui aborde aussi la topologie générale.
\end{itemize}

\paragraph{Théorie des ensembles}
La théorie des ensembles, née des travaux de Cantor, est connue pour permettre
d'y formaliser l'ensemble des mathématiques. Même la partie préliminaire, donnée
avant cette partie, peut en fait y être exprimée. En ce sens, elle introduit
donc des outils particulièrement généraux qui peuvent être utiles dans d'autres
domaines. Nous mettons donc, dans un premier temps, l'accent sur ces notions
incontournables de la théorie des ensembles qui peuvent être utilisées dans
d'autres domaines, et donnons des bases pour tout lecteur voulant acquérir un
bagage minimal pour parler de théorie des ensembles. La deuxième partie, plus
spécialisée, décrit trois parties plus isolées de la théorie des ensembles pour
donner un panorama de ce que l'on peut étudier grâce à cette théorie. Les livres
utilisés comme références propres à cette partie sont \cite{krivine1998théorie},
\cite{dehornoy2017théorie} et \cite{DBLP:books/daglib/0067194}~: ceux-ci
abordent des notions dépassant le cadre de notre ouvrage, décrivant en
particulier la méthode du forcing.

La structure des chapitres est la suivante~:
\begin{itemize}
\item Le \cref{chp.axiomes} se concentre sur l'aspect le plus pratique de la
  théorie des ensembles, en montrant comment il est possible de reconstruire
  toutes les mathématiques en utilisant la théorie ZFC. Le début de ce chapitre
  est donc une description des axiomes de ZFC, pour pouvoir les manipuler.
  Plutôt que de simplement donner une liste de ces axiomes, nous les
  introduisons les uns après les autres pour permettre au lecteur de se
  familiariser avec chacun et avec ses quelques conséquences directes. Cela nous
  permet de donner par exemple le lemme d'effondrement de Mostovski, résultat
  important en théorie des ensembles. Le reste du chapitre est dédié à la
  construction des ensembles les plus basiques que nous manipulons, des entiers
  jusqu'aux réels.

  Les constructions ensemblistes sont considérées dès les ouvrages de Bourbaki,
  dans \cite{bourbaki2007théorie} par exemple. Cependant, la construction des
  entiers n'est pas celle utilisée ici (qui est due à Von Neumann). La
  construction des nombres réels à travers la complétion idéale est adaptée
  de la présentation faite dans \cite{DBLP:books/daglib/0093287}.
\item Le \cref{chp.ordinaux} décrit les ordinaux, les cardinaux et la
  cofinalité. Ces notions sont en quelque sorte les points de départ de la
  théorie des ensembles à proprement parler. Nous y aborderons aussi l'intérêt
  de l'axiome du choix, qui est énoncé dans le chapitre précédent mais dont
  l'importance apparait grâce au lemme de Zorn, se basant sur la récursion
  transfinie. L'axiome du choix est ensuite utilisé dans l'ensemble du chapitre,
  car celui-ci est nécessaire pour assurer un bon comportement à la hiérarchie
  cardinale. Nous utilisons les cardinaux comme moyen d'aborder l'hypothèse du
  continu et le problème de définir $2^\kappa$ en tant que fonction cardinale,
  motivant l'étude de la cofinalité. Le chapitre se conclut alors sur les
  conditions d'Easton pour définir une fonction correspondant à l'exponentielle
  cardinale.
\item Le \cref{chp.combi} étudie la combinatoire des ordinaux. Le premier
  objectif est d'introduire la notion d'ensemble stationnaire et d'offrir
  quelques applications élémentaires permettant de comprendre leur
  fonctionnement. Nous y étudions en préliminaire la topologie de l'ordre pour
  les ordinaux, puis voyons la notion de club ainsi que les ensembles
  stationnaires et non stationnaires.

  Une fois la notion de club introduite, les questions d'intérêt du chapitre
  sont de chercher à quel point le filtre des clubs est stable par des
  intersections de grandes familles et combien on peut construire d'ensembles
  stationnaires disjoints. L'étude de ces questions est l'occasion d'introduire
  le lemme de Fodor sur les fonctions régressives.

  Nous voyons pour finir le chapitre une application du lemme de Fodor à un
  point important du \cref{chp.ordinaux}~: la détermination de l'exponentielle
  cardinale. Plus précisément, nous y voyons une démonstration combinatoire du
  théorème de Silver, décrivant le comportement de la fonction
  $\kappa\mapsto 2^\kappa$ pour des cardinaux singuliers.
\item Le \cref{chp.modZFC} est consacré à l'étude des modèles de ZFC. Comme la
  portée de la partie de théorie des ensembles ne couvre pas la méthode du
  forcing, l'objectif est surtout de donner une idée du fonctionnement des
  modèles internes de $\ZFC$ et des premières preuves de cohérence relative qui
  en découlent.

  Ce chapitre est aussi l'occasion de parler du schéma de théorèmes de
  réflexion, qui est un résultat majeur. Pour pouvoir en parler au mieux, nous
  introduisons ce théorème en décrivant la notion d'absoluité, qui nous est
  utile durant tout le chapitre. L'absoluité est accompagnée d'une étude
  sommaire de la hiérarchie de Lévy, décrivant comment on peut trouver
  efficacement des notions absolues.

  Enfin, nous nous attardons sur l'univers constructible de Gödel, $\mathbb L$,
  qui nous permet de justifier que $\AxC$ (sous une version particulièrement
  forte) ainsi que $\HC$ (là encore, sous une version forte) sont relativement
  cohérents avec $\ZF$.
\item Le \cref{chp.desc} se concentre sur la théorie descriptive des ensembles.
  Cette théorie, à mi chemin entre la topologie générale et la théorie des
  ensembles, étudie les propriétés des espaces polonais. Nous n'allons voir que
  les bases de cette théorie, en présentant les propriétés essentielles des
  espaces polonais ainsi que l'étude de deux espaces polonais parmi les plus
  importants~: l'espace de Cantor et l'espace de Baire.

  Nous abordons d'abord les propriétés de stabilité des espaces polonais~: il
  existe plusieurs opérations qui permettent, à partir d'espaces polonais déjà
  identifiés, d'en construire de nouveaux. Cela nous offrira aussi l'occasion
  de définir la compactification d'Alexandrov et quelques termes de topologie
  générale.

  Le but de la deuxième partie du chapitre est de caractériser les espaces de
  Cantor et de Baire (respectivement $2^\mathbb N$ et $\mathbb N^\mathbb N$
  munis de leur topologie produit), ce qui passe par la définition des schémas
  de Cantor et de Lusin.
\end{itemize}

\paragraph{Théorie des modèles}
A FAIRE

\paragraph{Calculabilité}
La calculabilité est un domaine riche, né directement de la crise des
fondements. L'objectif premier derrière ce domaine est de définir (puis étudier)
la notion de fonction que l'on peut effectivement calculer. Sa naissance découle
directement du programme de Hilbert, qui a mené au besoin de définir ce qu'est
une méthode systématique pour résoudre un problème. Cette idée de méthode
systématique a donné naissance à notre idée moderne d'algorithme, et dont un
exemple typique est celui de la résolution d'équation du second degré (d'où
son nom, provenant de l'auteur Al Khawarizmi et en référence à son traité
d'algèbre dans lequel il introduit la première résolution systématique d'une
équation du second degré).

La notion de fonction calculable a été formalisée par trois formalismes
importants~: par les fonctions récursives de Gödel en 1934, par les machines de
Turing introduites par Alan Turing dans son article \cite{turing1936a}, et
par le $\lambda$-calcul introduit par Alonzo Church dans les années 1930. Ces
trois formalismes sont ensuite prouvés équivalents en 1938 (d'après
\cite{4568032}), donnant ce qu'on appelle la thèse de Church-Turing.

Cette partie dédiée à la calculabilité est scindée en deux sous-parties
principales~: la mise en place de la calculabilité, et son utilisation. Dans
la première sous-partie (\cref{chp.auto}, \cref{chp.alg} et A ECRIRE), nous
parlons de la hiérarchie de Chomsky qui décrit les formalismes de langages
plus faibles ou équivalents à celui des langages récursivement énumérables.
La volonté derrière ces premiers chapitres est de permettre au lecteur de se
familiariser progressivement avec les automates pour plus facilement comprendre
les machines de Turing, qui est un formalisme essentiel pour la calculabilité.
Les références principalement utilisées dans cette sous-partie sont
\cite{carton2008langages} et \cite{wolper2001introduction}.
La deuxième sous-partie se concentre sur l'utilisation de la calculabilité, et
son utilisation première dans l'histoire~: les théorèmes d'incomplétude de
Gödel. Cette sous-partie se terminer avec des versions générales des théorèmes
d'incomplétude de Gödel et de l'indéfinissabilité de Tarski, justifiant
\latinexpr{a posteriori} les utilisations de ces théorèmes dans le
\cref{chp.modZFC}. Pour cette sous-partie, nous utilisons comme référence
principale \cite{MoninPatey2022}.

La structure des chapitres est la suivante~:
\begin{itemize}
\item Le \cref{chp.auto} présente les bases du formalisme lié à la
  calculabilité~: la notion de mot, de langage et de classe de langage, ainsi
  que la notion d'automate. Ce chapitre s'attarde principalement sur les
  langages rationnels en en donnant différentes caractérisations et en
  manipulant des automates finis. L'objectif principal de ce chapitre est de
  montrer la richesse de la théorie des automates et les résultats
  incontournables de cette théorie, et d'utiliser cette théorie pour donner une
  vision concrète de la théorie des langages.

  La première partie du chapitre est dédiée à cette présentation du formalisme,
  en donnant les définitions de mot et de langage, et en profite pour donner un
  point de vue algébrique à travers la propriété universelle du monoïde libre et
  la définition de langage reconnu par un monoïde. On cherche dans ce chapitre à
  présenter avant tout les notations et usages de base pour faciliter la lecture
  du reste du chapitre et des chapitres suivants.

  La deuxième partie présente la classe des langages rationnels à travers les
  expressions rationnelles et les automates finis. Nous voyons ensuite
  l'équivalence entre les deux formalismes, en montrant aussi l'équivalence
  entre plusieurs notions d'automates, et des constructions sur les automates.

  Enfin, la dernière partie du chapitre décrit des constructions plus abstraites
  et plus fortes~: le théorème de Myhill-Nérode et l'équivalence entre les
  automates et les monoïdes. Le premier résultat présenté est le lemme de
  l'étoile, permettant de montrer facilement que certains langages ne sont pas
  rationnels, pour ensuite présenter le théorème de Mihyll-Nérode en tant que
  théorème plus strict. Le théorème est ensuite approfondi à travers la
  construction explicite de l'automate minimal et de la congruence de Nérode.
  Enfin, le chapitre se termine en présentant la construction du monoïde des
  transitions, montrant que les langages reconnus par des monoïdes finis sont
  eux aussi des langages rationnels.
\end{itemize}

Remarquons que la théorie du $\lambda$-calcul n'est pas abordé dans cette
partie. Elle l'est dans la partie suivante, où l'ajout du typage apporte un
sens plus profond aux programmes.

\paragraph{Théorie de la démonstration}
A FAIRE

Ce livre est un livre de cours se voulant autonome. Pour permettre au lecteur de
travailler et d'approfondir ce cours, nous proposons donc des exercices. Ces
exercices sont de deux types~: les exercices au sein d'un chapitre et les
exercices à la fin d'un chapitre.

Les exercices au sein d'un chapitre sont des exercices relativement simples,
souvent calculatoires, cherchant à faire pratiquer les bases des définitions.
Certains résultats dont la démonstration est en exercice sont utilisés pour des
démonstrations dans la suite, mais ce sont des résultats jugés suffisamment
naturels pour ne pas être remis en question même sans démonstration. Un lecteur
n'ayant donc pas l'envie de faire ces exercices n'aura pas de gros souci à
suivre le reste, mais il n'aura moins l'occasion de vérifier qu'il maîtrise les
bases du cours.

Les exercices en fin de chapitre sont en général plus techniques. Certains sont
même des problèmes entiers. Ceux-ci sont tirés de différentes sources, en
premier lieu de TD ou de DM vus pendant le cursus des auteurs.

Cher lecteur, il ne nous reste qu'à vous souhaiter une bonne lecture.
